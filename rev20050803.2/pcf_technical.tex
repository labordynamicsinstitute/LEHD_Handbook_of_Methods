%TCIDATA{LaTeXparent=0,0,sw-edit.tex}
                      

% -*- latex -*- 
%
% Time-stamp: <02/04/03 20:39:49 vilhuber> 
%              Automatically adjusted if using Xemacs
%              Please adjust manually if using other editors
%
% pcf_technical.tex
% Responsible: Martha
% Part of QWI_methods.tex

\index{PCF|(mainref}

As part of the initial processing of state UI\index{UI} files, a list of
unique PIKs\index{PIK} is given to PRED\index{PRED} and extracts from the
master PCF and CPR\index{CPR} GEO files are created.  To this point, there
has been no standardized way of obtaining these extracts.  Sometimes LEHD
has prepared the list of PIKs and sent them to PRED.  Other times, PRED has
prepared the list of PIKs themselves.


\section{Files received}
\label{sec:pcf:files_received}


After the extracts have been created, two sets of files are returned to
LEHD.  The first set of files have the prefix CPRGEO\index{CPRGEO} and the
second set of files have the prefix LEHDPCF\index{LEHDPCF}.  Usually there
are 20 data sets for each prefix (although there have been exceptions to
this general rule as well).  These files are registered and then put in the
\begin{center}
  /data/master/pcfcpr/received
\end{center}
and the appropriate sub-directory (date received).  They are then
symbolically linked to 
\begin{center}
  /data/master/pcfcpr/rawdata/
\end{center}
and are given iteration
numbers.  For example, the first time we received one large CPRGEO file, it
was called  CPRGEO\_01.  The next time we received extracts, they came in
pieces and were mapped to CPRGEO\_02\_01-CPRGEO\_02\_20.  Each time we receive
extracts, the files are mapped to the next iteration number.

After the files have been registered and symbolically linked to the
appropriate names, a program is run to merge the extracts to the master
CPRGEO and LEHDPCF files.  The master files contain a list of all PIKs ever
found in any state or SIPP\index{SIPP} or CPS\index{CPS} survey as well as
the relevant geography or SSA Numident\index{Numident} information.  The
master files contain flags to indicate the states and/or surveys in which
the PIK appeared.

\section{Program sequence}

The merging programs call one of three possible macros: 
\begin{itemize}
\item 00.flagcreate.sas,
\item 00.flagexist.sas, and 
\item 00.flagmerge.sas. 
\end{itemize}
These macros are written to handle three different possible formats for the
arriving data.  Each macros standardizes the data and creates the
appropriate state and/or survey flags.  The resulting data set is then
merged by PIK to the master data base.  A brief description of each macro
follows.

\begin{description}
\item{00.flagcreate.sas:} The assumption in this macro is that all the data come from the same state or survey and no flag exists.  The user 
gives the macro the following parameters: 
\begin{itemize}
\item \texttt{batch = iteration number} assigned to the files received from PRED and included in the raw data file name 
\item \texttt{type = CPRGEO or LEHDPCF} 
\item \texttt{rawdatalib = /data/master/cprpcf/rawdata/} (usually)
\item \texttt{master=  /data/master/cprpcf/sasdata/} (usually). Defines home of the master CPRGEO and
LEHPCF files 
\item \texttt{piksource = 2-digit FIPS state code/ CPS/ SIPP}.
 Identifier for the source of the PIK.
\item \texttt{merged = yes} if all 20 files have been combined into 1 file; otherwise no.
\end{itemize}

\item{00.flagmerge.sas:} In this macro, the assumption is that the data
come from one or more states and surveys and that flags exist on a
previously created data set which indicate the source of each PIK.  The
program follows the same steps as in 00.flagcreate.sas but the flag is
merged on instead of being created.

\item{00.flagexist.sas:} Here, the assumption is that the flag indicating the source of each PIK is already on the file.
\end{description}

In all three macros, the first part ensures the existence of the source
flag, either by creating or merging on the flag, or by using an already
existing flag.  The second part sorts the resulting data set and then
merges this file onto the master CPRGEO /LEHDPCF file.  All variables are
kept during this merging process.

For each extract, the programs \texttt{create\_cpr\_master.sas} and
\texttt{create\_pcf\_master.sas} are run.  The first number on the program refers to
the absolute order in which all the programs in
/data/master/cpfpcf/conversions were run.  The second number refers to the
program number and is repeated for each extract.  Program 01. is for the
\texttt{cpr\_master} creation and 02. is for the \texttt{pcf\_master} creation.

If the extracts are sent to LEHD in a format not currently handled in one
of the three macros described above, the user will have to create a new
macro which creates the source flag and merges the extract onto the LEHD
master files.



\index{PCF|)mainref}

%%% Local Variables: 
%%% mode: latex
%%% TeX-master: "QWI_methods"
%%% End: 

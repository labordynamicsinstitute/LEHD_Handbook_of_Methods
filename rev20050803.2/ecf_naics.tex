%%%  +--< LEHD-QWI ecf 3.1.49 2005-01-26 schwa305            >--+  %%%
%%%  +--< Location: /programs/production/dev1/current/ecf    >--+  %%%
%%%  +--< File: doc/new_variables_naics_etc.tex              >--+  %%%
% Converted from Microsoft Word to LaTeX
% by Chikrii SoftLab Word2TeX converter (version 2.4)
% Copyright (C) 1999-2001 Kirill A. Chikrii, Anna V. Chikrii
% Copyright (C) 1999-2001 Chikrii SoftLab.
% All rights reserved.
% http://www.word2tex.com/
% mailto: info@word2tex.com, support@word2tex.com
%\subsection{Subject: ECF Enhancements}
%
%\subsubsection{Date: 2/20/2004}
%
%Author: Kevin McKinney


%\documentclass [10pt]{article}
%\usepackage {longtable}
%\usepackage{fullpage}

%\begin{document}





\subsection{NAICS codes on the ECF}
\label{sec:ecf:naics}

Enhanced NAICS variables on the ECF  can be differentiated  by the source(s) and coding
system used in their creation. There are two sources of data -- the ES202 and
the BLS-created LDB --  and two coding systems for NAICS --  NAICS1997 and
NAICS2002.  Every NAICS variable uses at least one
source and one coding system.

The ESO (ES202-only) and FNL (final) variables are of primary importance to the user community. 
The ESO variables use  information from the ES202 exclusively and ignore any 
information that may be available on the LDB. We provide in  Section~\ref{sec:backcoding} an analysis on 
why this may be preferred. The FNL variables incorporate information from 
both the ES202 and the LDB, with the LDB being the primary source. The QWI
uses FNL variables for its NAICS statistics. Neither ESO nor FNL variables
contain missing values.


\subsubsection{NAICS algorithm precedence ordering}


Four basic sources of industry information are available on the ECF: NAICS
and NAICS{\_}AUX as well as SIC from ES202 records, and the LDB-sourced
NAICS{\_}LDB codes. The NAICS, NAICS{\_}AUX, and 
NAICS{\_}LDB, when missing (no valid 6-digit industry code), are imputed based on the  following
algorithm. SIC is filled similarly. Depending on the imputation used, a
$miss$ variable is defined, which is used in building the ESO and FNL variables.


\begin{enumerate}
\item Valid 6 digit industry code ($miss=0$)

\item Imputed code based on first 3,4, or 5 digits when no valid six digit code is 
available in another period ($miss=0$)

\item Imputed code based on contemporaneous SIC if SIC changed prior to 2000 
($miss=1.5$)

\item Valid 6 digit code from another period ($miss=2$)

\item Valid code from another source (for example if NAICS1997 is missing, 
NAICS2002 or SIC may be available) ($miss=3$)

\item Use SEIN mode value ($miss=5$ if contemporaneous modal value, $miss=7$ if
  the modal value stems from another time period)

\item Unconditional impute ($miss=6$ if only the SEIN-level modal value is
  imputed unconditionally, $miss=11$ if the SEIN-level value was
  unconditionally imputed and propogated to all SEINUNITs.)
\end{enumerate}



\subsubsection{ESO and FNL variables}



The ESO and FNL variables are made up of combinations of the various sources 
of industry information. The ESO variable uses the NAICS and NAICS{\_}AUX 
variables as input. Information from the variable with the lowest MISS value 
is preferred although in case of a tie the NAICS{\_}AUX value is used.


The FNL variable uses the ESO and LDB variables. Information from the 
variable with the lowest MISS value is preferred although in case of a tie 
the NAICS{\_}LDB value is used. Keep in mind that although the source of an 
ESO or FNL variable may be equal to NCS, the actual source can only be 
ascertained by going back to the original.



\subsubsection{LDB versus LEHD NAICS backcoding}
\label{sec:backcoding}

The LDB algorithm is to some extent a black box and testing has shown that 
it does a relatively poor job of capturing  industry changes of SEINs that 
occurred during the 1990s. In fact, the LDB appears to be a simple
backfill that does not take into account an SEIN's entire SIC history.

% Kevin: email 2005-01-28
% Somewhere along the line something I said was lost in translation.
% Actually the LDB NAICS codes are too smooth over time (they never change)
% relative to the variation found in SIC codes.  The LDB looks like a
% simple backfill that does not take into account a firm's entire SIC
% history (prior to ~1999).  I know that some of the SIC changes may be
% spurious, but I also feel that there is valuable information in a firm's
% SIC code history that I therefore attempted to use in my impute
% algorithm.  However, overall the effect of the different approaches is
% relatively small since very few firms change industry, especially
% relative to, for example, the proportion of firms that change location. 
% 

Although some of the SIC changes over time may be spurious, an SEIN's SIC
code history contains valuable information that we have attempted to
preserve in our imputation algorithm. Overall, the effect of the different
approaches is relatively small, since very few SEINs change industry, in
particular relative to the proportion of SEINs that change geography.


In the following, we present  a summary of research done on a comparison of
the ESO and FNL NAICS codes on the Illinois ECF. 
%
%
The LDB-sourced NAICS variable is used for about 85{\%} of the records for
Illinois, the rest are filled with information from the ES202. It is
unclear why only 85{\%} of ES202 records are in the LDB. The results
weighted by employment are about the same suggesting that activity was not
a criterion for being included on the LDB. 

First and not surprisingly, in
later years and quarters (1999+) when NAICS is actively coded by the
states, the ESO and FNL codes look almost identical when available.

Second, there is little variation in the LDB NAICS codes over time compared 
with SIC. Among all of the active SEIN-SEINUNITs over the period covered by
the Illinois data, 
%\marginpar{\tiny fill in with actual time period spanned}
only slightly more than  8{\%} experience at least one SIC change compared with about 1.5{\%} on 
the LDB. Almost all NAICS code changes occur after 1999. While this is not entirely 
unexpected, it is something to keep in mind when comparing NAICS{\_}FNL 
versus SIC or NAICS{\_}ESO employment totals. Many of these changes in 
industry appear to be real and are not captured on the LDB.

One effect of this is that as we go back in time a larger portion of 
employment can be found in NAICS{\_}FNL codes that are different than one 
would expect given the SIC code on the ECF. For example, in 1990 about 
13{\%} of employment is in a NAICS{\_}FNL code that is different than what 
we would expect based on the SIC. By 2001, the proportion of employment
that is in a NAICS code outside of the set of possible values predicted by
the SIC-NAICS crosswalk falls to 3{\%}. The 
ES202-based NAICS variable does a better job tracking SIC, since more SIC 
information is used in putting it together.
% (about 3{\%} consistently over the period).
%\marginpar{\tiny what is 3 percent}

The main source of the discrepancy is due to entities that experience a 
change in their SIC code prior to 2000. The LDB appears to ignore this 
change, while the ES202-based NAICS variable uses an SIC-based impute for these 
SEINUNITS. The result is a series that exhibits similar patterns of change 
over time as SIC, while still preserving the value added in the NAICS codes 
for entities that did not experience a change.

Also, users should keep in mind that for early years ($<1997$) some of the 
NAICS industries have yet to come into existence. The
prevalence of this problem has not been investigated yet.



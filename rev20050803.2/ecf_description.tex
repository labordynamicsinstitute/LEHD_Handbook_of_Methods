%TCIDATA{LaTeXparent=0,0,sw-edit.tex}

% -*- latex -*- 
%
% Time-stamp: <05/04/08 02:43:44 vilhuber> 
%              Automatically adjusted if using Xemacs
%              Please adjust manually if using other editors
%
% ecf.tex
% Responsible: Kevin 
% Part of QWI_methods.tex

The Employer Characteristics File (ECF\index{ECF}) consolidates most firm level
information (size, location, industry, etc.) into two easily accessible
files. The firm or SEIN-level file contains one record for every
year-quarter an SEIN is present in either the ES-202 or the UI, with more
detailed information available for the establishments of multi-unit SEINs in the 
SEINUNIT-level file. The SEIN file is built up from the SEINUNIT file and contains
no additional information, but should be viewed merely as an easier and/or
more efficient way to access SEIN level data.

A number of inputs are used to build the ECF. 
%
The ES202 data  is the primary input to the ECF file
creation process.  
%
UI data is  used to supplement information on the ES202,
in particular SEIN-level employment. UI data is also used to extend
published BLS county-level employment data, which is used to construct
weights for later use in the QWI process. Geocoded address information from the
%
GAL\index{GAL} file contributes latitude-longitude coordinates of most
establishments, as well as updated WIB\index{WIB} and MSA
information. 
%
BLS-provided Longitudinal Database (LDB) extracts as well as
LEHD-developed imputation mechanisms are used to backfill NAICS information
for periods in which NAICS was not collected. 
%
Finally, the QWI disclosure
mechanism is initiated in the ECF. 
%
We will describe in
Section~\ref{sec:ecf_processing}, while the details of the NAICS imputation
algorithm are described in Section~\ref{sec:ecf:naics}, and the entire
disclosure-proofing mechanism described in Section~\ref{cha:disclosure_proofing}.


\subsection{Constructing the ECF}
\label{sec:ecf_processing}

%\subsection{Program Overview}

ECF processing starts by stacking yearly  ES202 files. 
General and state specific consistency checks are then performed.
The COUNTY, NAICS, SIC, and EIN data are checked for invalid values. The 
%\marginpar{\tiny is this correct? was SIC, expanded to NAICS}
 check for industry codes goes beyond a simple validity check. If a
4-digit SIC code or 5-digit NAICS code is
present, but is not valid, then the industry code undergoes a conditional impute
based on the first 2 and 3 (SIC) or 3,4 and 5 (NAICS) digits.%
%
\footnote{Both NAICS 1997 and NAICS 2002 are used. The same procedure is
  later used for LDB data.}%
%
If the resulting codes are not valid, then the industry code is set to
missing, and  imputed at a later stage of processing.

Based on the EHF, SEIN-level quarterly employment and payroll totals are
computed. UI data is used as an imputation source for either payroll or
employment in the following situations:

\begin{itemize}
\item if ES202 employment is missing, but ES202 payroll is reported, then UI
employment is used. 
\item if ES202 employment is zero, then UI employment is {\em not} used,
  since this may be a correct report of zero employment for an existing
  SEIN. The situation may arise when bonuses or benefits were retro-actively paid, even
  though no employees were actively employed.
\item if ES202 payroll is zero and ES202 employment is positive, then UI
  payroll is used. 
\item if  ES202 payroll and employment are both zero or both missing, then
  UI payroll and employment are  used.
\end{itemize}

The ES202 data contains a ``master'' record for multi-unit SEINs, which is
removed after preserving information not available in the establishment
records.
% (I initially allocate the data equally to each establishment). 
Various inconsistencies in the record structure are also dealt with, such
as two records (master and establishment) appearing for a
single-unit. 
%
Initially, information from the master records is used to
impute missing data items for the establishments.  A flat prior is
used in the allocation process: each establishment is assumed to have equal
employment and payroll. This is improved upon later in the process.

The allocation process implemented above (master to establishments) does not
incorporate any information on the structure of the SEIN. To  improve on
this, SEINs that are missing firm structure for some periods, but reported a
valid multi-unit structure in other periods, are inspected. The absence of
information on firm structure typically occurs when an SEIN record is
missing due to a data processing error. A SEIN with a
valid multi-unit structure in a previous period is a candidate for
structure imputation. The  firm
structure is then imputed using the last available record with a multi-unit
structure. Payroll and employment are allocated appropriately.

From this point on, the firm structure (number of establishments per SEIN)
is defined for all periods. Geocoded data from the GAL is incorporated to
obtain precise geographic information on all establishments. 

%\marginpar{\tiny NAICS impute here?}
Geographic data, industry codes (SIC and NAICS) and EIN data
 from  time periods with valid data are used to fill  missing data
in other periods for the same establishment (SEINUNIT). If at least one
industry variable among the several sources (SIC, NAICS1997, NAICS2002,
LDB) has valid data, it is used to impute missing values in other fields.
%\marginpar{\tiny algorithm here?} 
Geography, if still missing, is imputed
conditional on industry, if available. Counties with larger employment in a
SEINUNIT's industry have a higher probability of being selected.

For SEINs, the (employment and establishment-weighted) modal values
of county, industry codes, ownership codes, and EIN are calculated for each
SEIN and year-quarter. SEIN-level records with missing data are filled in
with data from the closest time period with valid data.

At this point, if an SEIN mode variable has a missing value, then no
information was ever available for that SEIN. Additional attention is
devoted to  industry codes, which
are critical for QWI processing. 
%SIC is imputed based on the
%distribution of employment across 4-digit SIC within states. 
% A draw from a uniform distribution is used to impute SIC. SIC industries with a larger
%share of employment are more at risk for assignment than those with a
%relatively small share. The SIC code is then used to impute NAICS.  
% 
% from Kevin
%
SIC and NAICS are randomly imputed with probability proportional to the
state-wide share of employment in 4-digit SIC code or 5-digit NAICS
code. SIC and NAICS codes with a larger share of employment have a higher
probability of selection. 
%
If an industry code is imputed, it is done so once for each SEIN and
remains constant across time. These industry codes are then propagated to
all SEINUNITs as well.

With most data items complete, weights are calculated. These weights are
discussed in the section on QWI (Section~\ref{sec:aggregate}). Furthermore,
the disclosure proofing is also prepared at the SEIN and SEINUNIT
level. This is discussed in detail in Section~\ref{sec:confidentiality}. 


\subsection{Imputations in the ECF}
\label{sec:ecf:impute}

Many data items, when missing, are imputed. The following is a summary list
of such imputations. Imputations can be of two types: algorithmic -- data
closest in time is copied into the  missing data items -- and probabilistic
-- the data is drawn from an empirical distribution, conditional on a
maximum of available information.

\begin{itemize}
\item Employment and payroll: can be imputed based on information in the
  SEIN master record, or based on information computed at the SEIN-level from UI
  data. Imputation is always algorithmic - no employment or payroll is ever
  imputed through probabilistic methods. 
\item Firm structure (relative size of establishments) can be imputed based
  on reported firm structure in other periods. Imputation is always algorithmic.
\item Geography, industry codes, ownership, and EIN are imputed
  algorithmically first, if possible.
\item Geography, if still missing, is imputed conditional on industry, if
  available, and unconditionally otherwise. Counties with more employment
  in an SEINUNIT's industry have a higher probability of being selected.
\item Industry codes are imputed probabilistically based on empirical
  correspondence tables conditional on the same unit's observed other
  industry data items. For instance, if SIC is missing, but NAICS1997 is
  available, the relative observed distribution of SIC-NAICS1997 pairs is
  used to impute the missing data item.
\item If all previous imputation mechanisms fail, SIC is imputed
  unconditionally based on the observed distribution of within-state
  employment across SIC industries. Once SIC is assigned, the previous
  conditional imputation mechanisms are again used to impute other industry
  data items.
\end{itemize}


%%% Local Variables: 
%%% mode: latex
%%% TeX-master: "qwi-overview"
%%% End: 

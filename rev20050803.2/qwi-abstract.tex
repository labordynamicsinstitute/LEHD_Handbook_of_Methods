%TCIDATA{LaTeXparent=0,0,sw-edit.tex}
% -*- latex -*-
% Time-stamp: <05/08/03 18:02:39 vilhuber>


The Longitudinal Employer Household Dynamics program at the U.S. Census
Bureau, with funding from several national funding agencies, has built a
set of infrastructure files using administrative data provided by state
agencies, enhanced with information culled from demographic and economic
(business) surveys and censuses. The LEHD Infrastructure Files provide a
detailed and comprehensive picture of workers, employers, and their
interaction in the U.S. economy. Building on this infrastructure, the
Quarterly Workforce Indicators (QWI), a new dataseries published since 2003
by the U.S. Census Bureau, are computed. The QWI offer unprecedented detail
on the local dynamics of labor markets. Despite the fine detail,
confidentiality is maintained due to the application of state-of-the-art
confidentiality protection methods. This article describes how the input
files are compiled and combined to create the infrastructure files. The
multiple imputation mechanisms that are used to fill in missing data, and
the statistical matching techniques used to combine data where a direct
match is not possible are both crucial to the success of the final product,
and described in detail here. Finally, special attention is paid to the
confidentiality protection mechanisms used to hide the identity of the
underlying entities in the final published data. A brief description of
public-use and restricted-access data files is also provided, with pointers
to further documentation for researchers  interested in using these data.


%%% Local Variables: 
%%% mode: latex
%%% TeX-master: "qwi-overview"
%%% End: 

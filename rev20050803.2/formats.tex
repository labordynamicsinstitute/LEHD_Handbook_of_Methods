%TCIDATA{LaTeXparent=0,0,QWI_methods.tex}
\newcommand{\mypath}{.}
\newcommand{\thisdoc}{QWI_methods}


%-------------------- Packages --------------------
\usepackage{natbib}    % for bibliography
%\newcommand{\mybibstyle}{elsart-harv} 
%\usepackage{harvard}     % format of bibliography
\newcommand{\mybibstyle}{dcu} % a harvard style 
                       %define bibliography to use
\usepackage{sas}       %required for SAS-produced tables
\usepackage {amssymb}  % AMS symbols
\usepackage {amsmath}  % AMS math
\usepackage{amsfonts}
\usepackage{pstricks}  % adds color to Postscript
\usepackage {graphicx} % for inclusion of EPS graphics
\usepackage {multicol} % Multicolumn formatting (in-line tables)
\usepackage{fullpage}  % enlarged margins
\usepackage{rotating}  % allows for rotated tables
\usepackage{epsfig}    % legacy package for inclusion of EPS
\usepackage{varioref}  % allows for references like ``Table x on page y''
\usepackage{float}     % ?
\usepackage{array}     % Modifications of tabular environment
%\usepackage{doublespace} % if double lines wanted
\usepackage{longtable} % allows for tables to wrap across pages.
\usepackage{supertabular} % allows for tables to wrap across pages.
\usepackage{fancyhdr}  % provides pagestyle fancy (page headers)
\usepackage{tocbibind} % provides for bibliography, List of tables, etc. to 
                       % be included in the Table of contents.
\usepackage{color}
\usepackage{theorem}
\usepackage{layout}
\usepackage{makeidx}
\usepackage{acronym}
%\usepackage{index}     % allows for multiple indexes
\makeindex

% defining colors for package hyperref
\definecolor{myblue}{rgb}{0,.2,1}

\usepackage{html}
\usepackage{hyperref}

\hypersetup{%
backref=true,%
naturalnames=true,%
bookmarksnumbered=true,%
bookmarksopen=false,%
plainpages=true,%
colorlinks=true,%
urlcolor=myblue,
linkcolor=myblue,%
filecolor=myblue,%
citecolor=black,%
pagecolor=myblue,%
pdftitle={\mytitle},%
pdfpagemode=UseOutlines,%
pdfauthor={\myauthors},%
pdfsubject={\myshorttitle}}%
%pdfstartview=FitW%
%dvips%
%pdftex
%latex2html%

%-------------------- choose the font here --------------------
\usepackage{times}
%\usepackage{newcent}
%\usepackage{helvet}
%\usepackage{helvetic}
%\usepackage{ncntrsbk}
%\usepackage{bookman}
%\usepackage{avantgar}




%-------------------- formatting of headers --------------------
\newrgbcolor{myblue}{0 .5 1}
\newrgbcolor{myred}{1 .1 0}

%\pagestyle{fancy}
%\fancyhead[LO,RE]{{\small \it \myred U.S. Bureau of the Census preliminary document and data:\\ not for attribution, publication, or redistribution}}
%\fancyhead[RO,LE]{{\small \it LEHD\\ \myshorttitle}}
%\fancyfoot[RO,LE]{\small \it - \thepage -}
%\fancyfoot[CO,CE]{\ \ }
%\fancyfoot[LO,RE]{\small \it \myversion}

\renewcommand{\headrulewidth}{0.4pt}
\renewcommand{\footrulewidth}{0.4pt}
\newcommand{\figurewidth}{\textwidth pt}

%-------------------- formatting of table of contents --------------------
\setcounter{tocdepth}{3}
\setcounter{secnumdepth}{3}

%-------------------- formatting of abbreviation index --------------------
\newcommand{\aindex}[1]{#1\index{#1}} % prints the name at the same time
\newcommand{\mainref}[1]{\textbf{\textit{\hyperpage{#1}}}}
\newcommand{\mindex}[1]{\index{#1|mainref}} % defines a main definition
\newcommand{\Mindex}[1]{#1\mindex{#1}}      % defines a visible main
\setlength{\columnsep}{60pt} % this generally resets the space between
                             % columns, but here in particular for the Index
% this next command redefines paragraph to put all paragraph entries into
% the index
%\renewcommand\paragraph{\@startsection{paragraph}{4}{\z@}%
%                                    {3.25ex \@plus1ex \@minus.2ex}%
%                                    {-1em}%
%                                    {\normalfont\normalsize\bfseries}}
% after having run the document, run 
%        makeindex -s QWI_methods.ist QWI_methods.idx


%-------------------- formatting of page layout --------------------
%\setlength{\baselineskip}{.8\baselineskip}
\setlength{\parskip}{0\baselineskip}
%\hyphenation{un-em-ploy-ment em-ploy-ment}

%-------------------- formatting of fancy references --------------------
%\newcommand{\Cite}{\citeasnoun} % this for harvard
\newcommand{\Cite}{\citet} % this for natbib

%\newcommand{\Vref}{\vref} % this refernces with page 
%\newcommand{\Vref}{\ref} % this refernces without page 
\newcommand{\tablefontsize}{\small}
\newcommand{\dotline}[1]{\multicolumn{#1}{c}{\dotfill}\\}
\newcommand{\Hline}{\dotline}
\newcolumntype{M}{>{$}c<{$}}
\newcolumntype{R}{>{$}r<{$}}

\newcommand{\updating}{
\begin{center}
 \fbox{\rule[-5mm]{0cm}{10mm}
\  Details being updated and currently not available.\ }
\end{center}
}

\theoremstyle{plain}
\theorembodyfont{\normalfont}
\theoremheaderfont{\itshape}
\newtheorem{identity}{Identity}
\newenvironment{steps}{\renewcommand{\theenumi}{Step \arabic{enumi}}%
  \begin{enumerate}}{\end{enumerate}\renewcommand{\theenumi}{\arabic{enumi}.}}
\newenvironment{notes}{\renewcommand{\theenumi}{Note \arabic{enumi}}%
  \begin{enumerate}}{\end{enumerate}\renewcommand{\theenumi}{\arabic{enumi}.}}

%-------------------- macros from TCILATEX.TEX --------------------------
% macros for user - defined functions
\def\limfunc#1{\mathop{\rm #1}}%
\def\func#1{\mathop{\rm #1}\nolimits}%
% macro for unit names
\def\unit#1{\mathop{\rm #1}\nolimits}%


%%% Local Variables: 
%%% mode: latex
%%% TeX-master: "qwi-overview"
%%% TeX-master: "qwi-overview"
%%% End: 

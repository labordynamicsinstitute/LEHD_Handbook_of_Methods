%TCIDATA{LaTeXparent=0,0,sw-edit.tex}

% -*- latex -*- 
%
% Time-stamp: <05/03/15 14:21:22 vilhuber> 

% publicly available files

In this section, we describe the publicly available files, and how they
differ from their internal correspondent files.


\section{Public use files}
\label{sec:public:public-use}

The only public use product currently available on a regular basis are the
QWI files proper. The public-use version differs from the Census-internal
versions only in that the public-use version has been subject to the
disclosure-proofing methods (coarsening and suppression) described in a
previous section. 


\section{Restricted-access files}
\label{sec:public:restricted}

A larger set of files are available within the protected environment
provided by the Census Research Data Centers (RDCs). The only information
missing on these files relative to their internal-use counterparts is any
information related to the confidential portions of the disclosure-proofing
methods. All of these files can be accessed for research purposes by
submitting a research proposal to the Center for Economic Studies at the
U.S. Census Bureau%
%
\footnote{\htmladdnormallink{http://www.ces.census.gov}{http://www.ces.census.gov}}%
%
.


\subsection{ECF}
\label{sec:LEHD-ECF}

The version of the ECF available in the RDC environment, referred to as
'LEHD-ECF'\index{LEHD-ECF|ECF}, differs only minimally from the internal
use ECF. Only variables used in the disclosure-proofing of the QWI have
been suppressed. More information, including a detailed description of the
LEHD-ECF, is available on the CES
website \citep{ip-lehd-ecf}.

\subsection{Unit Flow Files - Firm-level QWI}
\label{sec:LEHD-QWI}

The SEINUNIT-level input files to the final aggregation step of the QWI,
internally referred to as UFF(b)\index{UFF}, is available in the RDC
environment under the reference 'LEHD-QWI'\index{LEHD-QWI|QWI}. The actual
state-specific file is called {\tt qwi\_{STATE}\_SEINUNIT}. While the
internal-use version contains all information necessary to compute the
disclosable QWI statistics, these variables have been suppressed from the
RDC version. All statistics available at aggregated levels in the
public-use QWI are available on the LEHD-QWI for the establishment. More
information is available on the CES website.

\subsection{Business Register Bridge}
\label{sec:BRB}
\label{sec:LEHD-BRB}
The Census Bureau maintains a list of establishments 
to develop the frame for economic censuses and surveys. This list is called
the  Business Register (BR), and is updated annually. 
 The BR contains very reliable information on business 
identifiers, business organizational structure, and business
location. Unfortunately,  the establishment identification system for the Business Register 
differs from the LEHD establishment identifier (SEINUNIT). As a consequence, there is no 
single best way to form linkages between these data sources. 

The LEHD Business Register Bridge (LEHD-BRB) available in the RDC network
provides several ways to integrate the economic censuses and surveys with
LEHD-provided data. The choice of link record is left to researchers, and
the optimal choice will depend on the research objective. Available
identifiers on the LEHD-BRB are the EIN, geographic information, and
4-digit SIC, which are linked to SEIN and SEINUNIT at different levels of
precision. A more detailed guide is available on the CES website \citep{ip-lehd-brb}.

\subsection{Human Capital files}

These files will contain firm-level distributions of human capital measures
as initially developed in \citet{tp-2002-09}. It will become available in 2005.

%%% Local Variables: 
%%% mode: latex
%%% TeX-master: "qwi-overview"
%%% End: 

%TCIDATA{LaTeXparent=0,0,sw-edit.tex}
                      
%
% Time-stamp: <05/04/06 13:48:43 vilhuber>


%\section{Post-processing of UI and ES202 files}
%\label{sec:post-processing}

Once received, the UI and ES202 files are standardized.%
%
\footnote{%
The
ES202 files in particular have been received in a bewildering array of physical file
layouts and formats, reflecting the wide diversity in computer systems
installed in state agencies.}%
%
The UI files have been edited for longitudinal
consistency, and the SSN replaced by the Protected Identification Key
(PIK). Beyond that, no further processing has occurred.

The core Infrastructure Files are built from the core input files, and
augmented from a large number of additional Census-internal demographic and
economic (firm) surveys and censuses. 
%
The Employment History File (EHF\index{EHF}) provides a full time-series of
 earnings at all within-state jobs for all time periods covered by the
LEHD data, and activity calendars at a job, SEINUNIT, and SEIN level. 
%
The Individual Characteristics File
(ICF\index{ICF}) provides time-invariant personal characteristics and
some address information.%.
%
\footnote{A time-varying variant of the ICF is under development.}%
%
The Employer Characteristics File (ECF\index{ECF}) provides a complete 
database of firm and establishment characteristics, most of which are
time-varying. 
%
It includes a subset of the data available on the Geocoded Address List
(GAL\index{GAL}), which contains geocoded at the block-level  and
latitude/longitude coordinates for addresses from a large set of
administrative and survey data.
%
We will describe each in detail.


\section{Employment History File\label{cha:ehf}: EHF}
%TCIDATA{LaTeXparent=0,0,sw-edit.tex}

% -*- latex -*- 
%
% Time-stamp: <05/04/05 10:29:28 vilhuber> 
%              Automatically adjusted if using Xemacs
%              Please adjust manually if using other editors
%
% ehf.tex
% Responsible: Paul
% Part of QWI_methods.tex


%\subsection{Employment History File}
\label{sec:ehf_overview}
%\label{cha:ehf}


The \textit{Employment History File} (EHF)\mindex{EHF}%
\index{Employment History File|see{EHF}}
%\footnote{%
%See Appendix~\Vref{app:ehf_technical} for detailed information on the
%creation of the EHF.} 
is designed to store the complete in-state work
history for each individual that appears in the UI%
\index{UI} wage records. The EHF for each state contains one record for each
employee-employer combination~-- a job\index{job} --  in that state in each
year.
% Every individual who is employed during a given year will then have one observation per
%employer for that year.
  Both annual and quarterly earnings variables are available in
the EHF. Individuals who never have strictly positive earnings (a
theoretical possibility) are dropped. 

A re-ordering of the data into one observation per
job, with all quarterly earnings and activity records available within one record, is
also available (Person History File, PHF\index{PHF}). Activity is defined
as active employment within a quarter, requiring a strictly positive value
for quarterly earnings. A similar time-series of activity at the SEINUNIT level (UNIT History
File, UHF\index{UHF}) and the SEIN level (SEIN History File,
SHF\index{SHF}) is also computed at this time. 

A comparison of the earnings and employment information from the UI and
ES202 files is one of the core quality measures that are computed. Large
discrepancies are highlighted, and clarified with the data provider. Often,
a corrected data file can be imported into the LEHD system. Not
all data discrepancies can be easily resolved. In particular the historical
data sometimes are not correctable, because the data has been lost or
corrupted.%.
%
\footnote{A future extension currently being developed will allow to apply
imputation models to correct for large discrepancies.}
%


%%% Local Variables: 
%%% mode: latex
%%% TeX-master: "qwi-overview"
%%% End: 


%
%

\label{sec:files}
\section{Individual Characteristics File\label{cha:icf}: ICF}
%TCIDATA{LaTeXparent=0,0,sw-edit.tex}

% -*- latex -*- 
%
% Time-stamp: <05/04/06 23:54:19 vilhuber> 
%              Automatically adjusted if using Xemacs
%              Please adjust manually if using other editors
%
% icf.tex
% Responsible: Fredrik, Martha
% Part of QWI_methods.tex

%\subsection{Individual Characteristics File}
\label{sec:icf_overview}
%\label{cha:icf}

The \textit{Individual Characteristics File }(ICF)%
\mindex{ICF}\index{Individual Characteristics File|see{ICF}}
for each state contains one record for every
person who is ever employed in that state over the time period spanned by
the state's unemployment insurance records. 

The ICF is constructed in the following manner. First, the universe of
individuals is defined by compiling the list of unique PIKs from the
EHF. Demographic information from the PCF is then merged on by PIK, and
records without a valid match flagged. PIK\index{PIK}-survey identifier
crosswalks link the CPS and SIPP ID variables into the ICF, and gender and
age information from the CPS is used to complement and verify the
PCF-provided information. 
%\marginpar{\tiny Talk  about validated/non-validated crosswalks? Coverage per year?}

\subsection{Age and gender imputation}
\index{imputation!age}
\index{imputation!gender}

Approximately 3{\%} of the PIKs\index{PIK} found in the UI\index{UI}
wage records do not match to the PCF\index{PCF} file. Multiple imputation
methods are used to assign \index{missing values} date of birth and gender
to these individuals. To impute gender, the probability of being male is
estimated using a state-specific logit
model\index{imputation!logit}\index{logit}:

\begin{equation}
  \label{eq:logit:gender}
  P(male)=f(X_{is}\beta_s)
\end{equation}
where $X_{is}$ contains a  full set of yearly log earnings and squared log
earnings, and full set of employment indicators covering time period spanned
by the state's records, for each individual $i$ with strictly positive earnings
within state $s$ and non-missing PCF gender. The state-specific $\hat{\beta}_s$ as estimated from Equation
(\ref{eq:logit:gender}) is then used to predict the probability of
being male for individuals with missing gender  within state $s$, and gender is assigned as
\begin{equation}
  \label{eq:impute:gender}
  \text{male}~\text{if}~X_{is}\hat{\beta}_s \geq \mu_l 
\end{equation}
where $\mu_l \sim U[0,1]$ is one of  $l = 1,\dots,10$ independent draws
from the distribution. Thus, each individual with missing gender is
assigned ten independent implicates.

The  imputation of date of birth is done in a similar fashion using a multinomial logit
to predict the probability of being in one of eight age categories and then
assigning an age based on this probability and the distribution of ages
within the category. Again, the imputation occurs ten times.

It should be noted that if an individual is missing gender or age in the
PCF, but not in the CPS, then the CPS values are used, not the imputed
values. Also, before the imputation model for date of birth is implemented,
basic editing of the date of birth variable takes place to account for
obvious coding errors, such as a negative age at the time when UI earnings
is first reported for the individual. In those relatively rare cases where
the date of birth information is deemed unrealistic it is set to missing
and instead imputed based on the model described above.

\subsection{Place of residence imputation}
\label{sec:icf:place_impute}
\index{imputation!residence}

Place of residence information on the ICF is derived from the StARS
(Statistical Administrative Records System), which for the vast majority of
the individuals found in the UI wage records contains information on the
place of residence down to the exact geographical coordinates. However, in
some 10 
%[LARS: THIS NUMBER IS A ROUGH ESTIMATE BASED ON MY RECOLLECTION.
%CHECK WITH PRODUCTION FOR MORE PRECISION] 
percent of all cases this information is incomplete or missing. In
particular the  QWI computation relies on completed place of residence
information is because this is a key conditioning variable in the
unit-to-worker (U2W) imputation model (see Section~\ref{sec:u2w}). 

%In its current version the unit-to-worker
%imputation model needs place of residence information as of 1999 and,
%therefore, the completed place of residence information on the ICF reflects
%the same year, even though longitudinal information at an annual frequency
%is available from StARS.


County of residence is imputed based on a categorical model of data that
can be represented by a contingency table. In particular, separately for
each state, unique combinations of categories of gender, age, race, income
and county of work are used to form $i=1,\ldots ,I$ populations. For each
sample $i$, the probability of residing in a particular county as of 1999,
$\pi_{ij}$, is estimated by the sample proportion, $p_{ij}=n_{ij}/n_{i}$,
where $j=1,\ldots ,J$ indexes all the counties in the state plus an extra
category for out-of-state residents.

County of residence is then imputed based on
%
\begin{eqnarray}
  \label{eq:county_impute}
  county = j &\text{if}& P_{ij - 1} \le u_k < P_{ij} 
\end{eqnarray}
%
where $P_i $ is the CDF corresponding to $p_i $ for the $i$th population
and $\mu _{kl} \sim U[0,1]$ is one of $k = 1,\ldots,10$ independent draws for
the $l$th individual belonging to the $i$th population.

In its current version no geography below the county level is imputed and
in those cases where exact geographical coordinates are incomplete the
centroid of the finest geographical area is used. Thus, in cases where no
geography information is available this amounts to the centroid of the
imputed county. Geographical coordinates are not assigned to individuals
whose county of residence has been imputed to be out-of-state.

%
%
%09{\_}icf.sas - 15{\_}icf.sas: PCF, CPR match, who is imputed, what about
% people who move, precision of TIGER match.

\subsection{Education imputation}
\index{imputation!education}

%[ FREDRIK: describe, please. Verbal plus imputation model ]
%16{\_}icf.sas- 19{\_}icf.sas

The imputation model for education relies on a statistical match between
the Decennial Census 1990 and LEHD data. The probability of belonging to
one of 13 education categories is estimated using 1990 Decennial data
conditional on characteristics that are common to both Decennial and LEHD
data, using a state-specific logit
model\index{imputation!logit}\index{logit}:

\begin{equation}
  \label{eq:logit:educat}
  P(educat)=f(Z_{is}\gamma_s)
\end{equation}
where $Z_{is}$ contains age categories, earnings categories, and industry
dummies for individuals age 14 and older in the 1990 Census Long Form
residing in the state being estimated, and who reported strictly positive
wage earnings.

Education is then imputed based on
\begin{eqnarray}
  \label{eq:impute:educat}
  \text{educat}=j &~\text{if} & {cp}_{j-1} \leq \mu_l < {cp}_{j} 
\end{eqnarray}
where $cp_{j}=Z_{is}\hat{\gamma}_s$ and $\mu_l \sim U[0,1]$ is one of  $l =
11,\dots,20$ independent draws, and $i \in EHF$.


%%% Local Variables: 
%%% mode: latex
%%% TeX-master: "qwi-overview"
%%% End: 


%
%
\section{The Employer Characteristics File\label{cha:ecf}: ECF}
%TCIDATA{LaTeXparent=0,0,sw-edit.tex}

% -*- latex -*- 
%
% Time-stamp: <05/04/08 02:43:44 vilhuber> 
%              Automatically adjusted if using Xemacs
%              Please adjust manually if using other editors
%
% ecf.tex
% Responsible: Kevin 
% Part of QWI_methods.tex

The Employer Characteristics File (ECF\index{ECF}) consolidates most firm level
information (size, location, industry, etc.) into two easily accessible
files. The firm or SEIN-level file contains one record for every
year-quarter an SEIN is present in either the ES-202 or the UI, with more
detailed information available for the establishments of multi-unit SEINs in the 
SEINUNIT-level file. The SEIN file is built up from the SEINUNIT file and contains
no additional information, but should be viewed merely as an easier and/or
more efficient way to access SEIN level data.

A number of inputs are used to build the ECF. 
%
The ES202 data  is the primary input to the ECF file
creation process.  
%
UI data is  used to supplement information on the ES202,
in particular SEIN-level employment. UI data is also used to extend
published BLS county-level employment data, which is used to construct
weights for later use in the QWI process. Geocoded address information from the
%
GAL\index{GAL} file contributes latitude-longitude coordinates of most
establishments, as well as updated WIB\index{WIB} and MSA
information. 
%
BLS-provided Longitudinal Database (LDB) extracts as well as
LEHD-developed imputation mechanisms are used to backfill NAICS information
for periods in which NAICS was not collected. 
%
Finally, the QWI disclosure
mechanism is initiated in the ECF. 
%
We will describe in
Section~\ref{sec:ecf_processing}, while the details of the NAICS imputation
algorithm are described in Section~\ref{sec:ecf:naics}, and the entire
disclosure-proofing mechanism described in Section~\ref{cha:disclosure_proofing}.


\subsection{Constructing the ECF}
\label{sec:ecf_processing}

%\subsection{Program Overview}

ECF processing starts by stacking yearly  ES202 files. 
General and state specific consistency checks are then performed.
The COUNTY, NAICS, SIC, and EIN data are checked for invalid values. The 
%\marginpar{\tiny is this correct? was SIC, expanded to NAICS}
 check for industry codes goes beyond a simple validity check. If a
4-digit SIC code or 5-digit NAICS code is
present, but is not valid, then the industry code undergoes a conditional impute
based on the first 2 and 3 (SIC) or 3,4 and 5 (NAICS) digits.%
%
\footnote{Both NAICS 1997 and NAICS 2002 are used. The same procedure is
  later used for LDB data.}%
%
If the resulting codes are not valid, then the industry code is set to
missing, and  imputed at a later stage of processing.

Based on the EHF, SEIN-level quarterly employment and payroll totals are
computed. UI data is used as an imputation source for either payroll or
employment in the following situations:

\begin{itemize}
\item if ES202 employment is missing, but ES202 payroll is reported, then UI
employment is used. 
\item if ES202 employment is zero, then UI employment is {\em not} used,
  since this may be a correct report of zero employment for an existing
  SEIN. The situation may arise when bonuses or benefits were retro-actively paid, even
  though no employees were actively employed.
\item if ES202 payroll is zero and ES202 employment is positive, then UI
  payroll is used. 
\item if  ES202 payroll and employment are both zero or both missing, then
  UI payroll and employment are  used.
\end{itemize}

The ES202 data contains a ``master'' record for multi-unit SEINs, which is
removed after preserving information not available in the establishment
records.
% (I initially allocate the data equally to each establishment). 
Various inconsistencies in the record structure are also dealt with, such
as two records (master and establishment) appearing for a
single-unit. 
%
Initially, information from the master records is used to
impute missing data items for the establishments.  A flat prior is
used in the allocation process: each establishment is assumed to have equal
employment and payroll. This is improved upon later in the process.

The allocation process implemented above (master to establishments) does not
incorporate any information on the structure of the SEIN. To  improve on
this, SEINs that are missing firm structure for some periods, but reported a
valid multi-unit structure in other periods, are inspected. The absence of
information on firm structure typically occurs when an SEIN record is
missing due to a data processing error. A SEIN with a
valid multi-unit structure in a previous period is a candidate for
structure imputation. The  firm
structure is then imputed using the last available record with a multi-unit
structure. Payroll and employment are allocated appropriately.

From this point on, the firm structure (number of establishments per SEIN)
is defined for all periods. Geocoded data from the GAL is incorporated to
obtain precise geographic information on all establishments. 

%\marginpar{\tiny NAICS impute here?}
Geographic data, industry codes (SIC and NAICS) and EIN data
 from  time periods with valid data are used to fill  missing data
in other periods for the same establishment (SEINUNIT). If at least one
industry variable among the several sources (SIC, NAICS1997, NAICS2002,
LDB) has valid data, it is used to impute missing values in other fields.
%\marginpar{\tiny algorithm here?} 
Geography, if still missing, is imputed
conditional on industry, if available. Counties with larger employment in a
SEINUNIT's industry have a higher probability of being selected.

For SEINs, the (employment and establishment-weighted) modal values
of county, industry codes, ownership codes, and EIN are calculated for each
SEIN and year-quarter. SEIN-level records with missing data are filled in
with data from the closest time period with valid data.

At this point, if an SEIN mode variable has a missing value, then no
information was ever available for that SEIN. Additional attention is
devoted to  industry codes, which
are critical for QWI processing. 
%SIC is imputed based on the
%distribution of employment across 4-digit SIC within states. 
% A draw from a uniform distribution is used to impute SIC. SIC industries with a larger
%share of employment are more at risk for assignment than those with a
%relatively small share. The SIC code is then used to impute NAICS.  
% 
% from Kevin
%
SIC and NAICS are randomly imputed with probability proportional to the
state-wide share of employment in 4-digit SIC code or 5-digit NAICS
code. SIC and NAICS codes with a larger share of employment have a higher
probability of selection. 
%
If an industry code is imputed, it is done so once for each SEIN and
remains constant across time. These industry codes are then propagated to
all SEINUNITs as well.

With most data items complete, weights are calculated. These weights are
discussed in the section on QWI (Section~\ref{sec:aggregate}). Furthermore,
the disclosure proofing is also prepared at the SEIN and SEINUNIT
level. This is discussed in detail in Section~\ref{sec:confidentiality}. 


\subsection{Imputations in the ECF}
\label{sec:ecf:impute}

Many data items, when missing, are imputed. The following is a summary list
of such imputations. Imputations can be of two types: algorithmic -- data
closest in time is copied into the  missing data items -- and probabilistic
-- the data is drawn from an empirical distribution, conditional on a
maximum of available information.

\begin{itemize}
\item Employment and payroll: can be imputed based on information in the
  SEIN master record, or based on information computed at the SEIN-level from UI
  data. Imputation is always algorithmic - no employment or payroll is ever
  imputed through probabilistic methods. 
\item Firm structure (relative size of establishments) can be imputed based
  on reported firm structure in other periods. Imputation is always algorithmic.
\item Geography, industry codes, ownership, and EIN are imputed
  algorithmically first, if possible.
\item Geography, if still missing, is imputed conditional on industry, if
  available, and unconditionally otherwise. Counties with more employment
  in an SEINUNIT's industry have a higher probability of being selected.
\item Industry codes are imputed probabilistically based on empirical
  correspondence tables conditional on the same unit's observed other
  industry data items. For instance, if SIC is missing, but NAICS1997 is
  available, the relative observed distribution of SIC-NAICS1997 pairs is
  used to impute the missing data item.
\item If all previous imputation mechanisms fail, SIC is imputed
  unconditionally based on the observed distribution of within-state
  employment across SIC industries. Once SIC is assigned, the previous
  conditional imputation mechanisms are again used to impute other industry
  data items.
\end{itemize}


%%% Local Variables: 
%%% mode: latex
%%% TeX-master: "qwi-overview"
%%% End: 

%%%  +--< LEHD-QWI ecf 3.1.49 2005-01-26 schwa305            >--+  %%%
%%%  +--< Location: /programs/production/dev1/current/ecf    >--+  %%%
%%%  +--< File: doc/new_variables_naics_etc.tex              >--+  %%%
% Converted from Microsoft Word to LaTeX
% by Chikrii SoftLab Word2TeX converter (version 2.4)
% Copyright (C) 1999-2001 Kirill A. Chikrii, Anna V. Chikrii
% Copyright (C) 1999-2001 Chikrii SoftLab.
% All rights reserved.
% http://www.word2tex.com/
% mailto: info@word2tex.com, support@word2tex.com
%\subsection{Subject: ECF Enhancements}
%
%\subsubsection{Date: 2/20/2004}
%
%Author: Kevin McKinney


%\documentclass [10pt]{article}
%\usepackage {longtable}
%\usepackage{fullpage}

%\begin{document}





\subsection{NAICS codes on the ECF}
\label{sec:ecf:naics}

Enhanced NAICS variables on the ECF  can be differentiated  by the source(s) and coding
system used in their creation. There are two sources of data -- the ES202 and
the BLS-created LDB --  and two coding systems for NAICS --  NAICS1997 and
NAICS2002.  Every NAICS variable uses at least one
source and one coding system.

The ESO (ES202-only) and FNL (final) variables are of primary importance to the user community. 
The ESO variables use  information from the ES202 exclusively and ignore any 
information that may be available on the LDB. We provide in  Section~\ref{sec:backcoding} an analysis on 
why this may be preferred. The FNL variables incorporate information from 
both the ES202 and the LDB, with the LDB being the primary source. The QWI
uses FNL variables for its NAICS statistics. Neither ESO nor FNL variables
contain missing values.


\subsubsection{NAICS algorithm precedence ordering}


Four basic sources of industry information are available on the ECF: NAICS
and NAICS{\_}AUX as well as SIC from ES202 records, and the LDB-sourced
NAICS{\_}LDB codes. The NAICS, NAICS{\_}AUX, and 
NAICS{\_}LDB, when missing (no valid 6-digit industry code), are imputed based on the  following
algorithm. SIC is filled similarly. Depending on the imputation used, a
$miss$ variable is defined, which is used in building the ESO and FNL variables.


\begin{enumerate}
\item Valid 6 digit industry code ($miss=0$)

\item Imputed code based on first 3,4, or 5 digits when no valid six digit code is 
available in another period ($miss=0$)

\item Imputed code based on contemporaneous SIC if SIC changed prior to 2000 
($miss=1.5$)

\item Valid 6 digit code from another period ($miss=2$)

\item Valid code from another source (for example if NAICS1997 is missing, 
NAICS2002 or SIC may be available) ($miss=3$)

\item Use SEIN mode value ($miss=5$ if contemporaneous modal value, $miss=7$ if
  the modal value stems from another time period)

\item Unconditional impute ($miss=6$ if only the SEIN-level modal value is
  imputed unconditionally, $miss=11$ if the SEIN-level value was
  unconditionally imputed and propogated to all SEINUNITs.)
\end{enumerate}



\subsubsection{ESO and FNL variables}



The ESO and FNL variables are made up of combinations of the various sources 
of industry information. The ESO variable uses the NAICS and NAICS{\_}AUX 
variables as input. Information from the variable with the lowest MISS value 
is preferred although in case of a tie the NAICS{\_}AUX value is used.


The FNL variable uses the ESO and LDB variables. Information from the 
variable with the lowest MISS value is preferred although in case of a tie 
the NAICS{\_}LDB value is used. Keep in mind that although the source of an 
ESO or FNL variable may be equal to NCS, the actual source can only be 
ascertained by going back to the original.



\subsubsection{LDB versus LEHD NAICS backcoding}
\label{sec:backcoding}

The LDB algorithm is to some extent a black box and testing has shown that 
it does a relatively poor job of capturing  industry changes of SEINs that 
occurred during the 1990s. In fact, the LDB appears to be a simple
backfill that does not take into account an SEIN's entire SIC history.

% Kevin: email 2005-01-28
% Somewhere along the line something I said was lost in translation.
% Actually the LDB NAICS codes are too smooth over time (they never change)
% relative to the variation found in SIC codes.  The LDB looks like a
% simple backfill that does not take into account a firm's entire SIC
% history (prior to ~1999).  I know that some of the SIC changes may be
% spurious, but I also feel that there is valuable information in a firm's
% SIC code history that I therefore attempted to use in my impute
% algorithm.  However, overall the effect of the different approaches is
% relatively small since very few firms change industry, especially
% relative to, for example, the proportion of firms that change location. 
% 

Although some of the SIC changes over time may be spurious, an SEIN's SIC
code history contains valuable information that we have attempted to
preserve in our imputation algorithm. Overall, the effect of the different
approaches is relatively small, since very few SEINs change industry, in
particular relative to the proportion of SEINs that change geography.


In the following, we present  a summary of research done on a comparison of
the ESO and FNL NAICS codes on the Illinois ECF. 
%
%
The LDB-sourced NAICS variable is used for about 85{\%} of the records for
Illinois, the rest are filled with information from the ES202. It is
unclear why only 85{\%} of ES202 records are in the LDB. The results
weighted by employment are about the same suggesting that activity was not
a criterion for being included on the LDB. 

First and not surprisingly, in
later years and quarters (1999+) when NAICS is actively coded by the
states, the ESO and FNL codes look almost identical when available.

Second, there is little variation in the LDB NAICS codes over time compared 
with SIC. Among all of the active SEIN-SEINUNITs over the period covered by
the Illinois data, 
%\marginpar{\tiny fill in with actual time period spanned}
only slightly more than  8{\%} experience at least one SIC change compared with about 1.5{\%} on 
the LDB. Almost all NAICS code changes occur after 1999. While this is not entirely 
unexpected, it is something to keep in mind when comparing NAICS{\_}FNL 
versus SIC or NAICS{\_}ESO employment totals. Many of these changes in 
industry appear to be real and are not captured on the LDB.

One effect of this is that as we go back in time a larger portion of 
employment can be found in NAICS{\_}FNL codes that are different than one 
would expect given the SIC code on the ECF. For example, in 1990 about 
13{\%} of employment is in a NAICS{\_}FNL code that is different than what 
we would expect based on the SIC. By 2001, the proportion of employment
that is in a NAICS code outside of the set of possible values predicted by
the SIC-NAICS crosswalk falls to 3{\%}. The 
ES202-based NAICS variable does a better job tracking SIC, since more SIC 
information is used in putting it together.
% (about 3{\%} consistently over the period).
%\marginpar{\tiny what is 3 percent}

The main source of the discrepancy is due to entities that experience a 
change in their SIC code prior to 2000. The LDB appears to ignore this 
change, while the ES202-based NAICS variable uses an SIC-based impute for these 
SEINUNITS. The result is a series that exhibits similar patterns of change 
over time as SIC, while still preserving the value added in the NAICS codes 
for entities that did not experience a change.

Also, users should keep in mind that for early years ($<1997$) some of the 
NAICS industries have yet to come into existence. The
prevalence of this problem has not been investigated yet.




%
%

\section{The Geocoded Address List\label{cha:gal}: GAL}
% -*- latex -*-
% based on Marc's              Geocoded Address List Version 3.0
%

The Geocoded Address List (GAL) is a data set containing unique commercial and residential addresses
in a state geocoded to the Census Block and latitude/longitude coordinates. The file encompasses
addresses from the state ES202 data, the Census Bureau's Business Register (BR), the Census Bureau's Master
Address File (MAF), the American Community Survey Place of Work file (ACS-POW), and others.
Addresses from these source files go through geocoding software (Group1's
Code1), address standardizers (Ascential/Vality), and matching software
(Ascential/Vality) for
unduplication. 

% %, and several other steps in SAS. This document refers to one year's data from a source file
% %as a file-year (for example, the 1995 ES202). 
% 
% The job stream follows the steps below using the indicated software.
% 
% Step 1:   Create input (SAS).
% Step 2:   Standardize and geocode addresses (Code1).
% Step 3:   Parse and standardize address elements (Vality Standardize).
% Step 4:   Match addresses, flag masters and duplicates (Vality Unduplicate).
% Step 5:  Create preliminary crosswalk and unique address list with address identifier (SAS).
% Step 6:   Set file-year flags, create GAL Crosswalks containing the input identifier and address identifier (SAS).
% Step 7:   Retrieve and derive block codes and coordinates from the MAF (SAS).
% Step 8:   Impute block within known tract (SAS).
% Step 9:   Create GAL by adding higher-level geocodes by block (SAS). 
% Step 10: Delete intermediate data files and create links. 
% 
% The section entitled "Details of GAL Processing" gives details on each of these steps. 

The final output consists of the address list and a crosswalk for each
processed file-year. The GAL contains each unique address, identified by a
GAL identifier called {\tt galid}, its geocodes, a flag for each file-year
in which it appears, data quality indicators, and data processing
information, including the release date of the Geographic Reference File (GRF).
%
 The GAL Crosswalk contains the ID of each input entity and the
ID of its address ({\tt galid}).

% The following section describes the GAL's content. 
% 
% 
% \subsection{Important Variables}
% 
% \subsubsection{Unique identifier}
% 
% The variable {\tt galid} is the unique address identifier on the GAL, a 26-character string consisting of the
% letter 'A' in the first column followed by the 2-character state FIPS code and a zero-padded sequential
% number. {\tt galid}  is recreated each time a GAL is created. There's no consistency in the galid between
% versions or vintages of the GAL.
% 
% \subsubsection{Geographic vintage}
% 
% The release date (year) of the GRF identifies the geographic vintage
% variable {\tt a\_vintage}.
% 
\subsection{Geographic codes and their sources}

A geocode on the GAL  is constructed as 
\begin{center}
  {\tt FIPS-state(2)||FIPS-county(3)||Census-tract(6)}, 
\end{center}
and it uniquely identifies the
Census tract in the U.S. The tract is the lowest level of geography recommended for analysis. The Census
block within the tract is also available on the GAL, but the uncertainties in block-coding make block-level analysis
questionable. However, geocoding to the block allows us to add all the higher-level geocodes to the
addresses. 


\subsubsection{Block coding}


Block coding is achieved by a combination of geocoding software (Group1's
Code1), a match to the MAF, or an imputation based on addresses within the
tract. Table~\ref{tab:gal:a_block_src} describes the typical
distribution of geocode sources.

\begin{table}[htbp]
\footnotesize
  \centering
  \caption{Sources of geocodes on GAL}
  \label{tab:gal:a_block_src}
  \begin{tabular}{lrp{5in}}
%a_block_src
       &Typical\\
Value  &  Percent &  Meaning                                                                              \\  
\hline
\hline
C      &         12.20   &  Code1, or the address matches an address for which Code1 supplied the block code     \\
M      &         81.86   &  The MAF - the address is a MAF address or matches a MAF address                      \\
E      &          0.00   &  The MAF, the street address is exactly the same as a MAF address in the same tract   \\
W      &          0.03   &  The MAF, the street address is between 2 MAF addresses on the same block face        \\
O      &          1.23   &  Imputed using the distribution of commercial addresses in the tract                  \\
S      &          1.17   &  Imputed using the distribution of residential addresses in the tract                 \\
I      &          0.01   &  Imputed using the distribution of mixed-use addresses in the tract                   \\
D      &          0.00   &  Imputed using the distribution of all addresses in the tract                         \\
missing&          3.50   &  Block code is missing                                                                \\                
\hline
       &       100.00\\

  \end{tabular}
\end{table}

In all states observed so far except California, no address required the 'D' method. That is, almost every
tract where an address lacks a block code contains commercial, residential, and mixed-use addresses. 

The Census Bureau splits blocks to accommodate changes in political boundaries. Most commonly, these
are place boundaries (a place is a city, village, or similar municipality). The resulting block parts are
identified by 2 suffixes, each taking a value from A to Z. The GAL assigns the block part directly from
the MAF, or by adopting the one whose internal point is closest to the address by the straight-line
distance. 

% The variables a_block_suf1 and a_block_suf2 identify the block part, and a_block_suf_src
% generated in Step 9 describes the method used to assign it.
% 
% a_block_suf_src
% Value  Typical Percent   Meaning
% A               1.50          Assigned by distance
% M               4.18          The MAF - the address is a MAF address or matches a MAF address
% missing                94.32       Not a split block
%               100.00

The GAL also provides the following components of the geocodes as separate
variables, for convenience: FIPS code
(5 digits), FIPS state code (the first 2 digits of the FIPS code), FIPS
county code within state (the right-most 3 digits of the FIPS code), and
Census tract code (a tract within the county, a 6-digit code).
%
%a_ssccc        FIPS-state(2)||FIPS-county (3)
%a_st      FIPS state (2)
%a_cty          FIPS county within the state (3)
%a_tract        Census tract within the county (6)

Higher-level geographic codes originate from the Block Map File (BMF). 
% and attach to the GAL in Step 9. 
The BMF is an extract of the GRF-C (Geographic Reference File - Codes). All geocodes are
character variables. FIPS (Federal Information Processing Standard) codes are unique within the U.S.;
Census codes are not. Table~\ref{tab:gal:geocodes_hi} lists
 the available higher-level geocodes.

\begin{table}[htbp]
  \centering
  \caption{Higher-level geocodes on GAL}
  \label{tab:gal:geocodes_hi}
  \begin{tabular}{ll}
\hline
\hline
\tt a\_fipsmcd  &    5-digit FIPS Minor Civil Division (a division of a county)                    \\
\tt a\_mcd      &    3-digit Census Minor Civil Division (a division of a county)                  \\
\tt a\_fipspl   &    5-digit FIPS Place                                                            \\
\tt a\_place    &    4-digit Census Place                                                          \\
\tt a\_msapmsa  &    Metropolitan-Statistical-Area(4)||Primary-Metropolitan-Statistical-Area(4)    \\
\tt a\_wib      &    6-digit Workforce Investment Board area                                       \\ 
\hline
  \end{tabular}
\end{table}



\subsubsection{Geographic coordinates}

The geographic coordinates of each address available as latitude and
longitude
% are in the variables a_latitude and a_longitude. These variables are
% numeric 
with 6 implied decimals.
% (divide by 1,000,000 to convert them). 
The coordinates are not always as accurate as 6 decimal places implies. An
indicator flag of their quality is provided. Table~\ref{tab:gal:geoqual}
provides the typical distribution of codes, which range from $1$ (highest
quality) to $9$ (lowest quality).
%in the variable a_geoqual, a
%numeric variable taking values from 1 to 9 and generated in Steps 7, 8, and 9:

%a_geoqual
\begin{table}[htbp]
  \centering
  \caption{Quality of geographic coordinates}
  \label{tab:gal:geoqual}
  \begin{tabular}{lrl}
      &Typical\\
Value &  Percent&   Meaning\\
\hline
\hline
1     &         80.15  &        Rooftop or MAF (most accurate)\\
2     &          1.59  &        ZIP4 or block face, block face is certain\\
3     &         10.12  &        Block group is certain\\
4     &          4.65  &        Tract is certain\\
9     &          3.50  &        Coordinates are missing\\
\hline
      &        100.00  &\\
  \end{tabular}
\end{table}

%The format 'agqual' provided by 'format_geo.sas' in '/programs/projects/auxiliary/Formats' contains the
%meanings of the a_geoqual values listed above.
%

Variables indicating the source of the geographic coordinates (Block
internal point, geocoding software, MAF, or otherwise derived) are also
available. Most coordinates are provided by either commercial geocoding
software or the MAF.


%Two other variables give information about the coordinates. The flag a_latlong_src indicates their
%source: 
% 
% a_latlong_src
% Value  Typical Percent   Meaning
% B              14.77          Block (or block part) internal point
% C              70.04          Code1
% D               0.03          Derived
% M              11.66          the MAF
% missing                 3.50       Coordinates are missing
%               100.00
% Few addresses have a_latlong_src equal to 'D'. Deriving coordinates occurs only if they're still missing
% after Code1 processing and direct extraction from the MAF, but the tract is known. In this case, the flag
% a_latlong_drv generated in Step 7 describes the derivation method:
% 
% a_latlong_drv
% Value  Typical Percent   Meaning
% F               0.00          Adopted from the only address on the block face
% P               0.04          Extrapolated between 2 addresses on the block face
% missing                99.96       Derivation not performed
%               100.00
% 
% In GAL Version 1, deriving coordinates and block codes by these methods was an important means of
% block-coding. It rarely operates now, since Code1 began providing block codes. Nevertheless, GAL
% Version 3 still exhausts all methods of assigning block-codes and coordinates before resorting to
% imputation. 

%File-year flags

Finally, a set of flags also indicates, for each year and source file,
whether an address appears on that file. 

% A set of flags generated in Step 6 indicates what file-years an address appears in. The names of the flags
% conform to the naming convention [f][yyyy] for the source file [f] and year [yyyy], where [f] takes the
% following values:
% 
% Business Register (SSEL)                f = b
% ES202                              f = e
% Master Address File                     f = m
% American Community Survey - Place of Work    f = p
% American Housing Survey            f = h

For example, the flag variable {\tt b1997} equals 1 if the address is on the 1997 BR; otherwise it equals 0.
If a state partner supplies 1991 ES202 data with no address information,
{\tt e1991 } will be 0 for all
addresses. Typically,  between 3 and 6 percent of addresses are present on
any given year's ES202 files, 
%, the b[yyyy] flags equal 1 for 
between 4 and 10 percent are present on a specific BR year file, and 
% the m[yyyy] flag is 1 for 
between 80 and 90 percent are present on the MAF. 
%The p[yyyy] and h[yyyy] flags equal 1 for less 
Less than 1 percent of addresses are found on the ACS-POW and AHS
data, because these are sample surveys. 

% 
% 
%                                                         Other Variables
% 
% The variable occupant_type, recoded from the file-year flags in Step 8, indicates whether an address is
% commercial, residential, or mixed-use. 
% 
% The tracking ID bigsrcid, created in Step 1, uniquely identifies the entity that supplied the address. It
% consists of [f], [yyyy], the unique ID from the input file, zero-padding, and for some source files, a flag
% indicating which set of variables supplied the address. For addresses originating in the Business Register,
% another flag indicates the single-unit data set or the multi-unit data set. This tracking ID variable is useful
% for debugging. 
% 
% A diagnostic variable srcmast contains [f][yyyy], indicating the file-year that supplied this address. Bear
% in mind that it's often arbitrary which observation becomes the master address for a set of duplicates in
% Step 1 and Step 4, so bigsrcid and srcmast don't indicate anything special about an address or an entity.
% They simply identify the origin of an address that became a master address in unduplication. 
% 
% The names of Code1 variables contain the prefix c1_. They impart mostly diagnostic information from
% Code1 processing. They could be useful for development work or address research. The parsed address
% elements from Step 3 sit in the variables named with the prefix v_. They could be useful for development
% work, particularly in improving the parsing routine. 
% 
% 

\subsection{Accessing the GAL: the GAL Crosswalks}
\label{sec:gal:xwalks}

The GAL Crosswalks allow data users to extract geographic and address information about any entity whose
address went into the GAL. Each crosswalk contains the identifiers of the entity, its galid, and sometimes
flags. To attach geocodes, coordinates, or address information to an entity, merge the GAL Crosswalk to
the GAL by galid, outputting only observations existing on the GAL Crosswalk. Then merge the
resulting file to the entities of interest using the entity identifiers. An entity whose address wasn't
processed (because it's out of state or lacks address information) will
have blank GAL data. Table~\ref{tab:gal:xwalks} lists the entity
identifiers by dataset or survey.

\begin{table}[htbp]
  \centering
  \caption{GAL crosswalk entity identifiers}
  \label{tab:gal:xwalks}
  \begin{tabular}{lp{3in}p{2in}}
Dataset & Entity identifier variables & Note \\
\hline
\hline
AHS     &\tt \footnotesize control and year \\
ES202   &\tt \footnotesize sein, seinunit, year, and quarter & $e\_flag = p$ for physical
                                             addresses, $e\_flag = m$ for mailing addresses 
                                             as source of address info\\
ACS-POW &\tt  \footnotesize acsfileseq, cmid, seq, and pnum. \\
BR      &\tt  \footnotesize cfn, year, and singmult & $singmult$ indicates
                                     whether the entity resides in the single-unit ($su$) or the multi-unit
                                     ($mu$) data set.\\
        &\tt                          & $b\_flag =P$ if  physical address,
                                     $b\_flag=M $ for mailing address.\\
MAF     &\tt  \footnotesize mafid and year \\
\hline
  \end{tabular}
\end{table}

% For the AHS, the entity ID variables are control and year. 
% 
% For the ES202, the entity ID variables are sein, seinunit, year, and quarter. The flag variable e_flag
% indicates whether the address came from the address_street1, address_state, and address_zip9
% variables (e_flag=P for physical address) or from the ui_address_street1, ui_address_state, and
% ui_address_zip9 variables (e_flag=M for mailing address). 
% 
% For the ACS-POW data, the entity ID variables are acsfileseq, cmid, seq, and pnum. 
% 
% For the SSEL, the entity ID variables are cfn, year, and singmult. The flag variable singmult indicates
% whether the entity resides in the single-unit (su) or the multi-unit (mu) data set. Another flag variable
% b_flag indicates whether the address originated from the variables pstreet, pplce, pst, and pzip
% (b_flag=P for physical address) or street, plce, st, and zip (b_flag=M for mailing address). . 
% 
% For the MAF, mafid and year identify entities. 
%              Resources for geographic information
% 
% The best place for information about Census geography is 
% <http://www.census.gov/geo/www/reference.html>
% 
% Especially informative is the Geographic Areas Reference Manual (GARM), at
% <http://www.census.gov/geo/www/garm.html>
% 
% 
% 
%                    Details of GAL Processing
% 
%                         Control Programs
% 
% The control program is 'rungal.ksh', which documents itself. Simply type the name of the program to see
% instructions. A super-control program 'rungal_all.ksh' runs GALs in any number of states, limiting each
% step of GAL processing to 1 state a time. It actually launches another program 'rungal_line.ksh' for each
% state specified. The super-control program requires you to specify the states,  the steps, and other
% parameters in the program and confirm them before it will execute. Submit 'rungal_all.ksh' in the
% foreground to confirm the parameters.
% 
%                            Input Data
% 
% The input data consists of addresses, geocodes, and coordinates. Currently, the source files providing
% addresses consist of the following (future work will add the Non-employer file):
% ACS-POW   American Community Survey Place of Work (2001 and later)
% AHS       American Housing Survey (2002)
% ES202          Employment Security form 202 (all available years 1990 and later)
% SSEL      Business Register (Standard Statistical Establishment List 1990 and later)
% MAF       Master Address File (the year following the year of the desired geographic vintage)
% 
% The source files providing geocodes and coordinates are the following:
% GCP       the databases of Group1's Geographic Coding Plus software
% MAF       Master Address File
% GRF-C          Geographic Reference File, Codes (encompassed in the BMF)
% WIB-C          Workforce Investment Board, Codes (encompassed in the BMF)
% BMF       Block Map File
% 
% A flow-chart of how all these files relate to the GAL is in the document 'gal_schematic.pdf.'Below is a description of each step of GAL processing. Further information is available in the GAL
% programs, the documentation of the input data files, and the materials explaining the Code1 and Vality
% software packages. 
% 
%                  Step 1. Creating Address Input
% 
% This step creates all the input address data for the GAL. It launches a SAS job for each input file in
% parallel. The launched macro programs find observations with a unique identifier and address
% information in the specified state. The program creates one big data file from all the input data files, sorts
% it, outputs unique addresses separately from exact address duplicates, and splits the file of unique
% addresses into as many as 8 same-size chunks for parallel processing in the next step. 
% 
% The process keeps track of addresses by means of a tracking ID bigsrcid. This variable contains 5 digits
% to identify the file-year, IDs from the source file, sometimes a source-file-specific flag, and some zeroes
% as place holders. Quality Assurance (QA) for Step 1 checks bigsrcid for blank columns.
% 
% Address processing may exclude an address for any of the following reasons: it contains a duplicate
% source file identifier; it lacks a street address and zip; or it's in the wrong state. Only 1 observation enters
% address processing even if an input entity has more than 1 (for instance, a physical and mailing address).
% The following section lists the conditions required for processing an address from each source file. 
% 
% ES202
% 
% The source data set is state-specific. A flag indicates whether the address was a physical ('P') or mailing
% ('M') address. Later, Step 6 puts this flag on the crosswalk as the variable e_flag.
% 
% Conditions required to process an address are:
% 1-   a unique sein-seinunit-year-quarter
% 2-   the address is in state
% 3-   physical address street or zip is nonblank, or mailing address street or zip is nonblank
% 
% Business Register (SSEL)
% 
% The source data set covers all states. The flag 'singmult' indicates whether the address was from the
% single-unit ('S') or multi-unit ('M') source file. Additionally, for observations from the single-unit file, a
% flag indicates whether the address was a physical ('P', from variable pstreet) or mailing ('M', from
% variable street) address. All observations are 'M' from the multi-unit source file. Later, Step 6 puts this
% flag on the crosswalk as the variable b_flag. 
% 
% Conditions required to process an address are:
% 1-   a unique tracking ID
% 2-   the address is in state
% 3-   if from the multi-unit source file, the mailing address is nonblank; or if from the single-unit
%      source file, the physical address or mailing address is nonblank
% 
% Master Address File (MAF)
% 
% The source data set is state-specific. One condition suffices to process an address:
% 1-   the street or zip is nonblank
% 
% American Community Survey - Place of Work (ACS-POW)
% 
% The source data set covers all states. Conditions required to process an address are:
% 1-   a unique cmid-seq-pnum
% 2-   the address is in state
% 3-   the street or zip is nonblank
% 
%                     Step 2. Code1 Processing
% 
% This step standardizes and geocodes addresses using Group1 software. Here "standardize" really means
% correcting street names, zip codes, apartment numbers, and so forth. This step processes the chunks of
% data produced by Step 1 in parallel. The software license limits the number of jobs to 10. The GAL
% program sequence limits the number of jobs to 8, leaving 2 free for development work. Quality
% Assurance (QA) for this step checks the rates of best address matching and geocoding. 
% 
% Group1 recommendations and visual inspection of output indicate that medium strictness and a
% correctness score of 3 or lower achieves a good balance between the number and the quality of address
% corrections.  The following other options also apply. If Code1 can't find a matching address, it provides
% the input information as output. This ensures that the output field contains the best available information.
% If an input street name is an alias street name, then Code1 returns the base street name. If there are
% multiple address matches for a given input address, Code1 doesn't correct the address. The spreadsheet
% 'gal_fields_v3.xls' in '/data/doc/geocoding' provides further information on processing parameters, and
% can serve as a data dictionary for the Code1 output. 
% 
%                 Step 3. Parsing Address Elements
% 
% This step edits the Code1 address by parsing it into its elements and standardizing them.  This procedure
% prepares the address for unduplication matching. Address elements are number, street, directional,
% apartment number, and so forth. This step processes the chunks of data in parallel. 
% 
%                Step 4. Matching for Unduplication
% 
% This step creates input data files for each round of Vality matching, creating cuts of data without
% breaking apart blocking groups. It runs each round of Vality unduplication matching on each cut in
% sequence. In tests, parallel processing in this step didn't improve performance. The output data files
% consist of matched addresses with the master and duplicate addresses flagged as such.
% 
%                                 Step 5. Creating a Data Set of Unique Addresses and Preliminary Crosswalk
% 
% This step creates the unique address list and preliminary crosswalk by reading in the results of Vality
% unduplication matching and creating the unique address identifier galid. This step also builds the
% preliminary crosswalk containing the tracking ID and galid. The crosswalk is preliminary because it's a
% combined crosswalk for all file-years, it lacks the exact address duplicates set aside in Step 1, and it
% contains bigsrcid (not the source file ID). QA for this step checks the unduplication rates in each round
% of matching (production mode of GAL processing doesn't produce a report in Step 4). 
% 
% Step 6. Creating the GAL Crosswalks and Setting the File-Year Flags
% 
% This step combines the preliminary crosswalk and the address duplicates set aside in Step 1 and outputs
% the final crosswalks, each containing the source file identifiers of each input entity and its address
% identifier. SAS produces these crosswalks in parallel. Then the program sets the file-year flags to 1 or 0
% for each address on the GAL.
% 
% For QA, this step checks for any blank source-of-master-address (srcmast) and the proportion of
% addresses for which a MAF address is the master. QA also checks that each input entity appears on the
% appropriate crosswalk, and the frequency of each file-year flag. 
% 
%      Step 7. Retrieving and Deriving Geocodes from the MAF
% 
% This step retrieves and derives geocodes from the MAF and attaches them to any address possible. The
% program first retrieves MAF geocodes for MAF addresses. Then it determines block-codes and
% coordinates for any address that's the same as a MAF address or falling between 2 MAF addresses on the
% same block face. An address can be the same as a MAF address and fail to match it because of a different
% apartment number, missing data, or other reasons. 
% 
% The derivation routine affects very few addresses, but remains in the program sequence to exhaust all
% possible means of block-coding before resorting to imputation. Note that the program blanks out-of-state
% geocodes produced by Step 2. 
% 
%         Step 8. Imputing the Block Within a Known Tract
% 
% This step imputes a block when the tract is known but the block is still missing. The imputed block is a
% random draw from the distribution of residential addresses, commercial addresses, mixed addresses, or
% (as a last resort) all addresses in the tract. These 4 types of imputation, 2 methods of derivation from the
% MAF, retrieval from the MAF, and Code1 amount to 8 possible sources of the block code. For QA, this
% step checks that at least 7 of the 8 block-coding methods operated, and that the source-of-block-code
% variable (a_block_src) is never missing.
% 
%                                                  Step 9. Creating the Final GAL
% 
% This step adds higher-level geocodes such as Minor Civil Division, Place, and Metropolitan Statistical
% Area from the Block Map File (BMF) to all addresses possible. First it assigns the block suffix to
% addresses on blocks the Census Bureau has split to accomodate boundary changes. It adopts the
% coordinates of the block (or block part) internal point if the existing coordinates of an address are less
% precise than block face. This assures consistency between the coordinates and the geocode. 
% 
% In tests, a few anomalies appeared about which county a block falls into. Therefore the program now runs
% a patch when this occurs. 
% 
% For QA, this step checks how much the geocoding improved in Steps 7, 8, and 9, how many addresses
% adopted the coordinates of the block (or block part), and how many blocks needed patching.
%                                 
%                        Step 10. Clean-up
% 
% This step deletes all intermediate data files and creates links in the current/ directory pointing at the GAL
% and GAL Crosswalks in the version/geo-vintage directory.
% 
%%% Local Variables: 
%%% mode: latex
%%% TeX-master: "qwi-overview"
%%% End: 


%
%



%%% Local Variables: 
%%% mode: latex
%%% TeX-master: "qwi-overview"
%%% End: 

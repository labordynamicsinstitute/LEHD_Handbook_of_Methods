%TCIDATA{LaTeXparent=0,0,sw-edit.tex}

% -*- latex -*- 
%
% Time-stamp: <05/04/05 10:29:28 vilhuber> 
%              Automatically adjusted if using Xemacs
%              Please adjust manually if using other editors
%
% ehf.tex
% Responsible: Paul
% Part of QWI_methods.tex


%\subsection{Employment History File}
\label{sec:ehf_overview}
%\label{cha:ehf}


The \textit{Employment History File} (EHF)\mindex{EHF}%
\index{Employment History File|see{EHF}}
%\footnote{%
%See Appendix~\Vref{app:ehf_technical} for detailed information on the
%creation of the EHF.} 
is designed to store the complete in-state work
history for each individual that appears in the UI%
\index{UI} wage records. The EHF for each state contains one record for each
employee-employer combination~-- a job\index{job} --  in that state in each
year.
% Every individual who is employed during a given year will then have one observation per
%employer for that year.
  Both annual and quarterly earnings variables are available in
the EHF. Individuals who never have strictly positive earnings (a
theoretical possibility) are dropped. 

A re-ordering of the data into one observation per
job, with all quarterly earnings and activity records available within one record, is
also available (Person History File, PHF\index{PHF}). Activity is defined
as active employment within a quarter, requiring a strictly positive value
for quarterly earnings. A similar time-series of activity at the SEINUNIT level (UNIT History
File, UHF\index{UHF}) and the SEIN level (SEIN History File,
SHF\index{SHF}) is also computed at this time. 

A comparison of the earnings and employment information from the UI and
ES202 files is one of the core quality measures that are computed. Large
discrepancies are highlighted, and clarified with the data provider. Often,
a corrected data file can be imported into the LEHD system. Not
all data discrepancies can be easily resolved. In particular the historical
data sometimes are not correctable, because the data has been lost or
corrupted.%.
%
\footnote{A future extension currently being developed will allow to apply
imputation models to correct for large discrepancies.}
%


%%% Local Variables: 
%%% mode: latex
%%% TeX-master: "qwi-overview"
%%% End: 

%TCIDATA{LaTeXparent=0,0,sw-edit.tex}
                      

% -*- latex -*- 
%
% Time-stamp: <02/05/13 23:53:54 vilhuber> 
%              Automatically adjusted if using Xemacs
%              Please adjust manually if using other editors
%
% ehf_technical.tex
% Responsible: Paul
% Part of QWI_methods.tex

% Converted from Microsoft Word to LaTeX
% by Chikrii SoftLab Word2TeX converter (version 2.4)
% Copyright (C) 1999-2001 Kirill A. Chikrii, Anna V. Chikrii
% Copyright (C) 1999-2001 Chikrii SoftLab.
% All rights reserved.
% http://www.word2tex.com/
% mailto: info@word2tex.com, support@word2tex.com

\index{EHF|(mainref}
This chapter describes the Employment History File (EHF),
details each of the programs necessary to create it, and discusses where to
locate all input files, and where to store the output files. The purpose of
this documentation is to provide enough information for someone with no
prior experience--other than some familiarity with running SAS\index{SAS}
in a UNIX\index{UNIX} environment as well as a basic knowledge of SAS
macros---to process the EHF. Such processing will likely occur quite
frequently as current state partners send updates and new state
partnerships are formed. Hence, the information below should be used as a
primer to help guide the analyst charged with creating the EHF. It should
be detailed enough to provide the analyst with sufficient expertise to
implement minor modifications to the programming sequence should they
become necessary.





\section{Overview\label{sec:ehf_technical_overview}}

The EHF contains earnings information at the Protected Identity Key
(PIK)\index{PIK} - State Employer Identification Number
(SEIN)\index{SEIN}-Year level. Thus, one record exists for each
individual-employer-year combination. Earnings are recorded for each of the
four quarters in the year as well as for the entire year. All of this
information is generated solely from the state Unemployment Insurance (UI)\index{UI}
wage records shipped to LEHD from its state partners. No additional
information on individuals or employers from other data sources is ever
included.


At least one EHF file exists for each state the LEHD Program has
received UI\index{UI} records from. Over time, states send updates
(additional years of data), and when this occurs the EHF is reprocessed so
as to reflect this. As such, several EHFs may exist for a given state.
Along with the Individual Characteristics File (ICF)\index{ICF} and the Employer
Characteristics File (ECF)\index{ECF}, the EHF represents one of the most important
and frequently used data files at LEHD. In particular, it is the key input
file to the Employment Dynamics Estimates (QWI) generated by LEHD for its
state partners.





\section{Program files and notation\label{sec:ehf_technical_location}}


\subsection*{Location of program files}


The base directory for all EHF-related programs is 
\begin{center}
/programs/projects/Employment{\_}history/
\end{center}


The most current version of all program files necessary to construct the EHF 
can be found in ./jobstream200201230.   
Should changes ever be made to streamline or enhance this sequence, the 
resulting program files should be stored in a similar directory in 
the base directory, which reflects the date of the 
changes. 

The programs in ./jobstream200201230 
serve as a reference for the user and should never be run in this location. 
Instead, the user should create a new directory in 
the base directory that reflects the state of interest 
and the most recent year for which data are available for the state (for 
instance, ./MN2000) and then copy the program files there.



\subsection*{Notation}

The following notation is used for macro variables, and should be familiar
to users familiar with SAS\index{SAS} macros at LEHD:
\index{state@\&state.}

\begin{itemize}
\item  ``{\&}st'' refers to the two 
letter (postal) state abbreviation, in lower case  (e.g. ca, mn, fl, md, 
tx) 
\item  ``{\&}year'' refers to the 4 digit year, e.g. ``1999''
\end{itemize}






\section{Program Sequence\label{sec:ehf_technical_jobs}}

\begin{description}
\item Prior to running the job sequence, previous versions of the EHF
  should be backed up by renaming. The following  naming convention should be used: 
emphis{\&}st{\_}{\&}year{\&}q, where {\&}year refers to the last four digit 
year (for instance, 2001)  and {\&}q refers to 
the last one digit quarter (for instance, q4) for which data are available
on the previous version. The purpose of this is to keep available previous
versions. Downstream users should always access the most recent version.


\item To create the EHF the following jobs must be run sequentially:

  \begin{steps}

\item ui{\&}st-work-history-01.sas
  
  This program reads in the raw UI\index{UI} data files (one file for each
  year of available data), sorts them by \textit{PIK}\index{PIK},
  \textit{SEIN}\index{SEIN}, \textit{YEAR}\index{YEAR},
  \textit{QUARTER}\index{QUARTER}, and then interleaves them into a single
  (often very large) data file, called ui{\&}st.01. The two other variables
  on these input files are \textit{EIN}\index{EIN} and
  \textit{WAGE}\index{WAGE}.  In addition, an earnings variable based on
  the \textit{WAGE} variable and rounded to the nearest dollar is created.
  Each input files {\&}st{\&}year is stored in its own directory:
  \begin{center}
    /data/master/ui/{\&}st/{\&}year/sasdata.
\end{center}
In order to run the program
  successfully, the user must first determine how many years of data are
  available for the state of interest and manually edit
  ui{\&}st-work-history-01.sas to reflect the correct number (and names) of
  all input files. The program then calls all subsequent jobs listed below.
  Therefore, it is very important that all edits to subsequent programs are
  completed BEFORE running ui{\&}st-work-history01.sas. If the user is
  uncomfortable with this option, she may comment out the code telling SAS\index{SAS}
  to do this and process each subsequent program manually. The output file
  is stored in a user defined temporary workspace. Usually, this is one of
  the 12 /saswork directories on DSBU02\index{DSBU02}. The user needs to find a directory
  which is relatively empty 
  and edit the program so that the libname, ``working'' points to it.
  \begin{itemize}
  \item[TIP:] Use the ``df --k'' command at the UNIX prompt 
  \end{itemize}





\item ui{\&}st-work-history-02.sas

  
  This program sums the earnings (restricted to be positive) of all
  employees at each \textit{SEIN}\index{SEIN} for each year and quarter.
  The input file is ui{\&}st.01, stored in the user defined temporary
  workspace. The output file, sein{\_}employment{\_}{\&}st, is written to
  /data/master/Employer/sasdata/uitotals{\_}old.%
%
\footnote{In the
  future, the creation of this file is subsumed into the ECF sequence,
  which will reference the final EHF.}
%



\item ui{\&}st-work-history-03.sas
  
  This program creates all remaining variables in the EHF:
  quarterly earnings (4 variables, one for each quarter, denoted
  \textit{earn1}- \textit{earn4}), annual earnings (the sum of
  \textit{earn1}-\textit{earn4}, denoted \textit{earn{\_}ann}). All of
  these variables as well as \textit{PIK}\index{PIK},
  \textit{SEIN}\index{SEIN}, and \textit{YEAR}\index{YEAR} are subsequently
  labeled. Finally, the variable \textit{source} is created which assigns
  the state FIPS\index{FIPS} code (stored as a character variable). For
  instance the FIPS code for CA is ``06.''  The input file to this entire
  process is ui{\&}st.01.sas7bdat, which is stored in the user-defined
  temporary workspace. The output file, similarly stored, is
  ui{\&}st.03.sas7bdat.

\item ui{\&}st-work-history-04.sas
  
  This file creates the EHF and places it in
  /data/master/Employment{\_}history/sasdata. The input file, ui{\&}st.03
  is sorted by \textit{PIK}\index{PIK}, \textit{SEIN}\index{SEIN}, and
  \textit{YEAR}\index{YEAR} and then copied from the user defined temporary
  workspace ({\&}working) to its permanent location:
  /data/master/Employment{\_}history/sasdata. The file is subsequently
  renamed: emphis{\&}st.sas7bdat. This file is the EHF for the particular
  state being analyzed.

% However, if the EHF is an update to a previous version 
% for the same state a different naming convention should be used: 
% emphis{\&}st{\_}{\&}year{\&}q, where {\&}year refers to the last four digit 
% year (for instance, 2001) for which data are available and {\&}q refers to 
% the last one digit quarter (for instance, q4) for which data are available. 
% The purpose of this is to avoid writing over an existing EHF with its 
% updated version.

\item  ui{\&}st-work-history-05.sas

This file creates the indices for the EHF in 
/data/master/Employment{\_}history/sasdata based on the following 
combinations of variables: 
\index{index}
\index{PIK}
\index{YEAR}
\index{SEIN}

\begin{itemize}
\item \textit{PIK YEAR SEIN}
\item \textit{PIK SEIN YEAR}
\item \textit{SEIN YEAR}
\end{itemize}
Indices help make data processing more convenient for users of the EHF 
as they eliminate the need to subsequently sort the EHF along any of 
dimensions just listed. Indexation of the EHF results in the creation of an 
additional file emphis{\&}st.sas7bndx also stored in 
/data/master/Employment{\_}history/sasdata. 




% This has been moved to pff_technical.tex
% 
% \item ui{\&}st-work-history-06
% 
% This program is virtually identical to the first program in the QWI~v2.2 
% sequence, jobflow01.sas. The only difference is the output file contains the 
% variable PIK\index{PIK} in addition to other \textit{PIK-SEIN}-level\index{SEIN} job flow statistics. The 
% program combines information from the
% Individual Characteritics File (ICF)\index{ICF} as well as from the EHF. More 
% specifically, it reads in records from the employment history file and 
% attaches demographics from the ICF. Employment and earnings histories are 
% created for each individual which are subsequently used to create the 
% \textit{PIK-SEIN}-level job flow statistics. The output file, informally known as the 
% ``Person Flows File,'' is called jobflow01{\_}pik. Depending on the amount 
% of space available, it should be stored in either 
% /data/working5/person{\_}flows/{\&}st or 
% /data/working6/person{\_}flows/{\&}st. 
% 
% \begin{itemize}
% \item[TIP:] The user should first check to see whether these directories have
%   been created. In order for ui{\&}st-work-history-06.sas to run correctly,
%   the user must first edit the program macrolist.sas! 
% 
% \end{itemize}
% 
% This program has been written in a 
% very generic manner, meaning the user must select the appropriate state of 
% interest, enter an appropriate temporary working directory, enter the 
% appropriate person{\_}flows directory. The correct locations for the EHF and 
% ICF must also be entered. The location of the ICF\index{ICF} for each state can be 
% found in /data/doc/data{\_}availability/daf.xls. 
\end{steps}


\end{description}


\section{Notes on processing\label{sec:ehf:notes}}


A few final notes on the EHF are necessary in order to ensure the user knows 
enough to run the entire program sequence correctly. 
\begin{description}
\item \ 


  \begin{notes}
\item 
First, and most 
importantly, the programs ui{\&}st-work-history-02.sas -- 
ui{\&}st-work-history-05.sas are actually very short programs in which 
information such as the state and the location of the temporary working 
space needs to be entered. For some programs, the date specific extension to 
the name of the final EHF file (see the discussion for 
ui{\&}st-work-history-04 above) must also be entered. Each of these short 
programs calls a corresponding program in which the steps described above are actually 
coded. These programs are called state-work-history-02.sas --
state-work-history-05.sas. 
These programs have been written generically for all states in order 
to save editing time by the user. As such, all references to states and 
directories have been ``macroized'' and the macros are what get defined in 
ui{\&}st-work-history-02.sas -- ui{\&}st-work-history-05.sas. 

\item 
As stated above, at the end of ui{\&}st-work-history-01.sas all subsequent 
programs in the sequence are run automatically. As such, it is imperative 
that the user completes all edits to all programs in advance of running 
ui{\&}st-work-history01.sas. If the user is uncomfortable with this option, 
she may comment out the code telling SAS do this and process each subsequent 
program manually.

\item 
Following the completion of each job in the sequence, the user should check 
the .log file to ensure that no errors occurred. Since this sequence has 
been run many times before, it is highly likely that any errors that did 
occur are the result of the user failing to correctly edit the programs to 
reflect the appropriate state, date range, temporary work directory, et 
cetera. If problems persist, the user should contact a more experienced 
member of the LEHD team. 

\item

Once all programs have run successfully, two tasks remain. First, the user 
should go to the directory she specified as temporary workspace and delete 
all files. This will help free up disk space, which is often a scarce 
commodity on DSBU02\index{DSBU02}. 

\item Second, the user should update the Data availability 
file (DAF) stored in /data/doc/data{\_}availability as daf.xls. More specifically, 
the user should note the creation of the EHF as well as the person flows 
file, jobflow01{\_}pik, and fill in the appropriate information for all 
fields in the DAF.\index{DAF}
\end{notes}

\end{description}

\index{EHF|)mainref}

%%% Local Variables: 
%%% mode: latex
%%% TeX-master: "QWI_methods"
%%% End: 

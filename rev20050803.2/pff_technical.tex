%TCIDATA{LaTeXparent=0,0,sw-edit.tex}
                      

% -*- latex -*- 
%
% Time-stamp: <02/03/14 18:23:21 vilhuber> 
%              Automatically adjusted if using Xemacs
%              Please adjust manually if using other editors
%
% pff_technical.tex
% Responsible: Paul
% Part of QWI_methods.tex

% This was  moved here from ehf_technical.tex
% 

\index{PFF|(mainref}

This section describes the one program that creates the Person Flows File
(PFF). Since it is intricately linked to both the EHF\index{EHF}
(Chapter~\Vref{app:ehf_technical}) and the ECF\index{ECF}
(Chapter~\Vref{app:ecf_technical}), please revise the descriptions of that
processing as well. In particular, this file used to be part of the EHF
sequence, so all notes and tips in Section~\Vref{sec:ehf:notes} apply to
this file as well.

\begin{description}
\item \ 
  \begin{steps}
                                
\item ui{\&}st-work-history-06
 
  This program is virtually identical to the first program in the QWI~v2.2
  sequence, jobflow01.sas, and has in fact replaced it in the most current
  version of processing. The only difference is the output file contains
  the variable PIK\index{PIK} in addition to other
  \textit{PIK-SEIN}-level\index{SEIN} job flow statistics. The program
  combines information from the Individual Characteritics File
  (ICF)\index{ICF} as well as from the EHF. More specifically, it reads in
  records from the employment history file and attaches demographics from
  the ICF. Employment and earnings histories are created for each
  individual which are subsequently used to create the
  \textit{PIK-SEIN}-level job flow statistics. The output file is the
  ``Person Flows File,'' called jobflow01{\_}pik. Depending on the amount
  of space available, it should be stored in either
  /data/working5/person{\_}flows/{\&}st or
  /data/working6/person{\_}flows/{\&}st.
 
 \begin{itemize}
 \item[TIP:] The user should first check to see whether these directories have
   been created. In order for ui{\&}st-work-history-06.sas to run correctly,
   the user must first edit the program macrolist.sas! 
 
 \end{itemize}
 
 This program has been written in a 
 very generic manner, meaning the user must select the appropriate state of 
 interest, enter an appropriate temporary working directory, enter the 
 appropriate person{\_}flows directory. The correct locations for the EHF\index{EHF} and 
 ICF must also be entered. The location of the ICF\index{ICF} for each state can be 
 found in /data/doc/data{\_}availability/daf.xls. 
\end{steps}
\end{description}

\index{PFF|)mainref}

%%% Local Variables: 
%%% mode: latex
%%% TeX-master: "QWI_methods"
%%% End: 

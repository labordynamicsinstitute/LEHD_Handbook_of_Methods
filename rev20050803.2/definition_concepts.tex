%TCIDATA{Version=4.00.0.2321}
%TCIDATA{LaTeXparent=0,0,sw-edit.tex}
                      
%TCIDATA{ChildDefaults=chapter:1,page:1}

% -*- latex -*- 
%
% Time-stamp: <05/04/06 23:54:02 vilhuber> 
% Includes John Abowd mods 2005-04-05
%              Automatically adjusted if using Xemacs
%              Please adjust manually if using other editors
%



%\section{Fundamental Concepts}

\subsection{Dates}

\mindex{Dates}

The QWI is a quarterly data system with calendar year timing.  We use the
notation YYYY:Q to refer to a year and quarter combination. For example,
1999:4 refers to the fourth quarter of 1999, which includes the months
October, November, and December.

\subsection{Employer}

\mindex{Employer}

An employer in the QWI system consists of a single Unemployment Insurance (UI%
\index{UI}) account in a given state's UI wage reporting system. For
statistical purposes the QWI system creates an employer identifier called an
State Employer Identification Number (SEIN%
\index{SEIN}) from the UI-account number and information about the state
(FIPS%
\index{FIPS} code). Thus, within the QWI system, the SEIN is a unique
identifier within and across states but the entity to which it refers is a
UI account. All QWI statistics are produced at the establishment level.

\subsection{Establishment}

\mindex{Establishment}

For a given employer in the QWI system, an SEIN, each physical location
within the state is assigned a unit number, called the SEINUNIT. This
SEINUNIT is based on the reporting unit in the ES-202 files supplied by the
states. All QWI statistics are produced by aggregating statistics calculated
at the establishment level. Single-unit SEINs are UI accounts associated
with a single reporting unit in the state. Thus, single-unit SEINs have only
one associated SEINUNIT in every quarter. Multi-unit SEINs have two or more
SEINUNITS associated for some quarters. Since the UI wage records are not
coded down to the SEINUNIT, SEINUNITs are multiply imputed as described in
the section on unit-to-worker imputation above. A feature of this imputation
system is that it does not permit SEINUNIT to SEINUNIT movements within the
same SEIN. Thus, for multi-unit SEINs, the definitions below produce the
same flow estimates at the SEIN level whether the definition is applied to
the SEIN\ or the SEINUNIT.

\subsection{Employee}
\mindex{Employee}

Individual employees are identified by their Social Security Numbers (%
SSN\index{SSN}) on the UI wage records that provide the input to the QWI. To
protect privacy and confidentialty of the SSN and the individual's
name, a different branch of the Census Bureau removes the name and replaces
the SSN with an internal Census identifier called a Protected
Identity Key (PIK\index{PIK}).

\subsection{Job}
\mindex{Job}

The QWI system definition of a job is the association of an individual (PIK%
\index{PIK})\ with an establishment (SEINUNIT 
\index{SEINUNIT}) in a given year and quarter. The QWI system stores the
entire history of every job that an individual holds. Estimates are based on
the definitions presented below, which formalize how the QWI system
estimates the start of a job (accession), employment status (beginning- and
end-of-quarter employment), continuous employment (full-quarter employment),
the end of a job (separation), and average earnings for different groups.

\subsection{Unemployment Insurance wage records (the QWI system universe)}
\mindex{Universe}

The Quarterly Workforce Indicators are built upon concepts that begin with
the report of an individual's UI%
\index{UI}-covered earnings by an employing entity (SEIN%
\index{SEIN}). An individual's {UI} wage record enters the QWI system if at
least one employer reports earnings of at least one dollar for that
individual\ (PIK%
\index{PIK}) during the quarter. Thus, the job must produce at least one
dollar of {UI}-covered earnings during a given quarter to count in the QWI
system. The presence of this valid {UI} wage record in the QWI system
triggers the beginning of calculations that estimate whether that individual
was employed at the beginning of the quarter, at the end of the quarter, and
continuously throughout the quarter. These designations are discussed below.
Once these point-in-time employment measures have been estimated for the
individual, further analysis of the individual's wage records results in
estimates of full-quarter employment, accessions, separations (point-in-time
and full-quarter), job creations and destructions, and a variety of
full-quarter average earnings measures.

\subsection{Employment at a point in time}

\mindex{Employment!point in time}

Employment is estimated at two points in time during the quarter,
corresponding to the first and last calendar days. An individual is defined
as employed at the beginning of the quarter when that individual has valid %
UI\index{UI} wage records for the current quarter and the preceding
quarter.  Both records must apply to the same employer (SEIN\index{SEIN}).
An individul is defined as employed at the end of the quarter when that
individual has valid {UI} wage records for the current quarter and the
subsequent quarter. Again, both records must show the same employer. \ The
QWI system uses beginning and end of quarter employment as the basis for
constructing worker and job flows. In addition, these measures are used to
check the external consistency of the data, since a variety of employment
estimates are available as point-in-time measures. Many federal statistics
are based upon estimates of employment as of the 12th day of particular
months. The Census Bureau uses March 12 as the reference date for
employment measures contained in its Business Register and on the Economic
Censuses and Surveys.  The BLS\index{BLS} ``Covered Employment and Wages
(CEW\mindex{CEW})'' series, which is based on the ES-202\index{ES-202}
data, use the 12th of each month as the reference date for employment. The
QWI system cannot use exactly the same reference date as these other
systems because {UI} wage reports do not specify additional detail
regarding the timing of these payments. QWI research has shown that the
point-in-time definitions used to estimate beginning and end of quarter
employment track the CEW month one employment estimates well at the level
of an employer ({SEIN}). For single-unit SEINs, there is no difference between
an employer-based definition and an establishment-based definition of
point-in-time employment. For multi-unit SEINs, the unit-to-worker imputation model assumes that 
unit-to-unit transitions within the same SEIN cannot occur. So, point in 
time employment defined at either the SEIN or SEINUNIT level produces the 
same result.

\subsection{Employment for a full quarter}

\mindex{Employment!full quarter}

The concept of full quarter employment estimates individuals who are likely
to have been continuously employed throughout the quarter at a given
employer. An indivdual is defined as full-quarter-employed if that
individual has valid UI\index{UI}-wage records in the current quarter, the
preceding quarter, and the subsequent quarter at the same employer (%
SEIN\index{SEIN}). That is, in terms of the point-in-time definitions, if the
individual is employed at the same employer at both the beginning and end of
the quarter, then the individual is considered full-quarter employed in the
QWI system. 

Consider the following example. Suppose that an individual has
valid {UI} wage records at employer \textit{A} in 1999:2, 1999:3, and
1999:4. This individual does not have a valid {UI} wage record at
employer \textit{A} in 1999:1 or 2000:1. Then, according to the definitions
above, the individual is employed at the end of 1999:2, the beginning and
end of 1999:3, and the beginning of 1999:4 at employer \textit{A}. The QWI
system treats this individual as a full-quarter employee in 1999:3 but not
in 1999:2 or 1999:4. Full-quarter status is not defined for either the first
or last quarter of available data.

\subsection{Point-in-time estimates of accession and separation}

\mindex{Accessions!point-in-time} \mindex{Separations!point-in-time}

An accession occurs in the QWI system when it encounters the first valid %
UI\index{UI} wage record for a job (an individual (PIK\index{PIK})-employer
(SEIN\index{SEIN}) pair). Accessions are not defined for the first quarter
of available data from a given state. The QWI definition of an accession
can be interpreted as an estimate of the number of new employees added to
the payroll of the employer ({SEIN}) during the quarter. The individuals
who acceded to a particular employer were not employed by that employer
during the previous quarter but received at least one dollar of {UI}%
-covered earnings during the quarter of accession.

A separation occurs in the current quarter of the QWI system when it
encounters no valid {UI} wage record for an individual-employer pair
in the subsequent quarter. This definition of separation can be interpreted
as an estimate of the number of employees who left the employer during the
current quarter. \ These individuals received {UI}-covered earnings
during the current quarter but did not receive any {UI}-covered
earnings in the next quarter from this employer. Separations are not defined
for the last quarter of available data.

\subsection{Accession and separation from full-quarter employment}

\mindex{Accessions!full-quarter} \mindex{Separations!full-quarter}

Full-quarter employment is not a point-in-time concept. Full-quarter
accession refers to the quarter in which in individual first attains
full-quarter employment status at a given employer. Full-quarter separation
occurs in the last full-quarter that an individual worked for a given
employer.

As noted above, full-quarter employment refers to an estimate of the number
of employees who were employed at a given employer during the entire
quarter. An accession to full-quarter employment, then, involves two
additional conditions that are not relevant for ordinary accessions. First,
the individual (PIK\index{PIK}) must still be employed at the end of the
quarter at the same employer (SEIN\index{SEIN}) for which the ordinary
accession is defined. At this point (the end of the quarter where the
accession occured and the beginning of the next quarter) the individual has
acceded to continuing-quarter status. An accession to continuing-quarter
status means that the individual acceded in the current quarter and is
end-of-quarter employed. Next the QWI system must check for the possibility
that the individual becomes a full-quarter employee in the subsequent
quarter. An accession to full-quarter status occurs if the individual
acceded in the previous quarter, and is employed at both the beginning and
end of the current quarter. Consider the following example. An individual's
first valid UI\index{UI} wage record with employer \textit{A} occurs in
1999:2. The individual, thus acceded in 1999:2. The same individual has a
valid wage record with employer \textit{A} in 1999:3. The QWI system treats
this individual as end-of-quarter employed in 1999:2 and beginning of
quarter employed in 1999:3. The individual, thus, acceded to
continuing-quarter status in 1999:2. If the individual also has a valid %
{UI} wage record at employer \textit{A} in 1999:4, then the
individual is full-quarter employed in 1999:3. Since 1999:3 is the first
quarter of full-quarter employment, the QWI system considers this individual
an accession to full-quarter employment in 1999:3.

Full-quarter separation works much the same way. One must be careful about
the timing, however. If an individual separates in the current quarter, then
the QWI system looks at the preceding quarter to determine if the individual
was employed at the beginning of the current quarter. An individual who
separates in a quarter in which that person was employed at the beginning of
the quarter is a separation from continuing-quarter status in the current
quarter. Finally, the QWI system checks to see if the individual was a
full-quarter employee in the preceding quarter. An indivdidual who was a
full quarter employee in the previous quarter is treated as a full-quarter
separation in the quarter in which that person actually separates. Note,
therefore, that the definition of full-quarter separation preserves the
timing of the actual separation (current quarter) but restricts the estimate
to those individuals who were full-quarter status in the preceding quarter.
\ For example, suppose that an individual separates from employer \textit{A}
in 1999:3. This means that the individual had a valid {UI} wage
record at employer \textit{A} in 1999:3 but did not have a valid {UI}
wage record at employer \textit{A} in 1999:4. The separation is dated
1999:3. Suppose that the individual had a valid {UI} wage record at
employer \textit{A} in 1999:2. Then, a separation from continuing quarter
status occured in 1999:3. Finally, suppose that this individual had a valid %
{UI} wage record at employer \textit{A} in 1999:1. Then, this
individual was a full-quarter employee at employer \textit{A} in 1999:2. The
QWI system records a full-quarter separation in 1999:3.

\subsection{Point-in-time estimates of new hires and recalls}

\mindex{New hires!point-in-time} \mindex{Recalls!point-in-time}

The QWI system refines the concept of accession into two subcategories:\ new
hires and recalls. In order to do this, the QWI system looks at a full year
of wage record history prior to the quarter in which an accession occurs. If
there are no valid wage records for this job (PIK\index{PIK}-SEIN\index{SEIN})
during the four quarters preceding an accession, then the accession is
called a new hire; otherwise, the accession is called a recall. Thus, new
hires and recalls sum to accessions. For example, suppose that an individual
accedes to employer \textit{A} in 1999:3. Recall that this means that there
is a valid UI\index{UI} wage record for the individual 1 at employer \textit{A%
} in 1999:3 but not in 1999:2. If there are also no valid {UI} wage
records for individual 1 at employer \textit{A} for 1999:1, 1998:4 and
1998:3, then the QWI system designates this accession as a new hire of
individual 1 by employer \textit{A} in 1999:3. Consider a second example in
which individual 2 accedes to employer \textit{B} in 2000:2. Once again, the
accession implies that there is not a valid wage record for individual 2 at
employer \textit{B} in 2000:1. If there is a valid wage record for
individual 2 at employer \textit{B} in 1999:4, 1999:3, or 1999:2, then the
QWI system designates the accession of individual 2 to employer \textit{B}
as a recall in 2000:2. \ New hire and recall data, because they depend upon
having four quarters of historical data, only become available one year
after the data required to estimate accessions become available.

\subsection{New hires and recalls to and from full-quarter employment}

\mindex{New hires!full-quarter} \mindex{Recalls!full-quarter}

Accessions to full-quarter status can also be decomposed into new hires and
recalls. The QWI system accomplishes this decomposition by classifying all
accession to full-quarter status who were classified as new hires in the
previous quarter as new hires to full-quarter status in the current quarter.
Otherwise, the accession to full-quarter status is classified as a recall to
full-quarter status. For example, if individual 1 accedes to full-quarter
status at employer \textit{A} in 1999:4 then, according to the definitions
above, individual 1 acceded to employer \textit{A} in 1999:3 and reached
full-quarter status in 1999:4. Suppose that the accession to employer 
\textit{A} in 1999:3 was classified as a new hire, then the accession to
full quarter status in 1999:4 is classified as a full-quarter new hire. For
another example, consider individual 2 who accedes to full-quarter status at
employer \textit{B} in 2000:3. Suppose that the accession of individual 2 to
employer \textit{B} in 2000:2, which is implied by the full-quarter
accession in 2000:3, was classified by the QWI system as a recall in 2000:2;
then, the accession of individual 2 to full-quarter status at employer 
\textit{B} in 2000:3 is classified as a recall to full-quarter status.

\subsection{Job creations and destructions}

\mindex{Job creation} \mindex{Job destruction}
Job creations and destructions are defined at the employer (SEIN\index{SEIN})
level and not at the job (PIK\index{PIK}-{SEIN}) level. To construct an
estimate of job creations and destructions, the QWI system totals beginning
and ending employment for each quarter for every employer in the UI\index{UI}
wage record universe, that is, for an employer who has at least one valid %
{UI} wage record during the quarter. The QWI system actually uses the %
\Cite{DavisHaltiwangerSchuh} formulas for job creation and destruction
(see definitions in Appendix~\Vref{cha:definitions}). 
Here, we use a
simplified definition. If end-of-quarter employment is
greater than beginning-of-quarter employment, then the employer has created
jobs. The QWI system sets job creations in this case equal to end-of-quarter
employment less beginning-of-quarter employment. The estimate of job
destructions in this case is zero. On the other hand, if
beginning-of-quarter employment exceeds end-of-quarter employment, then this
employer has destroyed jobs. The QWI system computes job destructions in
this case as beginning-of-period employment less end-of-period employment.
The QWI system sets job creations to zero in this case. Notice that either
job creations are positive or job destructions are postive, but not both.
Job creations and job destructions can simultaneously be zero if
beginning-of-quarter employment equals end-of-quarter employment. There is
an important suptelty regarding job creations and destructions when they are
computed for different sex and age groups within the same employer. There
can be creation and destruction of jobs for certain demographic groups
within the employer without job creation or job destruction occuring
overall. \ That is, jobs can be created for some demographic groups and
destroyed for others even at enterprises that have no change in employment
as a whole.

Here is a simple example. Suppose employer \textit{A} has 250 employees at
the beginning of 2000:3 and 280 employees at the end of 2000:3. Then,
employer \textit{A} has 30 job creations and zero job destructions in
2000:3. Now suppose that of the 250 employees 100 are men and 150 are women
at the beginning of 2000:3. At the end of the quarter suppose that there are
135 men and 145 women. Then, job creations for men are 35 and job
destructions for men are 0 in 2000:3. For women in 2000:3 job creations are
0 and job destructions are 5. Notice that the sum of job creations for the
employer by sex (35 + 0) is not equal to job creations for the employer as a
whole (30) and that the sum of job destructions by sex (0 + 5) is not equal
to job destructions for the employer as a whole.

\subsection{Net job flows}

\mindex{Job flows!net} 
\index{Net job flows|see{Job flows}}

Net job flows are also only defined at the level of an employer (%
SEIN\index{SEIN}). They are the difference between job creations and job
destructions. Net job flows are, thus, always equal to end-of-quarter
employment less beginning of quarter employment.

Returning to the example in the description of job creations and
destructions. Employer \textit{A} has 250 employees at the beginning of
2000:3 and 280 employees at the end of 2000:3. Net job flows are 30 (job
creations less job destructions or beginning-of-quarter employment less
end-of-quarter employment). Suppose, once again that employment of men goes
from 100 to 135 from the beginning to the end of 2000:3 and employment of
women goes from 150 to 145. Notice, now, that net job flows for men (35)
plus net job flows for women ($-5$) equals net job flows for the employer as
a whole (30). Net job flows are additive across demographic groups even
though gross job flows (creations and destructions) are not.

Some useful relations among the worker and job flows include:

\begin{itemize}
\item Net job flows = job creations - job destructions 
\index{Job flows!net}%
\index{Job creation}%
\index{Job destruction}

\item Net job flows = end-of-quarter employment - beginning-of-period
employment 
\index{Job flows!net}%
\index{Employment!point in time}

\item Net job flows = accessions - separations 
\index{Job flows!net} 
\index{Accessions!point-in-time} 
\index{Separations!point-in-time}
\end{itemize}

These relations hold for every demographic group and for the employer as a
whole. Additional identities are shown in Appendix~\ref{cha:definitions}.

\subsection{Full-quarter job creations, job destructions and net job flows}

\mindex{Job flows!full-quarter} \mindex{Accessions!full-quarter} %
\mindex{Separations!full-quarter}

The QWI system applies the same job flow concepts to full-quarter employment
to generate estimates of full-quarter job creations, full-quarter job
destructions, and full-quarter net job flows. Full-quarter employment in the
current quarter is compared to full-quarter employment in the preceding
quarter. If full-quarter employment has increased between the preceding
quarter and the current quarter, then full-quarter job creations are equal
to full-quarter employment in the current quarter less full-quarter
employment in the preceding quarter. In this case full-quarter job
destructions are zero. If full-quarter employment has decreased between the
previous and current quarters, then full-quarter job destructions are equal
to full-quarter employment in the preceding quarter minus full-quarter
employment in the current quarter. In this case, full-quarter job
destructions are zero. Full-quarter net job flows equal full-quarter job
creations minus full-quarter job destructions. The same identities that hold
for the regular job flow concepts hold for the full-quarter concepts.

\subsection{Average earnings of end-of-period employees}

\mindex{Earnings!employees!end-of-period} 
\index{Average earnings|see{Earnings}}

The average earnings of end-of-period employees is estimated by first
totaling the UI\index{UI} wage records for all individuals who are
end-of-period employees at a given employer in a given quarter. Then the
total is divided by the number of end-of-period employees for that employer
and quarter.

\subsection{Average earnings of full-quarter employees}

\mindex{Earnings!employees!full-quarter}

Measuring earnings using UI\index{UI} wage records in the QWI system presents
some interesting challenges. The earnings of end-of-quarter employees who
are not present at the beginning of the quarter are the earnings of
accessions during the quarter. The QWI system does not provide any
information about how much of the quarter such individuals worked. The range
of possibilities goes from 1 day to every day of the quarter. Hence,
estimates of the average earnings of such individuals may not be comparable
from quarter to quarter unless one assumes that the average accession works
the same number of quarters regardless of other conditions in the economy.
Similarly, the earnings of beginning-of-quarter who are not present at the
end of the quarter represent the earnings of separations. These present the
same comparison problems as the average earnings of accessions; namely, it
is difficult to model the number of weeks worked during the quarter. If we
consider only those individuals employed at the firm in a given quarter who
were neither accessions nor separations during that quarter, we are left,
exactly, with the full-quarter employees, as discussed above.

The QWI system measures the average earnings of full-quarter employees by
summing the earnings on the {UI} wage records of all individuals at a
given employer who have full-quarter status in a given quarter then dividing
by the number of full-quarter employees. For example, suppose that in 2000:2
employer \textit{A} has 10 full-quarter employees and that their total
earnings are $\$300,000.$ Then, the average earnings of the full-quarter
employees at \textit{A} in 2000:2 is $\$30,000.$ Suppose, further that 6 of
these employees are men and that their total earnings are $\$150,000$. So,
the average earnings of full-quarter male employees is $\$25,000$ in 2000:2
and the average earnings of female full-quarter employees is $\$37,500$ $%
\left( =\$150,000/4\right) $.

\subsection{Average earnings of full-quarter accessions}

\mindex{Earnings!accessions!full-quarter}

As discussed above, a full-quarter accession is an individual who acceded in
the preceding quarter and acheived full-quarter status in the current
quarter. The QWI system measures the average earnings of full-quarter
accessions in a given quarter by summing the UI\index{UI}\ wage record
earnings of all full-quarter accessions during the quarter and dividing by
the number of full-quarter accessions in that quarter.

\subsection{Average earnings of full-quarter new hires}

\mindex{Earnings!new hires!full-quarter} 
\mindex{New
hires!Earnings!full-quarter}

Full-quarter new hires are accessions to full-quarter status who were also
new hires in the preceding quarter. The average earnings of full-quarter new
hires are measured as the sum of UI\index{UI} wage records for a given
employer for all full-quarter new hires in a given quarter divided by the
number of full-quarter new hires in that quarter.

\subsection{Average earnings of full-quarter separations}

\mindex{Earnings!separations!full-quarter}

Full-quarter separations are individuals who separate during the current
quarter who were full-quarter employees in the previous quarter. The QWI
system measures the average earnings of full-quarter separations by summing
the earnings for all individuals who are full-quarter status in the current
quarter and who separate in the subsequent quarter. This total is then
divided by full-quarter separations in the subsequent quarter. The average
earnings of full-quarter separations is, thus, the average earnings of
full-quarter employees in the current quarter who separated in the next
quarter. Note the dating of this variable.

\subsection{Average periods of non-employment for accessions, new hires, and
recalls}

\mindex{Non-employment!accessions} \mindex{Non-employment!new hires} %
\mindex{Non-employment!recalls} 
\index{Average periods of non-employment|see{Non-employment}}

As noted above an accession occurs when a job starts; that is, on the first
occurance of an SEIN\index{SEIN}-PIK\index{PIK} pair following the first quarter
of available data. When the QWI system detects an accession, it measures the
number of quarters (up to a maximum of four) that the individual spent
non-employed in the state prior to the accession. The QWI system estimates
the number of quarters spent non-employed by looking for all other jobs held
by the individual at any employer in the state in the preceding quarters up
to a maximum of four. If the QWI system doesn't find any other valid %
UI\index{UI}-wage records in a quarter preceding the accession it augments
the count of non-employed quarters for the individual who acceded, up to a
maximum of four. Total quarters of non-employment for all accessions is
divided by accessions to estimate average periods of non-employment for
accessions.

Here is a detailed example. Suppose individual 1 and individual 2 accede to
employer \textit{A} in 2000:1. In 1999:4, individual \textit{A} does not
work for any other employers in the state. In 1999:1 through 1999:3
individual 1 worked for employer \textit{B}. Individual 1 had one quarter of
non-employment preceding the accession to employer \textit{A} in 2000:1.
Individual 2 has no valid {UI} wage records for 1999:1 through
1999:4. Indivdiual 2 has four quarters of non-employment preceding the
accession to employer \textit{A} in 2000:1. The accessions to employer 
\textit{A} in 2000:1 had an average of 2.5 quarters of non-employment in the
state prior to accession.

Average periods of non-employment for new hires and recalls are estimated
using exactly analogous formulas except that the measures are estimated
separately for accesions who are also new hires as compared with accession
who are recalls.

\subsection{Average number of periods of non-employment for separations}

\mindex{Non-employment!separations}

Analogous to the average number of periods of non-employment for accessions
prior to the accession, the QWI system measures the average number of
periods of non-employment in the state for individuals who separated in the
current quarter, up to a maximum of four. When the QWI system detects a
separation, it looks forward for up to four quarters to find valid %
UI\index{UI} wage records for the individual who separated and other
employers in the state. Each quarter that it fails to detect any such jobs
is counted as a period of non-employment, up to a maximum of four. The
average number of periods of non-employment is estimated by dividing the
total number of periods of non-employment for separations in the current
quarter by the number of separations in the quarter.

\subsection{Average changes in total earnings for accessions and separations}

\mindex{Earnings!accessions!changes} \mindex{Earnings!separations!changes}

The QWI system measures the change in total earnings for individuals who
accede or separate in a given quarter. For an individual accession in a
given quarter, the QWI system computes total earnings from all valid wage
records for all of the individual's employers in the preceding quarter. The
system then computes the total earnings for the same individual for all
valid wage records and all employers in the current quarter. The acceding
individual's change in earnings is the difference between the current
quarter earnings from all employers and the preceding quarter earnings from
all employers. The average change in earnings for all accessions is the
total change in earnings for all accesions divided by the number of
accessions.

The QWI system computes the average change in earnings for separations in an
analogous manner. The system computes total earnings from all employers for
the separating indivdiual in the current quarter and subtracts total
earnings from all employers in the subsequent quarter. The average change in
earnings for all separations is the total change in earnings for all
separations divided by the number of separations.

Here is an example for the average change in earnings of accessions. Suppose
individual 1 accedes to employer \textit{A} in 2000:3. Earnings for
individual 1 at employer \textit{A} in 2000:3 are $\$8,000$. Individual 1
also worked for employer \textit{B} in 2000:2 and 2000:3. Individual 1's
earnings at employer \textit{B} were $\$7,000$ and $\$3,000$ in in 2000:2
and 2000:3, respectively. Individual 1's change in total earnings between
2000:3 and 2000:2 was $\$4,000$ $\left( =\$8,000+\$3,000-\$7,000\right) .$
Individual 2 also acceded to employer \textit{A} in 2000:3. Individual 2
earned $\$9,000$ from employer \textit{A} in 2000:3. Individual 2 had no
other employers during 2000:2 or 2000:3. Individual 2's change in total
earnings is $\$9,000.$ The average change in earnings for all of employer 
\textit{A}'s accessions is $\$6,500$ $\left( =\left( \$4,000+\$9,000\right)
/2\right) ,$ the average change in total earnings for individuals 1 and 2.


  

%%% Local Variables: 
%%% mode: latex
%%% TeX-master: "qwi-overview"
%%% TeX-master: "qwi-overview"
%%% End: 

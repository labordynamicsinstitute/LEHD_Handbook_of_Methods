%TCIDATA{Version=4.00.0.2321}
%TCIDATA{LaTeXparent=0,0,sw-edit.tex}
                      
%TCIDATA{ChildDefaults=chapter:1,page:1}


% -*- latex -*- 
%
% Time-stamp: <02/03/14 00:07:13 vilhuber> 
%              Automatically adjusted if using Xemacs
%              Please adjust manually if using other editors
%
% qwi-overview-intro.tex
% Responsible: John A
% Part of QWI_methods.tex

\section{Introduction}
\label{sec:intro}

The Longitudinal Employer Household Dynamics (LEHD) Program is a new 
state/federal partnership between the Census Bureau and ten states. Both 
sides gain from this partnership. States fulfill their mandate of providing 
high quality local labor market information to their customers. The Census 
Bureau uses state UI wage record and ES202 data to fulfill its Title 13 
mandate: improving the Census Bureau's economic and demographic censuses, 
surveys, and intercensal population estimates.





The Memoranda of Understanding (MOU) between the Census Bureau and the state 
partners specify that this is a voluntary partnership. Research beyond that 
specified in the MOU must have the express written authorization of the 
state data custodian. States receive three key products from Census: (1) 
quarterly employment indicators about the state economy at detailed industry 
and geography; (2) enhanced UI wage records; and (3) information about 
successor/predecessor firms.





The Census Bureau receives UI wage records and ES-202 establishment records
from the states. As a part of its Title~13 mission, the Bureau then uses
these products to integrate information about the individuals (place of
residence, sex, birth date, place of birth, race, education) with
information about the employer (place of work, industry, employment,
sales).  These integrated files are used to improve the Census Bureau's
demographic surveys, like the Current Population Survey, the Survey of
Income and Program Participation, and the American Community Survey. They
are also used to improve the Census Bureau's Business Register, which is
the sampling frame for all its economic data and the initial contact frame
for the Economic Census.





This document describes how the LEHD Program uses these integrated files to
produce the Employment Dynamics Estimates, which are the source of the
quarterly employment indicators returned to the states.  In the following
section, we provide definitions of employer, employee, job, and all
associated worker flow, job flow, and earnings measures. We then give an
overview of the different raw data inputs and how they are treated and
adjusted in Section~\ref{sec:input_files}. In a system that focusses on the
dynamics at the individual and firm level, proper identification of the
entities is important, and we briefly highlight the steps undertaken to
this end, although the finer detail is available in separate documents. The
raw data are then aggregated and standardized into a series of component
files, as described in Section~\ref{sec:files}. Finally,
Section~\ref{sec:aggregate} illustrates how they are brought together to
create the QWI statistics. It will soon become clear to the reader that the
level of detail potentially available with these statistics requires
special attention to the confidentiality of the the underlying entities.
How their identity is protected is described in Section
\ref{sec:confidentiality}. A glimpse at some examples from the final
statistics are provided in Section~\ref{sec:final}.

%%% Local Variables: 
%%% mode: latex
%%% TeX-master: "qwi-overview"
%%% End: 

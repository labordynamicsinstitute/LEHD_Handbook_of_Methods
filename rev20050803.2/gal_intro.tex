% -*- latex -*-
% based on Marc's              Geocoded Address List Version 3.0
%

The Geocoded Address List (GAL) is a data set containing unique commercial and residential addresses
in a state geocoded to the Census Block and latitude/longitude coordinates. The file encompasses
addresses from the state ES202 data, the Census Bureau's Business Register (BR), the Census Bureau's Master
Address File (MAF), the American Community Survey Place of Work file (ACS-POW), and others.
Addresses from these source files go through geocoding software (Group1's
Code1), address standardizers (Ascential/Vality), and matching software
(Ascential/Vality) for
unduplication. 

% %, and several other steps in SAS. This document refers to one year's data from a source file
% %as a file-year (for example, the 1995 ES202). 
% 
% The job stream follows the steps below using the indicated software.
% 
% Step 1:   Create input (SAS).
% Step 2:   Standardize and geocode addresses (Code1).
% Step 3:   Parse and standardize address elements (Vality Standardize).
% Step 4:   Match addresses, flag masters and duplicates (Vality Unduplicate).
% Step 5:  Create preliminary crosswalk and unique address list with address identifier (SAS).
% Step 6:   Set file-year flags, create GAL Crosswalks containing the input identifier and address identifier (SAS).
% Step 7:   Retrieve and derive block codes and coordinates from the MAF (SAS).
% Step 8:   Impute block within known tract (SAS).
% Step 9:   Create GAL by adding higher-level geocodes by block (SAS). 
% Step 10: Delete intermediate data files and create links. 
% 
% The section entitled "Details of GAL Processing" gives details on each of these steps. 

The final output consists of the address list and a crosswalk for each
processed file-year. The GAL contains each unique address, identified by a
GAL identifier called {\tt galid}, its geocodes, a flag for each file-year
in which it appears, data quality indicators, and data processing
information, including the release date of the Geographic Reference File (GRF).
%
 The GAL Crosswalk contains the ID of each input entity and the
ID of its address ({\tt galid}).

% The following section describes the GAL's content. 
% 
% 
% \subsection{Important Variables}
% 
% \subsubsection{Unique identifier}
% 
% The variable {\tt galid} is the unique address identifier on the GAL, a 26-character string consisting of the
% letter 'A' in the first column followed by the 2-character state FIPS code and a zero-padded sequential
% number. {\tt galid}  is recreated each time a GAL is created. There's no consistency in the galid between
% versions or vintages of the GAL.
% 
% \subsubsection{Geographic vintage}
% 
% The release date (year) of the GRF identifies the geographic vintage
% variable {\tt a\_vintage}.
% 
\subsection{Geographic codes and their sources}

A geocode on the GAL  is constructed as 
\begin{center}
  {\tt FIPS-state(2)||FIPS-county(3)||Census-tract(6)}, 
\end{center}
and it uniquely identifies the
Census tract in the U.S. The tract is the lowest level of geography recommended for analysis. The Census
block within the tract is also available on the GAL, but the uncertainties in block-coding make block-level analysis
questionable. However, geocoding to the block allows us to add all the higher-level geocodes to the
addresses. 


\subsubsection{Block coding}


Block coding is achieved by a combination of geocoding software (Group1's
Code1), a match to the MAF, or an imputation based on addresses within the
tract. Table~\ref{tab:gal:a_block_src} describes the typical
distribution of geocode sources.

\begin{table}[htbp]
\footnotesize
  \centering
  \caption{Sources of geocodes on GAL}
  \label{tab:gal:a_block_src}
  \begin{tabular}{lrp{5in}}
%a_block_src
       &Typical\\
Value  &  Percent &  Meaning                                                                              \\  
\hline
\hline
C      &         12.20   &  Code1, or the address matches an address for which Code1 supplied the block code     \\
M      &         81.86   &  The MAF - the address is a MAF address or matches a MAF address                      \\
E      &          0.00   &  The MAF, the street address is exactly the same as a MAF address in the same tract   \\
W      &          0.03   &  The MAF, the street address is between 2 MAF addresses on the same block face        \\
O      &          1.23   &  Imputed using the distribution of commercial addresses in the tract                  \\
S      &          1.17   &  Imputed using the distribution of residential addresses in the tract                 \\
I      &          0.01   &  Imputed using the distribution of mixed-use addresses in the tract                   \\
D      &          0.00   &  Imputed using the distribution of all addresses in the tract                         \\
missing&          3.50   &  Block code is missing                                                                \\                
\hline
       &       100.00\\

  \end{tabular}
\end{table}

In all states observed so far except California, no address required the 'D' method. That is, almost every
tract where an address lacks a block code contains commercial, residential, and mixed-use addresses. 

The Census Bureau splits blocks to accommodate changes in political boundaries. Most commonly, these
are place boundaries (a place is a city, village, or similar municipality). The resulting block parts are
identified by 2 suffixes, each taking a value from A to Z. The GAL assigns the block part directly from
the MAF, or by adopting the one whose internal point is closest to the address by the straight-line
distance. 

% The variables a_block_suf1 and a_block_suf2 identify the block part, and a_block_suf_src
% generated in Step 9 describes the method used to assign it.
% 
% a_block_suf_src
% Value  Typical Percent   Meaning
% A               1.50          Assigned by distance
% M               4.18          The MAF - the address is a MAF address or matches a MAF address
% missing                94.32       Not a split block
%               100.00

The GAL also provides the following components of the geocodes as separate
variables, for convenience: FIPS code
(5 digits), FIPS state code (the first 2 digits of the FIPS code), FIPS
county code within state (the right-most 3 digits of the FIPS code), and
Census tract code (a tract within the county, a 6-digit code).
%
%a_ssccc        FIPS-state(2)||FIPS-county (3)
%a_st      FIPS state (2)
%a_cty          FIPS county within the state (3)
%a_tract        Census tract within the county (6)

Higher-level geographic codes originate from the Block Map File (BMF). 
% and attach to the GAL in Step 9. 
The BMF is an extract of the GRF-C (Geographic Reference File - Codes). All geocodes are
character variables. FIPS (Federal Information Processing Standard) codes are unique within the U.S.;
Census codes are not. Table~\ref{tab:gal:geocodes_hi} lists
 the available higher-level geocodes.

\begin{table}[htbp]
  \centering
  \caption{Higher-level geocodes on GAL}
  \label{tab:gal:geocodes_hi}
  \begin{tabular}{ll}
\hline
\hline
\tt a\_fipsmcd  &    5-digit FIPS Minor Civil Division (a division of a county)                    \\
\tt a\_mcd      &    3-digit Census Minor Civil Division (a division of a county)                  \\
\tt a\_fipspl   &    5-digit FIPS Place                                                            \\
\tt a\_place    &    4-digit Census Place                                                          \\
\tt a\_msapmsa  &    Metropolitan-Statistical-Area(4)||Primary-Metropolitan-Statistical-Area(4)    \\
\tt a\_wib      &    6-digit Workforce Investment Board area                                       \\ 
\hline
  \end{tabular}
\end{table}



\subsubsection{Geographic coordinates}

The geographic coordinates of each address available as latitude and
longitude
% are in the variables a_latitude and a_longitude. These variables are
% numeric 
with 6 implied decimals.
% (divide by 1,000,000 to convert them). 
The coordinates are not always as accurate as 6 decimal places implies. An
indicator flag of their quality is provided. Table~\ref{tab:gal:geoqual}
provides the typical distribution of codes, which range from $1$ (highest
quality) to $9$ (lowest quality).
%in the variable a_geoqual, a
%numeric variable taking values from 1 to 9 and generated in Steps 7, 8, and 9:

%a_geoqual
\begin{table}[htbp]
  \centering
  \caption{Quality of geographic coordinates}
  \label{tab:gal:geoqual}
  \begin{tabular}{lrl}
      &Typical\\
Value &  Percent&   Meaning\\
\hline
\hline
1     &         80.15  &        Rooftop or MAF (most accurate)\\
2     &          1.59  &        ZIP4 or block face, block face is certain\\
3     &         10.12  &        Block group is certain\\
4     &          4.65  &        Tract is certain\\
9     &          3.50  &        Coordinates are missing\\
\hline
      &        100.00  &\\
  \end{tabular}
\end{table}

%The format 'agqual' provided by 'format_geo.sas' in '/programs/projects/auxiliary/Formats' contains the
%meanings of the a_geoqual values listed above.
%

Variables indicating the source of the geographic coordinates (Block
internal point, geocoding software, MAF, or otherwise derived) are also
available. Most coordinates are provided by either commercial geocoding
software or the MAF.


%Two other variables give information about the coordinates. The flag a_latlong_src indicates their
%source: 
% 
% a_latlong_src
% Value  Typical Percent   Meaning
% B              14.77          Block (or block part) internal point
% C              70.04          Code1
% D               0.03          Derived
% M              11.66          the MAF
% missing                 3.50       Coordinates are missing
%               100.00
% Few addresses have a_latlong_src equal to 'D'. Deriving coordinates occurs only if they're still missing
% after Code1 processing and direct extraction from the MAF, but the tract is known. In this case, the flag
% a_latlong_drv generated in Step 7 describes the derivation method:
% 
% a_latlong_drv
% Value  Typical Percent   Meaning
% F               0.00          Adopted from the only address on the block face
% P               0.04          Extrapolated between 2 addresses on the block face
% missing                99.96       Derivation not performed
%               100.00
% 
% In GAL Version 1, deriving coordinates and block codes by these methods was an important means of
% block-coding. It rarely operates now, since Code1 began providing block codes. Nevertheless, GAL
% Version 3 still exhausts all methods of assigning block-codes and coordinates before resorting to
% imputation. 

%File-year flags

Finally, a set of flags also indicates, for each year and source file,
whether an address appears on that file. 

% A set of flags generated in Step 6 indicates what file-years an address appears in. The names of the flags
% conform to the naming convention [f][yyyy] for the source file [f] and year [yyyy], where [f] takes the
% following values:
% 
% Business Register (SSEL)                f = b
% ES202                              f = e
% Master Address File                     f = m
% American Community Survey - Place of Work    f = p
% American Housing Survey            f = h

For example, the flag variable {\tt b1997} equals 1 if the address is on the 1997 BR; otherwise it equals 0.
If a state partner supplies 1991 ES202 data with no address information,
{\tt e1991 } will be 0 for all
addresses. Typically,  between 3 and 6 percent of addresses are present on
any given year's ES202 files, 
%, the b[yyyy] flags equal 1 for 
between 4 and 10 percent are present on a specific BR year file, and 
% the m[yyyy] flag is 1 for 
between 80 and 90 percent are present on the MAF. 
%The p[yyyy] and h[yyyy] flags equal 1 for less 
Less than 1 percent of addresses are found on the ACS-POW and AHS
data, because these are sample surveys. 

% 
% 
%                                                         Other Variables
% 
% The variable occupant_type, recoded from the file-year flags in Step 8, indicates whether an address is
% commercial, residential, or mixed-use. 
% 
% The tracking ID bigsrcid, created in Step 1, uniquely identifies the entity that supplied the address. It
% consists of [f], [yyyy], the unique ID from the input file, zero-padding, and for some source files, a flag
% indicating which set of variables supplied the address. For addresses originating in the Business Register,
% another flag indicates the single-unit data set or the multi-unit data set. This tracking ID variable is useful
% for debugging. 
% 
% A diagnostic variable srcmast contains [f][yyyy], indicating the file-year that supplied this address. Bear
% in mind that it's often arbitrary which observation becomes the master address for a set of duplicates in
% Step 1 and Step 4, so bigsrcid and srcmast don't indicate anything special about an address or an entity.
% They simply identify the origin of an address that became a master address in unduplication. 
% 
% The names of Code1 variables contain the prefix c1_. They impart mostly diagnostic information from
% Code1 processing. They could be useful for development work or address research. The parsed address
% elements from Step 3 sit in the variables named with the prefix v_. They could be useful for development
% work, particularly in improving the parsing routine. 
% 
% 

\subsection{Accessing the GAL: the GAL Crosswalks}
\label{sec:gal:xwalks}

The GAL Crosswalks allow data users to extract geographic and address information about any entity whose
address went into the GAL. Each crosswalk contains the identifiers of the entity, its galid, and sometimes
flags. To attach geocodes, coordinates, or address information to an entity, merge the GAL Crosswalk to
the GAL by galid, outputting only observations existing on the GAL Crosswalk. Then merge the
resulting file to the entities of interest using the entity identifiers. An entity whose address wasn't
processed (because it's out of state or lacks address information) will
have blank GAL data. Table~\ref{tab:gal:xwalks} lists the entity
identifiers by dataset or survey.

\begin{table}[htbp]
  \centering
  \caption{GAL crosswalk entity identifiers}
  \label{tab:gal:xwalks}
  \begin{tabular}{lp{3in}p{2in}}
Dataset & Entity identifier variables & Note \\
\hline
\hline
AHS     &\tt \footnotesize control and year \\
ES202   &\tt \footnotesize sein, seinunit, year, and quarter & $e\_flag = p$ for physical
                                             addresses, $e\_flag = m$ for mailing addresses 
                                             as source of address info\\
ACS-POW &\tt  \footnotesize acsfileseq, cmid, seq, and pnum. \\
BR      &\tt  \footnotesize cfn, year, and singmult & $singmult$ indicates
                                     whether the entity resides in the single-unit ($su$) or the multi-unit
                                     ($mu$) data set.\\
        &\tt                          & $b\_flag =P$ if  physical address,
                                     $b\_flag=M $ for mailing address.\\
MAF     &\tt  \footnotesize mafid and year \\
\hline
  \end{tabular}
\end{table}

% For the AHS, the entity ID variables are control and year. 
% 
% For the ES202, the entity ID variables are sein, seinunit, year, and quarter. The flag variable e_flag
% indicates whether the address came from the address_street1, address_state, and address_zip9
% variables (e_flag=P for physical address) or from the ui_address_street1, ui_address_state, and
% ui_address_zip9 variables (e_flag=M for mailing address). 
% 
% For the ACS-POW data, the entity ID variables are acsfileseq, cmid, seq, and pnum. 
% 
% For the SSEL, the entity ID variables are cfn, year, and singmult. The flag variable singmult indicates
% whether the entity resides in the single-unit (su) or the multi-unit (mu) data set. Another flag variable
% b_flag indicates whether the address originated from the variables pstreet, pplce, pst, and pzip
% (b_flag=P for physical address) or street, plce, st, and zip (b_flag=M for mailing address). . 
% 
% For the MAF, mafid and year identify entities. 
%              Resources for geographic information
% 
% The best place for information about Census geography is 
% <http://www.census.gov/geo/www/reference.html>
% 
% Especially informative is the Geographic Areas Reference Manual (GARM), at
% <http://www.census.gov/geo/www/garm.html>
% 
% 
% 
%                    Details of GAL Processing
% 
%                         Control Programs
% 
% The control program is 'rungal.ksh', which documents itself. Simply type the name of the program to see
% instructions. A super-control program 'rungal_all.ksh' runs GALs in any number of states, limiting each
% step of GAL processing to 1 state a time. It actually launches another program 'rungal_line.ksh' for each
% state specified. The super-control program requires you to specify the states,  the steps, and other
% parameters in the program and confirm them before it will execute. Submit 'rungal_all.ksh' in the
% foreground to confirm the parameters.
% 
%                            Input Data
% 
% The input data consists of addresses, geocodes, and coordinates. Currently, the source files providing
% addresses consist of the following (future work will add the Non-employer file):
% ACS-POW   American Community Survey Place of Work (2001 and later)
% AHS       American Housing Survey (2002)
% ES202          Employment Security form 202 (all available years 1990 and later)
% SSEL      Business Register (Standard Statistical Establishment List 1990 and later)
% MAF       Master Address File (the year following the year of the desired geographic vintage)
% 
% The source files providing geocodes and coordinates are the following:
% GCP       the databases of Group1's Geographic Coding Plus software
% MAF       Master Address File
% GRF-C          Geographic Reference File, Codes (encompassed in the BMF)
% WIB-C          Workforce Investment Board, Codes (encompassed in the BMF)
% BMF       Block Map File
% 
% A flow-chart of how all these files relate to the GAL is in the document 'gal_schematic.pdf.'Below is a description of each step of GAL processing. Further information is available in the GAL
% programs, the documentation of the input data files, and the materials explaining the Code1 and Vality
% software packages. 
% 
%                  Step 1. Creating Address Input
% 
% This step creates all the input address data for the GAL. It launches a SAS job for each input file in
% parallel. The launched macro programs find observations with a unique identifier and address
% information in the specified state. The program creates one big data file from all the input data files, sorts
% it, outputs unique addresses separately from exact address duplicates, and splits the file of unique
% addresses into as many as 8 same-size chunks for parallel processing in the next step. 
% 
% The process keeps track of addresses by means of a tracking ID bigsrcid. This variable contains 5 digits
% to identify the file-year, IDs from the source file, sometimes a source-file-specific flag, and some zeroes
% as place holders. Quality Assurance (QA) for Step 1 checks bigsrcid for blank columns.
% 
% Address processing may exclude an address for any of the following reasons: it contains a duplicate
% source file identifier; it lacks a street address and zip; or it's in the wrong state. Only 1 observation enters
% address processing even if an input entity has more than 1 (for instance, a physical and mailing address).
% The following section lists the conditions required for processing an address from each source file. 
% 
% ES202
% 
% The source data set is state-specific. A flag indicates whether the address was a physical ('P') or mailing
% ('M') address. Later, Step 6 puts this flag on the crosswalk as the variable e_flag.
% 
% Conditions required to process an address are:
% 1-   a unique sein-seinunit-year-quarter
% 2-   the address is in state
% 3-   physical address street or zip is nonblank, or mailing address street or zip is nonblank
% 
% Business Register (SSEL)
% 
% The source data set covers all states. The flag 'singmult' indicates whether the address was from the
% single-unit ('S') or multi-unit ('M') source file. Additionally, for observations from the single-unit file, a
% flag indicates whether the address was a physical ('P', from variable pstreet) or mailing ('M', from
% variable street) address. All observations are 'M' from the multi-unit source file. Later, Step 6 puts this
% flag on the crosswalk as the variable b_flag. 
% 
% Conditions required to process an address are:
% 1-   a unique tracking ID
% 2-   the address is in state
% 3-   if from the multi-unit source file, the mailing address is nonblank; or if from the single-unit
%      source file, the physical address or mailing address is nonblank
% 
% Master Address File (MAF)
% 
% The source data set is state-specific. One condition suffices to process an address:
% 1-   the street or zip is nonblank
% 
% American Community Survey - Place of Work (ACS-POW)
% 
% The source data set covers all states. Conditions required to process an address are:
% 1-   a unique cmid-seq-pnum
% 2-   the address is in state
% 3-   the street or zip is nonblank
% 
%                     Step 2. Code1 Processing
% 
% This step standardizes and geocodes addresses using Group1 software. Here "standardize" really means
% correcting street names, zip codes, apartment numbers, and so forth. This step processes the chunks of
% data produced by Step 1 in parallel. The software license limits the number of jobs to 10. The GAL
% program sequence limits the number of jobs to 8, leaving 2 free for development work. Quality
% Assurance (QA) for this step checks the rates of best address matching and geocoding. 
% 
% Group1 recommendations and visual inspection of output indicate that medium strictness and a
% correctness score of 3 or lower achieves a good balance between the number and the quality of address
% corrections.  The following other options also apply. If Code1 can't find a matching address, it provides
% the input information as output. This ensures that the output field contains the best available information.
% If an input street name is an alias street name, then Code1 returns the base street name. If there are
% multiple address matches for a given input address, Code1 doesn't correct the address. The spreadsheet
% 'gal_fields_v3.xls' in '/data/doc/geocoding' provides further information on processing parameters, and
% can serve as a data dictionary for the Code1 output. 
% 
%                 Step 3. Parsing Address Elements
% 
% This step edits the Code1 address by parsing it into its elements and standardizing them.  This procedure
% prepares the address for unduplication matching. Address elements are number, street, directional,
% apartment number, and so forth. This step processes the chunks of data in parallel. 
% 
%                Step 4. Matching for Unduplication
% 
% This step creates input data files for each round of Vality matching, creating cuts of data without
% breaking apart blocking groups. It runs each round of Vality unduplication matching on each cut in
% sequence. In tests, parallel processing in this step didn't improve performance. The output data files
% consist of matched addresses with the master and duplicate addresses flagged as such.
% 
%                                 Step 5. Creating a Data Set of Unique Addresses and Preliminary Crosswalk
% 
% This step creates the unique address list and preliminary crosswalk by reading in the results of Vality
% unduplication matching and creating the unique address identifier galid. This step also builds the
% preliminary crosswalk containing the tracking ID and galid. The crosswalk is preliminary because it's a
% combined crosswalk for all file-years, it lacks the exact address duplicates set aside in Step 1, and it
% contains bigsrcid (not the source file ID). QA for this step checks the unduplication rates in each round
% of matching (production mode of GAL processing doesn't produce a report in Step 4). 
% 
% Step 6. Creating the GAL Crosswalks and Setting the File-Year Flags
% 
% This step combines the preliminary crosswalk and the address duplicates set aside in Step 1 and outputs
% the final crosswalks, each containing the source file identifiers of each input entity and its address
% identifier. SAS produces these crosswalks in parallel. Then the program sets the file-year flags to 1 or 0
% for each address on the GAL.
% 
% For QA, this step checks for any blank source-of-master-address (srcmast) and the proportion of
% addresses for which a MAF address is the master. QA also checks that each input entity appears on the
% appropriate crosswalk, and the frequency of each file-year flag. 
% 
%      Step 7. Retrieving and Deriving Geocodes from the MAF
% 
% This step retrieves and derives geocodes from the MAF and attaches them to any address possible. The
% program first retrieves MAF geocodes for MAF addresses. Then it determines block-codes and
% coordinates for any address that's the same as a MAF address or falling between 2 MAF addresses on the
% same block face. An address can be the same as a MAF address and fail to match it because of a different
% apartment number, missing data, or other reasons. 
% 
% The derivation routine affects very few addresses, but remains in the program sequence to exhaust all
% possible means of block-coding before resorting to imputation. Note that the program blanks out-of-state
% geocodes produced by Step 2. 
% 
%         Step 8. Imputing the Block Within a Known Tract
% 
% This step imputes a block when the tract is known but the block is still missing. The imputed block is a
% random draw from the distribution of residential addresses, commercial addresses, mixed addresses, or
% (as a last resort) all addresses in the tract. These 4 types of imputation, 2 methods of derivation from the
% MAF, retrieval from the MAF, and Code1 amount to 8 possible sources of the block code. For QA, this
% step checks that at least 7 of the 8 block-coding methods operated, and that the source-of-block-code
% variable (a_block_src) is never missing.
% 
%                                                  Step 9. Creating the Final GAL
% 
% This step adds higher-level geocodes such as Minor Civil Division, Place, and Metropolitan Statistical
% Area from the Block Map File (BMF) to all addresses possible. First it assigns the block suffix to
% addresses on blocks the Census Bureau has split to accomodate boundary changes. It adopts the
% coordinates of the block (or block part) internal point if the existing coordinates of an address are less
% precise than block face. This assures consistency between the coordinates and the geocode. 
% 
% In tests, a few anomalies appeared about which county a block falls into. Therefore the program now runs
% a patch when this occurs. 
% 
% For QA, this step checks how much the geocoding improved in Steps 7, 8, and 9, how many addresses
% adopted the coordinates of the block (or block part), and how many blocks needed patching.
%                                 
%                        Step 10. Clean-up
% 
% This step deletes all intermediate data files and creates links in the current/ directory pointing at the GAL
% and GAL Crosswalks in the version/geo-vintage directory.
% 
%%% Local Variables: 
%%% mode: latex
%%% TeX-master: "qwi-overview"
%%% End: 

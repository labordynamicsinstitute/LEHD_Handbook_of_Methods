%TCIDATA{LaTeXparent=0,0,bryce.tex}
                      
%TCIDATA{ChildDefaults=chapter:1,page:1}


%
% Time-stamp: <05/04/07 22:19:32 vilhuber>

%\subsection{Overview}

 A primary objective of the QWI is to provide employment, job and
worker flow, and wage measures at a very fine level of geographic
(place-of-work) and industry detail. \ The structure of the administrative
data received by LEHD from state partners, however, poses a challenge to
achieving this goal. \ QWI measures are primarily based on the processing of
UI wage records which report, with the exception of Minnesota, only the
employing firm (SEIN) of workers. \ The QCEW micro-data, however, are
comprised of establishment-level records which provide the level of
geographic and industry detail needed to produce the QWI. \ For firms
operating only one establishment, the attachment of establishment-level
characteristics is trivial. \ However, approximately 30 to 40 percent of
state-level employment is concentrated in firms that operate more than one
establishment. \ For these multi-establishment firms, the SEIN on workers'
wage records identifies the employing firm in the QCEW data, though, not the
employing establishment.

In order to attach establishment-level characteristics to workers of
multi-establishment firms, a probability model for employment location and
imputation was developed. \ The model explains establishment-of-employment
using two key characteristics available in the LEHD\ data: 1) distance
between place-of-work and place-of-residence and 2) the distribution of
employment across establishments of multi-establishment firms. \ The model
is estimated using data from Minnesota, where both the firm (SEIN) and
establishment identifiers appear on a worker's UI wage record. \ Then,
parameters from this estimation are used to multiply impute
establishment-of-employment for workers in the data from other states. \
Emerging from this process is an output file, called the Unit-to-Worker
(U2W) file, containing ten imputed establishments for each worker of a
multi-establishment firm. \ These implicates are then used in the downstream
processing of the QWI.

 The U2W\ process relies on information from each of the four
infrastructure files -- ECF, GAL, EHF, and ICF -- as well as the auxiliary
SPF file. \ Within the ECF, the universe of multi-establishment firms is
identified. \ For these firms, the ECF\ also provides establishment-level
employment, date-of-birth, and location (which is acquired from the GAL). \
The SPF contains information on predecessor relationships which may lead to
the revision of date-of-birth implied by the ECF. \ Finally, individual work
histories in the EHF in conjunction with place-of-residence information
stored in the ICF provide the necessary worker information needed to
estimate and apply the imputation model.

\subsection{A Probability Model for Employment Location}

\subsubsection{Definitions}

Let $i=1,...,I$ index workers, $j=1,...,J$ index firms (SEINs), and $%
t=1,...,T$ index time (quarters). \ \ Let $R_{jt}$ denote the number of
active establishments at firm $j$ in quarter $t,$ let $\mathfrak{R}%
=\max_{j,t}R_{jt},$ and $r=1,...,\mathfrak{R}$ index establishments. Note
the index $r$ is nested within $j.$ Let $N_{jrt}$ denote the quarter $t$
employment of establishment $r$ in firm $j.$ Finally, if worker $i$ was
employed at firm $j$ in $t,$ denote by $y_{ijt}$ the establishment at which
she was employed$.$

Let $\mathcal{J}_{t}$ denote the set of firms active in quarter $t,$ let $%
\mathcal{I}_{jt}$ denote the set of individuals employed at firm $j$ in
quarter $t,$ let $\mathcal{R}_{jt}$ denote the set of active ($N_{jrt}>0$)
establishments at firm $j$ in $t,$ and let $\mathcal{R}_{jt}^{i}\subset $ $%
\mathcal{R}_{jt}$ denote the set of active establishments that are feasible
for worker $i$. Feasibility is defined as follows. An establishment $r\in 
\mathcal{R}_{jt}^{i}$ if $N_{jrs}>0$ for every quarter $s$ that $i$ was
employed at $j.$

\subsubsection{The Probability Model}

Let $p_{ijrt}=\Pr \left( y_{ijt}=r\right)$. At the core of the model is the
probability statement:%
%
\begin{equation}
p_{ijrt}=\frac{e^{\alpha _{jrt}+x_{ijrt}^{\prime }\beta }}{\sum_{s\in 
\mathcal{R}_{jt}^{i}}e^{\alpha _{jst}+x_{ijst}^{\prime }\beta }}
\label{u2w:p}
\end{equation}%
%
where $\alpha _{jrt}$ is a establishment- and quarter-specific effect, $%
x_{ijrt}$ is a time-varying vector of characteristics of the worker and
establishment, and $\beta $ measures the effect of characteristics on the
probability of being employed at a particular establishment. In the current
implementation, $x_{ijrt}$ is a linear spline in the (great-circle) distance
between worker $i$'s residence and the physical location of establishment $%
r. $ The spline has knots at 25, 50, and 100 miles.

Using (\ref{u2w:p}), the following likelihood is defined%
\begin{equation}
p\left( y|\alpha ,\beta ,x\right) =\prod_{t=1}^{T}\prod_{j\in \mathcal{J}%
_{t}}\prod_{i\in \mathcal{I}_{jt}}\prod_{r\in \mathcal{R}_{jt}^{i}}\left(
p_{ijrt}\right) ^{d_{ijrt}}  \label{u2w:likelihood}
\end{equation}%
where 
\begin{equation}
d_{ijrt}=\left\{ 
\begin{array}{cc}
1 & \text{if }y_{ijt}=r \\ 
0 & \text{otherwise}%
\end{array}%
\right.  \label{u2w:d}
\end{equation}%
and where $y$ is the appropriately-dimensioned vector of the outcome
variables $y_{ijt},$ $\alpha $ is the appropriately-dimensioned vector of
the $\alpha _{jrt},$ and $x$ is the appropriately-dimensioned matrix of
characteristics $x_{ijrt}$. \ For $\alpha _{jrt}$, a hierarchical Bayesian
model based on employment counts $N_{jrt}$ is specified.

The object of interest is the joint posterior distribution of $\alpha $ and $%
\beta .$ A uniform prior on $\beta ,$ $p\left( \beta \right) \varpropto 1$
is assumed. The characterization of $p\left( \alpha ,\beta |x,y,N\right) $
is based on the factorization 
\begin{eqnarray}
p\left( \alpha ,\beta |x,y,N\right)  &=&p\left( \alpha |N\right) p\left(
\beta |\alpha ,x,y\right)   \nonumber \\
&\varpropto &p\left( \alpha |N\right) p\left( \beta \right) p\left( y|\alpha
,\beta ,x\right)   \nonumber \\
&\varpropto &p\left( \alpha |N\right) p\left( y|\alpha ,\beta ,x\right) .
\label{u2w:p(a,b|x,y,N)}
\end{eqnarray}%
%
%\marginpar{{\footnotesize Is the reference here needed? Refers to
%    $p(a|N)$}}
%
Thus the joint posterior (\ref{u2w:p(a,b|x,y,N)}) is
completely characterized by the posterior of $\alpha $ and the likelihood of 
$y$ in (\ref{u2w:likelihood}). Note (\ref{u2w:likelihood}) and (\ref%
{u2w:p(a,b|x,y,N)}) assume that the employment counts $N$ affect employment
location $y$ only through the parameters $\alpha .$

\subsubsection{Estimation}

%\marginpar{{\footnotesize Refernce to (a-def) not resolved}} 
The joint
posterior $p\left( \alpha ,\beta |x,y,N\right) $ is approximated at the
posterior mode. In particular, we estimate the posterior mode of $p\left(
\beta |\alpha ,x,y\right) $ evaluated at the posterior mode of $\alpha $.
From these we compute the posterior modal values of the $\alpha _{jrt}$,
then, maximize the log posterior density 
%\marginpar{{\footnotesize Reference to (log-posterior) not resolved}} 
\begin{equation}
\log p\left( \beta |\alpha ,x,y\right) \varpropto \sum_{t=1}^{T}\sum_{j\in 
\mathcal{J}_{t}}\sum_{i\in \mathcal{I}_{jt}}\sum_{r\in \mathcal{R}%
_{jt}^{i}}d_{ijrt}\left( \alpha _{jrt}+x_{ijrt}^{\prime }\beta -\log \left(
\sum_{s\in \mathcal{R}_{jt}^{i}}e^{\alpha _{jst}+x_{ijst}^{\prime }\beta
}\right) \right)   \label{u2w:log-posterior}
\end{equation}%
which is evaluated at the posterior modal values of the $\alpha _{jrt},$
using a modified Newton-Raphson method. The mode-finding exercise is based
on the gradient and Hessian of (\ref{u2w:log-posterior}). \ In practice, (%
\ref{u2w:log-posterior}) is estimated for three firm employment size
classes: 1-100 employees, 101-500 employees, and greater than 500 employees,
using data for Minnesota.

\subsection{Imputing Place of Work}

After estimating the probability model using Minnesota data, the estimated
parameters are applied in the imputation process for other states. \ A brief
outline of the imputation method, as it relates to the probability model
previously discussed, is provided in this section. \ Emphasis is placed on
not only the imputation process itself, but also the preparation of input
data.

\subsubsection{Sketch of Imputation Method}

Ignoring temporal considerations, 10 implicates are generated as follows.
First, using the mean and variance of $\beta $ estimated from the Minnesota
data, we take 10 draws of $\beta $ from the normal approximation (at the
mode) to $p\left( \beta |\alpha ,x,y\right)$. Next, using QCEW employment
counts for the establishments, we compute 10 values of $\alpha _{jt}$ based
on the hierarchical model for these parameters. \ Note these are draws from
the exact posterior distribution of the $\alpha _{jrt}$. The drawn values of 
$\alpha $ and $\beta $ are used to draw 10 imputed values of place of work
from the normal approximation to the posterior predictive distribution%
\begin{equation}
p\left( \tilde{y}|x,y\right) =\int \int p\left( \tilde{y}|\alpha ,\beta
,x,y\right) p\left( \alpha |N\right) p\left( \beta |\alpha ,x,y\right)
d\alpha d\beta .  \label{u2w:post-pred}
\end{equation}

\subsubsection{Implementation}

\paragraph{Establishment Data}

 Using state-level \ micro-data, the set of firms (SEINs) that ever
operate more that one establishment in a given quarter are identified; these
SEINs represent the set of ever-multi-establishment firms\ defined above as
the set $\mathcal{J}_{t}$. \ For each of these firms, its
establishment-level records are identified. \ For each establishment,
latitude and longitude coordinates, which emerge from GAL\ processing,
parent firm (SEIN) employment, and QCEW first month employment\footnote{%
In rare instances where no QCEW employment is available, an alternative
employment measure based on UI\ wage record counts may be used.} for the
entire history of the establishment are retained. Those establishments with
positive first-month employment in a given quarter characterize $\mathcal{R}%
_{jt}$, the set of all active establishments. \ An establishment
date-of-birth is identified and, in most cases, is the first quarter in the
QCEW time series in which the establishment has positive first-month
employment. \ For some firms, predecessor relationships are identified in
the SPF; in those instances, the establishment date-of-birth is adjusted to
coincided with that of the predecessor's.

\paragraph{Worker Data}

 The EHF provides the earnings histories for employees of the
ever-multi-establishment firms. \ For each in-scope job (a worker-firm
pair), one observation is generated for the \textit{end} of each job spell,
where a job spell is defined as a continuum of quarters of positive earnings
for worker at a particular firm during which there are no more than 3
consecutive periods of non-positive earnings\footnote{%
A new hire is defined in the QWI as a worker who acceeds to a firm in the
current period but was not employed by the same firm in any of the 4
previous periods. \ A new job spell is created if, for example, a worker
leaves a firm for 4 or more quarters and is subsequently re-employed by the
same firm.}. \ The start-date of the job history is\ identified as the first
quarter of positive earnings; the end-date is the last date of positive
earnings\footnote{%
By definition, an end-date for a job spell is not assigned in cases where a
quarter of positive earnings at a firm is succeeded by fewer than 4 quarters
of non-employment and subsequent re-employment by the same firm.}. \ These
job spells characterize the set $\mathcal{I}_{jt}$

\paragraph{Candidates}

Once the universe of establishments and workers is identified, data are
combined and a priori restrictions and feasibility assumptions are imposed.
\ For each quarter of the date series, the history of every job spell that 
\textit{ends in that quarter} is compared to the history of \textit{every}
active (in terms of QCEW first month employment) establishment of the
employing firm (SEIN). \ The start date of the job spell is compared to the
birth date of each establishment. \ Establishments that were born after the
start of a job spell are immediately discarded from the set of candidate
establishments. \ The remaining establishments constitute the set $\mathcal{R%
}_{jt}^{i}\subset $ $\mathcal{R}_{jt}$ for a job spell (worker) at a given
firm\footnote{%
The sample of UI wage and QCEW data chosen for processing of the QWI is such
that the start and end dates are the same. \ Birth and death dates of
establishments are, more precisely, the dates associated with the beginning
and ending of employment activity observed in the data. \ The same is true
for the dates assigned to the job spells.}. \ \ 

Given the structure of the pairing of job spells with candidate
establishments, it is clear that within job spell changes of establishment
are ruled-out. \ An establishment is imputed once for each job spell%
\footnote{%
More specifically, an establishment is imputed to a job spell only once
within each implicate.}, thereby creating no false labor market transitions.

\paragraph{Imputation and Output Data}

 Once the input data are organized, a set of 10 imputed
establishment identifiers are generated for each job spell ending in every
quarter for which both QCEW and UI wage records exist. \ For each quarter,
implicate, and size class, $s=1,2,3$, the parameters on the linear spline in
distance between place-of-work and place-of-residence $\hat{\beta}^{s}$ are
sampled from the normal approximation of the posterior predictive
distribution of $\beta ^{s}$ conditional\ on Minnesota ($MN$) 
\begin{equation}
p(\beta ^{s}|\alpha _{MN},x_{MN},y_{MN})  \label{u2w:sample beta}
\end{equation}%
\ The draws from this distribution vary across implicates, but not across
time, firms, and individuals.
%
Next, for each firm $j$ at time $t$, a set of $\hat{\alpha}_{jrt}$ are drawn
from%
\begin{equation}
p\left( \alpha _{ST}|N_{ST}\right)  \label{u2w:sample alpha}
\end{equation}%
which are based on the QCEW first-month employment totals ($N_{jrt}$) for
all candidate establishments $r_{jt}\subset $ $\mathcal{R}_{jt}$at firm $j$
within the state ($ST$) being processed. \ The initial draws of $\hat{\alpha}%
_{jrt}$ from this distribution vary across time and firms but not across job
spells. \ \ 
%
Combining (\ref{u2w:sample beta}) and (\ref{u2w:sample alpha}) yields

\begin{eqnarray}
&&p\left( \alpha _{ST}|N_{ST}\right) p(\beta ^{s}|\alpha _{MN},x_{MN},y_{MN})
\label{u2w:post mode ST} \\
&\approx &p\left( \alpha _{ST}|N_{ST}\right) p(\beta ^{s}|\alpha
_{ST},x_{ST},y_{ST})  \nonumber \\
&=&p\left( \alpha _{ST},\beta _{ST}|x_{ST},y_{ST},N_{ST}\right),  \nonumber
\end{eqnarray}%
an approximation of the joint posterior distribution of $\alpha $ and $\beta
^{s}$ (\ref{u2w:p(a,b|x,y,N)}) conditional on data from the state being
processed.

The draws $\hat{\beta}^{s}$ and $\hat{\alpha}_{jrt}$ in conjunction with the
establishment, firm, and job spell data are used to construct the $p_{ijrt}$
in (\ref{u2w:p}) for all candidate establishments $r\in \mathcal{R}_{jt}^{i}$%
. \ For each job spell and candidate establishment combination, the $\hat{%
\beta}^{s}$ are applied to the calculated distance between
place-of-residence (of the worker holding the job spell) and the location of
the establishment, where the choice of $\hat{\beta}^{s}$depends on the size
class of the establishment's parent firm. \ For each combination an $\hat{%
\alpha}_{jrt}$ is drawn which is based primarily on the size (in terms of
employment) of the establishment relative to other active establishments at
the parent firm. \ In conjunction, these determine the conditional
probability $p_{ijrt} $ of a candidate establishment's assignment to a given
job spell. \ Finally, from this distribution of probabilities is drawn an
establishment of employment.

 Emerging from the imputation process is a data file containing a
set of 10 imputed establishment identifiers for each job spell. \ In a
minority of cases, the model fails to impute an establishment to a job
spell. \ This is often due to unanticipated idiosyncrasies in the underlying
administrative data. \ Furthermore, across states, the proportion of these
failures relative to successful imputation is well under 0.5\%. \ For these
job spells, a dummy establishment identifier is assigned and in downstream
processing, the employment-weighted modal firm-level characteristics are
used.



%%% Local Variables: 
%%% mode: latex
%%% TeX-master: "u2w_master"
%%% TeX-master: "qwi-overview"
%%% End: 

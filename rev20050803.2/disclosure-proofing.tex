%TCIDATA{Version=5.00.0.2570}
%TCIDATA{LaTeXparent=0,0,sw-edit.tex}
                      
%TCIDATA{ChildDefaults=chapter:1,page:1}


% -*- latex -*- 
%
% Time-stamp: <05/04/06 08:29:41 vilhuber> 
%              Automatically adjusted if using Xemacs
%              Please adjust manually if using other editors
%
% disclosure_proofing.tex
% Responsible: Simon
% Part of QWI_methods.tex

%\section{Disclosure Proofing the QWI}
%\label{sec:confidentiality}
\label{cha:disclosure_proofing}

\mindex{Disclosure proofing}
Disclosure proofing is the set of methods used by statistical agencies to
protect the confidentiality of the identity of and information about the
individuals and businesses that form the underlying data in the system. In
the QWI system, disclosure proofing is required to protect the information
about individuals and businesses that contribute to the UI\index{UI} wage
records, the ES-202 quarterly reports, and the Census Bureau demographic
data that have been integrated with these sources. There are two layers of %
\index{confidentiality protection} and disclosure proofing in the QWI
system.

The first layer occurs when workplace-level estimates are aggregated to
higher levels. At this stage, the QWI
system infuses specially constructed noise into the estimates of all of the
workplace-level measures. This noise\index{noise} is designed to have two very
important properties. First, for a given workplace, the data are always
distorted in the same direction (increased or decreased) by the same
percentage amount in every period. Second, the statistical properties%
\index{distortion!statistical properties} of this distortion are such that
when the estimates are aggregated, the effects of the distortion cancel out for the vast
majority of the estimates.

The second layer of confidentiality protection occurs after the
workplace-level measures are aggregated to the higher levels. The data from many individuals and businesses are combined
into a (relatively) few estimates. This aggregation helps to conceal the
exact information about any of the individuals or businesses that underlie
the estimate. At this level of confidentiality protection, some of the
estimates turn out to be based on fewer than three persons or firms. These estimates
are suppressed%
\index{suppression}. In addition, some of the estimates are based on data
that are still substantially influenced by the noise that was infused in the
first layer. These estimates are flagged as \Mindex{substantially
  distorted}. Table~\ref{tab:disclosure_flags} lists the possible flags for
these and other cases. Each observation on any one of the published QWI tables has an
associated flag. 

%TCIDATA{LaTeXparent=0,0,sw-edit.tex}
                      
%TCIDATA{ChildDefaults=chapter:1,page:1}


% -*- latex -*- 
%
% Time-stamp: <05/08/03 13:36:30 vilhuber> 

\begin{table}[htbp]
  \centering
  \caption{Disclosure flags in the QWI}
  \label{tab:disclosure_flags}
  \begin{tabular}{lp{4in}}
Flag & Explanation \\
\hline
 -2 &no data available in this category for this quarter\\
 -1 &data not available to compute this estimate\\
 0  &    no employment in this cell, or no positive denominator
         (OK to disclose a 0 for sum or count, missing for ratio)\\
 1  &    OK, distorted value released\\
% 2  &    less than 3 employees (value suppressed in publications)\\
% 3  &    less than 3 employers (value suppressed in publications)\\
% 4  &    for ratio and change variables, the value could not be computed
%        because a denominator rounds to zero. \\ %This is not a legal
%        code for sums and counts unless they are change variables.
5   &    Value suppressed because it does not meet US Census Bureau publication
         standards.\\
 9  &    data significantly distorted, distorted value released\\
\hline
  \end{tabular}
\end{table}

%%% Local Variables: 
%%% mode: latex
%%% TeX-master: "qwi-overview"
%%% End: 




\section{Multiplicative noise model\label{sec:disclosure:fuzz}}


\index{multiplicative noise model}
To implement  the multiplicative noise model, a
random fuzz factor%
\index{fuzz factor} $\delta _j$ is drawn for each employer $j$ according to
the following process:

\begin{equation*}
p\left( {\delta _j } \right) = \left\{ {{%
\begin{array}{*{20}c} {\mbox{ }{\left( {b - \delta } \right)} \mathord{\left/ {\vphantom {{\left( {b - \delta } \right)} {\left( {b - a} \right)^2}}} \right. \kern-\nulldelimiterspace} {\left( {b - a} \right)^2},\;\delta \in \mbox{ 
}\left[ {a,b} \right]} \\ {{\left( {b + \delta - 2} \right)} \mathord{\left/ {\vphantom {{\left( {b + \delta - 2} \right)} {\left( {b - a} \right)^2}}} \right. \kern-\nulldelimiterspace} {\left( {b - a} \right)^2},\;\delta \in \left[ {2 - b,2 - a} \right]\mbox{ }} \\ \end{array} 
}} \right.
\end{equation*}

\begin{equation*}
F\left( {\delta _j } \right) = \left\{ {{%
\begin{array}{*{20}c} {\mbox{0.5} + {\left[ {\left( {b - a} \right)^2 - \left( {b - \delta } \right)^2} \right]} \mathord{\left/ {\vphantom {{\left[ {\left( {b - a} \right)^2 - \left( {b - \delta } \right)^2} \right]} {\left[ {2\left( {b - a} \right)^2} \right]}}} \right. \kern-\nulldelimiterspace} {\left[ {2\left( {b - a} \right)^2} \right]},\;\delta \in \mbox{ }\left[ {a,b} \right]\mbox{ 
}} \\ {{\left[ {\left( {\delta + b - 2} \right)^2} \right]} \mathord{\left/ {\vphantom {{\left[ {\left( {\delta + b - 2} \right)^2} \right]} {\left[ {2\left( {b - a} \right)^2} \right]}}} \right. \kern-\nulldelimiterspace} {\left[ {2\left( {b - a} \right)^2} \right]},\;\delta \in \left[ {2 - b,2 - a} \right]\mbox{ }} \\ \end{array} 
}} \right.
\end{equation*}

\noindent where $a$ and $b$ are constants chosen such that $1 < a < b < 2$.%
%
\footnote{%
The exact numbers are confidential.}
%
This produces a random noise factor
centered around 1 with distortion of at least $a-1$ and at most $b-1$.



\paragraph{Distorting totals}

\index{fuzzing!totals}

The fuzz factor $\delta _{j}$ is used to distort all establishment
 totals by scaling of the true establishment level statistic

\[
\label{eq:fuzz_totals}X_{jt}^{\ast }=\delta _{j}X_{jt},
\]%
where $X_{jt}$ is an establishment level statistic among 
%. Statistics distorted by this method are 
beginning-of-quarter ($B$), end-of-quarter ($E$) employment, flow employment
($M$), full-quarter employment ($F$), accessions ($A$), separations ($S$),
new hires ($H$), recalls ($R$), flows into full-quarter status ($FA$), flows
out of full-quarter status ($FS$), new hires into full-quarter status ($FH$%
), total payroll ($W_{1}$), payroll associated with $E$ ($W_{2}$), with $B$ (%
$W_{3}$), with new hires ($WFH$), periods of non-employment for accessions ($%
NA$), for new hires ($NH$), for recalls ($NR$), and for separations ($NS$).

\paragraph{Distorting  averages of magnitude variables}

\index{fuzzing!averages}

Averages are constructed from distorted numerators (totals) with undistorted
denominators according to 
\[
ZY_{jt}^{\ast }=%
\frac{Y_{jt}^{\ast }}{B(Y)_{jt}}=\delta _{j}\frac{Y_{jt}}{B(Y)_{jt}},
\]%
where $ZY_{jt}$ is a statistic related to a total $Y_{jt}$, and $B(Y)$ is
the appropriate denominator for the calculation of the average. Statistics
distorted by this method are average earnings for various groups ($%
ZW_{2},ZW_{3},ZWFH,ZWA,ZWS$), and average periods of non-employment for
several groups ($ZNA,ZNH,ZNR$, and $ZNS$).

\paragraph{Distorting  differences of counts and magnitudes}

\index{fuzzing!differences}

Distorted net job flow%
\index{job flow} ($JF$) is computed at the aggregate ($k$ = geography,
industry, or combination of the two for the appropriate age and sex
categories) level as the product of the aggregated, undistorted rate of growth
and the aggregated distorted employment:%
\[
JF_{kt}^{\ast }=G_{kt}\times 
\bar{E}_{kt}^{\ast }=JF_{kt}\times \frac{\bar{E}_{kt}^{\ast }}{\bar{E}_{kt}}.
\]%
This method of distorting net job flow will consistently estimate net job flow
because it takes the product of two consistent estimators. The formulas for
distorting gross job creation ($JC$) and job destruction ($JD$) are similar: 
\index{job creation} 
\index{job destruction}%
\[
JC_{kt}^{\ast }=JCR_{kt}\times 
\bar{E}_{kt}^{\ast }=JC_{kt}\times \frac{\bar{E}_{kt}^{\ast }}{\bar{E}_{kt}}
\]%
and%
\[
JD_{kt}^{\ast }=JDR_{kt}\times \bar{E}_{kt}^{\ast }=JD_{kt}\times \frac{\bar{%
E}_{kt}^{\ast }}{\bar{E}_{kt}}.
\]%
where $JCR_{kt}$ and $JDR_{kt}$ are the aggregated growth rates for job
creations and destructions, respectively. Exactly analogous expressions
apply to full-quarter net job flows $\left( FJF\right) $, full-quarter job
creations $\left( FJC\right) ,$ and full-quarter job destructions $\left(
FJD\right) .$

The same logic was used to distort wage changes for subgroups (accessions,
separations, full-quarter accessions and separations). 
%: total change in earnings for
%accessions (all jobs)%
%\index{earnings!accessions}, total change in earnings for full-quarter
%accessions%
%\index{earnings!full-quarter accessions} (all jobs), total change in
%earnings for separations%
%\index{earnings!separations} (all jobs), and total change in earnings for
%full-quarter separations%
%\index{earnings!full-quarter separations} (all jobs). (Symbols used below: $%
%\Delta WA, \Delta WS$.) 
The undistorted total changes were divided by the undistorted denominators then
multiplied by the ratio of the distorted denominator to the undistorted
denominator for the computation of average change in earnings. 
% for accessions (all jobs), average change in earnings for
%full-quarter accessions (all jobs), average change in earnings for
%separations (all jobs), and average change in earnings for full-quarter
%separations (all jobs). (Symbols used below: $Z\Delta WA, Z \Delta WS.$)
Averages are distorted by multiplying by the ratio of the distorted denominator to
the true denominator. For example:%
\[
Z\Delta WY_{kt}^{\ast }=\frac{\Delta WY_{kt}}{Y_{kt}}\times \frac{%
Y_{kt}^{\ast }}{Y_{kt}}.
\]%
where, again, $Y$ denotes a particular count, and $Z\Delta WY$ the average
change in total earnings associated with that particular count.

\section{Item suppression}

\label{sec:disclosure:cell_suppression} 
\index{cell suppression}

Despite the noise infusion described in the previous sections, some
disclosure risk remains for counts based on very few entities in a cell. For
counts based on data from fewer than three individuals or employers, the
fuzz factors may not provide sufficient protection. This condition applies
to the variables $B$, $E$, $M$, $F$, $A$, $S$, $H$, $R$, $FA$, $FH$, $FS$, $%
JC$, $JD$, $JF$, $FJC$, $FJD$, $FJF$. The QWIs therefore also implement item
suppression based on the number of either workers or the number of employers
that contribute data for that item in a cell $k$ in time period $t$, where a
cell represents a particular combination of geography $\times $ industry $%
\times $ age $\times $ sex. Because of the noise infusion used previously,
however, no complementary suppressions are needed since all of the values
based on three or more individuals or employers are adequately protected.
Any estimate of the suppressed item computed by subtraction is also
protected.

The algorithm for item suppression for these variables 
%for the variables $B$, $E$, $M$, $F$, $A$,
%$S$, $H$, $R$, $FA$, $FH$, $FS$, $JC$, $JD$, and $JF$ 
is as follows:

%
% As of May 4. 2005, JohnAbowd
% 
% QWIPU: Replace codes disclosure status codes 2, 3, and 4 with code 5 for
% all instances for all variables. Do not change codes -2, -1, 0, and 9. 
%

\begin{itemize}
\item Check the conditions leading to a disclosure flag of -2 or -1 (data
availability). If met, set the item to missing in the release file.

\item Determine whether the value can be computed according to Census
standards:

\begin{itemize}
\item For the variables $JC$, $JD$, and $JF$, $\left( 
\text{respectively, }FJC,FJD,\text{ and }FJF\right) $ check whether the
denominator average employment ($\bar{E}_{kt}$; respectively, $\bar{F}_{kt}$%
) in the relevant cell $kt$ rounds to zero. %If so, set the
%disclosure status to $4$ and set the item to missing in the release file. 

\item Check whether the item in cell $kt$ rounds to zero. %If so, set the
%disclosure status to $0$ and set the item to $0$ in the release file.

\item Check whether the data used to construct the cell $kt$ value were
based on $1$ or $2$ individuals. 
%If so, set the disclosure status to $2$ and set
%the item to missing in the release file.

\item Check whether the data used to construct the cell $kt$ value were
based on $1$ or $2$ employers. 
%If so, set the disclosure status to $3$ and set the
%item to missing in the release file.
\end{itemize}

If any of these conditions are met, set the disclosure status to $5$ and set
the item to missing in the release file.

\item Check whether the distortion of cell $kt$ value exceeds the limit set
by the Census Disclosure Review Board\footnote{%
The precise value is confidential.}. If so, set the disclosure status to $9$
and copy the distorted value to the release file.

\item Otherwise, set the disclosure status to $1$ and copy the distorted value
to the release file.
\end{itemize}

%
%
%

\section{Analysis of the distortion due to the use of noise in the
disclosure proofing process}

Table~\Vref{tab:error_distribution} shows the distribution of the error in
the first order serial correlation%
\index{first order serial correlation} coefficient based on estimating an
AR(1) using the multiplicatively distorted data ($r^*$) and using the
undistorted data ($r$) for all counties in Illinois. The table shows that
none of our variables is seriously affected by the distortion. In
particular, the semi-interquartile range of the distortion is less than the
precision with which estimated serial correlation coefficients are normally
displayed---generally less than 2{\%}, which means that distortion is
economically meaningless.

%TCIDATA{LaTeXparent=0,0,QWI_methods.tex}
% -*- latex -*- 
%
% Time-stamp: <02/01/23 10:21:31 vilhu001> 
\begin{table}[htbp]
\tablefontsize
%\setlength{\tabcolsep}{1pt}
%\setlength{\extrarowheight}{1pt}
  \begin{center}
    \caption{Distribution of the Error in the First Order Serial Correlation}
    \label{tab:error_distribution}
    \begin{tabular}{rrrrrr}

\multicolumn{6}{c}{Coefficient Due to Multiplicative Noise Distortion ($r^* - r$)}\\

&Beginning &&&Full\\
& of Quarter&&& Quarter &Net Job\\
Quantile& Employment&Accessions&Separations& Employment& Flows\\
\hline
99\%&0.07894&0.07153&0.06711&0.06644&0.01104       \\  
95\%&0.04338&0.04253&0.04070&0.03465&0.00503       \\  
90\%&0.02610&0.03043&0.02826&0.01972&0.00314       \\  
75\%&0.00946&0.01387&0.01326&0.00718&0.00124       \\  
50\%&-0.00043&0.00103&0.00004&-0.00003&0.00000     \\  
25\%&-0.01026&-0.01271&-0.01179&-0.00641&-0.00096  \\  
10\%&-0.02520&-0.03012&-0.02592&-0.01720&-0.00281  \\  
5\%&-0.03695&-0.04100&-0.03569&-0.02806&-0.00471   \\  
1\%&-0.06984&-0.06863&-0.06645&-0.06185&-0.01038   \\  
\hline

    \end{tabular}
  \end{center}
\end{table}

%\begin{figure}[htbp]
%\centerline{\includegraphics[width=5.91in,height=2.81in]{%
%\mypath/GraphTable/QWI_methods_V2001011720}} \label{fig20}
%\end{figure}
%
%%% Local Variables: 
%%% mode: latex
%%% TeX-master: "QWI_methods"
%%% End: 


% 
%%% Local Variables: 
%%% mode: latex
%%% TeX-master: "QWI_methods"
%%% End: 

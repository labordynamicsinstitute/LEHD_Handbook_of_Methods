%TCIDATA{LaTeXparent=0,0,sw-edit.tex}

% -*- latex -*- 
%
% Time-stamp: <02/05/23 14:12:13 vilhu001> 
%              Automatically adjusted if using Xemacs
%              Please adjust manually if using other editors
%
% input_sources.tex
% Responsible: Lars/John
% Part of QWI_methods.tex

%\section{Description of initial data processing}

This section describes the data processing steps. Figure~\Vref{fig1} gives a
generic overview 
of that process. A more detailed flowchart is provided in
Figure~\Vref{fig:flowchart} in Appendix~\ref{app:flowchart}.

\begin{figure}[htbp]
\begin{center}
\centerline{\includegraphics[width=4.83in,height=3.84in]{\mypath/GraphTable/QWI_methods_V200101171}}
\caption{Overview of data processing\label{fig1}}
\end{center}
\end{figure}


\section{Receipt of data}

The data acquisition process starts with receipt of the data carrier (tape, 
CDROM) by the U.S. Census Bureau. If the data is on CDROM, readin is done
at LEHD, otherwise this task is performed by the 
Administrative Records Research Staff (ARRS\index{ARRS}).  Data entry, 
including method and date of receipt, and number of records, are recorded
both at ARRS\index{ARRS} and at LEHD in either case.

\section{Standardization}

The goal of much of LEHD processing is to create a homogeneous analytical 
product. Thus all data processing at LEHD standardizes variable concepts, 
names, and formats. Harmonization is necessary because different states have 
different ways of recording wages, \aindex{UI} account numbers, and other data items. 
In addition, while these identifiers may be unique within a state, they may 
not be so across different states. After harmonization, LEHD files contain a 
unique state firm identifier (\aindex{SEIN}) used in subsequent processing. Now that 
\aindex{BLS} reporting unit information is available on certain employer-level files, 
the reporting unit (\aindex{RUN}) will be incorporated into the business identifier 
permitting analyses at the establishment level. (LEHD program \aindex{QWI} project 
version 3 will incorporate the reporting unit.) Similar standardization 
treatment is given to other variables. 

\section{SSN editing}

SSN\index{SSN} can and do have coding errors. Since the QWI requires
a consistent longitudinal identifier for each individual, such coding
errors introduce bias into any of the measures computed. LEHD has
developed a process by which possibly miscoded records are matched back to
an otherwise consistent time-series for a given SSN. This processing is
done before anonymization, because it requires the original SSN. Not all
state data are processed this way. 

\section{Anonymization}

The first processing that the data receive, if applicable, is anonymization. 
The administrative data received from the states contain individual and firm 
identifiers. Identifiers include, but are not limited to, first and last 
name, Social Security Number (SSN\index{SSN}), state unemployment insurance account 
number \aindex{SEIN}, and federal EIN\index{EIN}. As per the current Memoranda of Understanding (\aindex{MOU}, 
also called Data Use Agreements%
\index{Data Use Agreements|see{MOU}} by some states), firm identifiers are 
carried along unchanged throughout LEHD processing. Personal identifiers, on 
the other hand, are either deleted or modified in such a manner as to mask 
the original identifier. Thus, the SSN\index{SSN} is replaced by a Census internal 
identifier (Protected Identity Key, PIK\index{PIK}). The original SSN\index{SSN} can not be 
re-inferred, since the algorithm used to associate PIK\index{PIK}s to SSN\index{SSN}s is not 
accessible to LEHD personnel. As an additional precaution, individual names 
are deleted from all files containing them. 

After having passed quality control within ARRS\index{ARRS}, the data are transferred to 
the LEHD computers. All processing from here on is performed exclusively 
within the secure computing environment at LEHD by LEHD
personnel, using PIKs as identifiers. This process is sometimes referred to
as ``PIKizing.\mindex{PIKizing}''


\section{Creation of state-specific characteristics files}

The data are next prepared for the process of extracting information on
jobs, firms, and individuals. It is at this stage that information
available within Census on both the individual and the employer is added to
the data.  Three related files are created for every state. The first of
these is the Individual Characteristics File (\aindex{ICF}), which contains
information on the individual, including demographic information added from
the Census Numident\index{Numident|see{PCF}}/\aindex{PCF} file and links to
any Census survey in which that individual may have participated.  The
actual survey respondent data are linked on a case-by-case basis. The
second file is the Employment History File (\aindex{EHF}), which contains a
detailed quarter-by-quarter time series of an individual's working activity
within the state. The third file is the Employer Characteristics File
(\aindex{ECF}), which contains both information on employers active within
the state as provided in the employer-based data received from the states
and indicators for the presence of that employer in Census data products
(business surveys, business censuses, etc.).

\subsection{Demographic products}

%Across all states presently on version 2, nearly xxxxxx%
%\marginpar{\tiny Has to be corrected for all states!}
Many
individuals have appeared in at least one of the eligible Census demographic 
products, and their detailed demographic information from those surveys can 
be linked to the extensive longitudinal data gleaned from the state records.


\subsection{Census PCF}
\index{SSA}
\index{Numident}
\index{PCF}

These data, which contain information on date of birth, place of birth,
race and sex, are maintained by ARRS\index{ARRS} under a Memorandum of
Understanding with the Social Security Administration\index{SSA}, being
based on the SSA Numident. The LEHD Program matches date of birth, sex,
race, and place of birth using the PIK\index{PIK}. This processing is done
on the ARRS\index{ARRS} system to protect the confidentiality of the
PIK\index{PIK}-SSN\index{SSN} cross walk.

\subsection{Economic censuses and annual surveys}
\index{Annual surveys!manufacturing}
\index{Annual surveys!trade}
\index{Annual surveys!service}
\index{Annual surveys!transportation}
\index{Annual surveys!communication}
\index{Economic census}

These data include the complete 1987, 1992 and 1997 economic censuses, all 
annual surveys of manufacturing, service, trade, transportation and 
communication industries and selected, approved fields from the Census 
Bureau's establishment master file.%
\index{Establishment master file}
Linkage to these data is based upon 
exact EIN\index{EIN} matches, supplemented with statistical matching to recover 
establishments\index{establishment}.

% \section{National master files}
% 
% All three types of master files are then aggregated to the national level. 
% For instance, an individual may have worked a number of years in state A, 
% but disappear periodically from that state's UI records. However, he or she 
% may actually have moved to a different state B and have continued working. 
% The national employment history file will contain the complete time series 
% of working information from all states in the LEHD data base. The same 
% applies for firms. The national employer master file will allow the activity 
% of a firm with a presence in several states to be used in a coherent 
% fashion. The process of producing the first national master files will begin 
% once certain critical statistical research has been completed on a few large 
% states.
% 
% As the number of states participating in the \aindex{QWI} project grows, the quality 
% of these linkages will improve. At the present, four states have already 
% contributed data to the LEHD Program. The number of potential linkages is 
% small. Nevertheless, over 500,000 individuals have records in at least two 
% of those states, a number much larger than all but the most comprehensive 
% surveys. 
%
%\section{Merge with other Census Bureau products}
%
%The structure of the files themselves will be virtually identical to the 
%state level files, so all the linkages to Census products available at the 
%state level carry over to the national files. Special projects that require 
%less numerous, but more detailed information on individuals and firms can be 
%carried out both at the national and the state level. 



%%% Local Variables: 
%%% mode: latex
%%% TeX-master: "QWI_methods"
%%% End: 

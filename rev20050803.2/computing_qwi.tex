%TCIDATA{Version=4.00.0.2321}
%TCIDATA{LaTeXparent=0,0,sw-edit.tex}
                      
%TCIDATA{ChildDefaults=chapter:1,page:1}
%
% Time-stamp: <05/04/06 08:07:32 vilhuber>
% Contains John Abowd mods 2005-04-05
%
%\section{Forming Aggregated Estimates}
%\label{sec:aggregate}


\section{What are the QWI statistics?}
\label{sec:qwi:what}

The Quarterly Workforce Indicators (QWI)%
\index{QWI} provide detailed local statistics for a variety of indicators.
Employment, earnings, gross job creation and destruction, and worker
turnover are available at different levels of geography, typically down to
the county or metro area. At each level of geography, they are available by
detailed industry (SIC and NAICS), sex, and age of workers. At the time of
writing of this article, QWI statistics for 31 states had been tabulated,
and the program is still expanding with the goal of national coverage.


\section{Computing the statistics}
\label{sec:qwi:compute}

The establishment of the LEHD Infrastructure Files was driven in large part,
although not exclusively, by the needs of the QWI statistics. Completed and
representative data were and are the primary concern for the QWI. The ICF
(Section~\ref{cha:icf}) and the ECF (Section~\ref{cha:ecf}) draw on a large
number of data sources, and use a set of imputation procedures, to provide a
complete and detailed picture of each economic actor. The ECF also provides
the input data for the weighting, which is explained in more detail in
Section~\ref{sec:weights}. The Wage Record Edit (Section~\ref{sec:ssnedit})
and the SPF (Section~\ref{sec:spf}) apply algorithmic and statistical
matching rules to the proper longitudinal linking of entities. The U2W
(Section~\ref{sec:u2w}) completes the picture, by attributing an employing
establishment to each individual employed at some point during the time
period covered by the multi-unit UI account under which the data were
reported.

These data are then combined and aggregated to compute the QWI statistics.
The aggregation is a four step process:

\begin{enumerate}
\item A ``job'' -- a unique PIK%
\index{PIK}-SEIN%
\index{SEIN}-SEINUNIT%
\index{SEINUNIT} combination -- is identified, and the job's complete
activity history -- when the worker worked, and when not -- recorded. Note
that each job history stems from an implicate of the U2W, and is weighted
accordingly.

\item Job-level variables are computed as a set of indicators. The
computation of each of these variables is described in detail in Section~\ref%
{sec:technical:individual}.

  
\item Job-level variables are aggregated to the establishment level
(SEINUNIT), using appropriate implicate weights. The aggregation is done
using formulae described in Section~\ref{sec:technical:employer}. For many
variables, aggregation to the establishment-level is achieved by summing the
job-level variables ( beginning-of-period employment, end-of-period
employment, accessions, new hires, recalls, separations, full-quarter
employment, full-quarter accessions, full-quarter new hires, total earnings
of full-quarter employees, total earnings of full-quarter accessions, and
total earnings of full-quarter new hires). Some aggregate flow variables are
computed using the beginning- and end-of-quarter employment estimates for
that workplace. Examples are net job flows (see Equation (\ref{eq:JFjt}) in
Appendix~\ref{cha:definitions}), average employment (\ref{eq:ebarjt}), job
creations (\ref{eq:JCjt}) and job destructions (\ref{eq:JDjt}).

The file created in this step, internally known as the Unit Flow File
(UFF\_B), is also available in the RDC system, see Section~\ref{sec:LEHD-QWI}
for details.

\item The variables necessary for disclosure-proofing -- SEINUNIT-specific
noise infusion called ``fuzz factors'' --
are attached, and the establishment-level file is summed to the desired
level of geographic and demographic detail, using the noise-infused values.
Some flow variables are computed directly from other aggregated variables
(see Section~\ref{sec:technical:aggregate_flows}). An undistorted version of
all aggregates is also created. All aggregations use weights (see Section~%
\ref{sec:weights}).

\item The tables created in the previous step are disclosure-proofed (see
Section~\ref{sec:disclosure-proofing}), by comparison with the undistorted
version and in comparison with cell counts. If appropriate, items in some
cells are suppressed, and noisy estimates are flagged as such.
\end{enumerate}
                                

\section{Weighting in the QWI}
\label{sec:weights}

The QWI statistics are weighted to conform, along one dimension, to
published BLS QCEW statistics. The fit is, however, not exact, since the
weights are applied before statistics are calculated from the
noise-distorted data. 

When building the ECF, weights are computed such that measured
beginning-of-quarter UI employment
of in-scope units, when properly weighted, is equal to published QCEW
state-wide employment in the first month of the quarter.
% from ECFv31
%Creation of the weights, step
%1.
%
% First, the sample over which the weights will be positive (and therefore
% which units will be included in any calculations) is determined. If a unit
% has ever appeared on the ES202 and is in the UI, then it is eligible for a
% positive weight. 
% %
% Additional restrictions are imposed for the actual calculation of the
% overall adjustment. 
The overall adjustment factor is calculated for private establishments and
later applied to public-sector establishments as well. % %
% The second component of the weights is based on the difference between UI
% and ES202 employment at the SEIN level. These differences are calculated and
% trimmed if necessary. 

Selection and longitudinal linking in the QWI changes the in-scope units
somewhat, and a weight-adjustment is recalculated. This weight is then used
for all published QWI statistics. For almost all states and periods, the
post-disclosure difference between the published QCEW statistic and the
appropriate statistic in the QWI system is less than 0.5\%. 


%%% Local Variables: 
%%% mode: latex
%%% TeX-master: "qwi-overview"
%%% End: 

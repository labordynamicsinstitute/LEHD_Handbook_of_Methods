%%%  +--< LEHD-QWI ecf 3.1.49 2005-01-26 schwa305            >--+  %%%
%%%  +--< Location: /programs/production/dev1/current/ecf    >--+  %%%
%%%  +--< File: doc/new_variables_naics_etc.tex              >--+  %%%
% Converted from Microsoft Word to LaTeX
% by Chikrii SoftLab Word2TeX converter (version 2.4)
% Copyright (C) 1999-2001 Kirill A. Chikrii, Anna V. Chikrii
% Copyright (C) 1999-2001 Chikrii SoftLab.
% All rights reserved.
% http://www.word2tex.com/
% mailto: info@word2tex.com, support@word2tex.com
%\subsection{Subject: ECF Enhancements}
%
%\subsubsection{Date: 2/20/2004}
%
%Author: Kevin McKinney


%\documentclass [10pt]{article}
%\usepackage {longtable}
%\usepackage{fullpage}

%\begin{document}





\section{NAICS codes on the ECF}

Enhanced NAICS variables are available on all ECF since February 2003. The
variable list below shows that there are 75 new variables for NAICS alone.
The variables can be differentiated mainly by the source(s) and coding
system used in their creation. There are two sources of data; the ES202 and
the LDB from the BLS: and two coding systems; NAICS1997 and NAICS2002 (see
the Census web site for more info.). Every NAICS variable uses at least one
source and one coding system.





The ESO and FNL variables are of primary importance to the user community. 
The ESO variables use ONLY information from the ES202 and ignore any 
information that may be available on the LDB (see Section\ref{sec:backcoding} for some analysis on 
why this may be preferred). The FNL variables incorporate information from 
both the ES202 and the LDB, with the LDB being the primary source. The 
ES{\_}NAICS{\_}FNL1997 and ES{\_}NAICS{\_}FNL2002 should be used to create 
the QWI estimates. Both the ESO and the FNL variables contain no missing 
values.


\subsection{LDB versus LEHD NAICS backcoding}
\label{sec:backcoding}

The LDB algorithm is to some extent a black box and testing has shown that 
it does a relatively poor job of capturing firm industry changes that 
occurred during the 1990's. In fact, the LDB appears to be a simple
backfill that does not take into account a firm's entire SIC history.

% Kevin: email 2005-01-28
% Somewhere along the line something I said was lost in translation.
% Actually the LDB NAICS codes are too smooth over time (they never change)
% relative to the variation found in SIC codes.  The LDB looks like a
% simple backfill that does not take into account a firm's entire SIC
% history (prior to ~1999).  I know that some of the SIC changes may be
% spurious, but I also feel that there is valuable information in a firm's
% SIC code history that I therefore attempted to use in my impute
% algorithm.  However, overall the effect of the different approaches is
% relatively small since very few firms change industry, especially
% relative to, for example, the proportion of firms that change location. 
% 

Although some of the SIC changes over time may be spurious, a firm's SIC
code history contains valuable information that we have attempted to
preserve in our imputation algorithm. Overall, the effect of the different
approaches is relatively small, since very few firms change industry, in
particular relative to the proportion of firms that change geography.


In the following, we present  a summary of research done on the ESO vs. FNL NAICS codes. 


The NAICS{\_}LDB variable is used for about 85{\%} of the records for 
Illinois, the rest are filled with information from the ES202 (not sure why 
only 85{\%} of the records on our ES202 files are in the LDB. The results 
weighted by employment are about the same suggesting that activity was not a 
criterion for being included on the LDB). First and not surprisingly, in 
later years and quarters (1999+) when NAICS is actively coded by the states, 
the codes look almost identical when available.


Second, there is little variation in the LDB NAICS codes over time compared 
with SIC. Among all of the active SEIN SEINUNITs over the period, a little 
over 8{\%} experience at least one SIC change compared with about 1.5{\%} on 
the LDB (almost all of these are 1999+). While this is not entirely 
unexpected, it is something to keep in mind when comparing NAICS{\_}FNL 
versus SIC or NAICS{\_}ESO employment totals. Many of these changes in 
industry appear to be real and are not captured on the LDB.

One effect of this is that as we go back in time a larger portion of 
employment can be found in NAICS{\_}FNL codes that are different than one 
would expect given the SIC code on the ECF. For example, in 1990 about 
13{\%} of employment is in a NAICS{\_}FNL code that is different than what 
we would expect based on the SIC. By 2001 this number falls to 3{\%}. The 
ES202 based NAICS variable does a better job tracking SIC, since more SIC 
information is used in putting it together (about 3{\%} consistently over 
the period).

The main source of the discrepancy is due to entities that experience a 
change in their SIC code prior to 2000. The LDB appears to ignore this 
change, while the ESO NAICS variable uses an SIC based impute for these 
SEINUNITS. The result is a series that exhibits similar patterns of change 
over time as SIC, while still preserving the value added in the NAICS codes 
for entities that did not experience a change.

Also, users should keep in mind that for early years (<1997) some of the 
NAICS industries have yet to come into existence. I have no estimates on the 
prevalence of this problem.

\subsection{Variable List}




\begin{longtable}[htbp]
{|p{161pt}|p{150pt}|p{120pt}|}
a & a & a  \kill
\hline
Variable Name& 
Source& 
Notes \\
\hline
\hline
\endhead
es{\_}naics{\_}aux1997& 
ES202 NAICS AUX variable& 
BLS coding of aux estabs \\
\hline
es{\_}naics{\_}aux1997{\_}flag& 
& 
 \\
\hline
es{\_}naics{\_}aux1997{\_}miss& 
& 
 \\
\hline
es{\_}naics{\_}aux1997{\_}src& 
& 
 \\
\hline
es{\_}naics{\_}aux2002& 
& 
 \\
\hline
es{\_}naics{\_}aux2002{\_}flag& 
& 
 \\
\hline
es{\_}naics{\_}aux2002{\_}miss& 
& 
 \\
\hline
es{\_}naics{\_}aux2002{\_}src& 
& 
 \\
\hline
es{\_}naics{\_}eso1997& 
ES202 NAICS AUX, NAICS, SIC& 
Only ES202 info used \\
\hline
es{\_}naics{\_}eso1997{\_}miss& 
& 
 \\
\hline
es{\_}naics{\_}eso1997{\_}src& 
& 
 \\
\hline
es{\_}naics{\_}eso2002& 
& 
 \\
\hline
es{\_}naics{\_}eso2002{\_}miss& 
& 
 \\
\hline
es{\_}naics{\_}eso2002{\_}src& 
& 
 \\
\hline
es{\_}naics{\_}fnl1997& 
BLS LDB and ESO Input vars.& 
All industry info used \\
\hline
es{\_}naics{\_}fnl1997{\_}2& 
& 
 \\
\hline
es{\_}naics{\_}fnl1997{\_}3& 
& 
 \\
\hline
es{\_}naics{\_}fnl1997{\_}4& 
& 
 \\
\hline
es{\_}naics{\_}fnl1997{\_}5& 
& 
 \\
\hline
es{\_}naics{\_}fnl1997{\_}miss& 
& 
 \\
\hline
es{\_}naics{\_}fnl1997{\_}src& 
& 
 \\
\hline
es{\_}naics{\_}fnl2002& 
BLS LDB and ESO input vars.& 
All industry info is used \\
\hline
es{\_}naics{\_}fnl2002{\_}2& 
& 
 \\
\hline
es{\_}naics{\_}fnl2002{\_}3& 
& 
 \\
\hline
es{\_}naics{\_}fnl2002{\_}4& 
& 
 \\
\hline
es{\_}naics{\_}fnl2002{\_}5& 
& 
 \\
\hline
es{\_}naics{\_}fnl2002{\_}miss& 
& 
 \\
\hline
es{\_}naics{\_}fnl2002{\_}src& 
& 
 \\
\hline
es{\_}naics{\_}imp1997& 
ES202 SIC code& 
Impute using only SIC \\
\hline
es{\_}naics{\_}imp1997{\_}miss& 
& 
 \\
\hline
es{\_}naics{\_}imp1997{\_}src& 
& 
 \\
\hline
es{\_}naics{\_}imp2002& 
& 
 \\
\hline
es{\_}naics{\_}imp2002{\_}miss& 
& 
 \\
\hline
es{\_}naics{\_}imp2002{\_}src& 
& 
 \\
\hline
es{\_}naics{\_}ldb1997& 
BLS LDB NAICS variable& 
 \\
\hline
es{\_}naics{\_}ldb1997{\_}flag& 
& 
 \\
\hline
es{\_}naics{\_}ldb1997{\_}miss& 
& 
 \\
\hline
es{\_}naics{\_}ldb1997{\_}src& 
& 
 \\
\hline
es{\_}naics{\_}ldb2002& 
& 
 \\
\hline
es{\_}naics{\_}ldb2002{\_}flag& 
& 
 \\
\hline
es{\_}naics{\_}ldb2002{\_}miss& 
& 
 \\
\hline
es{\_}naics{\_}ldb2002{\_}src& 
& 
 \\
\hline
es{\_}naics1997& 
ES202 NAICS Only& 
 \\
\hline
es{\_}naics1997{\_}flag& 
& 
 \\
\hline
es{\_}naics1997{\_}miss& 
& 
 \\
\hline
es{\_}naics1997{\_}src& 
& 
 \\
\hline
es{\_}naics2002& 
& 
 \\
\hline
es{\_}naics2002{\_}flag& 
& 
 \\
\hline
es{\_}naics2002{\_}miss& 
& 
 \\
\hline
es{\_}naics2002{\_}src& 
& 
 \\
\hline
mode{\_}es{\_}naics{\_}eso1997& 
Mode of ESO SEINUNIT var& 
 \\
\hline
mode{\_}es{\_}naics{\_}eso1997{\_}emp& 
& 
 \\
\hline
mode{\_}es{\_}naics{\_}eso1997{\_}emp{\_}flag& 
& 
 \\
\hline
mode{\_}es{\_}naics{\_}eso1997{\_}emp{\_}miss& 
& 
 \\
\hline
mode{\_}es{\_}naics{\_}eso1997{\_}flag& 
& 
 \\
\hline
mode{\_}es{\_}naics{\_}eso1997{\_}miss& 
& 
 \\
\hline
mode{\_}es{\_}naics{\_}eso2002& 
& 
 \\
\hline
mode{\_}es{\_}naics{\_}eso2002{\_}emp& 
& 
 \\
\hline
mode{\_}es{\_}naics{\_}eso2002{\_}emp{\_}flag& 
& 
 \\
\hline
mode{\_}es{\_}naics{\_}eso2002{\_}emp{\_}miss& 
& 
 \\
\hline
mode{\_}es{\_}naics{\_}eso2002{\_}flag& 
& 
 \\
\hline
mode{\_}es{\_}naics{\_}eso2002{\_}miss& 
& 
 \\
\hline
mode{\_}es{\_}naics{\_}fnl1997& 
Mode of FNL SEINUNIT var& 
 \\
\hline
mode{\_}es{\_}naics{\_}fnl1997{\_}emp& 
& 
 \\
\hline
mode{\_}es{\_}naics{\_}fnl1997{\_}emp{\_}flag& 
& 
 \\
\hline
mode{\_}es{\_}naics{\_}fnl1997{\_}emp{\_}miss& 
& 
 \\
\hline
mode{\_}es{\_}naics{\_}fnl1997{\_}flag& 
& 
 \\
\hline
mode{\_}es{\_}naics{\_}fnl1997{\_}miss& 
& 
 \\
\hline
mode{\_}es{\_}naics{\_}fnl2002& 
& 
 \\
\hline
mode{\_}es{\_}naics{\_}fnl2002{\_}emp& 
& 
 \\
\hline
mode{\_}es{\_}naics{\_}fnl2002{\_}emp{\_}flag& 
& 
 \\
\hline
mode{\_}es{\_}naics{\_}fnl2002{\_}emp{\_}miss& 
& 
 \\
\hline
mode{\_}es{\_}naics{\_}fnl2002{\_}flag& 
& 
 \\
\hline
mode{\_}es{\_}naics{\_}fnl2002{\_}miss& 
& 
\label{tab1}
 \\
\hline
\end{longtable}





\subsection{Coding of MISS and SRC}
\label{sec:coding_miss_src}




Each new NAICS variable has several associated variables of which the miss 
and src variable are the most important. 


\subsubsection{MISS Variable Codes}


If information from another period 
is used, the flag variable reports how many quarters away the NAICS value 
was found. Values greater than six should only appear in SEINUNIT level variables. If 
NAICS is missing for all quarters, then the SEINUNIT value has been filled 
with the SEIN value. The SEINUNIT codes  represent the SEIN value +5.




\begin{table}[htbp]
  \centering
\caption{MISS Variable Codes}
  \begin{tabular}{rcp{10cm}}
\\
0&=& Valid value available in that period\\
1&=& Missing\\
1.5&=& (1999 and earlier only) Filled using impute based on SIC due to an SIC
change over the period.\\
2&=& Filled using own code from another period\\
3&=& Filled from another source contemporaneously\\
5&=& Filled using the non-employ weight mode (SEIN mode var only)\\
6&=& Unconditionally imputed (SEIN mode var only)\\
6&=& NAICS imputed using SIC unconditional impute (SEIN mode var only)\\
7&=& Filled using the SEIN mode from another period (sic, fnl and eso vars 
only)\\
11&=& Filled using unconditional impute of SEIN value (sic, fnl and eso vars 
only)\\

  \end{tabular}
  \label{tab:miss}
\end{table}





\subsubsection{SRC Variable Codes}

The ESO and FNL variables use the following source codes. If more detail is 
desired about the source of the NAICS code, the user must look to the SRC 
code for that source. For example, if the ESO source code for 
ES{\_}NAICS{\_}ESO1997 says NCS, then the actual SRC information will be 
found in ES{\_}NAICS1997{\_}SRC.

\begin{table}[htbp]
  \centering
  \caption{SRC Variable: ESO, FNL}
  \label{tab:src1}
  \begin{tabular}{rcp{10cm}}
\\
AUX&=& Source is the ES202 NAICS AUX variable\\
LDB&=& Source is the LDB NAICS variable      \\
NCS&=& Source is the ES202 NAICS variable    \\
SIC&=& Source is the ES202 SIC code          \\  

  \end{tabular}
\end{table}




The AUX, LDB and standard NAICS codes have the following source variables. 

\begin{table}[htbp]
  \centering
  \caption{SRC Variable: AUX, LDB, NAICS}
  \label{tab:src2}
  \begin{tabular}{rcp{10cm}}
\\
SIC&=& Source is the ES202 SIC code\\
NO2&=&Source is a NAICS 2002 Code\\
N97&=& Source is a NAICS 1997 Code\\

  \end{tabular}
\end{table}




\subsection{NAICS algorithm precedence ordering}


Four basic sources of industry information are available on the ECF; NAICS, 
NAICS{\_}AUX, SIC, and the NAICS{\_}LDB. The NAICS, NAICS{\_}AUX, and 
NAICS{\_}LDB missing values were filled using the following preference 
ordering. SIC is filled similarly, except miss=1.5 is not used and NAICS, 
not SIC, would be the basis for the impute when miss=3.




\begin{enumerate}
\item Valid 6 digit industry code (miss=0)

\item Imputed code based on first 3,4, or 5 digits when no valid six digit code is 
available in another period (miss=0)

\item Imputed code based on contemporaneous SIC if SIC changed prior to 2000 
(miss=1.5)

\item Valid 6 digit code from another period (miss=2)

\item Valid code from another source (for example if NAICS1997 is missing, 
NAICS2002 or SIC may be available) (miss=3)

\item Use SEIN mode value (miss=5,7)

\item Unconditional impute (miss=6,11)
\end{enumerate}



\subsection{ESO and FNL variables}



The ESO and FNL variables are made up of combinations of the various sources 
of industry information. The ESO variable uses the NAICS and NAICS{\_}AUX 
variables as input. Information from the variable with the lowest MISS value 
is preferred although in case of a tie the NAICS{\_}AUX value is used.





The FNL variable uses the ESO and LDB variables. Information from the 
variable with the lowest MISS value is preferred although in case of a tie 
the NAICS{\_}LDB value is used. Keep in mind that although the source of an 
ESO or FNL variable may be equal to NCS, the actual source can only be 
ascertained by going back to the original.








\subsection{Employment Flag Variable Codes}

All current uses of the ECF have been forced to assume that employment and 
payroll information has been reported by the firm, although under certain 
conditions the ES202 processing specifications require imputation of missing 
values. The flag values below allow the user to determine when imputation 
has occurred.





The master record contains valuable information that has been preserved in 
the master{\_}empl{\_}month1{\_}flg --master{\_}total{\_}wages{\_}flg 
variables. For example, one should theoretically be able to distinguish 0 
prorated codes from 0 unknowns by looking at multi units with masters that 
reported (code=1) and subunits with a zero.





The following information stems from an email exchange between Kevin
McKinney (U.S. Census Bureau) and George Putnam (Illinois) on 12/15/2003.


{\it 
Prior to late 1995: 

\begin{tabular}{p{1cm}rcp{10cm}}
&0&=& unknown\\
&1&=& not imputed\\
&2&=& imputed (including prorated multiple worksite data)\\
\end{tabular}





Late 1995 or early 1996:

\begin{tabular}{p{1cm}rcp{10cm}}
&0&=& prorated data (multiple worksites)\\
&1&=& actual or not imputed data\\
&2&=& estimated data\\
\end{tabular}




1997 first quarter forward (ES202 processing manual, Appendix B):

\begin{tabular}{p{1cm}rcp{10cm}}
&Blank&=& reported data\\
&R &=& reported data\\
&A &=& estimated from CES report\\
&C &=& changed (re-reported)\\
&D &=& reported from missing data notice\\
&E &=& imputed single unit employment or imputed worksite employment prorated 
from imputed parent record\\
&H &=& hand-imputed (not system generated)\\
&L &=& late reported (overrides prior imputation)\\
&M &=& missing data\\
&N &=& zero-filled pending resolution of long-term delinquent reporter\\
&P &=& prorated from reported master to worksite\\
&S &=& aggregated master from reported MWR or EDI data \\
&W &=& estimated from wage record employment\\
&X &=& non-numeric employment zero-filled pending further action\\
\end{tabular}

}


\subsection{Multi-Unit Code or MEEI}





The MULTI{\_}UNIT variable on the ECF is determined by counting the number 
of SEINUNIT records for a given SEIN once the master records have been 
removed. However, some multiunit firms refuse to report detailed information 
for their sub-units and appear as single units on the ECF. The table below 
provides an estimate of the magnitude of multiunit firms refusing to report 
detailed unit information using data from Illinois.


\begin{tabular}{p{122pt}p{102pt}p{122pt}p{1pt}p{1pt}}
\\
&\multicolumn{4}{c}{MULTI{\_}UNIT}  \\
\hline
MULTI{\_}UNIT{\_}CODE& 
0& 
1& 
\multicolumn{2}{ p{3pt} }{}  \\
\hline
1& 
1,483,808& 
0& 
\multicolumn{2}{ p{3pt} }{}  \\
2& 
1& 
0& 
\multicolumn{2}{ p{3pt} }{}  \\
3& 
120& 
155859& 
\multicolumn{2}{ p{3pt} }{}  \\
4& 
5808& 
0& 
\multicolumn{2}{ p{3pt} }{}  \\
5& 
0& 
33& 
\multicolumn{2}{ p{3pt} }{}  \\
6& 
13899& 
0& 
\multicolumn{2}{ p{3pt} }{}  \\
\hline
\label{tab2}
\end{tabular}

Prior to 1997 (ES202 processing manual sent from George Putnam):

\begin{tabular}{p{1cm}rcp{10cm}}
&1 &=& Single establishment unit\\
&2 &=& Multi-unit master record\\
&3 &=& Subunit establishment level record for a multi-unit employer\\
&4 &=& Multi-establishment employer reporting as a single unit due to 
unavailability of data, including refusals\\
&5 &=& A subunit record that actually represents a combination 
of establishments; finer level breakouts are not yet available\\
&6 &=& Known multi establishment employer reporting as a single unit and not 
solicited for disaggregation because of small employment ($<$ 10) in all 
secondary establishments combined\\
\end{tabular}




1997 first quarter forward (ES202 processing manual, Appendix B):

\begin{tabular}{p{1cm}rcp{10cm}}
&1 &=& Single establishment unit\\
&2 &=& Multi-unit master record\\
&3 &=& Subunit establishment level record for a multi-unit employer\\
&4 &=& Multi-establishment employer reporting as a single unit due to 
unavailability of data, including refusals\\
&5 &=& A subunit record that actually represents a combination 
of establishments; finer level breakouts are not yet available\\
&6 &=& Known multi establishment employer reporting as a single unit and not 
solicited for disaggregation because of small employment (< 10) in all 
secondary establishments combined\\
\end{tabular}




\subsection {Auxiliary Code}





This variable gives detailed information about firm locations that do not 
directly engage in production related activities.





Prior to 1997 (ES202 processing manual sent from George Putnam):

\begin{tabular}{p{1cm}rcp{10cm}}
&0 &=& Unknown\\
&1 &=& Central administrative office\\
&2 &=& Performs research, development or testing services\\
&3 &=& Provides storage or warehouse services\\
&5 &=& Does not provide auxiliary services, it is an operating establishment\\
&9 &=& Performs auxiliary services that are not described above\\
\end{tabular}




1997 first quarter forward (ES202 processing manual, Appendix B):

\begin{tabular}{p{1cm}rcp{10cm}}
&0 &=& Auxiliary status not known\\
&1 &=& Central administrative office\\
&2 &=& Performs research, development or testing services\\
&3 &=& Provides storage or warehouse services\\
&5 &=& Does not provide auxiliary services, it is an operating establishment\\
&6 &=& Headquarters\\
&7 &=& Administrative, Other than Headquarters\\
&9 &=& Performs auxiliary services that are not described above\\
\end{tabular}




%\end{document}


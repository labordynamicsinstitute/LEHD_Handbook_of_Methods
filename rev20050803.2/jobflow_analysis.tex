%TCIDATA{LaTeXparent=0,0,sw-edit.tex}

% -*- latex -*- 
%
% Time-stamp: <02/07/26 13:43:51 vilhuber> 
%              Automatically adjusted if using Xemacs
%              Please adjust manually if using other editors
%
% jobflow_analysis.tex
% Responsible: John/Paul/Bryce
% Part of QWI_methods.tex

%\section{Overview}
%\index{employment status}
%\index{point-in-time}
%\index{full-quarter}
%\index{accession}
%\index{separation}
%\index{job flows}
%\index{new hires}
%\index{job creation}
%\index{job destruction}
%\index{earnings}
%\index{earnings!change}

The detailed description that follows makes explicit the links to the sets
of files described earlier.
The statistics calculated in this section are based on definitions  summarized in \Cite{AbowdCorbelKramarz99} and
\Cite{DavisHaltiwanger99}. 
As mentioned before, employment is measured at two points in
time (beginning and end of  
quarter) and according to two concepts (any employment status and 
full-quarter employment status). Worker flows are captured by accessions and 
separations with respect to both employment status concepts. Job flows are 
captured by gross job creation and destruction at the firm level, again 
according to both employment concepts. Accessions are further separated into 
new hires and recalls. Earnings and earnings change statistics are 
calculated for each of the worker flow categories as well as for both 
employment statuses.

%\subsection{Calculation of statistics}

The worker and employment flow statistics reported at the county and 
SIC division level are calculated through a multi-step 
\index{county}\index{SIC}
process.%
%
%\footnote{Details on the program sequence used to create the job 
%low statistics are available in Appendix~\Vref{app:jobflow_technical}.}
%
The EHF\index{EHF} (see Subsection~\ref{cha:ehf}), which contains individual
work and earnings histories, is combined with information from the
ICF\index{ICF} (see Subsection~\ref{cha:icf}) to incorporate demographic
characteristics of workers such as age and sex. For each worker in each
year and quarter, an array of jobs at various SEINs\index{SEIN} is stored.
The statistics listed in Subsection~\Vref{cha:definitions} are computed, when
appropriate, for each individual/job/quarter combination. The statistics
are then aggregated to the SEIN level by age and sex to create a file of
totals for each SEIN/year/quarter/agegroup/sexgroup combination. Both the
Workforce Investment Act (WIA)\index{WIA} and CPS\index{CPS} age groups are
used. The totals are stored by age/sex group as well as further aggregated
within SEIN over age and sex group to produce the overall total for the
SEIN as well as marginal totals for sex and age (for example, the total for
females of all ages).  All totals are then aggregated twice more: once to
the industry level and once to the county level. At this point the
statistics are in their final form except for the handling of disclosure
issues, as discussed below.

\subsection{Examples}

The following tables provide an example of how the flow statistics are
computed for four hypothetical individuals who work at three hypothetical
employers over a two year sample period. All individuals and firms in this
example are fictitious. Table~\Vref{table1} summarizes the earnings history
\index{earnings history} of each individual as it would appear in the
employment history file\index{EHF}.  Table~\Vref{table2} presents the
individual level employment flow statistics that can be computed from the
individual work histories. Note that individual 1 leaves employer
\textit{X} at some point during the second quarter of 1995, and that she
begins working for employer \textit{Y} during the same quarter. In
Table~\ref{table2}, employment flow statistics as defined in
Subsection~\ref{cha:definitions} have been computed for every quarter of every
job worked by Person~\textit{1}. Person~\textit{1} is considered to be
employed at employer \textit{X} from 1994:1 -- 1995:2.  Hence, \textit{e=1}
from 1994:1 through 1995:1 since she is still employed at \textit{X} at the
end of each of these quarters. Similarly, \textit{b=1} from 1994:2 through
1995:2 since she is employed at \textit{X} from the very beginning of these
quarters. Note that \textit{b} is missing in 1994:1. The first quarter of
the analysis is out-of-scope for \textit{b}, since it depends on employment
information from the previous quarter. Also note that for in-scope periods,
end-of-quarter employment\index{employment!point-in-time} at time
\textit{t} is equal to beginning-of quarter employment at time \textit{t +
  1}. In Subsection~\ref{cha:definitions}, this identity (Identity
\ref{identity:1}) is defined for aggregates, but as shown in the example it
holds at the individual level as well.

\input{\mypath/jobflow_analysis.table1.tex}

Moving on, \textit{f=1} for Person \textit{1}/Employer \textit{X} from
1994:2-1995:1, but \textit{f} is missing during 1994:1, which is
out-of-scope, and \textit{f=0} during 1995:2 because she is no longer
employed at \textit{X} in 1995:3. In 1995:2 \textit{s=1} and \textit{fs=1}
for Person~\textit{1}/ Employer~\textit{X} because she separates from
Employer~\textit{X} sometime during this quarter and appears to have been
in this job for the entire preceding quarter (1995:1).  In 1995:2,
\textit{a=1} for Person~\textit{1} and Employer~\textit{Y} because she
enters a relationship with Employer~\textit{Y} sometime during this
quarter, and \textit{fa=1} in 1995:3 because this is her first full quarter
\index{employment!full-quarter} at Employer~\textit{Y}. New hires\index{New
  hires}, \textit{h}, is also 1 because she has no previous relationship
with Employer~\textit{Y} in the last four quarters, and recalls\index{recall}
\textit{r=0} for the same reason. A variety of wage measures are also
calculated for each individual: \textit{w1} is simply the wage earned at
each job each quarter, while measures such as \textit{w2}, \textit{w3},
\textit{wa} are calculated as an individual's wage if he or she meets a
certain criteria (\textit{e=1} for \textit{w2}, \textit{f=1} for
\textit{w3}, \textit{etc}.).

\input{\mypath/jobflow_analysis.table2.tex}

In Table~\Vref{table3}, the individual statistics are aggregated to the
employer level by summing individual statistics by SEIN. \textit{E} for
Employer~\textit{X} in 95:1, then, is the sum of \textit{e} over all
individuals working at \textit{X} in 1995:1 (in this case individuals 1 and
4). Since \textit{e=1} for Individual~\textit{1}, who remains with
Employer~\textit{X} the next quarter, and \textit{e=0} for
Individual~\textit{4}, who has no wage record with Employer~\textit{X} the
next quarter, \textit{E=1}. Similarly, since \textit{a=0} for both
individuals this quarter (both worked at \textit{X} last quarter also),
\textit{A=0}. The job flow\mindex{job flow} at Employer~\textit{X}, defined
as the net increase in employment over that quarter, is calculated as the
difference between the number of end-of-quarter jobs held and the number of
beginning-of quarter jobs held. Thus, $JF = E -- B$, or 1 -- 2 = -1 in this
case. Because there was a negative net job flow of 1 this quarter, job
creation\index{job creation} \textit{JC= 0} and job destruction\index{job
  destruction} \textit{JD=1}. Total payroll \textit{W1} is also computed
for each employer; for Employer~\textit{X} in 1995:1 it is simply the sum
of the wages paid to individuals 1 and 4: {\$}5000 + {\$}4000 = {\$}9000.
Individual~1 also had end-of-quarter wages \textit{w2=5000} because she was
end-of-quarter employed at \textit{X} this period. For
Individual~\textit{4}, \textit{w2=0} because \textit{e=0} at \textit{X} in
1995:1. Total end of quarter wages \textit{W2} for Employer~\textit{X} in
1995:1 is then calculated as the sum of wages at all end-of-quarter jobs.
In this case, it is simply {\$}5000 since Individual~1 has the only
end-of-quarter job at \textit{X} in 1995:1.

\input{\mypath/jobflow_analysis.table3.tex}

%\marginpar{\tiny This paragraph (and chapter) might need to be tagged for the index!!}
Several identities from Subsection~\ref{cha:definitions} are illustrated in
Table~\ref{table3}.  Once again Identity~\ref{identity:1} ($B_{jt} = E_{jt
  - 1} )$ is noticeable just from glancing at the columns of numbers B and
E.  Identity~\ref{identity:3}, $E_{jt} = B_{jt} + A_{jt} - S_{jt} $ also
holds whenever all four variables are in-scope. For example, for
Employer~\textit{X} in 1995:1, \textit{E = 1 = 2 + 0 -- 1}. For this
employer in 1995:2, \textit{E = 0 = 1 + 0 -- 1}. Identity~\ref{identity:4},
$JF_{jt} = JC_{jt} - JD_{jt}$ is also true: for \textit{X} in 1995: 1
\textit{JF = -1 = 0 -- 1}. Identity~\ref{identity:5}, $E_{jt} = B_{jt} +
JC_{jt} - JD_{jt} $ , (\textit{X} in 1995:1 : \textit{E = 1 = 2 + 0 -- 1})
and Identity~\ref{identity:6}, $A_{jt} - S_{jt} = JC_{jt} - JD_{jt} $,
(\textit{X} in 1995: 1 : \textit{A -- S = 0 -- 1 = 0 -- 1 = JC -- JD}).
Finally, Identity~\ref{identity:15}, the total payroll identity ($W_{1jt} =
W_{2jt} + WS_{jt} )$ is met in all cases. For example, for SEIN \textit{X}
in 1995: 2, \textit{W1 = {\$}9000 = 5000 + 4000 = W2 + WS}. When \textit{WS
  is missing}, as in most cases, \textit{W1} and \textit{W2} are simply
equal because every wage is an end-of-quarter wage.

\input{\mypath/jobflow_analysis.table4.tex}

In Table~\Vref{table4}, the SEIN-level statistics are aggregated in a
similar way to create total flows and average wages. These can be thought
of as county totals if the hypothetical universe includes just a single
county. The total flows are computed exactly as the employer level flows in
Table~\ref{table3}. For 1995:1, total jobs at the end of quarter, total
\textit{E}, is just the sum of \textit{E} for \textit{X}, \textit{Y}, and
\textit{Z}: 1 + 0 + 1 = 2. Note that this is the same as the sum of all
individual \textit{e} in Table\ref{table2} for 1995:1. Total accessions are computed
similarly (\textit{A = 0 + 0 + 0}) as are total wages (\textit{W1 = 9000 +
  2000 + 6000 = 17000}).  Average wages (for example, \textit{Z\_W2},
\textit{Z\_WA}) are computed by summing total wages for \textit{X},
\textit{Y}, and \textit{Z}, and dividing by the total number of individuals
used to calculate the particular wage measure. For example, \textit{Z\_W2}
for 1995: 1 is computed as the sum of \textit{W2} for \textit{X},
\textit{Y}, and \textit{Z} where defined (5000 + 6000) divided by the total
number of end-of-quarter positions (\textit{E = 2}) for an average
end-of-quarter wage of \textit{Z\_W2} = {\$}5500. \textit{Z\_WA} is
undefined for this quarter because there are no accessions this quarter. In
1995:2 they are computed as the sum of \textit{WA} for \textit{X},
\textit{Y}, and \textit{Z} where defined (2000, since \textit{WA} is only defined
for \textit{Y} this quarter) divided by the total number of accessions this
quarter (1) so the average wage to accessions in 1995:2 is simply {\$}2000.




%%% Local Variables: 
%%% mode: latex
%%% TeX-master: "qwi-overview"
%%% End: 

%TCIDATA{LaTeXparent=0,0,sw-edit.tex}

% -*- latex -*- 
%
% Time-stamp: <05/04/06 23:54:19 vilhuber> 
%              Automatically adjusted if using Xemacs
%              Please adjust manually if using other editors
%
% icf.tex
% Responsible: Fredrik, Martha
% Part of QWI_methods.tex

%\subsection{Individual Characteristics File}
\label{sec:icf_overview}
%\label{cha:icf}

The \textit{Individual Characteristics File }(ICF)%
\mindex{ICF}\index{Individual Characteristics File|see{ICF}}
for each state contains one record for every
person who is ever employed in that state over the time period spanned by
the state's unemployment insurance records. 

The ICF is constructed in the following manner. First, the universe of
individuals is defined by compiling the list of unique PIKs from the
EHF. Demographic information from the PCF is then merged on by PIK, and
records without a valid match flagged. PIK\index{PIK}-survey identifier
crosswalks link the CPS and SIPP ID variables into the ICF, and gender and
age information from the CPS is used to complement and verify the
PCF-provided information. 
%\marginpar{\tiny Talk  about validated/non-validated crosswalks? Coverage per year?}

\subsection{Age and gender imputation}
\index{imputation!age}
\index{imputation!gender}

Approximately 3{\%} of the PIKs\index{PIK} found in the UI\index{UI}
wage records do not match to the PCF\index{PCF} file. Multiple imputation
methods are used to assign \index{missing values} date of birth and gender
to these individuals. To impute gender, the probability of being male is
estimated using a state-specific logit
model\index{imputation!logit}\index{logit}:

\begin{equation}
  \label{eq:logit:gender}
  P(male)=f(X_{is}\beta_s)
\end{equation}
where $X_{is}$ contains a  full set of yearly log earnings and squared log
earnings, and full set of employment indicators covering time period spanned
by the state's records, for each individual $i$ with strictly positive earnings
within state $s$ and non-missing PCF gender. The state-specific $\hat{\beta}_s$ as estimated from Equation
(\ref{eq:logit:gender}) is then used to predict the probability of
being male for individuals with missing gender  within state $s$, and gender is assigned as
\begin{equation}
  \label{eq:impute:gender}
  \text{male}~\text{if}~X_{is}\hat{\beta}_s \geq \mu_l 
\end{equation}
where $\mu_l \sim U[0,1]$ is one of  $l = 1,\dots,10$ independent draws
from the distribution. Thus, each individual with missing gender is
assigned ten independent implicates.

The  imputation of date of birth is done in a similar fashion using a multinomial logit
to predict the probability of being in one of eight age categories and then
assigning an age based on this probability and the distribution of ages
within the category. Again, the imputation occurs ten times.

It should be noted that if an individual is missing gender or age in the
PCF, but not in the CPS, then the CPS values are used, not the imputed
values. Also, before the imputation model for date of birth is implemented,
basic editing of the date of birth variable takes place to account for
obvious coding errors, such as a negative age at the time when UI earnings
is first reported for the individual. In those relatively rare cases where
the date of birth information is deemed unrealistic it is set to missing
and instead imputed based on the model described above.

\subsection{Place of residence imputation}
\label{sec:icf:place_impute}
\index{imputation!residence}

Place of residence information on the ICF is derived from the StARS
(Statistical Administrative Records System), which for the vast majority of
the individuals found in the UI wage records contains information on the
place of residence down to the exact geographical coordinates. However, in
some 10 
%[LARS: THIS NUMBER IS A ROUGH ESTIMATE BASED ON MY RECOLLECTION.
%CHECK WITH PRODUCTION FOR MORE PRECISION] 
percent of all cases this information is incomplete or missing. In
particular the  QWI computation relies on completed place of residence
information is because this is a key conditioning variable in the
unit-to-worker (U2W) imputation model (see Section~\ref{sec:u2w}). 

%In its current version the unit-to-worker
%imputation model needs place of residence information as of 1999 and,
%therefore, the completed place of residence information on the ICF reflects
%the same year, even though longitudinal information at an annual frequency
%is available from StARS.


County of residence is imputed based on a categorical model of data that
can be represented by a contingency table. In particular, separately for
each state, unique combinations of categories of gender, age, race, income
and county of work are used to form $i=1,\ldots ,I$ populations. For each
sample $i$, the probability of residing in a particular county as of 1999,
$\pi_{ij}$, is estimated by the sample proportion, $p_{ij}=n_{ij}/n_{i}$,
where $j=1,\ldots ,J$ indexes all the counties in the state plus an extra
category for out-of-state residents.

County of residence is then imputed based on
%
\begin{eqnarray}
  \label{eq:county_impute}
  county = j &\text{if}& P_{ij - 1} \le u_k < P_{ij} 
\end{eqnarray}
%
where $P_i $ is the CDF corresponding to $p_i $ for the $i$th population
and $\mu _{kl} \sim U[0,1]$ is one of $k = 1,\ldots,10$ independent draws for
the $l$th individual belonging to the $i$th population.

In its current version no geography below the county level is imputed and
in those cases where exact geographical coordinates are incomplete the
centroid of the finest geographical area is used. Thus, in cases where no
geography information is available this amounts to the centroid of the
imputed county. Geographical coordinates are not assigned to individuals
whose county of residence has been imputed to be out-of-state.

%
%
%09{\_}icf.sas - 15{\_}icf.sas: PCF, CPR match, who is imputed, what about
% people who move, precision of TIGER match.

\subsection{Education imputation}
\index{imputation!education}

%[ FREDRIK: describe, please. Verbal plus imputation model ]
%16{\_}icf.sas- 19{\_}icf.sas

The imputation model for education relies on a statistical match between
the Decennial Census 1990 and LEHD data. The probability of belonging to
one of 13 education categories is estimated using 1990 Decennial data
conditional on characteristics that are common to both Decennial and LEHD
data, using a state-specific logit
model\index{imputation!logit}\index{logit}:

\begin{equation}
  \label{eq:logit:educat}
  P(educat)=f(Z_{is}\gamma_s)
\end{equation}
where $Z_{is}$ contains age categories, earnings categories, and industry
dummies for individuals age 14 and older in the 1990 Census Long Form
residing in the state being estimated, and who reported strictly positive
wage earnings.

Education is then imputed based on
\begin{eqnarray}
  \label{eq:impute:educat}
  \text{educat}=j &~\text{if} & {cp}_{j-1} \leq \mu_l < {cp}_{j} 
\end{eqnarray}
where $cp_{j}=Z_{is}\hat{\gamma}_s$ and $\mu_l \sim U[0,1]$ is one of  $l =
11,\dots,20$ independent draws, and $i \in EHF$.


%%% Local Variables: 
%%% mode: latex
%%% TeX-master: "qwi-overview"
%%% End: 

%TCIDATA{Version=4.00.0.2321}
%TCIDATA{LaTeXparent=0,0,sw-edit.tex}
                      
% -*- latex -*- 
%
% Time-stamp: <05/04/06 23:54:45 vilhuber> 
% Incorporates JOhn abowd mods 2005-04-05
%              Automatically adjusted if using Xemacs
%              Please adjust manually if using other editors
%
% qwi-intro.tex

%\section{Introduction}
%\label{sec:intro}
%\marginpar{\footnotesize Needs final edit}
Since 2003, U.S. Census Bureau has published a new and novel statistical
series: the Quarterly Workforce Indicators (QWI). Compiled from
administrative records data collected by a large number of states for both
jobs and firms, and enhanced with information culled from other data sets at
the Census Bureau, these statistics offer unprecedented detail on the local
dynamics of labor markets. Despite the fine detail, confidentiality is
maintained due to the application of state-of-the-art protection methods.

The underlying data infrastructure was designed by the Longitudinal
Employer-Household Dynamics Program at the Census Bureau 
%Abowd-Haltiwanger-Lane AER 2004)
\citep{AbowdEtAl2004}. Although the QWI are the flagship
statistical product published from the LEHD infrastructure files, the latter
have found a much more widespread application. The infrastructure
constitutes an encompassing and almost universal data source for individuals
and firms of all 31 currently participating states.%
%
\footnote{%
The number of participating states still increases regularly as new Memoranda of
Understanding are signed and new states begin shipping data. As of April 5,
2005, there are 31 states in production (shipping data to Census) of which
27 are available at \htmladdnormallink{http://lehd.dsd.census.gov/}{http://lehd.dsd.census.gov/}}

In this article, we describe the primary input data underlying the the LEHD
Infrastructure Files, the methods by which the Infrastructure Files are
compiled, and how these files are integrated to create the Quarterly
Workforce Indicators. We also provide details about the statistical models
used to improve the basic administrative data, and describe enhancements and
limitations imposed by both data and legal constraints. Some of the
infrastructure and derivative microdata files have recently been made
available within the Research Data Centers of the U.S. Census Bureau, and we
point out these files during the discussion.

The QWI use a bewildering array of data sources, both from administrative
records and from survey and census data.  The Census Bureau receives UI
wage records and ES-202 establishment records from each state participating
in the program. The Bureau then uses these products to integrate
information about the individuals (place of residence, sex, birth date,
place of birth, race, education) with information about the employer (place
of work, industry, employment, sales). Not all of the integration methods
are straight one-to-one matches. In some cases, statistical matching
techniques are used, and in others variable values are imputed. Throughout,
critical imputations are done multiple times, 
improving the precision of the final estimates. 

It should be noted that the data integration is a two-way street. Not only
do the Census Bureau's surveys and censuses improve the detail on the
administrative files: As a part of its Title~13 mission, the Census Bureau
uses the integrated files to in turn improve the Census Bureau's
demographic surveys, like the Current Population Survey, the Survey of
Income and Program Participation, and the American Community Survey. They
are also used to improve the Census Bureau's Business Register, which is
the sampling frame for all its economic data and the initial contact frame
for the Economic Census.


We  give an
overview of the different raw data inputs and how they are treated and
adjusted in Section~\ref{sec:input_files}. In a system that focusses on the
dynamics at the individual and firm level, proper identification of the
entities is important, and we briefly highlight the steps undertaken to
this end. A more detailed analysis of the probabilistic editing of
individual record has been published elsewhere \citep{AbowdVilhuber2005}. The
raw data are then aggregated and standardized into a series of component
files, which we call the ``Infrastructure Files'', as described in Section~\ref{sec:files}. Finally,
Sections~\ref{sec:aux} and~\ref{sec:aggregate} illustrate how they are brought together to
create the QWI statistics. It will soon become clear to the reader that the
level of detail potentially available with these statistics requires
special attention to the confidentiality of the the underlying entities.
How their identity is protected is described in Section
\ref{sec:confidentiality}. Many of the files described in this paper are
accessible in either a public-use or restricted-access version, and a brief
description with pointers to more detailed documentation is provided in
Section~\ref{sec:public}. Section~\ref{sec:conclusion} concludes and
provides a glimpse at the ongoing research into improving the infrastructure files.

We should note that this paper has far too {\it few} authors. Over the years,
many individuals have contributed to the creation of these files. [~complete list here~]

%%% Local Variables: 
%%% mode: latex
%%% TeX-master: "qwi-overview"
%%% End: 

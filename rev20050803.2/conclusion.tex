%TCIDATA{Version=4.00.0.2321}
%TCIDATA{LaTeXparent=0,0,sw-edit.tex}
                      
%
% Time-stamp: <05/04/06 08:10:23 vilhuber>
% Contains John Abowd mods 2005-04-05

\section{Future projects}

This section describes some of the ongoing efforts to improve the LEHD\
Infrastructure Files.

\subsection{Planned improvements to the ICF}

Currently researchers at LEHD are developing an enhanced, longitudinal
version of the ICF, referred to as the LICF. It improves on the current
version of the ICF in a number of ways. The current ICF is a collection of
state-specific files. Individuals appearing in multiple states are treated
independently for each state in all missing data analyses and in the
computation of employment statistics. The LICF will be national in scope,
with a single set of missing data implicates for any PIK found on any of the
UI wage records, regardless of state and with integrated national geography.

Additional data sources will be integrated with the enhanced version of the
ICF using direct links. The statistical link to the 1990 Decennial Census
will be replaced by a direct link to the 2000 Decennial Census, and
additional links to the ACS will be incorporated. The existing education
impututation will greatly benefit from this enhancement. The additional
links, as well as improved links to currently integrated data, will also
allow for additional time-invariant characteristics to be incorporated and
completed, including information on race and ethnicity and additional
time-varying characteristics such as TANF\ recipiency.

Longitudinal residence information will be appended to the ICF based on the
information available from the StARS. Where appropriate, residence will be
imputed based on a change in residence imputation model and Bayesian methods
for imputing geography at the block level, replacing the current residential
address missing data imputation model. In fact, all imputation models will
be based on the most up-to-date imputation engines developed at LEHD.

\subsection{Planned improvements to the EHF}

The UI\ wage records in several states suffer from defects in the historical
records. These defects can be detected automatically when they produce a big
enough fluctuation in certain flow statistics, typically beginning of period
employment as compared to total flow employment. Algorithms have been
developed to detect the presence of missing wage records using the posterior
predictive distribution.of employment histories given the available data and
an informative prior on certain patterns. Once detected, the missing wage
records are imputed, again using appropriate Bayesian methods. The same
imputation engines are also being used to impute top-coded UI wages. These
improvements are in the testing stage and should be implemented within the
next year.

\subsection{Planned improvements to the ECF}

Two major enhancements to the ECF are in development. The first is a
probabilistic record link to the Census Bureau's Business Register in order
to improve the physical addresses on the ECF. This enhancement is currently
in the testing phase. The second improvement to the ECF is a long-term
project to incorporate information on non-employer businesses. The
non-employer enhancements will affect both the ECF and the EHF because the
information on non-employers also includes earnings from the non-employing
business.

\subsection{Creation of public-use synthetic data}

As a part of a new, National Science Foundation Information Technology
Research grant awarded to a consortium of Census Research Data Centers,
researchers at LEHD and other parts of Census will collaborate with
statisticians working in the RDCs to create and validate synthetic
micro-data from the LEHD infrastructure files. Such synthetic micro-data
will be confidentiality protected so that they may be released for public
use. They will also be inference valid--permitting the estimation of some
statistical models with results comparable to those obtained on the
confidential micro-data.

\subsection{The first 21st century statistical system}

The goal of the development of the Quarterly Workforce Indicators was to
create a 21st century statistical system. Without increasing respondent
burden, the LEHD\ infrastructure permits the creation of extremely detailed
statistics that, for the first time in the U.S., provide integrated
demographic and economic information about the local labor market. The same
techniques will work for other areas of interest--transportation dynamics
and welfare-to-work dynamics to name just two examples. The two essential
features of 21st century\ statistical systems will be their heavy reliance
on existing data instruments (surveys, censuses and administrative records
that are already in production) and their extensive use of data-intensive
statistical modeling to enhance and summarize this information. In these
regards, we think the LEHD infrastructure and the QWI system are worthy
pioneers.

%%% Local Variables: 
%%% mode: latex
%%% TeX-master: "qwi-overview"
%%% End: 

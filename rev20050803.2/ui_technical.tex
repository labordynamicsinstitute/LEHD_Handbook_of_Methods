%TCIDATA{LaTeXparent=0,0,sw-edit.tex}
                      
% -*- latex -*- 
%
% Time-stamp: <02/03/27 21:07:15 vilhuber> 
%              Automatically adjusted if using Xemacs
%              Please adjust manually if using other editors
%
% ui_technical.tex
% Responsible: Lars
% Part of QWI_methods.tex

The unemployment insurance (UI)\index{UI} wage records are and have been
received on a variety of media: CDROM\index{CDROM} and different
tape\index{tape} formats. On these media, files may come as single file
spanning all quarters of available data down to several files per quarter.
Formats of the actual data on the media vary as well, from flat
ASCII\index{ASCII} files that vary across vintages of data to other data
formats such as Fortran\index{Fortran} or Dbase files. Formats do vary
across states, and may vary within the date range of any given state.

LEHD has developed a set of SAS macros that take these diverse data and
standardize them. This also occurs for the \aindex{ES-202} data, but is documented
separately (see Chapter~\Vref{app:es202_technical}). The data analyst
reading in the data configures the \aindex{SAS} macros in order to accomodate the
varying formats. The suite of programs then reads in the data, standardizes
variable names and formats, sorts into a standard sort order, and saves the
file in a standardized location with a standardized name. All subsequent
processing, from the ICF\index{ICF} on, uses only these standardized data. It is this
suite of procedures and programs which will be described in the next sections.

\section{Data formats\label{app:ui_formats}}

The first step after receipt of the data is to ascertain the format the
data is in. Table~\Vref{tab:orig_layout} describes the standard format in
which data ideally is received. At the present date, 10 different
formats, comprising both ASCII\index{ASCII} and Dbase\index{Dbase} data descriptions, have been used
in the QWI processing. All formats are available in 
\begin{center}
/programs/projects/auxiliary/Layouts  ,
\end{center}
with legacy layouts labeled 
\begin{center}
 \index{state@\&state.}. \&state..ui.00.0x.layout.sas
\end{center}
and newer standardized layouts called
\begin{center}
  ui.00.0x.layout.sas.
\end{center}
If it is ascertained that the data for a given state fits one of the
available data descriptions, then that information will become one of the
macro parameters to be used in the readin files. If the data does not fit
any of the available descriptions, then a new standardized layout with
sequence number \textit{0x +1} is generated, and saved in the layout repository.

\section{Program sequence\label{app:ui_programs}}

The following list contains the names of all programs used to process the raw
data files. They are listed by location.


\paragraph{/data/master/ui/\&state/conversions/SSN}
 Contains state-specific macro calls. The following programs are typically
 found in this directory (see Section~\Vref{app:ui_step_by_step} for details).

\begin{itemize}                                %%%
\item {\tt config\&state..sas}\index{config\&state..sas}: Configuration file containing directory positions 
  and format definitions, read in by all SAS program files.
\item {\tt 01.01.readin.sas}: Reads the data in using the appropriate \mindex{layout}.
\item {\tt 02.02.readin-stats.sas}: Provides statistics and quality control on
  the readin files. Also creates a set of \LaTeX\index{LaTeX@\LaTeX} tables that can be used in
  the production of a codebook.
\item {\tt 03.03.uniquessn.sas}: Creates a file of unique \aindex{SSN}s
  and, if name information is on the file, a file of unique
  \Mindex{SSN-Name combinations}. These files are used for integrating
  other Census data.
\item {\tt 04.04.readin-sort.sas}: Yearly files are put into the standard
  sort order.
\item {\tt 05.05.pointer.sas}: Creates a \Mindex{pointer file} between the
  unique record identifier \Mindex{RID} and the unique SSN-Name identifier
  \Mindex{UID}.
\end{itemize}                                  %%%


\paragraph{/programs/projects/auxiliary/}

Contains generalized programs (macros):

\begin{itemize}
\item macro\_ui.readin.01.sas
\item macro\_ui.readin.02.sas
\item macro\_ui.analysis.01.sas
\item macro\_ui.analysis.02.sas
\item macro\_ui.sort.01.sas
\item macro\_ui.uniquessn.01.sas
\item macro\_ui.pointer.01.sas
\end{itemize}


\paragraph{/programs/projects/auxiliary/Layouts}
\mindex{layout}

Contains layouts. As of the receipt of North Carolina data, the following
format files have been defined:

\begin{description}
\item Legacy layouts
\begin{itemize}
\item ca.ui.00.01.layout.sas
\item fl.ui.00.01.layout.sas
\item fl.ui.00.02.layout.sas
\item md.ui.00.01.layout.sas
\end{itemize}
\item Standardized layouts
\begin{itemize}
\item ui.00.01.layout.sas
\item ui.00.02.layout.sas
\item ui.00.03.layout.sas
\item ui.00.04.layout.sas
\item ui.00.05.layout.sas
\end{itemize}
\end{description}

\section{Step-by-step instructions}
\label{app:ui_step_by_step}

\begin{description}
\item \ 


  \begin{steps}
  \item If not already in existence, create a state-specific subdirectory
    (or ideally, a separate partition) in /data/master/ui/, using the
    two-letter lowercase \aindex{postal abbreviation} for that state. That
    abbreviation will be referred to elsewhere in this chapter as
    \textit{\&state.}\index{state@\&state.}.
  \item Determine the format of the data. Verify with existing
    \aindex{layout}s. If a layout matches, retain the number (the second
    digit in the layout file name) for use in the relevant macros.
  \item If not already in existence, create data and program subdirectories
    of /data/master/ui/\&state.:

  \begin{itemize}
  \item ./conversions/SSN , which will contain programs that work on the
    original data
  \item ./ssn/rawdata , which will contain links to the original data
    (which is actually stored in ./received)
  \item ./ssn/sasdata , which will contain the standardized raw data
  \item ./conversions/UI , which will contain programs that work on the
    PIKized data
  \item ./\&year/sasdata , where \&year goes from the first observed year
    to the last available year. This will contain the final PIKized UI (and
    ES-202) SAS files.
\end{itemize}
\item If not done yet, create symbolic links in ./ssn/rawdata according to
  conventions as defined in the LEHD \aindex{Data standard} handbook
  (Datastand.pdf), available in /data/doc.
\item If not already in existence, copy the file
  \textit{../XX/conversions/configXX.sas} from a template state directory
  (a previous implementation of the readin programs, preferably the latest,
  where XX stands for some other state abbreviation) to ./conversions.
  Rename it to \textit{config\&state..sas}\index{config\&state..sas} and
  adjust its contents. In particular, adjust the beginning and end years
  and quarters of the data, since this information is used system-wide.
\item If not already in existence, copy the file
  \textit{../XX/conversions/SSN/configXX.sas} from a template state
  directory to ./conversions/SSN. Rename it to \textit{config\&state..sas}
  and adjust macro variables (but not SAS library names) to match the file
  created in the previous step. The only difference between the two files
  should be in the SAS library names!
\item Copy the first five files listed in the above program list (serial
  numbers 01.01-05.05) from  \textit{../XX/conversions/SSN} 
  to ./conversions/SSN. 
\item Adjust the the macro variable \textit{\&state.} in
  each. 
\item Further adjust \textit{01.01.readin.sas}. 
  \begin{itemize}
  \item[Tip:] comment out the line starting with ``\%readui'' by putting a
    ``*'' in front of it, and run the program. The file
    ``macro\_ui.readin.01.sas'' will print out instructions to the log
    file. Then uncomment the line again, and adjust the parameters
    according to the instructions.
\end{itemize}
The salient parameters are:
  \begin{itemize}
  \item \aindex{layout}=aa. This determines the layout used to read in the
    file. This should have been determined earlier. If the layout is the
    standard layout as defined by ui.00.01.layout.sas, then no adjustments
    need be made. For any other variable, add ``layout=aa'' to the end of
    the (comma-separated) parameter list of the macro call ``\%readui''.
  \item freq=xx. This parameter is related to how the data files are
    structured. If there is one file per quarter, then no adjustments need
    to be made. If there is one file per year, then use ``freq=yearly'',
    and if there is one global file, then use ``freq=global''.
  \end{itemize}
\item Further adjust \textit{03.03.uniquessn.sas}. The parameter
  ``name=yes'' needs only be set if the data will be SSN-edited. If not,
  then the default value of ``name=no'' is sufficient, and will save
  significant amounts of disk space.
\item No further adjustments should need to be made to the other 
  files, but double-check.
\item Run all five programs in sequence. Some of the programs can be long
  to run. 
  \begin{itemize}
  \item[Tip:] You can use the (LEHD-specific) command ``\Mindex{mailsas}''
    to have yourself sent a result code when the program has finished
    executing.
  \item[Tip:] Programs 05.05 is optional if the data will not be SSN-edited.
  \end{itemize}
\item Inspect the log and list files after each run, and adjust the
  programs if necessary. These programs have been tested many times, but
  are not guaranteed to be fool-proof.
\item Create a \Mindex{codebook} in /data/doc/latex. To start, run the
  script ``\Mindex{prepare-documentation.ksh}''. The project name should be
  ``ui\&state.''.
  \begin{itemize}
  \item Edit the file templates created by this program using \aindex{Xemacs}, other
    raw text editors, or \aindex{Scientific Word}/Workplace.
  \item[Tip:] When describing the variables, you can use, with minor modifications, the \LaTeX\index{LaTeX@\LaTeX} tables    created by 02.02 program.
  \item Create the Postscript version of the file by ``compiling'' the
    {\LaTeX} document (command: ``latex name\_of\_file[.tex]''). Multiple
    compilation runs may be necessary, and are indicated at the end of each
    compilation run. Do not use Scientific Word for this step.
  \item Create a preliminary PDF document by running ``\aindex{pdflatex}''
    (syntax is othewise the same as above).
  \item When satisfied, use \aindex{Acrobat Distiller} on your PC to
    convert the Postscript version of the file into that final PDF version
    (it is slightly nicer than the one created by pdflatex).
  \end{itemize}
\item Update the \aindex{DAF}.
\item Notify LEHD personnel of the availability of new or modified \aindex{UI}
  files. 
\end{steps}
\end{description}

Further processing depends on the necessity of a SSN edit\index{SSN!edit}:
\begin{itemize}
\item If there is an SSN edit, then the data stay on LEHD systems until the
  end of that processing.
\item If there is no SSN edit, then the data get transferred to \aindex{ARRS}/\aindex{PRED}
  immediately for replacement of the SSN by the \aindex{PIK}.
\end{itemize}


%%% Local Variables: 
%%% mode: latex
%%% TeX-master: "QWI_methods"
%%% End: 

%%%  +--< LEHD-QWI ecf 3.1.49 2005-01-26 schwa305            >--+  %%%
%%%  +--< Location: /programs/production/dev1/current/ecf    >--+  %%%
%%%  +--< File: doc/ECFv31.tex                               >--+  %%%
\subsection{General Overview}

The Employer Characteristics File (ECF) consolidates most firm level
information (size, location, industry, etc.) into two easily accessible
files. The firm or SEIN level file contains one record for every YEAR
QUARTER a firm is present in either the ES-202 or the UI, with more detailed
information available for the establishments of multi-unit firms in the SEIN
SEINUNIT file. The SEIN file is built up from the SEINUNIT file and contains
no additional information, but should be viewed merely as an easier and/or
more efficient way to access firm level data.

{\small
\begin{description}
\item[SEIN level file unique record identifier:] SEIN YEAR QUARTER
\item[SEIN level file sort order:] SEIN YEAR QUARTER
\item[SEIN level file indexes:] SEIN, SEIN{\_}YEAR{\_}QUARTER
\item[SEINUNIT level file unique record identifier:] SEIN SEINUNIT YEAR QUARTER
\item[SEINUNIT level file sort order:] SEIN SEINUNIT YEAR QUARTER
\item[SEINUNIT level file indexes:] SEIN,
  SEIN{\_}SEINUNIT{\_}YEAR{\_}QUARTER, \newline
SEIN{\_}YEAR{\_}QUARTER, SEIN{\_}YEAR{\_}QUARTER{\_}SEINUNIT
\end{description}
}



\subsection{Input Files}

\begin{itemize}
\item The ES202 data from the states is the primary input to the ECF file
creation process. 
%This data is stored in file locations of the format
%/data/master/ui/SS/YYYY/SSesYYYY.sas7bdat (SS= a two letter state
%abbreviation, YYYY= a 4 digit year, all in lower case). Brought in at
%program 01{\_}read{\_}all.sas

\item UI data is also used to supplement information on the ES202. As part of the
creation of the Employment History File (EHF), {\tt ehf\_sein\_employment}
is created.
%a file for each state is
%placed in /data/master/Employer/sasdata/uitotals/sein{\_}employment{\_}%
%SS.sas7bdat. 
This file contains E (end of period employment), B (beginning
of period employment), M (employed anytime in the quarter), and W1 ( total
wages) calculated similarly to the same measures on the QWI. 
%Brought in at 04{\_}sein{\_}totals.sas

% Longitudinal Employer Geography (LEG) file (leg{\_}{\&}st..sas7bdat),
% located in /data/master/geo/LEG. Contains lat/long coordinates of the
% establishments, plus county, wib and pmsa geo also. Brought in at 09{\_}%
% seinunit{\_}wide.sas

\item GAL data containing  lat/long coordinates of the
establishments, plus county, wib and pmsa geo also. 

\item Existing fuzz files must be available if data for the state has been
officially released. 
%Fuzz files( SEIN and SEINUNIT are located in
%/data/master/ECF/fuzz are brought in at 19{\_}seinunit{\_}fuzz.sas

\item SIC and NAICS impute datasets: 
  \begin{enumerate}
  \item  3 digit (parms\_us\_imp\_sic\_sic3, formerly called sic3{\_}impute) and 2 digit 
         (parms\_us\_imp\_sic\_sic2, formerly called sic2{\_}impute.sas7bdat) SIC impute datasets are used in 
          %01{\_}read{\_}all.sas,
         the first modules, 
  \item 6 digit NAICS to 4 digit SIC probabilistic crosswalk 
        (parms\_us\_imp\_ncs1997\_sic, formerly called naics{\_}sic{\_}9,
        and parms\_us\_imp\_sic\_ncs1997) is used 
%in 10{\_}fill{\_}sync.sas
after the merge to the GAL.
  \item NAICS 2002 to 1997 crosswalks are used 
    (parms\_naics\_imp\_ncs1997\_ncs2002 and 
    parms\_naics\_imp\_ncs2002\_ncs1997)
  \end{enumerate}
  

\item Various formats are used: 
  \begin{enumerate}
  \item multiformats for SIC and NAICS are used from
the format library, 
  \item  SIC division formats, 
  \item  EIN checking routine, 
  \item  county formats that vary from state to state. 
  \end{enumerate}
  Format files are all
located in /programs/projects/auxiliary/Formats.

\item BLS-derived control totals, produced by the EHF.
%BLS totals dataset created by Melissa. Melissa's dataset is read-in and
%restructured by a program named blstotals.sas. The data and program reside
%in /programs/projects/ECF/v3.1/inputs/BLS. Brought in at 18{\_}weight{\_}calc%
%{\_}02.sas.
\end{itemize}

% \subsection{Directory Structure}
% 
% Data is located in /data/master/ECF/v3.1 (subdirectories follow)
% 
% Fuzz -- contains the fuzz factor datasets
% 
% Leg{\_}structure -- contains the record structure and assorted other
% variables. These files are output by 08{\_}distribute.sas and are used by
% the LEG and pred/succ.
% 
% Uitotals -- SEIN YEAR QUARTER employment and payroll totals from the job
% flows sequence.
% 
% Uitotals{\_}old -- SEIN YEAR QUARTER employment and payroll totals from the
% creation of the ECF. Used in v1.0 and v2.x. Most v3.0 states use the files
% in the uitotals directory.
% 
% \noindent v1.0 -- My first efforts at building an ECF. Each state has a
% different set of programs depending on the level of development at the time
% the state's data was received. There are many valuable programs that
% investigate ES202 data in the address, duprecs, and multiunits
% subdirectories. The run{\_}expansion subdirectories contain my first
% attempts at version 3.0 (SEINUNIT based file).
% 
% \noindent v2.0 and v2.1 -- My first attempt at combining all the different
% versions of the ECF programs into one consistent set of code that is used
% for all states. This set of code is generally known as version 2.2 (also
% added employment weighted SIC and COUNTY).
% 
% \noindent v3.0 -- Represents the version of the ECF used for the first
% officially released version of the QWI
% 
% \noindent v3.1 -- The current version of the ECF
% 
% Links to v3.1 data files are located in /data/master/ECF/current
% 
% Programs and other files are located in /programs/projects/ECF/
% (subdirectories follow)
% 
% There is a separate subdirectory of the same name (v2.1,v3.0,\ldots .) as
% the directories in /data/master/ECF for the programs that create each
% version of the ECF.
% 
% Within this subdirectory are various other directories.
% 
% Aux -- anciliary programs used to investigate data quality issues
% 
% States -- contains the ECF code for each state
% 
% Inputs -- contains the NAICS and SIC impute datasets along with the BLS
% totals dataset used to create the weights.

\subsection{Program Overview}

First data is read in from the yearly ES202 files and stacked one on top of
the other. General and state specific consistency checks are then performed.
The COUNTY, NAICS, and EIN data are checked for invalid values. The SIC
invalid check is a little more sophisticated. If a 4 digit SIC code is
present, but is not valid, then the SIC code undergoes a conditional impute
based on the first 2 or 3 digits. If the first 2 or 3 digits are not valid
either, then SIC is set to missing (this value will eventually be filled).

The ES202 data contains a ``master'' record for multi-unit firms that must
be removed. Information in the master record is preserved if data is not
available in the establishment records (I initially allocate the data
equally to each establishment). Various inconsistencies in the record
structure are also dealt with, such as 2 records (master and establishment)
appearing for a single-unit.

The UI data is integrated with the ES202 data and totals are calculated at the
SEIN YEAR QUARTER level.

Using both UI and ES202 data I create a ``best'' series of variables for
payroll and employment.

The allocation process implemented above (master to establishments) does not
incorporate any information on the structure of the firm. A flat prior is
used in the allocation process (each establishment is assumed to have equal
employment and payroll). I improve on this by examining firms with allocated
data that previously reported as a multi-unit. I then use the structure of
their reports from a previous quarter to allocate payroll and employment.
The new records are integrated back into the data, hopefully improving
longitudinal consistency at the establishment level.

At this point, the SEIN YEAR QUARTER SEINUNIT dataset record structure is
finalized.
% and available for other jobs such as the LEG to access.
%
%The LEG is brought into the ECF.

The GAL is brought into the ECF (this used to be the separate LEG process).

The COUNTY, SIC, NAICS, and EIN data are transformed from long to wide
format for each SEINUNIT. I use this dataset to fill missing values in these
variables with information from other periods for the same establishment.

The modal COUNTY, SIC, NAICS, OWNER{\_}CODE, and EIN are calculated (both
establishment and employment weighted) for each SEIN in a given YEAR and
QUARTER.

The SEIN level mode variables (SIC, NAICS, etc) are then transformed from
long to wide and the missing values are filled with data from the closest
YEAR and QUARTER, if available.

At this point, if an SEIN mode variable has a missing value, then that
missing value must be present for every YEAR and QUARTER. The distribution
of employment across 4 digit SIC in 1997 is calculated and is used to impute
the industry code for each SEIN with missing SIC. These SIC codes are also
assigned to the SEINUNIT level data.

The weights are calculated, based on the expanded BLS controltotals
acquired from the EHF.

The final step is to apply fuzz factors to each dataset. The fuzz factor
process is done separately for the SEIN and the SEINUNIT data. Once this is
completed the datasets are written to their final location and the master
fuzz files are updated.

% \subsection{User's Guide}
% 
% The program sequence is designed to require the minimum possible interaction
% from the user. All configuration information is stored in config{\_}%
% dates.sas and config{\_}param.sas. If these files are set up properly than
% no further user intervention should be required, except for checking that
% the output looks sensible. Executing runall1.ksh starts the first stage of
% the program sequence. Once this stage has completed, the structure of the
% ECF is fixed and the LEG can be created. When LEG processing is finished
% stage two can be executed using runall2.ksh.
% 
% \subsubsection{Steps to Set-Up Programs for a new state}
% 
% Insure the EHF program sequence completed properly. The
% state-work-history-05.sas and state-work-history-06.sas programs produce
% useful information about the concordance between the UI and the ES202.
% 
% \begin{enumerate}
% \item Create a temporary working directory of your own choosing. Currently
% this is /saswork10/ECF{\_}intermediate/ss. A simple script named create{\_}%
% all.ksh in that directory creates subdirectories for each state. This script
% only needs to be run once if you move the temp directory.
% 
% \item Run the script new{\_}state.ksh with the 2 letter state postal
% abbreviation as the only argument. The script is located in the states
% subdirectory. The script creates a new directory, copies all the required
% files, and adds a line to the file update{\_}file.ksh in the master
% directory. ONLY CREATE A NEW DIRECTORY WITH THIS SCRIPT. If any other method
% is used to setup a state, then code changes made in master will not be
% propagated to each state when the update{\_}all.ksh script is run from the
% master directory.
% 
% \item The file config{\_}dates.sas contains all of the state specific
% configuration information. This file is NOT copied or updated when the update%
% {\_}all.ksh script is run, thus preventing you from accidentally erasing the
% configuration information. However, the file is copied when the new{\_}%
% state.ksh script is run.
% 
% \item Edit config{\_}dates.sas
% 
% \item Set up the macro variables ymin, ymax, st, yfirst, qfirst, ylast, and
% qlast.
% 
% \item Set the macro variable einavail=ein if EIN is available (empty string
% if not available).
% 
% \item Set the macro variable listsize={number}.
% Number should be large enough that at least some records for even small
% categories will show up in print statements (maybe 10-20{\%} of the data).
% 
% \item DO NOT alter the ltrim and utrim unless you are sure you know what you
% are doing.
% 
% \item Set up the following libnames
% 
% \noindent working (temp directory).
% 
% \noindent mastsein (final output for SEIN based file).
% 
% \noindent mastunit (final output for SEINUNIT based file).
% 
% \noindent uidata (location of UI SEIN YEAR QUARTER totals).
% 
% \noindent fuzz (location of master fuzz files for each SEIN/SEINUNIT)
% 
% \noindent emphist (location of the EHF)
% 
% \noindent geoout (location of ECF structure file)
% 
% \noindent geoin (location of the LEG)
% 
% \noindent sicimp (location of 4 digit SIC totals)
% 
% \noindent naicsic (location of NAICS to SIC crosswalk)
% 
% \noindent bls (location of BLS totals dataset)
% 
% The unedited ES202 data directories are automatically assigned a libname as
% long as the standard naming convention is used.
% 
% \item The program uses a format as a filter for invalid county codes. Every
% state has a set of formats in the file
% /programs/projects/auxiliary/Formats/format{\_}ss.sas. A format needs to be
% added to the file named cty{\_}temp. Look at an existing state file for the
% exact layout and make sure to have the value ``OTHER'' in the format so all
% invalid codes are assigned ``ZZZ''. These should be automatically be created
% as part of the GEO process.
% 
% \item If the ES202 read in completed properly, this step should not need to
% be performed, but the instructions are left in `just in case.' The next step
% may require some trial and error to correctly implement. The first program
% ``01{\_}read{\_}all.sas'' reads in the ES202 data, stacks each year file,
% and checks for any inconsistencies. Run this program by itself to check for
% problems before executing the whole sequence. Depending on the structure and
% quality of the data you may have to do some tweaking of config{\_}param.sas
% to get everything to work. Config{\_}param.sas contains the edits for ALL
% states. Unlike config{\_}dates.sas which is different for each state, there
% should be only one config{\_}param.sas. For example, if modifications need
% to be made for North Carolina, do not edit the config{\_}param.sas in North
% Carolina, edit the ``master'' config{\_}param.sas. The ``master'' config{\_}%
% param.sas currently resides in Illinois, but can reside anywhere. Once an
% edit is done in the ``master'' config{\_}param.sas then use update{\_}%
% all.ksh or update{\_}file.ksh to push the changes to ALL the states. If used
% properly, config{\_}param.sas becomes a central repository the
% idiosyncrasies of each state's data.
% 
% Virtually all macros used by the programs are located in config{\_}%
% param.sas. Sets of code specific to each state can be added to the following
% macro's to handle special situations.
% 
% For instance, for North Carolina we add the following modification in config{%
% \_}param.sas file:
% 
% \ \ {\%}if {\&}st=nc {\%}then {\%}do;
% 
% \ \ \ \ \ \ \noindent if trim(naics)=''0'' then naics=''999999'';
% 
% \ \ \ {\%}end;
% 
% Macro SET{\_}ES handles the data read in. If the unedited ES202 files have
% non-standard names for the key variables then they must be renamed
% appropriately. The standard statements that are executed conditional on a
% certain state can be used as a guide.
% 
% Macro CLEANUP removes any invalid values from COUNTY, SIC, NAICS, and EIN.
% Modify as necessary.
% 
% The first two digits of an EIN are compared with a list of known valid IRS
% revenue district codes. This list is currently up to date, but it would be a
% good idea to check periodically to see if new codes have been added.
% 
% Check the output of 01{\_}read{\_}all.sas and if everything talked about
% above looks OK, then check for duplicate records (duplicates should all be
% fixed in the ES202 read-in, but this method can be used if you are in a
% rush). The variable DUPREC identifies duplicate records (SEIN YEAR QUARTER
% SEINUNIT should be unique. If not then something went wrong with the ES202
% read in). Examine the list file and see if there is a common rule that can
% be used to eliminate the duplicates and investigate what went wrong with the
% ES202 read in.
% 
% Macro KEEPREC in config{\_}param.sas is where you would place the code that
% eliminates the duplicates. This step should no longer be needed. All
% duplicates should now be eliminated by the ES202 read in process.
% 
% \item The fuzz parameters are located in config{\_}param.sas, but they
% should never need to be changed. Execute runall1.ksh. The test.ksh script
% searches for errors in the log files. If no errors are found, then examine 08%
% {\_}distribute.lst to see if the data looks OK (make sure that employment
% and payroll totals are consistent across quarters). If everything looks good
% let the GEO team know so the LEG can be created. Once the LEG is ready, the
% second stage can be run using runall2.ksh. Check for errors and if
% everything is OK create new links in /data/master/ECF/current.
% 
% \item The script run{\_}info.ksh will give you an overview of the process
% and how the programs are preceeding. By default, output is sent to the
% screen, likely too fast for the user to assimilate. Instead, execute 
% {\tt run{\_}info.ksh > run{\_}info.ksh.log} to place the output into a file
% for easier viewing.
% 
% \item The script update{\_}links.ksh located in /data/master/ECF/current is
% used to maintain all the links to the ECF files. Create an entry in the
% script for each state. The entry calls the linkset function that creates the
% actual links (see the script for more information on how to call the
% function). Run the script every time you would like any new ECF's to be
% available to other users.
% \end{enumerate}
% 
% None of the intermediate files are deleted during normal processing. Once
% you are satisfied the programs have completed properly the files should be
% removed.

\subsection{SAS Program legacy name mapping}

Please note that program names have been changed since legacy version 3.1.0. Table~\ref{table:namechanges} lists a
correspondence of the old descriptive names to the numeric names in the
production system. The old descriptive names are also coded in the first
line of every program.

\begin{table}[htbp]
  \centering
  \caption{Name conversion between legacy programs, NAICS-adapted, and current programs}
  \label{table:namechanges}
  \begin{tabular}{rclcl}
\\
\hline
Legacy name (3.1.0) && NAICS (3.1.08-kev) && 3.1.15+ \\
\hline
\hline
                01\_read\_all & $\rightarrow$   & 01\_read\_es202               & $\rightarrow$ & 01\_ecf\\
                              & NEW             & 02\_naics\_clean              & $\rightarrow$ & 02\_ecf\\ 
                              & NEW             & 03\_naics\_aux\_clean         & $\rightarrow$ & 03\_ecf\\ 
                              & NEW             & 04\_sic\_clean                & $\rightarrow$ & 04\_ecf\\
                              & NEW             & 05\_merge\_ind                & $\rightarrow$ & 05\_ecf\\
             02\_num\_records & $\rightarrow$   & 06\_num\_records              & $\rightarrow$ & 06\_ecf\\          
               03\_rm\_master & $\rightarrow$   & 07\_rm\_master                & $\rightarrow$ & 07\_ecf\\  
             04\_sein\_totals & $\rightarrow$   & 08\_sein\_totals              & $\rightarrow$ & 08\_ecf\\          
               05\_best\_vars & $\rightarrow$   & 09\_best\_vars                & $\rightarrow$ & 09\_ecf\\  
          06\_select\_records & $\rightarrow$   & 10\_select\_records           & $\rightarrow$ & 10\_ecf\\          
          07\_special\_handle & $\rightarrow$   & 11\_special\_handle           & $\rightarrow$ & 11\_ecf\\          
              08\_distribute  & $\rightarrow$   & 12\_distribute                & $\rightarrow$ & 12\_ecf\\          
                 01\_leg      & $\rightarrow$   & 13\_leg1                      & $\rightarrow$ & 13\_ecf\\          
                 02\_leg      & $\rightarrow$   & 14\_leg2                      & $\rightarrow$ & 14\_ecf\\          
                 03\_leg      & $\rightarrow$   & 16\_leg3                      & $\rightarrow$ & 15\_ecf\\          
                 04\_leg      & $\rightarrow$   & 16\_leg4                      & $\rightarrow$ & 16\_ecf\\          
                              & NEW             & 17\_naics\_ldb\_clean         & $\rightarrow$ & 17\_ecf\\ 
           09\_seinunit\_wide & $\rightarrow$   & 18\_seinunit\_wide            & $\rightarrow$ & 18\_ecf\\          
               10\_fill\_sync & $\rightarrow$   & 19\_seinunit\_fill            & $\rightarrow$ & 19\_ecf\\          
                              & SPLIT           & 20\_seinunit\_sync            & $\rightarrow$ & 20\_ecf\\ 
         11\_sein\_mode\_calc & $\rightarrow$   & 21\_sein\_mode\_calc          & $\rightarrow$ & 21\_ecf\\          
               12\_sein\_wide & $\rightarrow$   & 22\_sein\_wide                & $\rightarrow$ & 22\_ecf\\  
               13\_sein\_fill & $\rightarrow$   & 23\_sein\_fill                & $\rightarrow$ & 23\_ecf\\  
   14\_impute\_sein\_industry & $\rightarrow$   & 24\_impute\_sein\_industry    & $\rightarrow$ & 24\_ecf\\         
15\_impute\_seinunit\_industry& $\rightarrow$   & 25\_impute\_seinunit\_industry& $\rightarrow$ & 25\_ecf\\         
      16\_seinunit\_yq\_chars & $\rightarrow$   & 26\_seinunit\_yq\_chars       & $\rightarrow$ & 26\_ecf\\         
         17\_weight\_calc\_01 & $\rightarrow$   & 27\_weight\_calc\_01          & $\rightarrow$ & 27\_ecf\\          
         18\_weight\_calc\_02 & $\rightarrow$   & 28\_weight\_calc\_02          & $\rightarrow$ & 28\_ecf\\          
           19\_seinunit\_fuzz & $\rightarrow$   & 29\_seinunit\_fuzz            & $\rightarrow$ & 29\_ecf\\          
               20\_sein\_file & $\rightarrow$   & 30\_sein\_file                & $\rightarrow$ & 30\_ecf\\  
             21\_fuzz\_update & $\rightarrow$   & 31\_fuzz\_update              & $\rightarrow$ & 31\_ecf\\         
\\
01\_legqa &&$\rightarrow$ &&01\_ecfqa\\
02\_legqa &&$\rightarrow$ &&02\_ecfqa\\
03\_legqa &&$\rightarrow$ &&03\_ecfqa\\
04\_legqa &&$\rightarrow$ &&04\_ecfqa\\
\hline
  \end{tabular}
\end{table}

% \subsubsection{Config{\_}dates.sas} 
% 
% Configuration parameters that vary by
% state are stored here.
% 
% See the User's Guide for more information.
% 
% \subsubsection{Config{\_}param.sas}
% 
% Macros are stored here.
% 
% This file is placed in a {\%}include statement at the beginning of each
% program below to make available a standard set of macros and macro variables.

% \subsubsection{01{\_}read{\_}all.sas}
% 
% Standardizes and reads in state ES202
% data.
% 
% The first step is to clean up and standardize the data. I allow both state
% specific and general standardization rules. Any state specific data cleanup
% rules are placed in config{\_}param.sas and the generic rules are contained
% in 01{\_}read{\_}all.sas. The idea is that all modifications are made to
% config{\_}param.sas and none of the actual programs should have to be
% modified.
% 
% A large portion of the data quality checking now occurs in the ES202 read in
% before the data ever reaches the ECF. Some of the error checking code could
% probably be removed from the ECF at some point in the future.
% 
% I check the following variables; SIC, NAICS, county, SEIN, and EIN.
% 
% SIC missing and 0000 are coded to 9999
% 
% NAICS missing and 000000 are coded 999999
% 
% County missing, 000, 900,994,995,996,998 are coded to 999
% 
% SEIN is checked for characters other then 0-9. Some SEIN's contain letters
% and the variable sein{\_}bad=1 allows you to determine when that is the case
% (these SEIN's are often related to another SEIN in the data).
% 
% EIN is also checked for characters other than 0-9. Letters and other ASCII
% codes are not allowed for the EIN. If any of these characters are
% encountered then ein{\_}bad=1. I also check to see if the first two digits
% of the EIN correspond with a valid 2 digit IRS Revenue district code. The
% current list of codes is derived from the SSEL and information from the IRS.
% The SSEL contains clean EIN's to the best of my knowledge. EIN's on our
% state data that are a member of the set of 2 digit IRS Revenue district
% codes have a positive probability of matching to the SSEL. This list may
% need to be updated as the IRS adds additional codes over time. I also create
% an ein{\_}defect variable that indicates what problems if any are found with
% the EIN.
% 
% \subsubsection{02{\_}num{\_}records.sas}
% 
%  Count the number of records for each
% SEIN in a given year and quarter. Check whether data is missing for all the
% records of an SEIN in a given year and quarter.
% 
% This program creates many of the measures that will be used later to
% determine the record structure for an SEIN. Unfortunately, the record
% structure of a multi-unit SEIN is not consistent. An seinunit equal to
% ``00000'' is either the only record for a single unit or the master record
% for a multi-unit. The sub-units of an SEIN should be in the range ``00001''
% to ``99999''.
% 
% However, the data we actual receive contains many variations on the above
% theme. The variables created here will allow me to create a consistent data
% structure from the variations on the intended structure that we actually
% receive from the states.
% 
% For the first record of an SEIN I set up initial values for the variables
% and check whether seinunit=''00000''. If seinunit$\sim $=''00000'' then I
% assume that there is no master record and the record represents information
% about an establishment. This distinction is important when creating the all{%
% \_}miss{\_}XX series of variables. These variables are intended to represent
% when all of the data for a given SEIN is missing except for the data in the
% master record. I will handle the data in the master record separately in the
% next program.
% 
% \subsubsection{03{\_}rm{\_}master.sas} 
% 
% Eliminate the master record while
% preserving any information in the subunit records. Create multi-unit
% indicator and establishment counts.
% 
% This program is extremely important since it creates the record structure
% that will be used from this point forward. If this process is not done
% carefully then important information about the SEIN may be destroyed.
% 
% I break the processing tasks up into three distinct groups.
% 
% 1. SEIN's with only one record
% 
% 2. The first record for SEIN's with more than one record
% 
% 3. The second record and beyond for SEIN's with more than one record
% 
% Detailed Explanation
% 
% SEIN's with only one record must be single units. Therefore I set the num{\_}%
% estabs=1 and initialize all other variables. If all{\_}miss{\_}XX=1 then I
% set no{\_}XX=1 ( the subunit is the master record in this case). There is
% also no need to allocate master record information to the subunits.
% 
% For the other cases, initialize all variables and if the master record is
% present then store any information in the master record in the master{\_}XX
% variables. Calculate the number of establishments if there is a master
% record present by subtracting one from the number of records calculated
% previously. If the number of records is equal to two and one of the records
% is a master record then this situation is inconsistent with the defined
% record structure. The number of establishments is set equal to one in this
% case. This situation likely occurs when a multi-unit shrinks and only one
% unit is left. The master record is not removed immediately since it is
% unclear whether the SEIN will remain a single unit or return to multi-unit
% status. If there is information in the master record and there is no
% information in the subunits then set the impute{\_}XX flag=1. If there is no
% information in the master and the subunits then set the no{\_}XX flag=1.
% Finally, if there is no master record then calculate the number of
% establishments as equal to the number of records. If the number of
% establishments is greater than one then set multi{\_}unit=1.
% 
% The second record and beyond are output without modification unless the
% impute{\_}XX flag=1. In this case, information from the master record is
% allocated to the subunits. Since I have no additional information at this
% point about the structure of the SEINUNITS I allocate any payroll or
% employment equally across the subunits. Further along in the sequence I use
% additional information available in years and quarters when the subunits
% report payroll and employment to improve the allocation.
% 
% \subsubsection{04{\_}sein{\_}totals.sas} 
% 
% Create the SEIN level totals,
% indicators for the appearance of wage and employment data, and merge the
% result with the UI SEIN level data.
% 
% In the previous program I created an establishment level dataset. Now I am
% able to sum over the establishment level records and create SEIN totals. I
% also want to check and see if the SEIN was ever active during the time
% period that we observe the firm. This information will be used later when I
% improve my allocation of master record information to the subunits.
% 
% Next, I merge UI SEIN totals with SEIN totals from the ES202 and create
% indicators for each SEIN YEAR QUARTER showing what data is available.
% 
% Finally, I merge the combined UI and ES202 data with a list of the SEIN's that
% appear on the ES202. This allows me to create an indicator showing whether an
% SEIN ever appeared on the ES202. The ES202, not the UI contains information on
% the structure of employment at the subunits, thus it is important to know
% whether that information may be available.
% 
% \subsubsection{05{\_}best{\_}vars.sas} 
% 
% Create a set of best{\_}XXX variables.
% 
% The best{\_}(wages and employment) variables incorporate information from
% both the UI and the ES202. This analysis is done at the SEINUNIT level.
% 
% I examine each record to find out what type of information is available on
% the ES202. I classify a record into one of four possible categories using
% the variable info{\_}202.
% 
% I look at single-unit and multi-unit records separately. For single-units it
% is relatively easy to fill wages and employment with information from the
% UI. For multi-units I do a naive allocation that is improved upon for some
% firms in 07{\_}special{\_}handle.sas.
% 
% UI data is used in the following situations
% 
% ES202 employment is missing, but ES202 payroll is reported, then UI
% employment is used (if ES202 employment is zero, then UI employment is NOT
% used since this may be a correct report and I do not have enough information
% to determine if it is incorrect).
% 
% ES202 payroll is zero and ES202 employment is positive then UI payroll is
% used.
% 
% If ES202 payroll and employment are zero then UI payroll and employment are
% used.
% 
% If both ES202 payroll and employment are missing then UI payroll and
% employment are used.
% 
% \subsubsection{06{\_}select{\_}records.sas} 
% 
% Select the records that have
% enough information available to enable subunit structure to be imputed.
% 
% There are four main classes of records that may benefit from using
% information on subunit structure in other quarters.
% 
% Records that appear only in the UI in a given year and quarter but appear
% previously in the ES202 as a multi unit.
% 
% Records that appear only in the ES202 and master record data was allocated to
% the subunits.
% 
% Records that appear in both the UI and the ES202 but have no information in
% the ES202 and are a multi unit.
% 
% Records that appear in both the UI and the ES202 and master record data was
% allocated to the subunits.
% 
% However, not every case that meets the conditions above will have enough
% prior information to allow an impute to take place. The firm must have been
% a multi-unit prior to the date that a record met one of the conditions
% above. The firm must also have valid data in another quarter than can be
% used for the impute. Only about half of the candidate records contain valid
% data in Illinois.
% 
% All SEINUNIT records selected for imputation are consolidated to SEIN YEAR
% QUARTER records.
% 
% The complete firm history for any SEIN that meets the criteria for special
% handling is kept in a separate dataset and is indexed for easy access
% (special{\_}handle{\_}history{\_}06.sas7bdat).
% 
% \subsubsection{07{\_}special{\_}handle.sas} 
% 
% Look for subunit structure in off
% years and quarters.
% 
% For each SEIN YEAR QUARTER record selected for structure imputation I look
% in both previous and future quarters for a period where the firm reported
% payroll and/or employment for the SEINUNITs. The closest quarter is kept for
% further processing. In Illinois about 80{\%} are found within +/- four
% quarters.
% 
% \subsubsection{08{\_}distribute.sas} 
% 
% Use the record structure identified in
% the previous program to allocate payroll and employment. Interleave the new
% record structure with the existing data.
% 
% The process of imputing the record structure of an SEIN in a given year and
% quarter is completed here. I use SEIN level information on employment and
% payroll from the CURRENT quarter combined with distribution of employment /
% payroll across the SEINUNIT's information learned from off quarter reports
% to allocate payroll and employment to the subunits. I combine the newly
% imputed record structure with the old data to create the final record
% structure.
% 
% NOTE: The best{\_}flag variable when combined with the structure{\_}fix
% variable can be used to identify the type of edits and data source of the
% best{\_}xx variables.
% 
% The record structure file {\&}st.{\_}leg{\_}structure.sas7bdat is created
% here. This file is used to determine the final structure of the LEG and by
% the pred/succ programs.
% 
% \subsubsection{09{\_}seinunit{\_}wide.sas} 
% 
% Read in the geography data and
% transform the data from long into wide format.
% 
% Geography data is brought into the ECF program sequence and checked to
% ensure the proper record structure is returned. Each GEO variable is also
% output into a separate file so that the mode value can be calculated in step
% 11.
% 
% Filling missing values requires a transformation of the data into wide
% format (one record for each SEIN SEINUNIT with the YEAR QUARTER values for a
% variable stored in arrays). The wide file is used in later programs to make
% available the complete history for each SEIN SEINUNIT (used to fill missing
% values).
% 
% The following variables are transformed into wide format; SIC, NAICS,
% county, ownership code, and EIN.
% 
% \subsubsection{10{\_}fill{\_}sync.sas} 
% 
% Missing values are filled and NAICS
% values are used to impute SIC.
% 
% The variables in program 09 that were transformed into wide format are
% filled, if missing, with the closest value from another YEAR QUARTER.
% 
% If SIC is missing and NAICS is available then impute the SIC value using the
% NAICS code. A probability based impute is used to assign the SIC if a NAICS
% code maps to multiple SIC codes.
% 
% \subsubsection{11{\_}sein{\_}mode{\_}calc.sas} 
% 
% Calculate the modal SIC,
% NAICS, and county for each SEIN YEAR QUARTER.
% 
% In programs 09 and 10, each variable that will have the mode calculated was
% placed into a separate fill along with a measure of employment. The
% resulting datasets are relatively narrow and thus less costly to sort then
% the complete dataset (each dataset or variable requires a different sort
% order).
% 
% Using SIC as an example, the data is first sorted by SEIN YEAR QUARTER SIC.
% The number of SEINUNIT's or employment is calculated for each set of
% subunits that are in the same industry. The SIC code with the most
% SEINUNIT's or EMPLOYMENT is the mode.
% 
% The unit and employment weighted modal values are calculated over a
% different universe since not every SEINUNIT in the ES202 has positive
% employment. Thus, the unit weighted mode is still available even when the
% firm is inactive.
% 
% The modal values for each SEIN YEAR QUARTER are merged together to create
% one dataset containing all the mode variables.
% 
% \subsubsection{12{\_}sein{\_}wide.sas} 
% 
% Transform the modes calculated in
% program 11 into wide format.
% 
% Filling any remaining missing values requires a transformation of the data
% into wide format (one record for each SEIN with the YEAR QUARTER values for
% a variable stored in arrays). The wide file is then merged by SEIN with the
% data in the standard ``long'' format. The complete history for each SEIN
% SEINUNIT is then available to fill any missing values.
% 
% The following variables are transformed into wide format; mode{\_}es{\_}sic,
% mode{\_}es{\_}naics, mode{\_}es{\_}county, mode{\_}es{\_}owner{\_}code, mode{%
% \_}es{\_}ein, mode{\_}leg{\_}wib, mode{\_}leg{\_}msapmsa, mode{\_}leg{\_}%
% state, mode{\_}leg{\_}county, mode{\_}leg{\_}subctygeo. There are actually
% two variables for each variable in the previous list, one for the unit
% weighted mode and an additional one for the employment weighted mode.
% 
% \subsubsection{13{\_}sein{\_}fill.sas} 
% 
% Fill missing values in the SEIN level
% mode variables.
% 
% The variables in program 12 that were transformed into wide format are
% filled, if missing, with the closest value from another YEAR QUARTER.
% 
% The same logic is used in this program as in program 10. Missing values need
% to be filled for the GEO variables at the SEIN level even though they
% contain no missing values at the SEINUNIT level. For quarters where no
% SEINUNITS are active, the employment weighted mode will be missing. This
% value is filled, if available, with information from another quarter.
% 
% \subsubsection{14{\_}impute{\_}sein{\_}industry.sas} 
% 
% Fill missing SEIN values
% with the unit weighted mode if available. SIC, if still missing is imputed.
% 
% Missing SEIN YEAR QUARTER values for SIC, NAICS, county, ownership code, and
% EIN are filled with the unit weighted mode.
% 
% SIC, must not have any missing values for QWI processing to proceed. Any SIC
% missing values that remain are for firms that never reported either a NAICS
% or SIC code over the entire time period. Since we have little information,
% the distribution of employment across 4 digit SIC is used. A random number
% is drawn that determines the actual imputed SIC. SIC codes with a larger
% share of employment are more at risk for assignment than those with a
% relatively small share. The SIC is imputed once for each SEIN and thus
% remains constant across time.
% 
% \subsubsection{15{\_}impute{\_}seinunit{\_}industry.sas} 
% 
% Fill the missing
% SEINUNIT records using the SEIN values.
% 
% This is a simple program that fills the SEINUNIT missing values with the
% SEIN mode. Except for SIC, some missing values will still remain for firms
% that never reported any information for that variable.
% 
% \subsubsection{16{\_}seinunit{\_}yq{\_}chars.sas} Bring all the data
% together, apply variable labels, and perform some minor clean-up.
% 
% Programs 09 through 16 have been focused on filling industry and location
% missing values. The results of this effort are merged back onto the
% employment and earnings variables from step 08.
% 
% Variable lengths are set to minimize storage space
% 
% Variable labels are defined.
% 
% Create new aggregated industry classifications based on the detailed 4 digit
% SIC and 6 digit NAICS
% 
% Perform final clean up on ownership code and some of the geography variables.
% 
% \subsubsection{17{\_}weight{\_}calc{\_}01.sas} 
% 
% Creation of the weights, step
% 1.
% 
% First, the sample over which the weights will be positive (and therefore
% which units will be included in any calculations) is determined. If a unit
% has ever appeared on the ES202 and is in the UI then it is eligible for a
% positive weight.
% 
% Additional restrictions are imposed for the actual calculation of the
% overall adjustment. The overall adjustment factor is calculated for private
% firms (es{\_}owner{\_}code=5 and the first digit of es{\_}sic not equal to
% 9). This adjustment is later applied to public firms as well.
% 
% The second component of the weights is based on the difference between UI
% and ES202 employment at the SEIN level. These differences are calculated and
% trimmed if necessary. The bounds for the trim at the time of this documents
% creation are .7 and 1.3.
% 
% See the technical paper on the weights for more information.
% 
% \subsubsection{18{\_}weight{\_}calc{\_}02.sas} 
% 
% Creation of the weights, step
% 2.
% 
% Calculate the UI totals necessary to create the overall adjustment.
% 
% The UI totals are combined with the BLS totals and the SEIN level adjustment
% to create the final weight.
% 
% See the technical paper on the weights for more information.
% 
% \subsubsection{19{\_}seinunit{\_}fuzz.sas} 
% 
% Attach fuzz factors to the ECF.
% 
% An SEIN and SEINUNIT has a constant fuzz factor over time. The two are
% related such that if the SEIN fuzz factor is less than one then all SEINUNIT
% fuzz factors for that firm are also less then one. A fuzz factor, once
% assigned, never changes.
% 
% When applying fuzz factors, if the SEIN SEINUNIT was previously part of the
% ECF then the existing fuzz factor is used. If not, then the fuzz factor is
% calculated using both the triangle (ramp) and beta distributions and stored
% in a temporary file for use by program 21.
% 
% The fuzz factor for each SEIN and SEIN SEINUNIT are stored in separate files
% in the fuzz directory and should NEVER be removed. The first time this
% program is run for a state a new empty file is created. This file will be
% filled in program 21.
% 
% \subsubsection{20{\_}sein{\_}file.sas} 
% 
% Output an SEIN level file.
% 
% Some data users only require information at the firm level. To make data
% access easier, a subset of the variables that do not vary within an SEIN
% YEAR QUARTER are output to a separate file.
% 
% \subsubsection{21{\_}fuzz{\_}update.sas} 
% 
% Update the master fuzz factor
% dataset.
% 
% The first step is to calculate the largest update number on the fuzz factor
% dataset. The new additions to the fuzz file receive an update number one
% larger than the current max value, unless the file is empty and the number
% for all records is set to 0.
% 
% In addition to the update number, the date the fuzz factor was created is
% added to the master fuzz file. These two pieces allow for the auditing and
% tracking of all additions to the master fuzz file.
% 
% The new fuzz factors are merged onto the master fuzz factor file. If a new
% fuzz factor has the same SEIN or SEIN SEINUNIT as an existing record then
% processing is immediately halted, thus preventing any pollution of the
% existing fuzz factors.

%\subsection{Scripts}
%
%\subsubsection{new{\_}state.ksh}
%
%Sets up the run directory for a new state. Give the 2 letter state postal
%abbreviation in lower case as the only argument. All necessary files and
%scripts are copied to the appropriate directory in ECF/v3.1/states/ss. THIS
%SCRIPT IS THE ONLY WAY TO SET UP A DIRECTORY FOR A NEW STATE. An entry in
%update{\_}file.ksh that insures any code changes made to master are
%correctly pushed down to the state run directories. The config{\_}dates.sas
%file must still be edited before the ECF programs can be run.
%
%\subsubsection{runall.ksh}
%
%Executes all the SAS programs necessary to create the ECF. Temp files are
%NOT cleaned up and links are NOT created in /data/master/ECF/current.
%
%\subsubsection{runall1.ksh}
%
%Executes stage 1 of the ECF program sequence.
%
%\subsubsection{runall2.ksh}
%
%Executes stage 2 of the ECF program sequence.
%
%\subsubsection{size.ksh}
%
%Reports the size in lines, words, and bytes of each program file. Useful for
%determining if a program has been modified.
%
%\subsubsection{test.ksh}
%
%Reports any errors in the log files
%
%\subsubsection{run{\_}info.ksh}
%
%Produces a summary report of each program's run results. This script should
%be run after the SAS programs have completed. All output is sent to the
%screen by default. This will likely scroll by too fast for easy viewing so
%redirect the output to a file (run{\_}info.ksh > run{\_}info.ksh.log).
%
%\subsubsection{update{\_}all.ksh}
%
%Copies all the SAS program files in states/master automatically to each
%state directory. If a code change is necessary, this script can easily
%update the code in each state's run directory.
%
%\subsubsection{update{\_}file.ksh}
%
%Called by update{\_}all.ksh. The only required argument is the name of the
%file to update. Copies the requested file from /states/master to each state
%subdirectory. The list of state directories is kept current by the use of
%the new{\_}state.ksh script.
%
%\subsubsection{update{\_}links.ksh}
%
%Used to create the links in /data/master/ECF/current. The program first
%removes ALL existing links. New links are created and a contents file is
%created for each file link. A call to the function linkset must be present
%in the script for each state. The function requires two arguments. The name
%of the directory within /data/master/ECF where the ECF is located and the
%name of the directory where the fuzz files are located.
%
%\subsubsection{update{\_}contents.ksh}
%
%Called by update{\_}links.ksh. Requires one argument, the data file name. A
%proc contents is run for each file along with a short proc print of the
%first 50 observations.

\subsection{Files Created}

Most temporary SAS datasets have the number of the program that created them
in the name. Some final output datasets were renamed, see
Table~\ref{table:datachanges}.

\begin{table}[htbp]
  \centering
\caption{Changed dataset names}  \label{table:datachanges}
  \begin{tabular}{rcl}
\\
\hline
sic3\_impute                             & $\rightarrow$  & parms\_us\_sic3impute               \\
                                         & $\rightarrow$  & parms\_us\_imp\_sic\_sic3               \\
sic2\_impute                             & $\rightarrow$  & parms\_us\_sic2impute               \\
                                         & $\rightarrow$  & parms\_us\_imp\_sic\_sic2               \\
naics\_sic\_9                             & $\rightarrow$  & parms\_us\_naics\_sic                \\
                                          & $\rightarrow$  & parms\_us\_imp\_ncs1997\_sic                \\
INPUTS.\&state.\_master\_sein\_fuzz         & $\rightarrow$  & INPUTS.ecf\_\&state.\_fuzz\_sein      \\
INPUTS.\&state.\_master\_seinunit\_fuzz     & $\rightarrow$  & INPUTS.ecf\_\&state.\_fuzz\_seinunit  \\
OUTPUTS.\&state.\_master\_sein\_fuzz        & $\rightarrow$  & OUTPUTS.ecf\_\&state.\_fuzz\_sein     \\
OUTPUTS.\&state.\_master\_seinunit\_fuzz    & $\rightarrow$  & OUTPUTS.ecf\_\&state.\_fuzz\_seinunit \\
OUTPUTS.ecf\_seinunit\_\&state.            & $\rightarrow$  & OUTPUTS.ecf\_\&state.\_seinunit      \\
OUTPUTS.ecf\_sein\_\&state.                & $\rightarrow$  & OUTPUTS.ecf\_\&state.\_sein          \\
\&state.\_leg\_structure                   & $\rightarrow$  & OUTPUTS.ecf\_\&state.\_leg\_structure \\
                                          & $\rightarrow$  & eliminated (3.1.12)\\
leg\_\&state.                             & $\rightarrow$  & OUTPUTS.ecf\_\&state.\_leg                   \\
xwlk\_e\&year                             & $\rightarrow$  & gal\_\&state.\_xwlk\_e\&year           \\
gal\_us\_subcty                           & $\rightarrow$  & parms\_us\_wib\_subcty                     \\         
\hline
  \end{tabular}

\end{table}


\subsubsection{ecf{\_}ST{\_}seinunit.sas7bdat}

Created by 19{\_}seinunit{\_}fuzz.sas

The final version of the SEIN SEINUNIT YEAR QUARTER ECF.

\subsubsection{ecf{\_}ST{\_}sein.sas7bdat}

Created by 20{\_}sein{\_}file.sas

The final version of the SEIN YEAR QUARTER ECF.

\subsubsection{ecf{\_}ST{\_}fuzz{\_}sein.sas7bdat}

Created by 21{\_}fuzz{\_}update.sas

If the SEIN file does not exist it is created, otherwise the file is updated
through a merge with new{\_}sein{\_}fuzz{\_}19.sas7bdat.

\subsubsection{ecf{\_}ST{\_}fuzz{\_}seinunit.sas7bdat}

Created by 21{\_}fuzz{\_}update.sas

If the SEIN SEINUNIT file does not exist it is created, otherwise the file
is updated through a merge with new{\_}seinunit{\_}fuzz{\_}19.sas7bdat.

\subsubsection{ecf{\_}ST{\_}leg.sas7bdat}

The old LEG. Useful for research.

\subsection{Details on intermediate files}

\subsubsection{ecf{\_}stacked{\_}01.sas7bdat}

Stacked data from the ES202. Contains master and subunits

\subsubsection{count{\_}data{\_}02.sas7bdat}

\noindent adds number of record counts and data availability flags to ecf{\_}%
stacked{\_}01.sas7bdat

\subsubsection{no{\_}master{\_}03.sas7bdat}

Removes the master record. File is now in the form SEIN YEAR QUARTER SEINUIT

\subsubsection{sein{\_}totals{\_}04.sas7bdat}

SEIN YEAR QUARTER totals from the ES202

\subsubsection{sein{\_}list{\_}ui{\_}04.sas7bdat}

A list of the SEIN's that ever appear on the UI

\subsubsection{sein{\_}list{\_}202{\_}04.sas7bdat}

A list of the SEIn's that ever appear on the ES202

\subsubsection{seinunit{\_}202{\_}UI{\_}04.sas7bdat}

SEIN YEAR QUARTER SEINUNIT file containing payroll and employment totals
from both the UI and ES202. If an observation is only from the UI then there
is no SEINUNIT information.

\subsubsection{best{\_}vars{\_}05.sas7bdat}

Improved payroll and employment measures at the SEIN YEAR QUARTER SEINUNIT
level.

\subsubsection{special{\_}handle{\_}list{\_}06.sas7bdat}

SEIN YEAR QUARTER records that will have SEINUNIT record structure imputed

\subsubsection{alldata{\_}06.sas7bdat}

A copy of best{\_}vars{\_}05.sas7bdat with several new variables added.

\subsubsection{special{\_}handle{\_}history{\_}06.sas7bdat}

SEIN YEAR QUARTER SEINUNIT dataset for each SEIN that has a record in special%
{\_}handle{\_}list{\_}06.sas7bdat. The complete history of the SEIN is
included in this dataset.

\subsubsection{special{\_}handle{\_}07.sas7bdat}

The new record structure created for records in special{\_}handle{\_}list{\_}%
06.sas7bdat by using information from another quarter. The number of SEIN
YEAR QUARTER records is unchanged.

\subsubsection{ST{\_}employer{\_}char{\_}unit.sas7bdat}

Created by 08{\_}distribute.sas

The structure of the ECF is set when this file is created. The employment
and payroll variables have been completed.

% \subsubsection{ST{\_}leg{\_}structure.sas7bdat}
% 
% \noindent created by 08{\_}distribute.sas
% 
% The ECF structure file used by the LEG and pred/succ.

\subsubsection{ST{\_}employer{\_}char{\_}unit2.sas7bdat}

Created by 09{\_}seinunit{\_}wide.sas

The geography variables have been brought in from the LEG.

\subsubsection{seinunit{\_}county{\_}09.sas7bdat}

SEIN SEINUNIT YEAR QUARTER file with LEG county and employment variables.
Used to calculate the mode.

\subsubsection{seinunit{\_}msapmsa{\_}09.sas7bdat}

SEIN SEINUNIT YEAR QUARTER file with LEG msapmsa and employment variables.
Used to calculate the mode.

\subsubsection{seinunit{\_}state{\_}09.sas7bdat}

SEIN SEINUNIT YEAR QUARTER file with LEG state and employment variables.
Used to calculate the mode.

\subsubsection{seinunit{\_}subctygeo{\_}09.sas7bdat}

SEIN SEINUNIT YEAR QUARTER file with LEG subctygeo and employment variables.
Used to calculate the mode.

\subsubsection{seinunit{\_}wib{\_}09.sas7bdat}

SEIN SEINUNIT YEAR QUARTER file with LEG wib codes and employment variables.
Used to calculate the mode.

\subsubsection{seinunit{\_}wide{\_}09.sas7bdat}

SEIN SEINUNIT file with SIC, NAICS, county, ownership code, and EIN values
placed in arrays. Used to fill missing values.

\subsubsection{seinunit{\_}long{\_}09.sas7bdat}

SEIN SEINUNIT YEAR QUARTER file containing the general structure of the data
and employment variables. Used in program 10 to rebuild the structure from
the wide file.

\subsubsection{seinunit{\_}county{\_}10.sas7bdat}

SEIN SEINUNIT YEAR QUARTER file with county codes and employment variables.
Used to calculate the mode.

\subsubsection{seinunit{\_}ein{\_}10.sas7bdat}

SEIN SEINUNIT YEAR QUARTER file with the EIN and employment variables. Used
to calculate the mode.

\subsubsection{seinunit{\_}naics{\_}10.sas7bdat}

SEIN SEINUNIT YEAR QUARTER file with NAICS codes and employment variables.
Used to calculate the mode.

\subsubsection{seinunit{\_}owner{\_}code{\_}10.sas7bdat}

SEIN SEINUNIT YEAR QUARTER file with ownership codes and employment
variables. Used to calculate the mode.

\subsubsection{seinunit{\_}sic{\_}10.sas7bdat}

SEIN SEINUNIT YEAR QUARTER file with SIC codes and employment variables.
Used to calculate the mode.

\subsubsection{seinunit{\_}long{\_}10.sas7bdat}

SEIN SEINUNIT YEAR QUARTER file containing the missing filled data for SIC,
NAICS, county, ownership code, and EIN.

\subsubsection{sein{\_}modes{\_}11.sas7bdat}

SEIN YEAR QUARTER file containing the modal values for SIC, NAICS, county,
ownership code, EIN, wib, msapmsa, LEG state, LEG county, and subctygeo.

\subsubsection{sein{\_}wide{\_}12.sas7bdat}

SEIN file containing the mode variable values in arrays.

\subsubsection{sein{\_}long{\_}12.sas7bdat}

SEIN YEAR QUARTER file containing the structure of the data.

\subsubsection{sein{\_}missing{\_}13.sas7bdat}

File containing the list of SEIN's with missing values for SIC.

\subsubsection{sein{\_}mode{\_}fill{\_}13.sas7bdat}

SEIN YEAR QUARTER file containing the mode variables with filled missing
values.

\subsubsection{sein{\_}missing{\_}14.sas7bdat}

Same record count as sein{\_}missing{\_}13.sas7bdat

SEIN file containing imputed SIC.

\subsubsection{sein{\_}long{\_}14.sas7bdat}

SEIN YEAR QUARTER file containing missing filled mode variables.

\subsubsection{sicfreq{\_}14.sas7bdat}

File containing the employment in each 4 digit SIC

\subsubsection{sicfreq2{\_}14.sas7bdat}

Slightly transformed version of sicfreq{\_}14.sas7bdat. Contains the CDF of
employment by SIC.

\subsubsection{seinunit{\_}long{\_}15.sas7bdat}

SEIN YEAR QUARTER SEINUNIT file containing the missing value cleaned SIC,
NAICS, county, ownership code, and EIN.

\subsubsection{il{\_}employer{\_}char{\_}unit3.sas7bdat}

Created by 16{\_}seinunit{\_}yq{\_}chars.sas

SEIN YEAR QUARTER SEINUNIT file containing all variables.

\subsubsection{weights{\_}sein{\_}17.sas7bdat}

YEAR QUARTER SEIN file containing the results of the first stage of the
weights process.

\subsubsection{weight{\_}totals{\_}ss{\_}18.sas7bdat}

YEAR QUARTER summary file containing various totals necessary for the
weights calculation.

\subsubsection{weights{\_}sein{\_}18.sas7bdat}

SEIN YEAR QUARTER file containing the final weights

\subsubsection{il{\_}employer{\_}char{\_}unit4.sas7bdat}

Created by 18{\_}weights{\_}calc{\_}02.sas

SEIN SEINUNIT YEAR QUARTER file containing all variables plus the new
weights.

\subsubsection{new{\_}sein{\_}fuzz{\_}19.sas7bdat}

SEIN file containing the new fuzz factors

\subsubsection{new{\_}seinunit{\_}fuzz{\_}19.sas7bdat}

SEIN SEINUNIT file containing the new fuzz factors


%
%------------------------------ VARIABLES ------------------------------
%

\subsection{Variables Created}
Please note that program names in this section refer to the programs as
present in the legacy version 3.1.0. Table~\ref{table:namechanges} lists a
correspondence of the old descriptive names to the numeric names in the
production system.

NOTE to LARS: this should be replaced/augmented by PROC CONTENTS of the
actual files. See the data codebook generating programs.

\subsubsection{01{\_}read{\_}all.sas}

\paragraph{sein}

Variables read in from the ES202 yearly files.

12 digit firm identifier (first 2 digits are the state FIPS code)

\paragraph{year}

\paragraph{quarter}

\paragraph{seinunit}

5 digit code identifying the establishment. Generally used in combination
with the SEIN to uniquely identify an establihment. The identifier itself is
only unique within a firm or SEIN.

\paragraph{owner{\_}code}

\noindent see ES{\_}OWNER{\_}CODE

\paragraph{EIN}

\paragraph{county}

\paragraph{SIC}

\paragraph{NAICS}

\paragraph{empl{\_}month1}

\paragraph{empl{\_}month2}

\paragraph{empl{\_}month3}

\paragraph{total{\_}wages}

End of variables read in from the ES202 yearly files.

\paragraph{Sein{\_}bad}

0 = SEIN contains only characters 0-9

1 = SEIN contains a character outside the above range

\paragraph{Ein{\_}bad}

0 = EIN contains only characters 0-9

1 = EIN contains a character outside the above range

\paragraph{Valid{\_}ein}

0 = first 2 digits of EIN do not represent a valid IRS Revenue district code

1 = first 2 digits are valid

\paragraph{Ein{\_}defect}

0 = no defect found

1 = EIN it is all nines or all zeros

2 = ein{\_}bad=1, EIN contains characters outside the range 0-9

3 = EIN is a 7 digit or less number. An EIN must be at least eight characters

4 = valid{\_}ein=0, the first two digits of the EIN do not represent a valid
IRS Revenue district code

\paragraph{Sic{\_}invalid}

0 = SIC is OK

1 = SIC not valid

2 = first 2 digits valid, last 2 digits imputed

3 = first 3 digits valid, last digit imputed

\subsubsection{02{\_}num{\_}records.sas}

\paragraph{NUM{\_}RECORDS}

1-N = the number of records for each SEIN in a given year and quarter

\paragraph{All{\_}miss{\_}(pay,emp1,emp2,emp3,sic,county)}

0 = at least one or more subunits has data

1 = all subunits have missing data

\subsubsection{03{\_}rm{\_}master.sas}

\paragraph{num{\_}estabs}

1-N = the number of establishments for each SEIN in a given year and quarter

\paragraph{multi{\_}unit}

0 = not a multi unit

1 = multi unit

\paragraph{impute{\_}(wage,emp1,emp2,emp3,sic,county)}

0 = data not available or imputation unnecessary

1 = data available in master record and no data in subunits

\paragraph{no{\_}(wages,emp1,emp2,emp3,sic,county)}

0 = data available in either master record or subunits

1 = no data in either master record or subunits

\paragraph{master{\_}(wage,emp1,emp2,emp3,sic,county)}

Information contained in the master record is stored here

\paragraph{seinunit{\_}type}

0 = seinunit=''00000''

1 = seinunit$\sim $=''00000''

\subsubsection{04{\_}sein{\_}totals.sas}

\paragraph{seinsize{\_}m}

\noindent variables read in from the UI SEIN YEAR QUARTER summary file.

Count of PIK level wage records that appear at an SEIN in a given YEAR
QUARTER.

\paragraph{seinsize{\_}b}

Count of PIK level wage records that appear at an SEIN in both the current
and previous YEAR QUARTER

\paragraph{seinsize{\_}e}

Count of PIK level wage records that appear at an SEIN in both the current
and subsequent YEAR QUARTER.

\paragraph{Payroll}

Sum of earnings for PIK level wage records at the SEIN in a given YEAR
QUARTER.

\paragraph{ever{\_}(multi,wages,emp1,emp2,emp3)}

0 = the SEIN never reports data on the ES202

1 = the SEIN is a multi unit at some time or reports payroll or employment
at some time during the observed period on the ES202.

\paragraph{sein{\_}(emp1,emp2,emp3,wages)}

SEIN level totals for payroll and employment from the ES202

\paragraph{multi{\_}first{\_}year}

The first year when an SEIN appears as a multi unit on the ES202

\paragraph{multi{\_}first{\_}quarter}

The first quarter when an SEIN appears as a multi unit on the ES202

\paragraph{in{\_}UI}

0 = SEIN is not on the UI in a given year and quarter

1 = SEIN appears on the UI in given year and quarter

\paragraph{in{\_}202}

0 = SEIN is not on the ES202 a given year and quarter

1 = SEIN appears on the ES202 in a given year and quarter

\paragraph{source}

1 = UI only

2 = ES202 only

3 = both UI and ES202

\paragraph{ever{\_}202}

0 = not on ES202

1 = SEIN appears on the ES202 at some time during observed period

\paragraph{yr{\_}qtr}

A 6 character sequential year variable. Format is YYYY:Q. A 4 digit year, a
colon, and a 1 digit quarter.

\subsubsection{05{\_}best{\_}vars.sas}

\paragraph{emp(1,2,3){\_}UI}

Attempt to create the best possible approximation of ES202 employment and
payroll using UI data.

Emp1{\_}UI = seinsize{\_}b if available, then seinsize{\_}e, and finally
seinsize{\_}m.

Emp2{\_}UI = seinsize{\_}b if available, then seinsize{\_}e, and finally
seinsize{\_}m.

Emp1{\_}UI = seinsize{\_}e if available, then seinsize{\_}b, and finally
seinsize{\_}m.

\paragraph{best{\_}(wages,emp1,emp2,emp3)}

My best estimate of payroll and employment for a subunit using as much
information available in the UI and ES202. I use both contemporaneous
information and information about the firm in other years and quarters. If
information is available in the ES202 then that data takes precedence over
information in the UI.

\paragraph{best{\_}flag}

NOTE: The best{\_}flag variable when combined with the structure{\_}fix
variable can be used to identify the type of edits and data source of the
best{\_}xx variables.

0 = no wage or employment information on the ES202 or UI

1 = SU, ES202 wages, but ES202 employment is zero

2 = SU, ES202 wages , but ES202 employment is missing

3 = SU, no ES202 wages, but ES202 employment is available

4 = SU, ES202 wages and employment is greater than zero

5 = SU, no ES202 wages or employment, UI available

6 = SU, not in ES202 and UI available

7 = MU, ES202 wages, but ES202 employment is zero

8 = MU, ES202 wages , but ES202 employment is missing

9 = MU, no ES202 wages, but ES202 employment is available

10 = MU, ES202 wages and employment is greater than zero

11 = MU, no ES202 wages or employment, UI available

\paragraph{info{\_}202}

0 = no wages and employment on ES202

1 = wages and no employment on ES202

2 = wages and no employment on ES202

3 = wages and employment on ES202

\paragraph{noemp{\_}202}

0 = positive ES202 employment

1 = employment is not >0 on the ES202

\paragraph{emp{\_}202{\_}miss}

0 = not in the ES202 and non-missing ES202 employment

1 = in the ES202 and all ES202 employment is missing.

\subsubsection{06{\_}select{\_}records.sas}

\paragraph{special{\_}handle}

0 = no special handling required

1 = in{\_}UI=1 and in{\_}202=0 and ever{\_}multi=1

2 = in{\_}UI=0 and in{\_}202=1 and impute{\_}data=1

3 = in{\_}UI=1 and in{\_}202=1 and no{\_}data=1 and multi{\_}unit=1

4 = in{\_}UI=1 and in{\_}202=1 and impute{\_}data=1

\paragraph{no{\_}get{\_}data}

0 = get{\_}XX=1 for at least one variable

1 = get{\_}XX=0 for all variables

\paragraph{data{\_}avail}

0 = no data available

1 = in{\_}202=1 and some subunit data available that period

\paragraph{impute{\_}data}

0 = no allocation of master to subunit that period

1 = allocation of master to subunit that period

\paragraph{no{\_}data}

0 = data available

1 = no data in master or subunit available that period

\paragraph{get{\_}(wages,emp1,emp2,emp3)}

0 = special{\_}handle=0 or special{\_}handle=1 and no subunit wages
available in other periods

1 = special{\_}handle>0 and subunit data is available in other periods

\paragraph{(wages,emp1,emp2,emp3){\_}202}

Renamed sein{\_}XX variables on the special{\_}handle{\_}06.sas7bdat
dataset. This is necessary in the next program when I match a record with
missing subunit information the to another record in another year and
quarter.

\paragraph{Wages{\_}UI}

Payroll is renamed similarly to emp(1,2,3,){\_}UI variables.

\subsubsection{07{\_}special{\_}handle.sas}

\paragraph{qtime{\_}master}

Continuous quarter time from1985 quarter 1 for the record for which I am
trying to determine subunit structure.

\paragraph{qtime{\_}first}

The first quarter in continuous time that an SEIN appears as a multi unit

\paragraph{year{\_}found}

The closest year that contains subunit structure

\paragraph{quarter{\_}found}

The closest quarter that contains subunit structure

\paragraph{Stop}

0 = record not found

1 = record with subunit structure found

\subsubsection{08{\_}distribute.sas}

\paragraph{best{\_}(wages,emp1,emp2,emp3)}

Update of original values computed in 05{\_}best{\_}vars.sas. My best
estimate of payroll and employment for a subunit using as much information
available in the UI and ES202. I use both contemporaneous information and
information about the firm in other years and quarters. If information is
available in the ES202 then that data takes precedence over information in the
UI.

\paragraph{sein{\_}best{\_}(wages, emp1, emp2, emp3)}

SEIN YEAR QUARTER summaries of the best{\_}XX variables.

\paragraph{structure{\_}fix}

NOTE: The best{\_}flag variable when combined with the structure{\_}fix
variable can be used to identify the type of edits and data source of the
best{\_}xx variables.

0 = record not selected for structure imputation

1 = record selected for structure imputation

\subsubsection{09{\_}seinunit{\_}wide.sas}

\paragraph{leg{\_}state}

See the LEG documentation for more information on these variables

\paragraph{leg{\_}county}

\paragraph{leg{\_}wib}

\paragraph{leg{\_}msapmsa}

\paragraph{leg{\_}geo{\_}qual}

\paragraph{leg{\_}longitude}

\paragraph{leg{\_}latitude}

\paragraph{leg{\_}flag{\_}geo}

\paragraph{es{\_}state}

FIPS code of the state

\subsubsection{10{\_}fill{\_}sync.sas}

\paragraph{es{\_}ein}

\noindent cleaned SEINUNIT EIN

9 digit federal firm identifier. Generally not unique within a state. There
may be multiple state level firms for a given federal firm identifier.

\paragraph{es{\_}county}

\noindent cleaned SEINUNIT county

3 digit FIPS county code.

\paragraph{es{\_}naics}

\noindent cleaned SEINUNIT NAICS

\paragraph{es{\_}owner{\_}code}

\noindent cleaned SEINUNIT ownership code

1 = Federal Government

2 = State Government

3 = Local Government

5 = Private Sector

\paragraph{es{\_}sic}

\noindent cleaned SEINUNIT SIC

\paragraph{es{\_}(sic, naics, county, owner{\_}code, ein){\_}miss}

0 = Variable is not missing

1 = Variable is missing before using information from other quarters.

2 = Variable is not missing after search for off quarter information.

\noindent mode{\_}es{\_}XXX{\_}emp+4 = Variable is missing, filled with the
SEIN employment weighted mode value.

\paragraph{es{\_}(sic, naics, county, owner{\_}code, ein){\_}flag}

Missing = No information in other quarters

0 = Variable is not missing in current quarter

>0 = quarter after the current quarter where replacement value is found

<0 = quarter before the current quarter where replacement value is found

\subsubsection{11{\_}sein{\_}mode{\_}calc.sas}

\paragraph{mode{\_}(es{\_}sic, es{\_}naics, es{\_}county, es{\_}owner{\_}%
code, es{\_}ein, leg{\_}wib, leg{\_}msapmsa, leg{\_}state, leg{\_}county, leg%
{\_}subctygeo)}

The modal value of the variable in an SEIN YEAR QUARTER (unit weighted)

\paragraph{mode{\_}( es{\_}sic, es{\_}naics, es{\_}county, es{\_}owner{\_%
}code, es{\_}ein, leg{\_}wib, leg{\_}msapmsa, leg{\_}state, leg{\_}county,
leg{\_}subctygeo){\_}emp}

The modal value of the variable in an SEIN YEAR QUARTER (employment weighted)

\subsubsection{12{\_}sein{\_}wide.sas}

\paragraph{Place SIC, NAICS, COUNTY, ownership code, EIN and LEG SEIN
level variables in arrays}

\subsubsection{13{\_}sein{\_}fill.sas}

\paragraph{mode{\_}es{\_}(sic, naics, county, owner{\_}code, ein){\_}miss%
}

0 = Variable is not missing

1 = Variable is missing before using information from other quarters.

2 = Variable is not missing after search for off quarter information.

6 = Variable is missing, filled with imputed value. Currently only used for
SIC.

11 = variable missing, but value set to 5. Currently only used for owner{\_}%
code. Assume records with missing ownership codes are private firms.

\paragraph{mode{\_}es{\_}(sic, naics, county, owner{\_}code, ein){\_}emp{%
\_}miss}

0 = Variable is not missing

1 = Variable is missing before using information from other quarters.

2 = Variable is missing, filled with off quarter information.

5 = Variable is missing, filled with the corresponding unit weighted value

6 = Variable is missing, filled with imputed value. Currently only used for
SIC.

11 = variable missing, but value set to 5. Currently only used for owner{\_}%
code. Assume records with missing ownership codes are private firms.

\paragraph{mode{\_}es{\_}(sic, naics, county, owner{\_}code, ein){\_}flag%
}

Missing = No information in other quarters

0 = Variable is not missing in current quarter

>0 = quarter after the current quarter where replacement value is found

<0 = quarter before the current quarter where replacement value is found

\subsubsection{14{\_}impute{\_}sein{\_}industry.sas}

SEIN mode variables missing values are replaced. Missing codes are adjusted.
See program 13 for an explanation of valid values.

\subsubsection{15{\_}impute{\_}seinunit{\_}industry.sas}

SEINUNIT mode variables missing values are replaced. Missing codes are
adjusted. See program 10 for an explanation of valid values.

\subsubsection{16{\_}seinunit{\_}yq{\_}chars.sas}

\paragraph{es{\_}sic{\_}div}

SIC divisions (A, B, C, \ldots ., Z)

\paragraph{ES{\_}SIC{\_}2}

First 2 digits of the 4 digit SIC

\paragraph{ES{\_}SIC{\_}3}

First 3 digits of the 4 digit SIC

\paragraph{ES{\_}NAICS{\_}2}

First 2 digits of the 6 digit NAICS

\paragraph{ES{\_}NAICS{\_}3}

First 3 digits of the 6 digit NAICS

\subsubsection{17{\_}weight{\_}calc{\_}01.sas}

Only temporary variables used in the calculation of the weights are created.

\subsubsection{18{\_}weight{\_}calc{\_}02.sas}

\noindent qwi{\_}unit{\_}weight = Final ECF weight. See technical
documentation for the weights for detailed information.

\subsubsection{19{\_}seinunit{\_}fuzz.sas}

\ifthenelse{\equal{\confidential}{yes}}%
{%
\paragraph{uniform{\_}(sein, seinunit)}

\noindent a random draw from a [0,1] uniform distribution

\paragraph{ramp{\_}(sein, seinunit)}

\noindent a random draw from the ramp (triangle) distribution.

\paragraph{beta{\_}(sein,seinunit)}

\noindent a random draw from the BETA distribution.
}{%
\begin{center}\begin{tabular}{|c|}
\hline
Suppressed for confidentiality\\
\hline
\end{tabular}\end{center}}

\subsubsection{20{\_}sein{\_}file.sas}

No new variables are created.

\subsubsection{21{\_}fuzz{\_}update.sas}

\paragraph{DATE{\_}(SEIN, SEINUNIT){\_}FUZZ}

SAS date value for when the fuzz factor was created.

\paragraph{UPDATE{\_}NUMBER{\_}(SEIN, SEINUNIT)}

Sequential update number. The first time the ECF is created all fuzz factors
receive a value of 0. The value is incremented by 1 each time any fuzz
factors are added to the master file.


%%% Local Variables: 
%%% mode: latex
%%% TeX-master: "ecf_master"
%%% End: 


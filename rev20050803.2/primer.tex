%TCIDATA{Version=4.00.0.2321}
%TCIDATA{LaTeXparent=0,0,sw-edit.tex}
                      
%TCIDATA{ChildDefaults=chapter:1,page:1}


% -*- latex -*- 
%
% Time-stamp: <02/03/14 00:07:13 vilhuber> 
%              Automatically adjusted if using Xemacs
%              Please adjust manually if using other editors
%
% primer.tex
% Responsible: John A
% Part of QWI_methods.tex



\section{Fundamental Concepts}

\subsection{Dates}

\mindex{Dates}

The QWI is a quarterly data system with calendar year timing.  We use the
notation YYYY:Q to refer to a year and quarter combination. For example,
1999:4 refers to the fourth quarter of 1999, which includes the months
October, November, and December.

\subsection{Employer}

\mindex{Employer}

An employer in the QWI system consists of a single Unemployment Insurance (%
UI\index{UI}) account in a given state's UI wage reporting system.  For
statistical purposes the QWI system creates an employer identifier called
an State Employer Identification Number (SEIN\index{SEIN}) from the
UI-account number and information about the state (FIPS\index{FIPS} code). 
Thus, within the QWI system, the SEIN is a unique identifier within and
across states but the entity to which it refers is a UI account.

\subsection{Employee}
\mindex{Employee}

Individual employees are identified by their Social Security Numbers (%
SSN\index{SSN}) on the UI wage records that provide the input to the QWI. To
protect privacy and confidentialty of the SSN and the individual's
name, a different branch of the Census Bureau removes the name and replaces
the SSN with an internal Census identifier called a Protected
Identity Key (PIK\index{PIK}).

\subsection{Job}
\mindex{Job}

The QWI system definition of a job is the association of an individual (%
PIK\index{PIK})\ with an employer (SEIN\index{SEIN}) in a given year and
quarter.  The QWI system stores the entire history of every job that an
individual holds. Estimates are based on the definitions presented below,
which formalize how the QWI system estimates the start of a job
(accession), employment status (beginning- and end-of-quarter employment),
continuous employment (full-quarter employment), the end of a job
(separation), and average earnings for different groups.

\subsection{Unemployment Insurance wage records (the QWI system universe)}
\mindex{Universe}

The Employment Dynamics Estimates are built upon concepts that begin with
the report of an individual's UI\index{UI}-covered earnings by an employing
entity (SEIN\index{SEIN}). An individual's {UI} wage record enters the
QWI system if at least one employer reports earnings of at least one dollar
for that individual\ (PIK\index{PIK}) during the quarter. Thus, the job must
produce at least one dollar of {UI}-covered earnings during a given
quarter to count in the QWI system. The presence of this valid {UI}
wage record in the QWI system triggers the beginning of calculations that
estimate whether that individual was employed at the beginning of the
quarter, at the end of the quarter, and continuously throughout the quarter.
These designations are discussed below. Once these point-in-time employment
measures have been estimated for the individual, further analysis of the
individual's wage records results in estimates of full-quarter employment,
accessions, separations (point-in-time and full-quarter), job creations and
destructions, and a variety of full-quarter average earnings measures.

\subsection{Employment at a point in time}

\mindex{Employment!point in time}

Employment is estimated at two points in time during the quarter,
corresponding to the first and last calendar days. An individual is defined
as employed at the beginning of the quarter when that individual has valid %
UI\index{UI} wage records for the current quarter and the preceding
quarter.  Both records must apply to the same employer (SEIN\index{SEIN}).
An individul is defined as employed at the end of the quarter when that
individual has valid {UI} wage records for the current quarter and the
subsequent quarter. Again, both records must show the same employer. \ The
QWI system uses beginning and end of quarter employment as the basis for
constructing worker and job flows. In addition, these measures are used to
check the external consistency of the data, since a variety of employment
estimates are available as point-in-time measures. Many federal statistics
are based upon estimates of employment as of the 12th day of particular
months. The Census Bureau uses March 12 as the reference date for
employment measures contained in its Business Register and on the Economic
Censuses and Surveys.  The BLS\index{BLS} ``Covered Employment and Wages
(CEW\mindex{CEW})'' series, which is based on the ES-202\index{ES-202}
data, use the 12th of each month as the reference date for employment. The
QWI system cannot use exactly the same reference date as these other
systems because {UI} wage reports do not specify additional detail
regarding the timing of these payments. QWI research has shown that the
point-in-time definitions used to estimate beginning and end of quarter
employment track the CEW month one employment estimates well at the level
of an employer ({SEIN}).

\subsection{Employment for a full quarter}

\mindex{Employment!full quarter}

The concept of full quarter employment estimates individuals who are likely
to have been continuously employed throughout the quarter at a given
employer. An indivdual is defined as full-quarter-employed if that
individual has valid UI\index{UI}-wage records in the current quarter, the
preceding quarter, and the subsequent quarter at the same employer (%
SEIN\index{SEIN}). That is, in terms of the point-in-time definitions, if the
individual is employed at the same employer at both the beginning and end of
the quarter, then the individual is considered full-quarter employed in the
QWI system. 

Consider the following example. Suppose that an individual has
valid {UI} wage records at employer \textit{A} in 1999:2, 1999:3, and
1999:4. This individual does not have a valid {UI} wage record at
employer \textit{A} in 1999:1 or 2000:1. Then, according to the definitions
above, the individual is employed at the end of 1999:2, the beginning and
end of 1999:3, and the beginning of 1999:4 at employer \textit{A}. The QWI
system treats this individual as a full-quarter employee in 1999:3 but not
in 1999:2 or 1999:4. Full-quarter status is not defined for either the first
or last quarter of available data.

\subsection{Point-in-time estimates of accession and separation}

\mindex{Accessions!point-in-time} \mindex{Separations!point-in-time}

An accession occurs in the QWI system when it encounters the first valid %
UI\index{UI} wage record for a job (an individual (PIK\index{PIK})-employer
(SEIN\index{SEIN}) pair). Accessions are not defined for the first quarter
of available data from a given state. The QWI definition of an accession
can be interpreted as an estimate of the number of new employees added to
the payroll of the employer ({SEIN}) during the quarter. The individuals
who acceded to a particular employer were not employed by that employer
during the previous quarter but received at least one dollar of {UI}%
-covered earnings during the quarter of accession.

A separation occurs in the current quarter of the QWI system when it
encounters no valid {UI} wage record for an individual-employer pair
in the subsequent quarter. This definition of separation can be interpreted
as an estimate of the number of employees who left the employer during the
current quarter. \ These individuals received {UI}-covered earnings
during the current quarter but did not receive any {UI}-covered
earnings in the next quarter from this employer. Separations are not defined
for the last quarter of available data.

\subsection{Accession and separation from full-quarter employment}

\mindex{Accessions!full-quarter} \mindex{Separations!full-quarter}

Full-quarter employment is not a point-in-time concept. Full-quarter
accession refers to the quarter in which in individual first attains
full-quarter employment status at a given employer. Full-quarter separation
occurs in the last full-quarter that an individual worked for a given
employer.

As noted above, full-quarter employment refers to an estimate of the number
of employees who were employed at a given employer during the entire
quarter. An accession to full-quarter employment, then, involves two
additional conditions that are not relevant for ordinary accessions. First,
the individual (PIK\index{PIK}) must still be employed at the end of the
quarter at the same employer (SEIN\index{SEIN}) for which the ordinary
accession is defined. At this point (the end of the quarter where the
accession occured and the beginning of the next quarter) the individual has
acceded to continuing-quarter status. An accession to continuing-quarter
status means that the individual acceded in the current quarter and is
end-of-quarter employed. Next the QWI system must check for the possibility
that the individual becomes a full-quarter employee in the subsequent
quarter. An accession to full-quarter status occurs if the individual
acceded in the previous quarter, and is employed at both the beginning and
end of the current quarter. Consider the following example. An individual's
first valid UI\index{UI} wage record with employer \textit{A} occurs in
1999:2. The individual, thus acceded in 1999:2. The same individual has a
valid wage record with employer \textit{A} in 1999:3. The QWI system treats
this individual as end-of-quarter employed in 1999:2 and beginning of
quarter employed in 1999:3. The individual, thus, acceded to
continuing-quarter status in 1999:2. If the individual also has a valid %
{UI} wage record at employer \textit{A} in 1999:4, then the
individual is full-quarter employed in 1999:3. Since 1999:3 is the first
quarter of full-quarter employment, the QWI system considers this individual
an accession to full-quarter employment in 1999:3.

Full-quarter separation works much the same way. One must be careful about
the timing, however. If an individual separates in the current quarter, then
the QWI system looks at the preceding quarter to determine if the individual
was employed at the beginning of the current quarter. An individual who
separates in a quarter in which that person was employed at the beginning of
the quarter is a separation from continuing-quarter status in the current
quarter. Finally, the QWI system checks to see if the individual was a
full-quarter employee in the preceding quarter. An indivdidual who was a
full quarter employee in the previous quarter is treated as a full-quarter
separation in the quarter in which that person actually separates. Note,
therefore, that the definition of full-quarter separation preserves the
timing of the actual separation (current quarter) but restricts the estimate
to those individuals who were full-quarter status in the preceding quarter.
\ For example, suppose that an individual separates from employer \textit{A}
in 1999:3. This means that the individual had a valid {UI} wage
record at employer \textit{A} in 1999:3 but did not have a valid {UI}
wage record at employer \textit{A} in 1999:4. The separation is dated
1999:3. Suppose that the individual had a valid {UI} wage record at
employer \textit{A} in 1999:2. Then, a separation from continuing quarter
status occured in 1999:3. Finally, suppose that this individual had a valid %
{UI} wage record at employer \textit{A} in 1999:1. Then, this
individual was a full-quarter employee at employer \textit{A} in 1999:2. The
QWI system records a full-quarter separation in 1999:3.

\subsection{Point-in-time estimates of new hires and recalls}

\mindex{New hires!point-in-time} \mindex{Recalls!point-in-time}

The QWI system refines the concept of accession into two subcategories:\ new
hires and recalls. In order to do this, the QWI system looks at a full year
of wage record history prior to the quarter in which an accession occurs. If
there are no valid wage records for this job (PIK\index{PIK}-SEIN\index{SEIN})
during the four quarters preceding an accession, then the accession is
called a new hire; otherwise, the accession is called a recall. Thus, new
hires and recalls sum to accessions. For example, suppose that an individual
accedes to employer \textit{A} in 1999:3. Recall that this means that there
is a valid UI\index{UI} wage record for the individual 1 at employer \textit{A%
} in 1999:3 but not in 1999:2. If there are also no valid {UI} wage
records for individual 1 at employer \textit{A} for 1999:1, 1998:4 and
1998:3, then the QWI system designates this accession as a new hire of
individual 1 by employer \textit{A} in 1999:3. Consider a second example in
which individual 2 accedes to employer \textit{B} in 2000:2. Once again, the
accession implies that there is not a valid wage record for individual 2 at
employer \textit{B} in 2000:1. If there is a valid wage record for
individual 2 at employer \textit{B} in 1999:4, 1999:3, or 1999:2, then the
QWI system designates the accession of individual 2 to employer \textit{B}
as a recall in 2000:2. \ New hire and recall data, because they depend upon
having four quarters of historical data, only become available one year
after the data required to estimate accessions become available.

\subsection{New hires and recalls to and from full-quarter employment}

\mindex{New hires!full-quarter} \mindex{Recalls!full-quarter}

Accessions to full-quarter status can also be decomposed into new hires and
recalls. The QWI system accomplishes this decomposition by classifying all
accession to full-quarter status who were classified as new hires in the
previous quarter as new hires to full-quarter status in the current quarter.
Otherwise, the accession to full-quarter status is classified as a recall to
full-quarter status. For example, if individual 1 accedes to full-quarter
status at employer \textit{A} in 1999:4 then, according to the definitions
above, individual 1 acceded to employer \textit{A} in 1999:3 and reached
full-quarter status in 1999:4. Suppose that the accession to employer 
\textit{A} in 1999:3 was classified as a new hire, then the accession to
full quarter status in 1999:4 is classified as a full-quarter new hire. For
another example, consider individual 2 who accedes to full-quarter status at
employer \textit{B} in 2000:3. Suppose that the accession of individual 2 to
employer \textit{B} in 2000:2, which is implied by the full-quarter
accession in 2000:3, was classified by the QWI system as a recall in 2000:2;
then, the accession of individual 2 to full-quarter status at employer 
\textit{B} in 2000:3 is classified as a recall to full-quarter status.

\subsection{Job creations and destructions}

\mindex{Job creation} \mindex{Job destruction}
Job creations and destructions are defined at the employer (SEIN\index{SEIN})
level and not at the job (PIK\index{PIK}-{SEIN}) level. To construct an
estimate of job creations and destructions, the QWI system totals beginning
and ending employment for each quarter for every employer in the UI\index{UI}
wage record universe, that is, for an employer who has at least one valid %
{UI} wage record during the quarter. The QWI system actually uses the %
\Cite{DavisHaltiwangerSchuh} formulas for job creation and destruction
(see definitions in Appendix~\Vref{cha:definitions}). 
Here, we use a
simplified definition. If end-of-quarter employment is
greater than beginning-of-quarter employment, then the employer has created
jobs. The QWI system sets job creations in this case equal to end-of-quarter
employment less beginning-of-quarter employment. The estimate of job
destructions in this case is zero. On the other hand, if
beginning-of-quarter employment exceeds end-of-quarter employment, then this
employer has destroyed jobs. The QWI system computes job destructions in
this case as beginning-of-period employment less end-of-period employment.
The QWI system sets job creations to zero in this case. Notice that either
job creations are positive or job destructions are postive, but not both.
Job creations and job destructions can simultaneously be zero if
beginning-of-quarter employment equals end-of-quarter employment. There is
an important suptelty regarding job creations and destructions when they are
computed for different sex and age groups within the same employer. There
can be creation and destruction of jobs for certain demographic groups
within the employer without job creation or job destruction occuring
overall. \ That is, jobs can be created for some demographic groups and
destroyed for others even at enterprises that have no change in employment
as a whole.

Here is a simple example. Suppose employer \textit{A} has 250 employees at
the beginning of 2000:3 and 280 employees at the end of 2000:3. Then,
employer \textit{A} has 30 job creations and zero job destructions in
2000:3. Now suppose that of the 250 employees 100 are men and 150 are women
at the beginning of 2000:3. At the end of the quarter suppose that there are
135 men and 145 women. Then, job creations for men are 35 and job
destructions for men are 0 in 2000:3. For women in 2000:3 job creations are
0 and job destructions are 5. Notice that the sum of job creations for the
employer by sex (35 + 0) is not equal to job creations for the employer as a
whole (30) and that the sum of job destructions by sex (0 + 5) is not equal
to job destructions for the employer as a whole.

\subsection{Net job flows}

\mindex{Job flows!net} 
\index{Net job flows|see{Job flows}}

Net job flows are also only defined at the level of an employer (%
SEIN\index{SEIN}). They are the difference between job creations and job
destructions. Net job flows are, thus, always equal to end-of-quarter
employment less beginning of quarter employment.

Returning to the example in the description of job creations and
destructions. Employer \textit{A} has 250 employees at the beginning of
2000:3 and 280 employees at the end of 2000:3. Net job flows are 30 (job
creations less job destructions or beginning-of-quarter employment less
end-of-quarter employment). Suppose, once again that employment of men goes
from 100 to 135 from the beginning to the end of 2000:3 and employment of
women goes from 150 to 145. Notice, now, that net job flows for men (35)
plus net job flows for women ($-5$) equals net job flows for the employer as
a whole (30). Net job flows are additive across demographic groups even
though gross job flows (creations and destructions) are not.

Some useful relations among the worker and job flows include:

\begin{itemize}
\item Net job flows = job creations - job destructions 
\index{Job flows!net}%
\index{Job creation}%
\index{Job destruction}

\item Net job flows = end-of-quarter employment - beginning-of-period
employment 
\index{Job flows!net}%
\index{Employment!point in time}

\item Net job flows = accessions - separations 
\index{Job flows!net} 
\index{Accessions!point-in-time} 
\index{Separations!point-in-time}
\end{itemize}

These relations hold for every demographic group and for the employer as a
whole. Additional identities are shown in Appendix~\ref{cha:definitions}.

\subsection{Full-quarter job creations, job destructions and net job flows}

\mindex{Job flows!full-quarter} \mindex{Accessions!full-quarter} %
\mindex{Separations!full-quarter}

The QWI system applies the same job flow concepts to full-quarter employment
to generate estimates of full-quarter job creations, full-quarter job
destructions, and full-quarter net job flows. Full-quarter employment in the
current quarter is compared to full-quarter employment in the preceding
quarter. If full-quarter employment has increased between the preceding
quarter and the current quarter, then full-quarter job creations are equal
to full-quarter employment in the current quarter less full-quarter
employment in the preceding quarter. In this case full-quarter job
destructions are zero. If full-quarter employment has decreased between the
previous and current quarters, then full-quarter job destructions are equal
to full-quarter employment in the preceding quarter minus full-quarter
employment in the current quarter. In this case, full-quarter job
destructions are zero. Full-quarter net job flows equal full-quarter job
creations minus full-quarter job destructions. The same identities that hold
for the regular job flow concepts hold for the full-quarter concepts.

\subsection{Average earnings of end-of-period employees}

\mindex{Earnings!employees!end-of-period} 
\index{Average earnings|see{Earnings}}

The average earnings of end-of-period employees is estimated by first
totaling the UI\index{UI} wage records for all individuals who are
end-of-period employees at a given employer in a given quarter. Then the
total is divided by the number of end-of-period employees for that employer
and quarter.

\subsection{Average earnings of full-quarter employees}

\mindex{Earnings!employees!full-quarter}

Measuring earnings using UI\index{UI} wage records in the QWI system presents
some interesting challenges. The earnings of end-of-quarter employees who
are not present at the beginning of the quarter are the earnings of
accessions during the quarter. The QWI system does not provide any
information about how much of the quarter such individuals worked. The range
of possibilities goes from 1 day to every day of the quarter. Hence,
estimates of the average earnings of such individuals may not be comparable
from quarter to quarter unless one assumes that the average accession works
the same number of quarters regardless of other conditions in the economy.
Similarly, the earnings of beginning-of-quarter who are not present at the
end of the quarter represent the earnings of separations. These present the
same comparison problems as the average earnings of accessions; namely, it
is difficult to model the number of weeks worked during the quarter. If we
consider only those individuals employed at the firm in a given quarter who
were neither accessions nor separations during that quarter, we are left,
exactly, with the full-quarter employees, as discussed above.

The QWI system measures the average earnings of full-quarter employees by
summing the earnings on the {UI} wage records of all individuals at a
given employer who have full-quarter status in a given quarter then dividing
by the number of full-quarter employees. For example, suppose that in 2000:2
employer \textit{A} has 10 full-quarter employees and that their total
earnings are $\$300,000.$ Then, the average earnings of the full-quarter
employees at \textit{A} in 2000:2 is $\$30,000.$ Suppose, further that 6 of
these employees are men and that their total earnings are $\$150,000$. So,
the average earnings of full-quarter male employees is $\$25,000$ in 2000:2
and the average earnings of female full-quarter employees is $\$37,500$ $%
\left( =\$150,000/4\right) $.

\subsection{Average earnings of full-quarter accessions}

\mindex{Earnings!accessions!full-quarter}

As discussed above, a full-quarter accession is an individual who acceded in
the preceding quarter and acheived full-quarter status in the current
quarter. The QWI system measures the average earnings of full-quarter
accessions in a given quarter by summing the UI\index{UI}\ wage record
earnings of all full-quarter accessions during the quarter and dividing by
the number of full-quarter accessions in that quarter.

\subsection{Average earnings of full-quarter new hires}

\mindex{Earnings!new hires!full-quarter} 
\mindex{New
hires!Earnings!full-quarter}

Full-quarter new hires are accessions to full-quarter status who were also
new hires in the preceding quarter. The average earnings of full-quarter new
hires are measured as the sum of UI\index{UI} wage records for a given
employer for all full-quarter new hires in a given quarter divided by the
number of full-quarter new hires in that quarter.

\subsection{Average earnings of full-quarter separations}

\mindex{Earnings!separations!full-quarter}

Full-quarter separations are individuals who separate during the current
quarter who were full-quarter employees in the previous quarter. The QWI
system measures the average earnings of full-quarter separations by summing
the earnings for all individuals who are full-quarter status in the current
quarter and who separate in the subsequent quarter. This total is then
divided by full-quarter separations in the subsequent quarter. The average
earnings of full-quarter separations is, thus, the average earnings of
full-quarter employees in the current quarter who separated in the next
quarter. Note the dating of this variable.

\subsection{Average periods of non-employment for accessions, new hires, and
recalls}

\mindex{Non-employment!accessions} \mindex{Non-employment!new hires} %
\mindex{Non-employment!recalls} 
\index{Average periods of non-employment|see{Non-employment}}

As noted above an accession occurs when a job starts; that is, on the first
occurance of an SEIN\index{SEIN}-PIK\index{PIK} pair following the first quarter
of available data. When the QWI system detects an accession, it measures the
number of quarters (up to a maximum of four) that the individual spent
non-employed in the state prior to the accession. The QWI system estimates
the number of quarters spent non-employed by looking for all other jobs held
by the individual at any employer in the state in the preceding quarters up
to a maximum of four. If the QWI system doesn't find any other valid %
UI\index{UI}-wage records in a quarter preceding the accession it augments
the count of non-employed quarters for the individual who acceded, up to a
maximum of four. Total quarters of non-employment for all accessions is
divided by accessions to estimate average periods of non-employment for
accessions.

Here is a detailed example. Suppose individual 1 and individual 2 accede to
employer \textit{A} in 2000:1. In 1999:4, individual \textit{A} does not
work for any other employers in the state. In 1999:1 through 1999:3
individual 1 worked for employer \textit{B}. Individual 1 had one quarter of
non-employment preceding the accession to employer \textit{A} in 2000:1.
Individual 2 has no valid {UI} wage records for 1999:1 through
1999:4. Indivdiual 2 has four quarters of non-employment preceding the
accession to employer \textit{A} in 2000:1. The accessions to employer 
\textit{A} in 2000:1 had an average of 2.5 quarters of non-employment in the
state prior to accession.

Average periods of non-employment for new hires and recalls are estimated
using exactly analogous formulas except that the measures are estimated
separately for accesions who are also new hires as compared with accession
who are recalls.

\subsection{Average number of periods of non-employment for separations}

\mindex{Non-employment!separations}

Analogous to the average number of periods of non-employment for accessions
prior to the accession, the QWI system measures the average number of
periods of non-employment in the state for individuals who separated in the
current quarter, up to a maximum of four. When the QWI system detects a
separation, it looks forward for up to four quarters to find valid %
UI\index{UI} wage records for the individual who separated and other
employers in the state. Each quarter that it fails to detect any such jobs
is counted as a period of non-employment, up to a maximum of four. The
average number of periods of non-employment is estimated by dividing the
total number of periods of non-employment for separations in the current
quarter by the number of separations in the quarter.

\subsection{Average changes in total earnings for accessions and separations}

\mindex{Earnings!accessions!changes} \mindex{Earnings!separations!changes}

The QWI system measures the change in total earnings for individuals who
accede or separate in a given quarter. For an individual accession in a
given quarter, the QWI system computes total earnings from all valid wage
records for all of the individual's employers in the preceding quarter. The
system then computes the total earnings for the same individual for all
valid wage records and all employers in the current quarter. The acceding
individual's change in earnings is the difference between the current
quarter earnings from all employers and the preceding quarter earnings from
all employers. The average change in earnings for all accessions is the
total change in earnings for all accesions divided by the number of
accessions.

The QWI system computes the average change in earnings for separations in an
analogous manner. The system computes total earnings from all employers for
the separating indivdiual in the current quarter and subtracts total
earnings from all employers in the subsequent quarter. The average change in
earnings for all separations is the total change in earnings for all
separations divided by the number of separations.

Here is an example for the average change in earnings of accessions. Suppose
individual 1 accedes to employer \textit{A} in 2000:3. Earnings for
individual 1 at employer \textit{A} in 2000:3 are $\$8,000$. Individual 1
also worked for employer \textit{B} in 2000:2 and 2000:3. Individual 1's
earnings at employer \textit{B} were $\$7,000$ and $\$3,000$ in in 2000:2
and 2000:3, respectively. Individual 1's change in total earnings between
2000:3 and 2000:2 was $\$4,000$ $\left( =\$8,000+\$3,000-\$7,000\right) .$
Individual 2 also acceded to employer \textit{A} in 2000:3. Individual 2
earned $\$9,000$ from employer \textit{A} in 2000:3. Individual 2 had no
other employers during 2000:2 or 2000:3. Individual 2's change in total
earnings is $\$9,000.$ The average change in earnings for all of employer 
\textit{A}'s accessions is $\$6,500$ $\left( =\left( \$4,000+\$9,000\right)
/2\right) ,$ the average change in total earnings for individuals 1 and 2.


\section{Input files for the QWI}
\label{sec:input_files}

The QWI are constructed from a limited
number of standardized files. The underlying data, as mentioned above, are
extracted from UI administrative files from each participating state, as
well as from the (typically independently created) files from the ES-202
system. Further information comes from Census-internal files, such as the
Census Personal Characteristics File (PCF). 

Once the longitudinal identification has been improved, data from the
original adminstrative files as well as  internal Census files
is parsed, and saved in the form of standardized
files. For each entity level (individual or firm), two files are
constructed, one containing time-aggregated and time-invariant information,
the other containing detailed time-varying information.%
%
%\footnote{The firm-level time-varying information is currently being developed.}
%
%
Table~\Vref{tab:files_QWI} provides an overview of these
files. 

  
% Time-stamp: <05/02/26 10:45:32 vilhuber>

\begin{table}[htbp]
\begin{center}
  \caption{Files used for QWI}
  \label{tab:files_QWI}
  \begin{tabular}{lp{2in}}
{\bf Input data} &\it Dimension \\

\\[-.3cm]
\hline
\\[-.3cm]
UI wage data  &(quarterly, active individuals)\\
ES-202 reports&(quarterly, active firms)\\
Census PCF    &(no. of individuals)\\
\\
\\
\multicolumn{1}{l}{\bf Infrastructure files}&\it Dimension\\

\\[-.3cm]
\hline
\\[-.3cm]
ICF &(no. of individuals) \\
EHF &(no. of individuals {\it x}\\
    &\ \ years present in data)\\
ECF &(no. of firms)\\
    &\ \ years present in data)\\
  \end{tabular}
\end{center}
\end{table}
%%% Local Variables: 
%%% mode: latex
%%% TeX-master: "qwi-overview"
%%% End: 



\subsection{Longitudinal consistency of identifiers}
\label{sec:coding}


Both the UI and the ES-202 data files in the state administrations
are built up from a mix of paper and electronic records. Firms in the
ES-202 system are identified by a (UI tax) account number attributed by the
state. As with all firm identities, an account number can change for a
number of reasons over time, not all of which are distinguishable economic
entities for the purpose of these statistics. States take great care to
follow the legal entities in their system, but account numbers may
nevertheless change for reasons which economists may not consider
legitimate for the purposes of the QWI.
%
On the other hand, the UI data concern a large number of individuals with a
legally unique identifier, the Social Security Number, and individuals are
normally not allowed to change their Social Security Number. However,
coding errors are less well identified by the UI administration, and are
thus more prevalent than for the firm identifiers.
%
In both cases, the longitudinal integrity of the identifier is
compromised, creating spurious job creation and job destruction, and
biasing all other point-in-time estimates as well.

 At LEHD, each problem has been addressed by a specially
designed process. SSN coding errors are addressed by a process which
reattaches miscoded records to the relevant within-firm work history of a
person, relying on name and earnings information.
\Cite{AbowdVilhuber2005} provide a detailed description and an analysis of
the effects of this correction. 

The identification of a
firm's economic successor is achieved by identification of large inter-firm
employment transfers, which is only possible through the combination of
ES-202 and UI records(see \Cite{tp2002-04} for an overview).
%
%
              [EXPAND]
%
%


\section{Component files for the QWI process}
\label{sec:files}
%\chapter{Input sources}\label{cha:input}
%
%%TCIDATA{LaTeXparent=0,0,sw-edit.tex}

% -*- latex -*- 
%
% Time-stamp: <02/05/23 14:12:13 vilhu001> 
%              Automatically adjusted if using Xemacs
%              Please adjust manually if using other editors
%
% input_sources.tex
% Responsible: Lars/John
% Part of QWI_methods.tex

%\section{Description of initial data processing}

This section describes the data processing steps. Figure~\Vref{fig1} gives a
generic overview 
of that process. A more detailed flowchart is provided in
Figure~\Vref{fig:flowchart} in Appendix~\ref{app:flowchart}.

\begin{figure}[htbp]
\begin{center}
\centerline{\includegraphics[width=4.83in,height=3.84in]{\mypath/GraphTable/QWI_methods_V200101171}}
\caption{Overview of data processing\label{fig1}}
\end{center}
\end{figure}


\section{Receipt of data}

The data acquisition process starts with receipt of the data carrier (tape, 
CDROM) by the U.S. Census Bureau. If the data is on CDROM, readin is done
at LEHD, otherwise this task is performed by the 
Administrative Records Research Staff (ARRS\index{ARRS}).  Data entry, 
including method and date of receipt, and number of records, are recorded
both at ARRS\index{ARRS} and at LEHD in either case.

\section{Standardization}

The goal of much of LEHD processing is to create a homogeneous analytical 
product. Thus all data processing at LEHD standardizes variable concepts, 
names, and formats. Harmonization is necessary because different states have 
different ways of recording wages, \aindex{UI} account numbers, and other data items. 
In addition, while these identifiers may be unique within a state, they may 
not be so across different states. After harmonization, LEHD files contain a 
unique state firm identifier (\aindex{SEIN}) used in subsequent processing. Now that 
\aindex{BLS} reporting unit information is available on certain employer-level files, 
the reporting unit (\aindex{RUN}) will be incorporated into the business identifier 
permitting analyses at the establishment level. (LEHD program \aindex{QWI} project 
version 3 will incorporate the reporting unit.) Similar standardization 
treatment is given to other variables. 

\section{SSN editing}

SSN\index{SSN} can and do have coding errors. Since the QWI requires
a consistent longitudinal identifier for each individual, such coding
errors introduce bias into any of the measures computed. LEHD has
developed a process by which possibly miscoded records are matched back to
an otherwise consistent time-series for a given SSN. This processing is
done before anonymization, because it requires the original SSN. Not all
state data are processed this way. 

\section{Anonymization}

The first processing that the data receive, if applicable, is anonymization. 
The administrative data received from the states contain individual and firm 
identifiers. Identifiers include, but are not limited to, first and last 
name, Social Security Number (SSN\index{SSN}), state unemployment insurance account 
number \aindex{SEIN}, and federal EIN\index{EIN}. As per the current Memoranda of Understanding (\aindex{MOU}, 
also called Data Use Agreements%
\index{Data Use Agreements|see{MOU}} by some states), firm identifiers are 
carried along unchanged throughout LEHD processing. Personal identifiers, on 
the other hand, are either deleted or modified in such a manner as to mask 
the original identifier. Thus, the SSN\index{SSN} is replaced by a Census internal 
identifier (Protected Identity Key, PIK\index{PIK}). The original SSN\index{SSN} can not be 
re-inferred, since the algorithm used to associate PIK\index{PIK}s to SSN\index{SSN}s is not 
accessible to LEHD personnel. As an additional precaution, individual names 
are deleted from all files containing them. 

After having passed quality control within ARRS\index{ARRS}, the data are transferred to 
the LEHD computers. All processing from here on is performed exclusively 
within the secure computing environment at LEHD by LEHD
personnel, using PIKs as identifiers. This process is sometimes referred to
as ``PIKizing.\mindex{PIKizing}''


\section{Creation of state-specific characteristics files}

The data are next prepared for the process of extracting information on
jobs, firms, and individuals. It is at this stage that information
available within Census on both the individual and the employer is added to
the data.  Three related files are created for every state. The first of
these is the Individual Characteristics File (\aindex{ICF}), which contains
information on the individual, including demographic information added from
the Census Numident\index{Numident|see{PCF}}/\aindex{PCF} file and links to
any Census survey in which that individual may have participated.  The
actual survey respondent data are linked on a case-by-case basis. The
second file is the Employment History File (\aindex{EHF}), which contains a
detailed quarter-by-quarter time series of an individual's working activity
within the state. The third file is the Employer Characteristics File
(\aindex{ECF}), which contains both information on employers active within
the state as provided in the employer-based data received from the states
and indicators for the presence of that employer in Census data products
(business surveys, business censuses, etc.).

\subsection{Demographic products}

%Across all states presently on version 2, nearly xxxxxx%
%\marginpar{\tiny Has to be corrected for all states!}
Many
individuals have appeared in at least one of the eligible Census demographic 
products, and their detailed demographic information from those surveys can 
be linked to the extensive longitudinal data gleaned from the state records.


\subsection{Census PCF}
\index{SSA}
\index{Numident}
\index{PCF}

These data, which contain information on date of birth, place of birth,
race and sex, are maintained by ARRS\index{ARRS} under a Memorandum of
Understanding with the Social Security Administration\index{SSA}, being
based on the SSA Numident. The LEHD Program matches date of birth, sex,
race, and place of birth using the PIK\index{PIK}. This processing is done
on the ARRS\index{ARRS} system to protect the confidentiality of the
PIK\index{PIK}-SSN\index{SSN} cross walk.

\subsection{Economic censuses and annual surveys}
\index{Annual surveys!manufacturing}
\index{Annual surveys!trade}
\index{Annual surveys!service}
\index{Annual surveys!transportation}
\index{Annual surveys!communication}
\index{Economic census}

These data include the complete 1987, 1992 and 1997 economic censuses, all 
annual surveys of manufacturing, service, trade, transportation and 
communication industries and selected, approved fields from the Census 
Bureau's establishment master file.%
\index{Establishment master file}
Linkage to these data is based upon 
exact EIN\index{EIN} matches, supplemented with statistical matching to recover 
establishments\index{establishment}.

% \section{National master files}
% 
% All three types of master files are then aggregated to the national level. 
% For instance, an individual may have worked a number of years in state A, 
% but disappear periodically from that state's UI records. However, he or she 
% may actually have moved to a different state B and have continued working. 
% The national employment history file will contain the complete time series 
% of working information from all states in the LEHD data base. The same 
% applies for firms. The national employer master file will allow the activity 
% of a firm with a presence in several states to be used in a coherent 
% fashion. The process of producing the first national master files will begin 
% once certain critical statistical research has been completed on a few large 
% states.
% 
% As the number of states participating in the \aindex{QWI} project grows, the quality 
% of these linkages will improve. At the present, four states have already 
% contributed data to the LEHD Program. The number of potential linkages is 
% small. Nevertheless, over 500,000 individuals have records in at least two 
% of those states, a number much larger than all but the most comprehensive 
% surveys. 
%
%\section{Merge with other Census Bureau products}
%
%The structure of the files themselves will be virtually identical to the 
%state level files, so all the linkages to Census products available at the 
%state level carry over to the national files. Special projects that require 
%less numerous, but more detailed information on individuals and firms can be 
%carried out both at the national and the state level. 



%%% Local Variables: 
%%% mode: latex
%%% TeX-master: "QWI_methods"
%%% End: 

%
%\chapter{The Individual Characteristics File\label{cha:icf}}
%
\input{\mypath/icf.tex}
%
%\chapter{The Employment History File\label{cha:ehf}}
%
\input{\mypath/ehf.tex}
%
%\chapter{The Employer Characteristics File\label{cha:ecf}}
%
\input{\mypath/ecf.tex}
%
xxx Talk about files here: inhale the short descriptions from the QWI
methods file. xxx

\section{Forming Aggregated Estimates}
\label{sec:aggregate}

Aggregating the QWI data is a four step process, which can be summarized as
follows:

\begin{itemize}
\item[\ ] 

\begin{enumerate}
\item The basic variables, as discussed above, are created for each
employment history (PIK\index{PIK}-SEIN\index{SEIN} pair) and for every quarter
that the pair exits.

\item Unit-to-Worker impute 
%
%
  [ EXPAND ]
%
%

\item The QWI system sums for each employer's workplace the following variables:
beginning-of-period employment, end-of-period employment, accessions, new
hires, recalls, separations, full-quarter employment, full-quarter
accessions, full-quarter new hires, total earnings of full-quarter
employees, total earnings of full-quarter accessions, and total earnings of
full-quarter new hires. Job creations, job destructions, and net job flows
are estimated for each workplace using the beginning and end of quarter
employment estimates for that workplace. The first-layer of
disclosure-proofing is also applied at this step.

\item The workplace-level variables in the list above are summed over the
relevant aggregating unit (county or SIC) for each quarter. Average
earnings of full-quarter employees, full-quarter accessions, and
full-quarter new hires are estimated by taking the ratio of total earnings
of the relevant category to the total number of individuals in that
category. \ For example, avearge earnings of full-quarter men ages 55-64 for
a given year, quarter and county is the ratio of total earnings of
full-quarter men ages 55-64 to the number of full-quarter men ages 55-64 in
that year, quarter, and county.

\item The beginning-of-quarter employment for each county or SIC division is
controlled (raked) to the BLS estimate of total county employment in month
one of that quarter from the Covered Employment and Wages series. At this
point the other estimates and the demographic groups are also raked to
preserve the underlying relations among the variables.
\end{enumerate}
\end{itemize}

%\chapter{Worker and Job Flow Analysis\label{cha:jobflow}}
%
%%TCIDATA{LaTeXparent=0,0,sw-edit.tex}

% -*- latex -*- 
%
% Time-stamp: <02/07/26 13:43:51 vilhuber> 
%              Automatically adjusted if using Xemacs
%              Please adjust manually if using other editors
%
% jobflow_analysis.tex
% Responsible: John/Paul/Bryce
% Part of QWI_methods.tex

%\section{Overview}
%\index{employment status}
%\index{point-in-time}
%\index{full-quarter}
%\index{accession}
%\index{separation}
%\index{job flows}
%\index{new hires}
%\index{job creation}
%\index{job destruction}
%\index{earnings}
%\index{earnings!change}

The detailed description that follows makes explicit the links to the sets
of files described earlier.
The statistics calculated in this section are based on definitions  summarized in \Cite{AbowdCorbelKramarz99} and
\Cite{DavisHaltiwanger99}. 
As mentioned before, employment is measured at two points in
time (beginning and end of  
quarter) and according to two concepts (any employment status and 
full-quarter employment status). Worker flows are captured by accessions and 
separations with respect to both employment status concepts. Job flows are 
captured by gross job creation and destruction at the firm level, again 
according to both employment concepts. Accessions are further separated into 
new hires and recalls. Earnings and earnings change statistics are 
calculated for each of the worker flow categories as well as for both 
employment statuses.

%\subsection{Calculation of statistics}

The worker and employment flow statistics reported at the county and 
SIC division level are calculated through a multi-step 
\index{county}\index{SIC}
process.%
%
%\footnote{Details on the program sequence used to create the job 
%low statistics are available in Appendix~\Vref{app:jobflow_technical}.}
%
The EHF\index{EHF} (see Subsection~\ref{cha:ehf}), which contains individual
work and earnings histories, is combined with information from the
ICF\index{ICF} (see Subsection~\ref{cha:icf}) to incorporate demographic
characteristics of workers such as age and sex. For each worker in each
year and quarter, an array of jobs at various SEINs\index{SEIN} is stored.
The statistics listed in Subsection~\Vref{cha:definitions} are computed, when
appropriate, for each individual/job/quarter combination. The statistics
are then aggregated to the SEIN level by age and sex to create a file of
totals for each SEIN/year/quarter/agegroup/sexgroup combination. Both the
Workforce Investment Act (WIA)\index{WIA} and CPS\index{CPS} age groups are
used. The totals are stored by age/sex group as well as further aggregated
within SEIN over age and sex group to produce the overall total for the
SEIN as well as marginal totals for sex and age (for example, the total for
females of all ages).  All totals are then aggregated twice more: once to
the industry level and once to the county level. At this point the
statistics are in their final form except for the handling of disclosure
issues, as discussed below.

\subsection{Examples}

The following tables provide an example of how the flow statistics are
computed for four hypothetical individuals who work at three hypothetical
employers over a two year sample period. All individuals and firms in this
example are fictitious. Table~\Vref{table1} summarizes the earnings history
\index{earnings history} of each individual as it would appear in the
employment history file\index{EHF}.  Table~\Vref{table2} presents the
individual level employment flow statistics that can be computed from the
individual work histories. Note that individual 1 leaves employer
\textit{X} at some point during the second quarter of 1995, and that she
begins working for employer \textit{Y} during the same quarter. In
Table~\ref{table2}, employment flow statistics as defined in
Subsection~\ref{cha:definitions} have been computed for every quarter of every
job worked by Person~\textit{1}. Person~\textit{1} is considered to be
employed at employer \textit{X} from 1994:1 -- 1995:2.  Hence, \textit{e=1}
from 1994:1 through 1995:1 since she is still employed at \textit{X} at the
end of each of these quarters. Similarly, \textit{b=1} from 1994:2 through
1995:2 since she is employed at \textit{X} from the very beginning of these
quarters. Note that \textit{b} is missing in 1994:1. The first quarter of
the analysis is out-of-scope for \textit{b}, since it depends on employment
information from the previous quarter. Also note that for in-scope periods,
end-of-quarter employment\index{employment!point-in-time} at time
\textit{t} is equal to beginning-of quarter employment at time \textit{t +
  1}. In Subsection~\ref{cha:definitions}, this identity (Identity
\ref{identity:1}) is defined for aggregates, but as shown in the example it
holds at the individual level as well.

\input{\mypath/jobflow_analysis.table1.tex}

Moving on, \textit{f=1} for Person \textit{1}/Employer \textit{X} from
1994:2-1995:1, but \textit{f} is missing during 1994:1, which is
out-of-scope, and \textit{f=0} during 1995:2 because she is no longer
employed at \textit{X} in 1995:3. In 1995:2 \textit{s=1} and \textit{fs=1}
for Person~\textit{1}/ Employer~\textit{X} because she separates from
Employer~\textit{X} sometime during this quarter and appears to have been
in this job for the entire preceding quarter (1995:1).  In 1995:2,
\textit{a=1} for Person~\textit{1} and Employer~\textit{Y} because she
enters a relationship with Employer~\textit{Y} sometime during this
quarter, and \textit{fa=1} in 1995:3 because this is her first full quarter
\index{employment!full-quarter} at Employer~\textit{Y}. New hires\index{New
  hires}, \textit{h}, is also 1 because she has no previous relationship
with Employer~\textit{Y} in the last four quarters, and recalls\index{recall}
\textit{r=0} for the same reason. A variety of wage measures are also
calculated for each individual: \textit{w1} is simply the wage earned at
each job each quarter, while measures such as \textit{w2}, \textit{w3},
\textit{wa} are calculated as an individual's wage if he or she meets a
certain criteria (\textit{e=1} for \textit{w2}, \textit{f=1} for
\textit{w3}, \textit{etc}.).

\input{\mypath/jobflow_analysis.table2.tex}

In Table~\Vref{table3}, the individual statistics are aggregated to the
employer level by summing individual statistics by SEIN. \textit{E} for
Employer~\textit{X} in 95:1, then, is the sum of \textit{e} over all
individuals working at \textit{X} in 1995:1 (in this case individuals 1 and
4). Since \textit{e=1} for Individual~\textit{1}, who remains with
Employer~\textit{X} the next quarter, and \textit{e=0} for
Individual~\textit{4}, who has no wage record with Employer~\textit{X} the
next quarter, \textit{E=1}. Similarly, since \textit{a=0} for both
individuals this quarter (both worked at \textit{X} last quarter also),
\textit{A=0}. The job flow\mindex{job flow} at Employer~\textit{X}, defined
as the net increase in employment over that quarter, is calculated as the
difference between the number of end-of-quarter jobs held and the number of
beginning-of quarter jobs held. Thus, $JF = E -- B$, or 1 -- 2 = -1 in this
case. Because there was a negative net job flow of 1 this quarter, job
creation\index{job creation} \textit{JC= 0} and job destruction\index{job
  destruction} \textit{JD=1}. Total payroll \textit{W1} is also computed
for each employer; for Employer~\textit{X} in 1995:1 it is simply the sum
of the wages paid to individuals 1 and 4: {\$}5000 + {\$}4000 = {\$}9000.
Individual~1 also had end-of-quarter wages \textit{w2=5000} because she was
end-of-quarter employed at \textit{X} this period. For
Individual~\textit{4}, \textit{w2=0} because \textit{e=0} at \textit{X} in
1995:1. Total end of quarter wages \textit{W2} for Employer~\textit{X} in
1995:1 is then calculated as the sum of wages at all end-of-quarter jobs.
In this case, it is simply {\$}5000 since Individual~1 has the only
end-of-quarter job at \textit{X} in 1995:1.

\input{\mypath/jobflow_analysis.table3.tex}

%\marginpar{\tiny This paragraph (and chapter) might need to be tagged for the index!!}
Several identities from Subsection~\ref{cha:definitions} are illustrated in
Table~\ref{table3}.  Once again Identity~\ref{identity:1} ($B_{jt} = E_{jt
  - 1} )$ is noticeable just from glancing at the columns of numbers B and
E.  Identity~\ref{identity:3}, $E_{jt} = B_{jt} + A_{jt} - S_{jt} $ also
holds whenever all four variables are in-scope. For example, for
Employer~\textit{X} in 1995:1, \textit{E = 1 = 2 + 0 -- 1}. For this
employer in 1995:2, \textit{E = 0 = 1 + 0 -- 1}. Identity~\ref{identity:4},
$JF_{jt} = JC_{jt} - JD_{jt}$ is also true: for \textit{X} in 1995: 1
\textit{JF = -1 = 0 -- 1}. Identity~\ref{identity:5}, $E_{jt} = B_{jt} +
JC_{jt} - JD_{jt} $ , (\textit{X} in 1995:1 : \textit{E = 1 = 2 + 0 -- 1})
and Identity~\ref{identity:6}, $A_{jt} - S_{jt} = JC_{jt} - JD_{jt} $,
(\textit{X} in 1995: 1 : \textit{A -- S = 0 -- 1 = 0 -- 1 = JC -- JD}).
Finally, Identity~\ref{identity:15}, the total payroll identity ($W_{1jt} =
W_{2jt} + WS_{jt} )$ is met in all cases. For example, for SEIN \textit{X}
in 1995: 2, \textit{W1 = {\$}9000 = 5000 + 4000 = W2 + WS}. When \textit{WS
  is missing}, as in most cases, \textit{W1} and \textit{W2} are simply
equal because every wage is an end-of-quarter wage.

\input{\mypath/jobflow_analysis.table4.tex}

In Table~\Vref{table4}, the SEIN-level statistics are aggregated in a
similar way to create total flows and average wages. These can be thought
of as county totals if the hypothetical universe includes just a single
county. The total flows are computed exactly as the employer level flows in
Table~\ref{table3}. For 1995:1, total jobs at the end of quarter, total
\textit{E}, is just the sum of \textit{E} for \textit{X}, \textit{Y}, and
\textit{Z}: 1 + 0 + 1 = 2. Note that this is the same as the sum of all
individual \textit{e} in Table\ref{table2} for 1995:1. Total accessions are computed
similarly (\textit{A = 0 + 0 + 0}) as are total wages (\textit{W1 = 9000 +
  2000 + 6000 = 17000}).  Average wages (for example, \textit{Z\_W2},
\textit{Z\_WA}) are computed by summing total wages for \textit{X},
\textit{Y}, and \textit{Z}, and dividing by the total number of individuals
used to calculate the particular wage measure. For example, \textit{Z\_W2}
for 1995: 1 is computed as the sum of \textit{W2} for \textit{X},
\textit{Y}, and \textit{Z} where defined (5000 + 6000) divided by the total
number of end-of-quarter positions (\textit{E = 2}) for an average
end-of-quarter wage of \textit{Z\_W2} = {\$}5500. \textit{Z\_WA} is
undefined for this quarter because there are no accessions this quarter. In
1995:2 they are computed as the sum of \textit{WA} for \textit{X},
\textit{Y}, and \textit{Z} where defined (2000, since \textit{WA} is only defined
for \textit{Y} this quarter) divided by the total number of accessions this
quarter (1) so the average wage to accessions in 1995:2 is simply {\$}2000.




%%% Local Variables: 
%%% mode: latex
%%% TeX-master: "qwi-overview"
%%% End: 

%%TCIDATA{LaTeXparent=0,0,sw-edit.tex}

% -*- latex -*- 
%
% Time-stamp: <02/07/26 09:44:20 vilhuber> 
%              Automatically adjusted if using Xemacs
%              Please adjust manually if using other editors
%
% jobflow_analysis_data_consistency.tex



\section{Data Consistency}

Figure~\Vref{figure2}  shows the consistency of
beginning-of-quarter employment totals based on the LEHD processing with
the BLS\index{BLS} data from the Covered Worker series (CEW)\index{CEW},
for the state of Illinois.
The difference in overall levels reflects systematic differences in the
assumptions of the QWI system and the BLS CEW system. Single-quarter
discrepancies reflect data difficulties with the unemployment insurance
wage records in the historical databases. 


% \begin{figure}[htbp]
% \begin{center}
% \caption{Data consistency: California\label{figure2}}
% \centerline{\includegraphics[width=6.45in]{\mypath/GraphTable/gplotca}}
% \end{center}
% \end{figure}

%\begin{figure}[htbp]
%\begin{center}
%\caption{Data consistency: Florida\label{figure2fl}}
%\centerline{\includegraphics[width=6.45in]{\mypath/GraphTable/gplotfl}}
%\end{center}
%\end{figure}
%
\begin{figure}[htbp]
\begin{center}
\caption{Data consistency: Illinois\label{figure2}}
\centerline{\includegraphics[width=6.45in]{\mypath/GraphTable/gplotil}}
\end{center}
\end{figure}
%
%\begin{figure}[htbp]
%\begin{center}
%\caption{Data consistency: Maryland\label{figure2md}}
%\centerline{\includegraphics[width=6.45in]{\mypath/GraphTable/gplotmd}}
%\end{center}
%\end{figure}
%
%\begin{figure}[htbp]
%\begin{center}
%\caption{Data consistency: Minnesota\label{figure2mn}}
%\centerline{\includegraphics[width=6.45in]{\mypath/GraphTable/gplotmn}}
%\end{center}
%\end{figure}
%
%\begin{figure}[htbp]
%\begin{center}
%\caption{Data consistency: Texas\label{figure2tx}}
%\centerline{\includegraphics[width=6.45in]{\mypath/GraphTable/gplottx}}
%\end{center}
%\end{figure}

%%% Local Variables: 
%%% mode: latex
%%% TeX-master: "qwi-overview"
%%% End: 


\section{Disclosure Proofing the QWI}
\label{sec:confidentiality}

\mindex{Disclosure proofing}

Disclosure proofing is the set of methods used by statistical agencies to
protect the confidentiality of the identity of and information about the
individuals and businesses that form the underlying data in the system. In
the QWI system, disclosure proofing is required to protect the information
about individuals and businesses that contribute to the UI\index{UI} wage
records, the ES-202 quarterly reports, and the Census Bureau demographic
data that have been integrated with these sources. There are three layers of %
\index{confidentiality protection} and disclosure proofing in the QWI
system.

The first layer occurs at stage two in the production of the estimates, the
stage at which workplace-level estimates are made. At this stage, the QWI
system infuses specially constructed noise into the estimates of all of the
workplace-level measures. This noise\index{noise} is designed to have two very
important properties. First, for a given workplace, the data are always
distorted in the same direction (increased or decreased) by the same
percentage amount in every period. Second, the statistical properties%
\index{distortion!statistical properties} of this distortion are such that
when the estimates are aggregated to the county\index{county} or SIC\index{SIC}
division level the effects of the distortion cancel out for the vast
majority of the estimates.

The second layer of confidentiality protection occurs when the
workplace-level measures are aggregated to the county\index{county} or SIC\index{SIC}
division level. The data from many individuals and businesses are combined
into a (relatively) few estimates. This aggregation helps to conceal the
exact information about any of the individuals or businesses that underlie
the estimate. At this level of confidentiality protection, some of the
estimates turn out to be based on fewer than three persons. These estimates
are suppressed%
\index{suppression}. In addition, some of the estimates are based on data
that are still substantially influenced by the noise that was infused in the
first layer. These estimates are flagged as \Mindex{substantially distorted}.

The final layer of confidentiality protection occurs when the
beginning-of-quarter employment estimate for the county\index{county} or %
SIC\index{SIC} division as a whole is raked to the published BLS\index{BLS}
estimates from the Covered Employment and Wages%
\index{Covered Employment and Wages|see{CEW}} series. This raking\mindex{raking} has two
effects. First, aggregates that the BLS\index{BLS} suppresses (because of
conditions in its disclosure-proofing system) are also suppressed by the QWI
system. Second, the QWI system does not produce an independent estimate of
overall employment for the aggregate. The QWI system, thus supplements the
Covered Employment and Wages program by providing worker flows, job flows,
full-quarter employment estimates, and demographic detail, none of which can
be easily estimated from the ES-202%
\index{ES-202} quarterly reports themselves. The final layer of
confidentiality protection is not applied to the full-quarter estimates
(employment, flows and average earnings) because there is no comparable
estimate produced by the BLS from ES-202 data.


\section{The Final product}
\label{sec:final}

\subsection{Summary Variable Definitions}

\subsubsection{Timing and Category Variables}

\index{Timing Variables|see{Variables, timing}} 
\index{Category Variables|see{Variables, category}} \mindex{Variables!timing}
\mindex{Variables!category}

Timing and categorical variables are used to describe the population and
time period that the content variables cover. The first such variable is 
\textsf{STATE}\mindex{STATE}, which is the two-digit FIPS%
\index{FIPS} code for the state upon which the employment dynamics estimates
are based. The next two variables (\textsf{YEAR}\mindex{YEAR} and \textsf{%
QUARTER}\mindex{QUARTER}) refer to the calendar year and quarter covered by
the content variables. The \textsf{COUNTY}\mindex{COUNTY} variable
(county-level data file) is the three-digit FIPS%
\index{FIPS} code for the county (within the state). The \textsf{%
SIC\_DIVISION}\mindex{SIC\_DIVISION} variable (sic-division-level data file)
is the one-character SIC\ (1987) major industry group. The \textsf{SEX}%
\mindex{SEX} variable indicates whether the data cover men or women. The 
\textsf{AGEGROUP}\mindex{AGEGROUP} variable indicates which of the eight age
categories the data cover.

\subsubsection{Content Variables}

\index{Content Variables|see{Variables, content}} \mindex{Variables!content}

The quarterly employment estimates for beginning of quarter employment are
contained the variable \textsf{B} and the estimates for end of quarter
employment are found in the variable \textsf{E}. Accessions are reported in
the variable \textsf{A}. New hires are in \textsf{H} and recalls are
reported in \textsf{R}. Separations are reported in the variable \textsf{S}.

Because of the confidentiality protection system used for the Employment
Dynamics Estimates, the estimate of beginning-of-quarter employment for both
sexes (\textsf{SEX}=0) and all age groups (\textsf{AGEGROUP}=0) is exactly
equal to the BLS-published Covered Employment and Wages estimate of
employment on the 12th day of the first month of the quarter for the
relevant geographic and industrial category. For example, in California the
QWI estimate for beginning-of-quarter employment in the entire state in
1999:3 is 14,440,000 (\textsf{B}=14,440,000 for \textsf{STATE}=``06'', 
\textsf{YEAR}=1999, \textsf{QUARTER}=3, \textsf{COUNTY}=``000'' (or \textsf{%
SIC\_DIVISION}=(blank)),\textsf{SEX}=0, \textsf{AGEGROUP}=0), which exactly
equals the BLS CEW estimate for month 1 in 1999:3 for the entire state,
combining all establishment sizes and all ownership categories. \ Similarly,
the QWI estimate of end-of-quarter employment is controlled for the category
both sexes (\textsf{SEX}=0) and all age groups (\textsf{AGEGROUP}=0) to
equal the BLS-published CEW estimate of employment on the 12th day of the
first month of the succeeding quarter. Again, considering California, the
QWI estimate for end-of-quarter employment in 1999:3 is 14,660,000, which
exactly equals the BLS\ CEW estimate for month 1 in 1999:4 for the entire
state (\textsf{E}=14,660,000 for \textsf{STATE}=``06'', \textsf{YEAR}=1999, 
\textsf{QUARTER}=3, \textsf{COUNTY}=``000'' (or \textsf{SIC\_DIVISION}%
=(blank)), \textsf{SEX}=0, \textsf{AGEGROUP}=0). 
%See the tables in Appendix %
%\ref{app:raking_data} for a complete list of the BLS\ series used in this
%control.

Quarterly employment estimates are also provided on a full-quarter basis.
These estimates are reported in the variable \textsf{F}. Full-quarter
accessions are reported in \textsf{FA}. Full-quarter separations are
reported in \textsf{FS}. Full-quarter new hires are in \textsf{H3. }The
raking step of the QWI confidentiality protection system used to disclosure
proof the variables \textsf{B} and \textsf{E} (and related variables) does
not affect the estimates of full-quarter employment and related flows.

Job creations and destructions are reported in the variables \textsf{JC} and 
\textsf{JD}, respectively. Net job flows are reported in the variable 
\textsf{JF}. Full-quarter job creations and destructions are reported in 
\textsf{FJC} and \textsf{FJD}, respectively. Full-quarter net job flows are
in \textsf{FJF}.

Average earnings of full-quarter employees can be found in \textsf{Z\_W3}.
Average earnings of full-quarter new hires are reported in \textsf{Z\_WH3}.

%See the table of contents at the end of this primer for a list of other
%variables and definitions in the QWI data files.

\subsubsection{Status Flag Variables}

\index{Status Flag Variables|see{Variables, status flag}} %
\mindex{Variables!status flag}

Every variable in QWI data files has an associated status flag. These
variables are called \textsf{[varname]\_status}. The status flag variables
are also shown in the contents tables at the end of this primer. The status
flag has three distinct values:

\begin{itemize}
\item[$*$] indicates significant distortion is necessary to preserve
confidentiality

\item[$d$] indicates an estimate is based on $<3$ employees in the at-risk group.

\item[$n$] indicates an estimate is not defined because no employees are in
the relevant category
\end{itemize}

%\newpage

\subsection{Data structure\label{primer_tables}}

Appendix~\Vref{app:data_structure} describes the contents of a typical
output statistics file, in this case for the state of Texas, and aggregated
to the county level. A similar file exists at the industry aggregation
level, and the same pair of files is constructed for every available state.

%\input{\mypath/ca_county_v23_fuzzed.tex} \newpage \input{\mypath/ca_county_v23_fuzzed.freq.tex}
%\newpage \input{\mypath/ca_sic_division_v23_fuzzed.tex} \newpage \input{%
%\mypath/ca_sic_division_v23_fuzzed.freq.tex}

% \newpage
% 
% \subsection{Florida}
% 
% \input{\mypath/fl_county_v23_fuzzed.tex} \newpage \input{\mypath/fl_county_v23_fuzzed.freq.tex}
% \newpage \input{\mypath/fl_sic_division_v23_fuzzed.tex} \newpage \input{%
% \mypath/fl_sic_division_v23_fuzzed.freq.tex}
% 
% \newpage
% 
% \subsection{Illinois}
% 
% \input{\mypath/il_county_v23_fuzzed.tex} \newpage \input{\mypath/il_county_v23_fuzzed.freq.tex}
% \newpage \input{\mypath/il_sic_division_v23_fuzzed.tex} \newpage \input{%
% \mypath/il_sic_division_v23_fuzzed.freq.tex}
% 
% \newpage
% 
% \subsection{Maryland}
% 
% %    Generated by SAS
%    http://www.sas.com
% created by=vilhu001
% sasversion=8.2
% date=2002-05-23
% time=00:39:34
% encoding=iso-8859-1
% ====================begin of output====================
% \begin{document}

% An external file needs to be included, as specified% in latexlong.sas. This can be called sas.sty,
% in which case you want to include a line like
% \usepackage{sas}
% or it can be a simple (La)TeX file, which you 
% include by typing 
% %%
%% This is file `sas.sty',
%% generated with the docstrip utility.
%%
%% 
\NeedsTeXFormat{LaTeX2e}
\ProvidesPackage{sas}
        [2002/01/18 LEHD version 0.1
    provides definition for tables generated by SAS%
                   ]
\@ifundefined{array@processline}{\RequirePackage{array}}{}
\@ifundefined{longtable@processline}{\RequirePackage{longtable}}{}
 \def\ContentTitle{\small\it\sffamily}
 \def\Output{\small\sffamily}
 \def\HeaderEmphasis{\small\it\sffamily}
 \def\NoteContent{\small\sffamily}
 \def\FatalContent{\small\sffamily}
 \def\Graph{\small\sffamily}
 \def\WarnContentFixed{\footnotesize\tt}
 \def\NoteBanner{\small\sffamily}
 \def\DataStrong{\normalsize\bf\sffamily}
 \def\Document{\small\sffamily}
 \def\BeforeCaption{\normalsize\bf\sffamily}
 \def\ContentsDate{\small\sffamily}
 \def\Pages{\small\sffamily}
 \def\TitlesAndFooters{\footnotesize\bf\it\sffamily}
 \def\IndexProcName{\small\sffamily}
 \def\ProcTitle{\normalsize\bf\it\sffamily}
 \def\IndexAction{\small\sffamily}
 \def\Data{\small\sffamily}
 \def\Table{\small\sffamily}
 \def\FooterEmpty{\footnotesize\bf\sffamily}
 \def\SysTitleAndFooterContainer{\footnotesize\sffamily}
 \def\RowFooterEmpty{\footnotesize\bf\sffamily}
 \def\ExtendedPage{\small\it\sffamily}
 \def\FooterFixed{\footnotesize\tt}
 \def\RowFooterStrongFixed{\footnotesize\bf\tt}
 \def\RowFooterEmphasis{\footnote\it\sffamily}
 \def\ContentFolder{\small\sffamily}
 \def\Container{\small\sffamily}
 \def\Date{\small\sffamily}
 \def\RowFooterFixed{\footnotesize\tt}
 \def\Caption{\normalsize\bf\sffamily}
 \def\WarnBanner{\small\sffamily}
 \def\Frame{\small\sffamily}
 \def\HeaderStrongFixed{\footnotesize\bf\tt}
 \def\IndexTitle{\small\it\sffamily}
 \def\NoteContentFixed{\footnotesize\tt}
 \def\DataEmphasisFixed{\footnotesize\it\tt}
 \def\Note{\small\sffamily}
 \def\Byline{\normalsize\bf\sffamily}
 \def\FatalBanner{\small\sffamily}
 \def\ProcTitleFixed{\footnotesize\bf\tt}
 \def\ByContentFolder{\small\sffamily}
 \def\PagesProcLabel{\small\sffamily}
 \def\RowHeaderFixed{\footnotesize\tt}
 \def\RowFooterEmphasisFixed{\footnotesize\it\tt}
 \def\WarnContent{\small\sffamily}
 \def\DataEmpty{\small\sffamily}
 \def\Cell{\small\sffamily}
 \def\Header{\normalsize\bf\sffamily}
 \def\PageNo{\normalsize\bf\sffamily}
 \def\ContentProcLabel{\small\sffamily}
 \def\HeaderFixed{\footnotesize\tt}
 \def\PagesTitle{\small\it\sffamily}
 \def\RowHeaderEmpty{\normalsize\bf\sffamily}
 \def\PagesProcName{\small\sffamily}
 \def\Batch{\footnotesize\tt}
 \def\ContentItem{\small\sffamily}
 \def\Body{\small\sffamily}
 \def\PagesDate{\small\sffamily}
 \def\Index{\small\sffamily}
 \def\HeaderEmpty{\normalsize\bf\sffamily}
 \def\FooterStrong{\footnotesize\bf\sffamily}
 \def\FooterEmphasis{\footnotesize\it\sffamily}
 \def\ErrorContent{\small\sffamily}
 \def\DataFixed{\footnotesize\tt}
 \def\HeaderStrong{\normalsize\bf\sffamily}
 \def\GraphBackground{}
 \def\DataEmphasis{\small\it\sffamily}
 \def\TitleAndNoteContainer{\small\sffamily}
 \def\RowFooter{\footnotesize\bf\sffamily}
 \def\IndexItem{\small\sffamily}
 \def\BylineContainer{\small\sffamily}
 \def\FatalContentFixed{\footnotesize\tt}
 \def\BodyDate{\small\sffamily}
 \def\RowFooterStrong{\footnotesize\bf\sffamily}
 \def\UserText{\small\sffamily}
 \def\HeadersAndFooters{\footnotesize\bf\sffamily}
 \def\RowHeaderEmphasisFixed{\footnotesize\it\tt}
 \def\ErrorBanner{\small\sffamily}
 \def\ContentProcName{\small\sffamily}
 \def\RowHeaderStrong{\normalsize\bf\sffamily}
 \def\FooterEmphasisFixed{\footnotesize\it\tt}
 \def\Contents{\small\sffamily}
 \def\FooterStrongFixed{\footnotesize\bf\tt}
 \def\PagesItem{\small\sffamily}
 \def\RowHeader{\normalsize\bf\sffamily}
 \def\AfterCaption{\normalsize\bf\sffamily}
 \def\RowHeaderStrongFixed{\footnotesize\bf\tt}
 \def\RowHeaderEmphasis{\small\it\sffamily}
 \def\DataStrongFixed{\footnotesize\bf\tt}
 \def\Footer{\footnotesize\bf\sffamily}
 \def\FolderAction{\small\sffamily}
 \def\HeaderEmphasisFixed{\footnotesize\it\tt}
 \def\SystemTitle{\large\bf\it\sffamily}
 \def\ErrorContentFixed{\footnotesize\tt}
 \def\SystemFooter{\footnotesize\it\sffamily}
% Set cell padding 
\renewcommand{\arraystretch}{1.3}
% Headings
\newcommand{\heading}[2]{\csname#1\endcsname #2}
\newcommand{\proctitle}[2]{\csname#1\endcsname #2}
% Declare new column type
\newcolumntype{S}[2]{>{\csname#1\endcsname}#2}
% Set warning box style
\newcommand{\msg}[2]{\fbox{%
   \begin{minipage}{\textwidth}#2\end{minipage}}%
}

\begin{center}\heading{ProcTitle}{The CONTENTS Procedure}\end{center}
\begin{center}\begin{longtable}
{llll}\hline % colspecs
   \multicolumn{1}{S{RowHeader}{l}}{Data Set Name:} & 
   \multicolumn{1}{S{Data}{l}}{STATE.MD{\textunderscore}COUNTY{\textunderscore}V23{\textunderscore}FUZZED} & 
   \multicolumn{1}{S{RowHeader}{l}}{Observations:} & 
   \multicolumn{1}{S{Data}{l}}{26271}
\\
   \multicolumn{1}{S{RowHeader}{l}}{Member Type:} & 
   \multicolumn{1}{S{Data}{l}}{DATA} & 
   \multicolumn{1}{S{RowHeader}{l}}{Variables:} & 
   \multicolumn{1}{S{Data}{l}}{60}
\\
   \multicolumn{1}{S{RowHeader}{l}}{Engine:} & 
   \multicolumn{1}{S{Data}{l}}{V8} & 
   \multicolumn{1}{S{RowHeader}{l}}{Indexes:} & 
   \multicolumn{1}{S{Data}{l}}{0}
\\
   \multicolumn{1}{S{RowHeader}{l}}{Created:} & 
   \multicolumn{1}{S{Data}{l}}{18:40 Thursday, May 16, 2002} & 
   \multicolumn{1}{S{RowHeader}{l}}{Observation Length:} & 
   \multicolumn{1}{S{Data}{l}}{288}
\\
   \multicolumn{1}{S{RowHeader}{l}}{Last Modified:} & 
   \multicolumn{1}{S{Data}{l}}{18:40 Thursday, May 16, 2002} & 
   \multicolumn{1}{S{RowHeader}{l}}{Deleted Observations:} & 
   \multicolumn{1}{S{Data}{l}}{0}
\\
   \multicolumn{1}{S{RowHeader}{l}}{Protection:} & 
   \multicolumn{1}{S{Data}{l}}{ } & 
   \multicolumn{1}{S{RowHeader}{l}}{Compressed:} & 
   \multicolumn{1}{S{Data}{l}}{NO}
\\
   \multicolumn{1}{S{RowHeader}{l}}{Data Set Type:} & 
   \multicolumn{1}{S{Data}{l}}{ } & 
   \multicolumn{1}{S{RowHeader}{l}}{Sorted:} & 
   \multicolumn{1}{S{Data}{l}}{NO}
\\
   \multicolumn{1}{S{RowHeader}{l}}{Label:} & 
   \multicolumn{1}{S{Data}{l}}{ } & 
   \multicolumn{1}{S{RowHeader}{l}}{ } & 
   \multicolumn{1}{S{Data}{l}}{ }
\\
\end{longtable}
\end{center}
\begin{center}\begin{longtable}
{rllrrl}\hline % colspecs
% table_head start
   \multicolumn{6}{S{Header}{c}}{-----Variables Ordered by Position-----}
\\
   \multicolumn{1}{S{Header}{r}}{\#} & 
   \multicolumn{1}{S{Header}{l}}{Variable} & 
   \multicolumn{1}{S{Header}{l}}{Type} & 
   \multicolumn{1}{S{Header}{r}}{Len} & 
   \multicolumn{1}{S{Header}{r}}{Pos} & 
   \multicolumn{1}{S{Header}{l}}{Label}
\\
\hline 
\endhead % table_head end
\hline 
\multicolumn{1}{r}{(cont.)}\\
\endfoot 
\hline 
\endlastfoot % table_head end
   \multicolumn{1}{S{RowHeader}{r}}{1} & 
   \multicolumn{1}{S{Data}{l}}{state} & 
   \multicolumn{1}{S{Data}{l}}{Char} & 
   \multicolumn{1}{S{Data}{r}}{2} & 
   \multicolumn{1}{S{Data}{r}}{216} & 
   \multicolumn{1}{S{Data}{l}}{FIPS State}
\\
   \multicolumn{1}{S{RowHeader}{r}}{2} & 
   \multicolumn{1}{S{Data}{l}}{year} & 
   \multicolumn{1}{S{Data}{l}}{Num} & 
   \multicolumn{1}{S{Data}{r}}{3} & 
   \multicolumn{1}{S{Data}{r}}{275} & 
   \multicolumn{1}{S{Data}{l}}{Year}
\\
   \multicolumn{1}{S{RowHeader}{r}}{3} & 
   \multicolumn{1}{S{Data}{l}}{quarter} & 
   \multicolumn{1}{S{Data}{l}}{Num} & 
   \multicolumn{1}{S{Data}{r}}{3} & 
   \multicolumn{1}{S{Data}{r}}{278} & 
   \multicolumn{1}{S{Data}{l}}{Quarter}
\\
   \multicolumn{1}{S{RowHeader}{r}}{4} & 
   \multicolumn{1}{S{Data}{l}}{county} & 
   \multicolumn{1}{S{Data}{l}}{Char} & 
   \multicolumn{1}{S{Data}{r}}{3} & 
   \multicolumn{1}{S{Data}{r}}{218} & 
   \multicolumn{1}{S{Data}{l}}{FIPS county}
\\
   \multicolumn{1}{S{RowHeader}{r}}{5} & 
   \multicolumn{1}{S{Data}{l}}{sex} & 
   \multicolumn{1}{S{Data}{l}}{Num} & 
   \multicolumn{1}{S{Data}{r}}{3} & 
   \multicolumn{1}{S{Data}{r}}{281} & 
   \multicolumn{1}{S{Data}{l}}{Sex}
\\
   \multicolumn{1}{S{RowHeader}{r}}{6} & 
   \multicolumn{1}{S{Data}{l}}{agegroup} & 
   \multicolumn{1}{S{Data}{l}}{Num} & 
   \multicolumn{1}{S{Data}{r}}{3} & 
   \multicolumn{1}{S{Data}{r}}{284} & 
   \multicolumn{1}{S{Data}{l}}{Age group}
\\
   \multicolumn{1}{S{RowHeader}{r}}{7} & 
   \multicolumn{1}{S{Data}{l}}{A} & 
   \multicolumn{1}{S{Data}{l}}{Num} & 
   \multicolumn{1}{S{Data}{r}}{8} & 
   \multicolumn{1}{S{Data}{r}}{0} & 
   \multicolumn{1}{S{Data}{l}}{Accessions}
\\
   \multicolumn{1}{S{RowHeader}{r}}{8} & 
   \multicolumn{1}{S{Data}{l}}{B} & 
   \multicolumn{1}{S{Data}{l}}{Num} & 
   \multicolumn{1}{S{Data}{r}}{8} & 
   \multicolumn{1}{S{Data}{r}}{8} & 
   \multicolumn{1}{S{Data}{l}}{Beginning-of-period employment}
\\
   \multicolumn{1}{S{RowHeader}{r}}{9} & 
   \multicolumn{1}{S{Data}{l}}{E} & 
   \multicolumn{1}{S{Data}{l}}{Num} & 
   \multicolumn{1}{S{Data}{r}}{8} & 
   \multicolumn{1}{S{Data}{r}}{16} & 
   \multicolumn{1}{S{Data}{l}}{End-of-period employment}
\\
   \multicolumn{1}{S{RowHeader}{r}}{10} & 
   \multicolumn{1}{S{Data}{l}}{F} & 
   \multicolumn{1}{S{Data}{l}}{Num} & 
   \multicolumn{1}{S{Data}{r}}{8} & 
   \multicolumn{1}{S{Data}{r}}{24} & 
   \multicolumn{1}{S{Data}{l}}{Full-quarter employment}
\\
   \multicolumn{1}{S{RowHeader}{r}}{11} & 
   \multicolumn{1}{S{Data}{l}}{FA} & 
   \multicolumn{1}{S{Data}{l}}{Num} & 
   \multicolumn{1}{S{Data}{r}}{8} & 
   \multicolumn{1}{S{Data}{r}}{32} & 
   \multicolumn{1}{S{Data}{l}}{Flow into full-quarter employment}
\\
   \multicolumn{1}{S{RowHeader}{r}}{12} & 
   \multicolumn{1}{S{Data}{l}}{FJC} & 
   \multicolumn{1}{S{Data}{l}}{Num} & 
   \multicolumn{1}{S{Data}{r}}{8} & 
   \multicolumn{1}{S{Data}{r}}{40} & 
   \multicolumn{1}{S{Data}{l}}{Full-quarter job creation}
\\
   \multicolumn{1}{S{RowHeader}{r}}{13} & 
   \multicolumn{1}{S{Data}{l}}{FJD} & 
   \multicolumn{1}{S{Data}{l}}{Num} & 
   \multicolumn{1}{S{Data}{r}}{8} & 
   \multicolumn{1}{S{Data}{r}}{48} & 
   \multicolumn{1}{S{Data}{l}}{Full-quarter job destruction}
\\
   \multicolumn{1}{S{RowHeader}{r}}{14} & 
   \multicolumn{1}{S{Data}{l}}{FJF} & 
   \multicolumn{1}{S{Data}{l}}{Num} & 
   \multicolumn{1}{S{Data}{r}}{8} & 
   \multicolumn{1}{S{Data}{r}}{56} & 
   \multicolumn{1}{S{Data}{l}}{Net change in full-quarter employment}
\\
   \multicolumn{1}{S{RowHeader}{r}}{15} & 
   \multicolumn{1}{S{Data}{l}}{FS} & 
   \multicolumn{1}{S{Data}{l}}{Num} & 
   \multicolumn{1}{S{Data}{r}}{8} & 
   \multicolumn{1}{S{Data}{r}}{64} & 
   \multicolumn{1}{S{Data}{l}}{Flow out of full-quarter employment}
\\
   \multicolumn{1}{S{RowHeader}{r}}{16} & 
   \multicolumn{1}{S{Data}{l}}{H} & 
   \multicolumn{1}{S{Data}{l}}{Num} & 
   \multicolumn{1}{S{Data}{r}}{8} & 
   \multicolumn{1}{S{Data}{r}}{72} & 
   \multicolumn{1}{S{Data}{l}}{New hires}
\\
   \multicolumn{1}{S{RowHeader}{r}}{17} & 
   \multicolumn{1}{S{Data}{l}}{H3} & 
   \multicolumn{1}{S{Data}{l}}{Num} & 
   \multicolumn{1}{S{Data}{r}}{8} & 
   \multicolumn{1}{S{Data}{r}}{80} & 
   \multicolumn{1}{S{Data}{l}}{Full-quarter new hires}
\\
   \multicolumn{1}{S{RowHeader}{r}}{18} & 
   \multicolumn{1}{S{Data}{l}}{JC} & 
   \multicolumn{1}{S{Data}{l}}{Num} & 
   \multicolumn{1}{S{Data}{r}}{8} & 
   \multicolumn{1}{S{Data}{r}}{88} & 
   \multicolumn{1}{S{Data}{l}}{Job creation}
\\
   \multicolumn{1}{S{RowHeader}{r}}{19} & 
   \multicolumn{1}{S{Data}{l}}{JD} & 
   \multicolumn{1}{S{Data}{l}}{Num} & 
   \multicolumn{1}{S{Data}{r}}{8} & 
   \multicolumn{1}{S{Data}{r}}{96} & 
   \multicolumn{1}{S{Data}{l}}{Job destruction}
\\
   \multicolumn{1}{S{RowHeader}{r}}{20} & 
   \multicolumn{1}{S{Data}{l}}{JF} & 
   \multicolumn{1}{S{Data}{l}}{Num} & 
   \multicolumn{1}{S{Data}{r}}{8} & 
   \multicolumn{1}{S{Data}{r}}{104} & 
   \multicolumn{1}{S{Data}{l}}{Net job flows}
\\
   \multicolumn{1}{S{RowHeader}{r}}{21} & 
   \multicolumn{1}{S{Data}{l}}{R} & 
   \multicolumn{1}{S{Data}{l}}{Num} & 
   \multicolumn{1}{S{Data}{r}}{8} & 
   \multicolumn{1}{S{Data}{r}}{112} & 
   \multicolumn{1}{S{Data}{l}}{Recalls}
\\
   \multicolumn{1}{S{RowHeader}{r}}{22} & 
   \multicolumn{1}{S{Data}{l}}{S} & 
   \multicolumn{1}{S{Data}{l}}{Num} & 
   \multicolumn{1}{S{Data}{r}}{8} & 
   \multicolumn{1}{S{Data}{r}}{120} & 
   \multicolumn{1}{S{Data}{l}}{Separations}
\\
   \multicolumn{1}{S{RowHeader}{r}}{23} & 
   \multicolumn{1}{S{Data}{l}}{Z{\textunderscore}NA} & 
   \multicolumn{1}{S{Data}{l}}{Num} & 
   \multicolumn{1}{S{Data}{r}}{8} & 
   \multicolumn{1}{S{Data}{r}}{128} & 
   \multicolumn{1}{S{Data}{l}}{Average periods of non-employment for accessions}
\\
   \multicolumn{1}{S{RowHeader}{r}}{24} & 
   \multicolumn{1}{S{Data}{l}}{Z{\textunderscore}NH} & 
   \multicolumn{1}{S{Data}{l}}{Num} & 
   \multicolumn{1}{S{Data}{r}}{8} & 
   \multicolumn{1}{S{Data}{r}}{136} & 
   \multicolumn{1}{S{Data}{l}}{Average periods of non-employment for new hires}
\\
   \multicolumn{1}{S{RowHeader}{r}}{25} & 
   \multicolumn{1}{S{Data}{l}}{Z{\textunderscore}NR} & 
   \multicolumn{1}{S{Data}{l}}{Num} & 
   \multicolumn{1}{S{Data}{r}}{8} & 
   \multicolumn{1}{S{Data}{r}}{144} & 
   \multicolumn{1}{S{Data}{l}}{Average periods of non-employment for recalls}
\\
   \multicolumn{1}{S{RowHeader}{r}}{26} & 
   \multicolumn{1}{S{Data}{l}}{Z{\textunderscore}NS} & 
   \multicolumn{1}{S{Data}{l}}{Num} & 
   \multicolumn{1}{S{Data}{r}}{8} & 
   \multicolumn{1}{S{Data}{r}}{152} & 
   \multicolumn{1}{S{Data}{l}}{Average periods of non-employment for separations}
\\
   \multicolumn{1}{S{RowHeader}{r}}{27} & 
   \multicolumn{1}{S{Data}{l}}{Z{\textunderscore}W2} & 
   \multicolumn{1}{S{Data}{l}}{Num} & 
   \multicolumn{1}{S{Data}{r}}{8} & 
   \multicolumn{1}{S{Data}{r}}{160} & 
   \multicolumn{1}{S{Data}{l}}{Average earnings of end-of-period employees}
\\
   \multicolumn{1}{S{RowHeader}{r}}{28} & 
   \multicolumn{1}{S{Data}{l}}{Z{\textunderscore}W3} & 
   \multicolumn{1}{S{Data}{l}}{Num} & 
   \multicolumn{1}{S{Data}{r}}{8} & 
   \multicolumn{1}{S{Data}{r}}{168} & 
   \multicolumn{1}{S{Data}{l}}{Average earnings of full-quarter employees}
\\
   \multicolumn{1}{S{RowHeader}{r}}{29} & 
   \multicolumn{1}{S{Data}{l}}{Z{\textunderscore}WFA} & 
   \multicolumn{1}{S{Data}{l}}{Num} & 
   \multicolumn{1}{S{Data}{r}}{8} & 
   \multicolumn{1}{S{Data}{r}}{176} & 
   \multicolumn{1}{S{Data}{l}}{Average earnings of transits to full-quarter status}
\\
   \multicolumn{1}{S{RowHeader}{r}}{30} & 
   \multicolumn{1}{S{Data}{l}}{Z{\textunderscore}WFS} & 
   \multicolumn{1}{S{Data}{l}}{Num} & 
   \multicolumn{1}{S{Data}{r}}{8} & 
   \multicolumn{1}{S{Data}{r}}{184} & 
   \multicolumn{1}{S{Data}{l}}{Average earnings of separations from full-quarter status}
\\
   \multicolumn{1}{S{RowHeader}{r}}{31} & 
   \multicolumn{1}{S{Data}{l}}{Z{\textunderscore}WH3} & 
   \multicolumn{1}{S{Data}{l}}{Num} & 
   \multicolumn{1}{S{Data}{r}}{8} & 
   \multicolumn{1}{S{Data}{r}}{192} & 
   \multicolumn{1}{S{Data}{l}}{Average earnings of full-quarter new hires}
\\
   \multicolumn{1}{S{RowHeader}{r}}{32} & 
   \multicolumn{1}{S{Data}{l}}{Z{\textunderscore}dWA} & 
   \multicolumn{1}{S{Data}{l}}{Num} & 
   \multicolumn{1}{S{Data}{r}}{8} & 
   \multicolumn{1}{S{Data}{r}}{200} & 
   \multicolumn{1}{S{Data}{l}}{Average change in total earnings for accessions}
\\
   \multicolumn{1}{S{RowHeader}{r}}{33} & 
   \multicolumn{1}{S{Data}{l}}{Z{\textunderscore}dWS} & 
   \multicolumn{1}{S{Data}{l}}{Num} & 
   \multicolumn{1}{S{Data}{r}}{8} & 
   \multicolumn{1}{S{Data}{r}}{208} & 
   \multicolumn{1}{S{Data}{l}}{Average change in total earnings for separations}
\\
   \multicolumn{1}{S{RowHeader}{r}}{34} & 
   \multicolumn{1}{S{Data}{l}}{A{\textunderscore}status} & 
   \multicolumn{1}{S{Data}{l}}{Char} & 
   \multicolumn{1}{S{Data}{r}}{2} & 
   \multicolumn{1}{S{Data}{r}}{221} & 
   \multicolumn{1}{S{Data}{l}}{Status: accessions}
\\
   \multicolumn{1}{S{RowHeader}{r}}{35} & 
   \multicolumn{1}{S{Data}{l}}{B{\textunderscore}status} & 
   \multicolumn{1}{S{Data}{l}}{Char} & 
   \multicolumn{1}{S{Data}{r}}{2} & 
   \multicolumn{1}{S{Data}{r}}{223} & 
   \multicolumn{1}{S{Data}{l}}{Status: beginning-of-period employment}
\\
   \multicolumn{1}{S{RowHeader}{r}}{36} & 
   \multicolumn{1}{S{Data}{l}}{E{\textunderscore}status} & 
   \multicolumn{1}{S{Data}{l}}{Char} & 
   \multicolumn{1}{S{Data}{r}}{2} & 
   \multicolumn{1}{S{Data}{r}}{225} & 
   \multicolumn{1}{S{Data}{l}}{Status: end-of-period employment}
\\
   \multicolumn{1}{S{RowHeader}{r}}{37} & 
   \multicolumn{1}{S{Data}{l}}{F{\textunderscore}status} & 
   \multicolumn{1}{S{Data}{l}}{Char} & 
   \multicolumn{1}{S{Data}{r}}{2} & 
   \multicolumn{1}{S{Data}{r}}{227} & 
   \multicolumn{1}{S{Data}{l}}{Status: full-quarter employment}
\\
   \multicolumn{1}{S{RowHeader}{r}}{38} & 
   \multicolumn{1}{S{Data}{l}}{FA{\textunderscore}status} & 
   \multicolumn{1}{S{Data}{l}}{Char} & 
   \multicolumn{1}{S{Data}{r}}{2} & 
   \multicolumn{1}{S{Data}{r}}{229} & 
   \multicolumn{1}{S{Data}{l}}{Status: flow into full-quarter employment}
\\
   \multicolumn{1}{S{RowHeader}{r}}{39} & 
   \multicolumn{1}{S{Data}{l}}{FJC{\textunderscore}status} & 
   \multicolumn{1}{S{Data}{l}}{Char} & 
   \multicolumn{1}{S{Data}{r}}{2} & 
   \multicolumn{1}{S{Data}{r}}{231} & 
   \multicolumn{1}{S{Data}{l}}{Status: full-quarter job creation}
\\
   \multicolumn{1}{S{RowHeader}{r}}{40} & 
   \multicolumn{1}{S{Data}{l}}{FJD{\textunderscore}status} & 
   \multicolumn{1}{S{Data}{l}}{Char} & 
   \multicolumn{1}{S{Data}{r}}{2} & 
   \multicolumn{1}{S{Data}{r}}{233} & 
   \multicolumn{1}{S{Data}{l}}{Status: full-quarter job destruction}
\\
   \multicolumn{1}{S{RowHeader}{r}}{41} & 
   \multicolumn{1}{S{Data}{l}}{FJF{\textunderscore}status} & 
   \multicolumn{1}{S{Data}{l}}{Char} & 
   \multicolumn{1}{S{Data}{r}}{2} & 
   \multicolumn{1}{S{Data}{r}}{235} & 
   \multicolumn{1}{S{Data}{l}}{Status: net change in full-quarter employment}
\\
   \multicolumn{1}{S{RowHeader}{r}}{42} & 
   \multicolumn{1}{S{Data}{l}}{FS{\textunderscore}status} & 
   \multicolumn{1}{S{Data}{l}}{Char} & 
   \multicolumn{1}{S{Data}{r}}{2} & 
   \multicolumn{1}{S{Data}{r}}{237} & 
   \multicolumn{1}{S{Data}{l}}{Status: flow out of full-quarter employment}
\\
   \multicolumn{1}{S{RowHeader}{r}}{43} & 
   \multicolumn{1}{S{Data}{l}}{H{\textunderscore}status} & 
   \multicolumn{1}{S{Data}{l}}{Char} & 
   \multicolumn{1}{S{Data}{r}}{2} & 
   \multicolumn{1}{S{Data}{r}}{239} & 
   \multicolumn{1}{S{Data}{l}}{Status: new hires}
\\
   \multicolumn{1}{S{RowHeader}{r}}{44} & 
   \multicolumn{1}{S{Data}{l}}{H3{\textunderscore}status} & 
   \multicolumn{1}{S{Data}{l}}{Char} & 
   \multicolumn{1}{S{Data}{r}}{2} & 
   \multicolumn{1}{S{Data}{r}}{241} & 
   \multicolumn{1}{S{Data}{l}}{Status: full-quarter new hires}
\\
   \multicolumn{1}{S{RowHeader}{r}}{45} & 
   \multicolumn{1}{S{Data}{l}}{JC{\textunderscore}status} & 
   \multicolumn{1}{S{Data}{l}}{Char} & 
   \multicolumn{1}{S{Data}{r}}{2} & 
   \multicolumn{1}{S{Data}{r}}{243} & 
   \multicolumn{1}{S{Data}{l}}{Status: job creation}
\\
   \multicolumn{1}{S{RowHeader}{r}}{46} & 
   \multicolumn{1}{S{Data}{l}}{JD{\textunderscore}status} & 
   \multicolumn{1}{S{Data}{l}}{Char} & 
   \multicolumn{1}{S{Data}{r}}{2} & 
   \multicolumn{1}{S{Data}{r}}{245} & 
   \multicolumn{1}{S{Data}{l}}{Status: job destruction}
\\
   \multicolumn{1}{S{RowHeader}{r}}{47} & 
   \multicolumn{1}{S{Data}{l}}{JF{\textunderscore}status} & 
   \multicolumn{1}{S{Data}{l}}{Char} & 
   \multicolumn{1}{S{Data}{r}}{2} & 
   \multicolumn{1}{S{Data}{r}}{247} & 
   \multicolumn{1}{S{Data}{l}}{Status: net job flows}
\\
   \multicolumn{1}{S{RowHeader}{r}}{48} & 
   \multicolumn{1}{S{Data}{l}}{R{\textunderscore}status} & 
   \multicolumn{1}{S{Data}{l}}{Char} & 
   \multicolumn{1}{S{Data}{r}}{2} & 
   \multicolumn{1}{S{Data}{r}}{249} & 
   \multicolumn{1}{S{Data}{l}}{Status: recalls}
\\
   \multicolumn{1}{S{RowHeader}{r}}{49} & 
   \multicolumn{1}{S{Data}{l}}{S{\textunderscore}status} & 
   \multicolumn{1}{S{Data}{l}}{Char} & 
   \multicolumn{1}{S{Data}{r}}{2} & 
   \multicolumn{1}{S{Data}{r}}{251} & 
   \multicolumn{1}{S{Data}{l}}{Status: separations}
\\
   \multicolumn{1}{S{RowHeader}{r}}{50} & 
   \multicolumn{1}{S{Data}{l}}{Z{\textunderscore}NA{\textunderscore}status} & 
   \multicolumn{1}{S{Data}{l}}{Char} & 
   \multicolumn{1}{S{Data}{r}}{2} & 
   \multicolumn{1}{S{Data}{r}}{253} & 
   \multicolumn{1}{S{Data}{l}}{Status: average periods of non-employment for accessions}
\\
   \multicolumn{1}{S{RowHeader}{r}}{51} & 
   \multicolumn{1}{S{Data}{l}}{Z{\textunderscore}NH{\textunderscore}status} & 
   \multicolumn{1}{S{Data}{l}}{Char} & 
   \multicolumn{1}{S{Data}{r}}{2} & 
   \multicolumn{1}{S{Data}{r}}{255} & 
   \multicolumn{1}{S{Data}{l}}{Status: average periods of non-employment for new hires}
\\
   \multicolumn{1}{S{RowHeader}{r}}{52} & 
   \multicolumn{1}{S{Data}{l}}{Z{\textunderscore}NR{\textunderscore}status} & 
   \multicolumn{1}{S{Data}{l}}{Char} & 
   \multicolumn{1}{S{Data}{r}}{2} & 
   \multicolumn{1}{S{Data}{r}}{257} & 
   \multicolumn{1}{S{Data}{l}}{Status: average periods of non-employment for recalls}
\\
   \multicolumn{1}{S{RowHeader}{r}}{53} & 
   \multicolumn{1}{S{Data}{l}}{Z{\textunderscore}NS{\textunderscore}status} & 
   \multicolumn{1}{S{Data}{l}}{Char} & 
   \multicolumn{1}{S{Data}{r}}{2} & 
   \multicolumn{1}{S{Data}{r}}{259} & 
   \multicolumn{1}{S{Data}{l}}{Status: average periods of non-employment for separations}
\\
   \multicolumn{1}{S{RowHeader}{r}}{54} & 
   \multicolumn{1}{S{Data}{l}}{Z{\textunderscore}W2{\textunderscore}status} & 
   \multicolumn{1}{S{Data}{l}}{Char} & 
   \multicolumn{1}{S{Data}{r}}{2} & 
   \multicolumn{1}{S{Data}{r}}{261} & 
   \multicolumn{1}{S{Data}{l}}{Status: average earnings of end-of-period employees}
\\
   \multicolumn{1}{S{RowHeader}{r}}{55} & 
   \multicolumn{1}{S{Data}{l}}{Z{\textunderscore}W3{\textunderscore}status} & 
   \multicolumn{1}{S{Data}{l}}{Char} & 
   \multicolumn{1}{S{Data}{r}}{2} & 
   \multicolumn{1}{S{Data}{r}}{263} & 
   \multicolumn{1}{S{Data}{l}}{Status: average earnings of full-quarter employees}
\\
   \multicolumn{1}{S{RowHeader}{r}}{56} & 
   \multicolumn{1}{S{Data}{l}}{Z{\textunderscore}WFA{\textunderscore}status} & 
   \multicolumn{1}{S{Data}{l}}{Char} & 
   \multicolumn{1}{S{Data}{r}}{2} & 
   \multicolumn{1}{S{Data}{r}}{265} & 
   \multicolumn{1}{S{Data}{l}}{Status: average earnings of transits to full-quarter status}
\\
   \multicolumn{1}{S{RowHeader}{r}}{57} & 
   \multicolumn{1}{S{Data}{l}}{Z{\textunderscore}WFS{\textunderscore}status} & 
   \multicolumn{1}{S{Data}{l}}{Char} & 
   \multicolumn{1}{S{Data}{r}}{2} & 
   \multicolumn{1}{S{Data}{r}}{267} & 
   \multicolumn{1}{S{Data}{l}}{Status: average earnings of separations from full-quarter status}
\\
   \multicolumn{1}{S{RowHeader}{r}}{58} & 
   \multicolumn{1}{S{Data}{l}}{Z{\textunderscore}WH3{\textunderscore}status} & 
   \multicolumn{1}{S{Data}{l}}{Char} & 
   \multicolumn{1}{S{Data}{r}}{2} & 
   \multicolumn{1}{S{Data}{r}}{269} & 
   \multicolumn{1}{S{Data}{l}}{Status: average earnings of full-quarter new hires}
\\
   \multicolumn{1}{S{RowHeader}{r}}{59} & 
   \multicolumn{1}{S{Data}{l}}{Z{\textunderscore}dWA{\textunderscore}status} & 
   \multicolumn{1}{S{Data}{l}}{Char} & 
   \multicolumn{1}{S{Data}{r}}{2} & 
   \multicolumn{1}{S{Data}{r}}{271} & 
   \multicolumn{1}{S{Data}{l}}{Status: average change in total earnings for accessions}
\\
   \multicolumn{1}{S{RowHeader}{r}}{60} & 
   \multicolumn{1}{S{Data}{l}}{Z{\textunderscore}dWS{\textunderscore}status} & 
   \multicolumn{1}{S{Data}{l}}{Char} & 
   \multicolumn{1}{S{Data}{r}}{2} & 
   \multicolumn{1}{S{Data}{r}}{273} & 
   \multicolumn{1}{S{Data}{l}}{Status: average change in total earnings for separations}
\\
\end{longtable}
\end{center}

% ====================end of output====================
 \newpage %    Generated by SAS
%    http://www.sas.com
% created by=vilhu001
% sasversion=8.2
% date=2002-05-23
% time=00:50:04
% encoding=iso-8859-1
% ====================begin of output====================
% \begin{document}

% An external file needs to be included, as specified% in latexlong.sas. This can be called sas.sty,
% in which case you want to include a line like
% \usepackage{sas}
% or it can be a simple (La)TeX file, which you 
% include by typing 
% %%
%% This is file `sas.sty',
%% generated with the docstrip utility.
%%
%% 
\NeedsTeXFormat{LaTeX2e}
\ProvidesPackage{sas}
        [2002/01/18 LEHD version 0.1
    provides definition for tables generated by SAS%
                   ]
\@ifundefined{array@processline}{\RequirePackage{array}}{}
\@ifundefined{longtable@processline}{\RequirePackage{longtable}}{}
 \def\ContentTitle{\small\it\sffamily}
 \def\Output{\small\sffamily}
 \def\HeaderEmphasis{\small\it\sffamily}
 \def\NoteContent{\small\sffamily}
 \def\FatalContent{\small\sffamily}
 \def\Graph{\small\sffamily}
 \def\WarnContentFixed{\footnotesize\tt}
 \def\NoteBanner{\small\sffamily}
 \def\DataStrong{\normalsize\bf\sffamily}
 \def\Document{\small\sffamily}
 \def\BeforeCaption{\normalsize\bf\sffamily}
 \def\ContentsDate{\small\sffamily}
 \def\Pages{\small\sffamily}
 \def\TitlesAndFooters{\footnotesize\bf\it\sffamily}
 \def\IndexProcName{\small\sffamily}
 \def\ProcTitle{\normalsize\bf\it\sffamily}
 \def\IndexAction{\small\sffamily}
 \def\Data{\small\sffamily}
 \def\Table{\small\sffamily}
 \def\FooterEmpty{\footnotesize\bf\sffamily}
 \def\SysTitleAndFooterContainer{\footnotesize\sffamily}
 \def\RowFooterEmpty{\footnotesize\bf\sffamily}
 \def\ExtendedPage{\small\it\sffamily}
 \def\FooterFixed{\footnotesize\tt}
 \def\RowFooterStrongFixed{\footnotesize\bf\tt}
 \def\RowFooterEmphasis{\footnote\it\sffamily}
 \def\ContentFolder{\small\sffamily}
 \def\Container{\small\sffamily}
 \def\Date{\small\sffamily}
 \def\RowFooterFixed{\footnotesize\tt}
 \def\Caption{\normalsize\bf\sffamily}
 \def\WarnBanner{\small\sffamily}
 \def\Frame{\small\sffamily}
 \def\HeaderStrongFixed{\footnotesize\bf\tt}
 \def\IndexTitle{\small\it\sffamily}
 \def\NoteContentFixed{\footnotesize\tt}
 \def\DataEmphasisFixed{\footnotesize\it\tt}
 \def\Note{\small\sffamily}
 \def\Byline{\normalsize\bf\sffamily}
 \def\FatalBanner{\small\sffamily}
 \def\ProcTitleFixed{\footnotesize\bf\tt}
 \def\ByContentFolder{\small\sffamily}
 \def\PagesProcLabel{\small\sffamily}
 \def\RowHeaderFixed{\footnotesize\tt}
 \def\RowFooterEmphasisFixed{\footnotesize\it\tt}
 \def\WarnContent{\small\sffamily}
 \def\DataEmpty{\small\sffamily}
 \def\Cell{\small\sffamily}
 \def\Header{\normalsize\bf\sffamily}
 \def\PageNo{\normalsize\bf\sffamily}
 \def\ContentProcLabel{\small\sffamily}
 \def\HeaderFixed{\footnotesize\tt}
 \def\PagesTitle{\small\it\sffamily}
 \def\RowHeaderEmpty{\normalsize\bf\sffamily}
 \def\PagesProcName{\small\sffamily}
 \def\Batch{\footnotesize\tt}
 \def\ContentItem{\small\sffamily}
 \def\Body{\small\sffamily}
 \def\PagesDate{\small\sffamily}
 \def\Index{\small\sffamily}
 \def\HeaderEmpty{\normalsize\bf\sffamily}
 \def\FooterStrong{\footnotesize\bf\sffamily}
 \def\FooterEmphasis{\footnotesize\it\sffamily}
 \def\ErrorContent{\small\sffamily}
 \def\DataFixed{\footnotesize\tt}
 \def\HeaderStrong{\normalsize\bf\sffamily}
 \def\GraphBackground{}
 \def\DataEmphasis{\small\it\sffamily}
 \def\TitleAndNoteContainer{\small\sffamily}
 \def\RowFooter{\footnotesize\bf\sffamily}
 \def\IndexItem{\small\sffamily}
 \def\BylineContainer{\small\sffamily}
 \def\FatalContentFixed{\footnotesize\tt}
 \def\BodyDate{\small\sffamily}
 \def\RowFooterStrong{\footnotesize\bf\sffamily}
 \def\UserText{\small\sffamily}
 \def\HeadersAndFooters{\footnotesize\bf\sffamily}
 \def\RowHeaderEmphasisFixed{\footnotesize\it\tt}
 \def\ErrorBanner{\small\sffamily}
 \def\ContentProcName{\small\sffamily}
 \def\RowHeaderStrong{\normalsize\bf\sffamily}
 \def\FooterEmphasisFixed{\footnotesize\it\tt}
 \def\Contents{\small\sffamily}
 \def\FooterStrongFixed{\footnotesize\bf\tt}
 \def\PagesItem{\small\sffamily}
 \def\RowHeader{\normalsize\bf\sffamily}
 \def\AfterCaption{\normalsize\bf\sffamily}
 \def\RowHeaderStrongFixed{\footnotesize\bf\tt}
 \def\RowHeaderEmphasis{\small\it\sffamily}
 \def\DataStrongFixed{\footnotesize\bf\tt}
 \def\Footer{\footnotesize\bf\sffamily}
 \def\FolderAction{\small\sffamily}
 \def\HeaderEmphasisFixed{\footnotesize\it\tt}
 \def\SystemTitle{\large\bf\it\sffamily}
 \def\ErrorContentFixed{\footnotesize\tt}
 \def\SystemFooter{\footnotesize\it\sffamily}
% Set cell padding 
\renewcommand{\arraystretch}{1.3}
% Headings
\newcommand{\heading}[2]{\csname#1\endcsname #2}
\newcommand{\proctitle}[2]{\csname#1\endcsname #2}
% Declare new column type
\newcolumntype{S}[2]{>{\csname#1\endcsname}#2}
% Set warning box style
\newcommand{\msg}[2]{\fbox{%
   \begin{minipage}{\textwidth}#2\end{minipage}}%
}

\begin{center}\heading{SystemTitle}{Maryland            }\end{center}
\begin{center}\heading{ProcTitle}{The FREQ Procedure}\end{center}
\begin{center}\begin{longtable}
{lrrrr}\hline % colspecs
% table_head start
   \multicolumn{5}{S{Header}{c}}{FIPS State}
\\
   \multicolumn{1}{S{Header}{l}}{state} & 
   \multicolumn{1}{S{Header}{r}}{Frequency} & 
   \multicolumn{1}{S{Header}{r}}{ Percent} & 
   \multicolumn{1}{S{Header}{r}}{Cumulative\linebreak  Frequency} & 
   \multicolumn{1}{S{Header}{r}}{Cumulative\linebreak   Percent}
\\
\hline 
\endhead % table_head end
\hline 
\multicolumn{1}{r}{(cont.)}\\
\endfoot 
\hline 
\endlastfoot % table_head end
   \multicolumn{1}{S{RowHeader}{l}}{24 MARYLAND} & 
   \multicolumn{1}{S{Data}{r}}{26271} & 
   \multicolumn{1}{S{Data}{r}}{100.00} & 
   \multicolumn{1}{S{Data}{r}}{26271} & 
   \multicolumn{1}{S{Data}{r}}{100.00}
\\
\end{longtable}
\end{center}
\begin{center}\begin{longtable}
{lrrrr}\hline % colspecs
% table_head start
   \multicolumn{5}{S{Header}{c}}{FIPS county}
\\
   \multicolumn{1}{S{Header}{l}}{county} & 
   \multicolumn{1}{S{Header}{r}}{Frequency} & 
   \multicolumn{1}{S{Header}{r}}{ Percent} & 
   \multicolumn{1}{S{Header}{r}}{Cumulative\linebreak  Frequency} & 
   \multicolumn{1}{S{Header}{r}}{Cumulative\linebreak   Percent}
\\
\hline 
\endhead % table_head end
\hline 
\multicolumn{1}{r}{(cont.)}\\
\endfoot 
\hline 
\endlastfoot % table_head end
   \multicolumn{1}{S{RowHeader}{l}}{000 MARYLAND} & 
   \multicolumn{1}{S{Data}{r}}{1053} & 
   \multicolumn{1}{S{Data}{r}}{4.01} & 
   \multicolumn{1}{S{Data}{r}}{1053} & 
   \multicolumn{1}{S{Data}{r}}{4.01}
\\
   \multicolumn{1}{S{RowHeader}{l}}{001 ALLEGANY} & 
   \multicolumn{1}{S{Data}{r}}{1053} & 
   \multicolumn{1}{S{Data}{r}}{4.01} & 
   \multicolumn{1}{S{Data}{r}}{2106} & 
   \multicolumn{1}{S{Data}{r}}{8.02}
\\
   \multicolumn{1}{S{RowHeader}{l}}{003 ANNE ARUNDEL} & 
   \multicolumn{1}{S{Data}{r}}{1053} & 
   \multicolumn{1}{S{Data}{r}}{4.01} & 
   \multicolumn{1}{S{Data}{r}}{3159} & 
   \multicolumn{1}{S{Data}{r}}{12.02}
\\
   \multicolumn{1}{S{RowHeader}{l}}{005 BALTIMORE} & 
   \multicolumn{1}{S{Data}{r}}{1053} & 
   \multicolumn{1}{S{Data}{r}}{4.01} & 
   \multicolumn{1}{S{Data}{r}}{4212} & 
   \multicolumn{1}{S{Data}{r}}{16.03}
\\
   \multicolumn{1}{S{RowHeader}{l}}{009 CALVERT} & 
   \multicolumn{1}{S{Data}{r}}{1053} & 
   \multicolumn{1}{S{Data}{r}}{4.01} & 
   \multicolumn{1}{S{Data}{r}}{5265} & 
   \multicolumn{1}{S{Data}{r}}{20.04}
\\
   \multicolumn{1}{S{RowHeader}{l}}{011 CAROLINE} & 
   \multicolumn{1}{S{Data}{r}}{1053} & 
   \multicolumn{1}{S{Data}{r}}{4.01} & 
   \multicolumn{1}{S{Data}{r}}{6318} & 
   \multicolumn{1}{S{Data}{r}}{24.05}
\\
   \multicolumn{1}{S{RowHeader}{l}}{013 CARROLL} & 
   \multicolumn{1}{S{Data}{r}}{1053} & 
   \multicolumn{1}{S{Data}{r}}{4.01} & 
   \multicolumn{1}{S{Data}{r}}{7371} & 
   \multicolumn{1}{S{Data}{r}}{28.06}
\\
   \multicolumn{1}{S{RowHeader}{l}}{015 CECIL} & 
   \multicolumn{1}{S{Data}{r}}{1053} & 
   \multicolumn{1}{S{Data}{r}}{4.01} & 
   \multicolumn{1}{S{Data}{r}}{8424} & 
   \multicolumn{1}{S{Data}{r}}{32.07}
\\
   \multicolumn{1}{S{RowHeader}{l}}{017 CHARLES} & 
   \multicolumn{1}{S{Data}{r}}{1053} & 
   \multicolumn{1}{S{Data}{r}}{4.01} & 
   \multicolumn{1}{S{Data}{r}}{9477} & 
   \multicolumn{1}{S{Data}{r}}{36.07}
\\
   \multicolumn{1}{S{RowHeader}{l}}{019 DORCHESTER} & 
   \multicolumn{1}{S{Data}{r}}{1053} & 
   \multicolumn{1}{S{Data}{r}}{4.01} & 
   \multicolumn{1}{S{Data}{r}}{10530} & 
   \multicolumn{1}{S{Data}{r}}{40.08}
\\
   \multicolumn{1}{S{RowHeader}{l}}{021 FRQWIRICK} & 
   \multicolumn{1}{S{Data}{r}}{1053} & 
   \multicolumn{1}{S{Data}{r}}{4.01} & 
   \multicolumn{1}{S{Data}{r}}{11583} & 
   \multicolumn{1}{S{Data}{r}}{44.09}
\\
   \multicolumn{1}{S{RowHeader}{l}}{023 GARRETT} & 
   \multicolumn{1}{S{Data}{r}}{1053} & 
   \multicolumn{1}{S{Data}{r}}{4.01} & 
   \multicolumn{1}{S{Data}{r}}{12636} & 
   \multicolumn{1}{S{Data}{r}}{48.10}
\\
   \multicolumn{1}{S{RowHeader}{l}}{025 HARFORD} & 
   \multicolumn{1}{S{Data}{r}}{1053} & 
   \multicolumn{1}{S{Data}{r}}{4.01} & 
   \multicolumn{1}{S{Data}{r}}{13689} & 
   \multicolumn{1}{S{Data}{r}}{52.11}
\\
   \multicolumn{1}{S{RowHeader}{l}}{027 HOWARD} & 
   \multicolumn{1}{S{Data}{r}}{1053} & 
   \multicolumn{1}{S{Data}{r}}{4.01} & 
   \multicolumn{1}{S{Data}{r}}{14742} & 
   \multicolumn{1}{S{Data}{r}}{56.12}
\\
   \multicolumn{1}{S{RowHeader}{l}}{029 KENT} & 
   \multicolumn{1}{S{Data}{r}}{1053} & 
   \multicolumn{1}{S{Data}{r}}{4.01} & 
   \multicolumn{1}{S{Data}{r}}{15795} & 
   \multicolumn{1}{S{Data}{r}}{60.12}
\\
   \multicolumn{1}{S{RowHeader}{l}}{031 MONTGOMERY} & 
   \multicolumn{1}{S{Data}{r}}{1053} & 
   \multicolumn{1}{S{Data}{r}}{4.01} & 
   \multicolumn{1}{S{Data}{r}}{16848} & 
   \multicolumn{1}{S{Data}{r}}{64.13}
\\
   \multicolumn{1}{S{RowHeader}{l}}{033 PRINCE GEORGE'S} & 
   \multicolumn{1}{S{Data}{r}}{1053} & 
   \multicolumn{1}{S{Data}{r}}{4.01} & 
   \multicolumn{1}{S{Data}{r}}{17901} & 
   \multicolumn{1}{S{Data}{r}}{68.14}
\\
   \multicolumn{1}{S{RowHeader}{l}}{035 QUEEN ANNE'S} & 
   \multicolumn{1}{S{Data}{r}}{1053} & 
   \multicolumn{1}{S{Data}{r}}{4.01} & 
   \multicolumn{1}{S{Data}{r}}{18954} & 
   \multicolumn{1}{S{Data}{r}}{72.15}
\\
   \multicolumn{1}{S{RowHeader}{l}}{037 SOMERSET} & 
   \multicolumn{1}{S{Data}{r}}{1053} & 
   \multicolumn{1}{S{Data}{r}}{4.01} & 
   \multicolumn{1}{S{Data}{r}}{20007} & 
   \multicolumn{1}{S{Data}{r}}{76.16}
\\
   \multicolumn{1}{S{RowHeader}{l}}{039 ST. MARY'S} & 
   \multicolumn{1}{S{Data}{r}}{999} & 
   \multicolumn{1}{S{Data}{r}}{3.80} & 
   \multicolumn{1}{S{Data}{r}}{21006} & 
   \multicolumn{1}{S{Data}{r}}{79.96}
\\
   \multicolumn{1}{S{RowHeader}{l}}{041 TALBOT} & 
   \multicolumn{1}{S{Data}{r}}{1053} & 
   \multicolumn{1}{S{Data}{r}}{4.01} & 
   \multicolumn{1}{S{Data}{r}}{22059} & 
   \multicolumn{1}{S{Data}{r}}{83.97}
\\
   \multicolumn{1}{S{RowHeader}{l}}{043 WASHINGTON} & 
   \multicolumn{1}{S{Data}{r}}{1053} & 
   \multicolumn{1}{S{Data}{r}}{4.01} & 
   \multicolumn{1}{S{Data}{r}}{23112} & 
   \multicolumn{1}{S{Data}{r}}{87.98}
\\
   \multicolumn{1}{S{RowHeader}{l}}{045 WICOMICO} & 
   \multicolumn{1}{S{Data}{r}}{1053} & 
   \multicolumn{1}{S{Data}{r}}{4.01} & 
   \multicolumn{1}{S{Data}{r}}{24165} & 
   \multicolumn{1}{S{Data}{r}}{91.98}
\\
   \multicolumn{1}{S{RowHeader}{l}}{047 WORCESTER} & 
   \multicolumn{1}{S{Data}{r}}{1053} & 
   \multicolumn{1}{S{Data}{r}}{4.01} & 
   \multicolumn{1}{S{Data}{r}}{25218} & 
   \multicolumn{1}{S{Data}{r}}{95.99}
\\
   \multicolumn{1}{S{RowHeader}{l}}{510 BALTIMORE CITY} & 
   \multicolumn{1}{S{Data}{r}}{1053} & 
   \multicolumn{1}{S{Data}{r}}{4.01} & 
   \multicolumn{1}{S{Data}{r}}{26271} & 
   \multicolumn{1}{S{Data}{r}}{100.00}
\\
\end{longtable}
\end{center}
\begin{center}\begin{longtable}
{rrrrr}\hline % colspecs
% table_head start
   \multicolumn{5}{S{Header}{c}}{Sex}
\\
   \multicolumn{1}{S{Header}{r}}{sex} & 
   \multicolumn{1}{S{Header}{r}}{Frequency} & 
   \multicolumn{1}{S{Header}{r}}{ Percent} & 
   \multicolumn{1}{S{Header}{r}}{Cumulative\linebreak  Frequency} & 
   \multicolumn{1}{S{Header}{r}}{Cumulative\linebreak   Percent}
\\
\hline 
\endhead % table_head end
\hline 
\multicolumn{1}{r}{(cont.)}\\
\endfoot 
\hline 
\endlastfoot % table_head end
   \multicolumn{1}{S{RowHeader}{r}}{0 : All} & 
   \multicolumn{1}{S{Data}{r}}{8757} & 
   \multicolumn{1}{S{Data}{r}}{33.33} & 
   \multicolumn{1}{S{Data}{r}}{8757} & 
   \multicolumn{1}{S{Data}{r}}{33.33}
\\
   \multicolumn{1}{S{RowHeader}{r}}{1 : Men} & 
   \multicolumn{1}{S{Data}{r}}{8757} & 
   \multicolumn{1}{S{Data}{r}}{33.33} & 
   \multicolumn{1}{S{Data}{r}}{17514} & 
   \multicolumn{1}{S{Data}{r}}{66.67}
\\
   \multicolumn{1}{S{RowHeader}{r}}{2 : Women} & 
   \multicolumn{1}{S{Data}{r}}{8757} & 
   \multicolumn{1}{S{Data}{r}}{33.33} & 
   \multicolumn{1}{S{Data}{r}}{26271} & 
   \multicolumn{1}{S{Data}{r}}{100.00}
\\
\end{longtable}
\end{center}
\begin{center}\begin{longtable}
{rrrrr}\hline % colspecs
% table_head start
   \multicolumn{5}{S{Header}{c}}{Age group}
\\
   \multicolumn{1}{S{Header}{r}}{agegroup} & 
   \multicolumn{1}{S{Header}{r}}{Frequency} & 
   \multicolumn{1}{S{Header}{r}}{ Percent} & 
   \multicolumn{1}{S{Header}{r}}{Cumulative\linebreak  Frequency} & 
   \multicolumn{1}{S{Header}{r}}{Cumulative\linebreak   Percent}
\\
\hline 
\endhead % table_head end
\hline 
\multicolumn{1}{r}{(cont.)}\\
\endfoot 
\hline 
\endlastfoot % table_head end
   \multicolumn{1}{S{RowHeader}{r}}{0 : All Ages} & 
   \multicolumn{1}{S{Data}{r}}{2919} & 
   \multicolumn{1}{S{Data}{r}}{11.11} & 
   \multicolumn{1}{S{Data}{r}}{2919} & 
   \multicolumn{1}{S{Data}{r}}{11.11}
\\
   \multicolumn{1}{S{RowHeader}{r}}{1 : 14-18} & 
   \multicolumn{1}{S{Data}{r}}{2919} & 
   \multicolumn{1}{S{Data}{r}}{11.11} & 
   \multicolumn{1}{S{Data}{r}}{5838} & 
   \multicolumn{1}{S{Data}{r}}{22.22}
\\
   \multicolumn{1}{S{RowHeader}{r}}{2 : 19-21} & 
   \multicolumn{1}{S{Data}{r}}{2919} & 
   \multicolumn{1}{S{Data}{r}}{11.11} & 
   \multicolumn{1}{S{Data}{r}}{8757} & 
   \multicolumn{1}{S{Data}{r}}{33.33}
\\
   \multicolumn{1}{S{RowHeader}{r}}{3 : 22-24} & 
   \multicolumn{1}{S{Data}{r}}{2919} & 
   \multicolumn{1}{S{Data}{r}}{11.11} & 
   \multicolumn{1}{S{Data}{r}}{11676} & 
   \multicolumn{1}{S{Data}{r}}{44.44}
\\
   \multicolumn{1}{S{RowHeader}{r}}{4 : 25-34} & 
   \multicolumn{1}{S{Data}{r}}{2919} & 
   \multicolumn{1}{S{Data}{r}}{11.11} & 
   \multicolumn{1}{S{Data}{r}}{14595} & 
   \multicolumn{1}{S{Data}{r}}{55.56}
\\
   \multicolumn{1}{S{RowHeader}{r}}{5 : 35-44} & 
   \multicolumn{1}{S{Data}{r}}{2919} & 
   \multicolumn{1}{S{Data}{r}}{11.11} & 
   \multicolumn{1}{S{Data}{r}}{17514} & 
   \multicolumn{1}{S{Data}{r}}{66.67}
\\
   \multicolumn{1}{S{RowHeader}{r}}{6 : 45-54} & 
   \multicolumn{1}{S{Data}{r}}{2919} & 
   \multicolumn{1}{S{Data}{r}}{11.11} & 
   \multicolumn{1}{S{Data}{r}}{20433} & 
   \multicolumn{1}{S{Data}{r}}{77.78}
\\
   \multicolumn{1}{S{RowHeader}{r}}{7 : 55-64} & 
   \multicolumn{1}{S{Data}{r}}{2919} & 
   \multicolumn{1}{S{Data}{r}}{11.11} & 
   \multicolumn{1}{S{Data}{r}}{23352} & 
   \multicolumn{1}{S{Data}{r}}{88.89}
\\
   \multicolumn{1}{S{RowHeader}{r}}{8 : 65+} & 
   \multicolumn{1}{S{Data}{r}}{2919} & 
   \multicolumn{1}{S{Data}{r}}{11.11} & 
   \multicolumn{1}{S{Data}{r}}{26271} & 
   \multicolumn{1}{S{Data}{r}}{100.00}
\\
\end{longtable}
\end{center}
\begin{center}\begin{longtable}
{llllll}\hline % colspecs
% table_head start
   \multicolumn{6}{S{Header}{c}}{Table of year by quarter}
\\
   \multicolumn{1}{S{Header}{c}}{year(Year)} & 
   \multicolumn{4}{S{Header}{c}}{quarter(Quarter)} & 
   \multicolumn{1}{S{Header}{r}}{Total}
\\
   \multicolumn{1}{l}{~} & 
   \multicolumn{1}{S{Header}{r}}{      1     } & 
   \multicolumn{1}{S{Header}{r}}{      2     } & 
   \multicolumn{1}{S{Header}{r}}{      3     } & 
   \multicolumn{1}{S{Header}{r}}{      4     } & 
   \multicolumn{1}{l}{~}
\\
\hline 
\endhead % table_head end
\hline 
\multicolumn{1}{r}{(cont.)}\\
\endfoot 
\hline 
\endlastfoot % table_head end
   \multicolumn{1}{S{Header}{r}}{1990        } & 
   \multicolumn{1}{S{Data}{r}}{   675} & 
   \multicolumn{1}{S{Data}{r}}{   675} & 
   \multicolumn{1}{S{Data}{r}}{   675} & 
   \multicolumn{1}{S{Data}{r}}{   675} & 
   \multicolumn{1}{S{Data}{r}}{  2700}
\\
   \multicolumn{1}{S{Header}{r}}{1991        } & 
   \multicolumn{1}{S{Data}{r}}{   675} & 
   \multicolumn{1}{S{Data}{r}}{   675} & 
   \multicolumn{1}{S{Data}{r}}{   675} & 
   \multicolumn{1}{S{Data}{r}}{   648} & 
   \multicolumn{1}{S{Data}{r}}{  2673}
\\
   \multicolumn{1}{S{Header}{r}}{1992        } & 
   \multicolumn{1}{S{Data}{r}}{   648} & 
   \multicolumn{1}{S{Data}{r}}{   675} & 
   \multicolumn{1}{S{Data}{r}}{   675} & 
   \multicolumn{1}{S{Data}{r}}{   675} & 
   \multicolumn{1}{S{Data}{r}}{  2673}
\\
   \multicolumn{1}{S{Header}{r}}{1993        } & 
   \multicolumn{1}{S{Data}{r}}{   675} & 
   \multicolumn{1}{S{Data}{r}}{   675} & 
   \multicolumn{1}{S{Data}{r}}{   675} & 
   \multicolumn{1}{S{Data}{r}}{   675} & 
   \multicolumn{1}{S{Data}{r}}{  2700}
\\
   \multicolumn{1}{S{Header}{r}}{1994        } & 
   \multicolumn{1}{S{Data}{r}}{   675} & 
   \multicolumn{1}{S{Data}{r}}{   675} & 
   \multicolumn{1}{S{Data}{r}}{   675} & 
   \multicolumn{1}{S{Data}{r}}{   675} & 
   \multicolumn{1}{S{Data}{r}}{  2700}
\\
   \multicolumn{1}{S{Header}{r}}{1995        } & 
   \multicolumn{1}{S{Data}{r}}{   675} & 
   \multicolumn{1}{S{Data}{r}}{   675} & 
   \multicolumn{1}{S{Data}{r}}{   675} & 
   \multicolumn{1}{S{Data}{r}}{   675} & 
   \multicolumn{1}{S{Data}{r}}{  2700}
\\
   \multicolumn{1}{S{Header}{r}}{1996        } & 
   \multicolumn{1}{S{Data}{r}}{   675} & 
   \multicolumn{1}{S{Data}{r}}{   675} & 
   \multicolumn{1}{S{Data}{r}}{   675} & 
   \multicolumn{1}{S{Data}{r}}{   675} & 
   \multicolumn{1}{S{Data}{r}}{  2700}
\\
   \multicolumn{1}{S{Header}{r}}{1997        } & 
   \multicolumn{1}{S{Data}{r}}{   675} & 
   \multicolumn{1}{S{Data}{r}}{   675} & 
   \multicolumn{1}{S{Data}{r}}{   675} & 
   \multicolumn{1}{S{Data}{r}}{   675} & 
   \multicolumn{1}{S{Data}{r}}{  2700}
\\
   \multicolumn{1}{S{Header}{r}}{1998        } & 
   \multicolumn{1}{S{Data}{r}}{   675} & 
   \multicolumn{1}{S{Data}{r}}{   675} & 
   \multicolumn{1}{S{Data}{r}}{   675} & 
   \multicolumn{1}{S{Data}{r}}{   675} & 
   \multicolumn{1}{S{Data}{r}}{  2700}
\\
   \multicolumn{1}{S{Header}{r}}{1999        } & 
   \multicolumn{1}{S{Data}{r}}{   675} & 
   \multicolumn{1}{S{Data}{r}}{   675} & 
   \multicolumn{1}{S{Data}{r}}{   675} & 
   \multicolumn{1}{S{Data}{r}}{     0} & 
   \multicolumn{1}{S{Data}{r}}{  2025}
\\
   \multicolumn{1}{S{Header}{l}}{Total           } & 
   \multicolumn{1}{S{Data}{r}}{   6723} & 
   \multicolumn{1}{S{Data}{r}}{   6750} & 
   \multicolumn{1}{S{Data}{r}}{   6750} & 
   \multicolumn{1}{S{Data}{r}}{   6048} & 
   \multicolumn{1}{S{Data}{r}}{  26271}
\\
\end{longtable}
\end{center}

% ====================end of output====================

% \newpage %    Generated by SAS
%    http://www.sas.com
% created by=vilhu001
% sasversion=8.2
% date=2002-05-23
% time=00:39:34
% encoding=iso-8859-1
% ====================begin of output====================
% \begin{document}

% An external file needs to be included, as specified% in latexlong.sas. This can be called sas.sty,
% in which case you want to include a line like
% \usepackage{sas}
% or it can be a simple (La)TeX file, which you 
% include by typing 
% %%
%% This is file `sas.sty',
%% generated with the docstrip utility.
%%
%% 
\NeedsTeXFormat{LaTeX2e}
\ProvidesPackage{sas}
        [2002/01/18 LEHD version 0.1
    provides definition for tables generated by SAS%
                   ]
\@ifundefined{array@processline}{\RequirePackage{array}}{}
\@ifundefined{longtable@processline}{\RequirePackage{longtable}}{}
 \def\ContentTitle{\small\it\sffamily}
 \def\Output{\small\sffamily}
 \def\HeaderEmphasis{\small\it\sffamily}
 \def\NoteContent{\small\sffamily}
 \def\FatalContent{\small\sffamily}
 \def\Graph{\small\sffamily}
 \def\WarnContentFixed{\footnotesize\tt}
 \def\NoteBanner{\small\sffamily}
 \def\DataStrong{\normalsize\bf\sffamily}
 \def\Document{\small\sffamily}
 \def\BeforeCaption{\normalsize\bf\sffamily}
 \def\ContentsDate{\small\sffamily}
 \def\Pages{\small\sffamily}
 \def\TitlesAndFooters{\footnotesize\bf\it\sffamily}
 \def\IndexProcName{\small\sffamily}
 \def\ProcTitle{\normalsize\bf\it\sffamily}
 \def\IndexAction{\small\sffamily}
 \def\Data{\small\sffamily}
 \def\Table{\small\sffamily}
 \def\FooterEmpty{\footnotesize\bf\sffamily}
 \def\SysTitleAndFooterContainer{\footnotesize\sffamily}
 \def\RowFooterEmpty{\footnotesize\bf\sffamily}
 \def\ExtendedPage{\small\it\sffamily}
 \def\FooterFixed{\footnotesize\tt}
 \def\RowFooterStrongFixed{\footnotesize\bf\tt}
 \def\RowFooterEmphasis{\footnote\it\sffamily}
 \def\ContentFolder{\small\sffamily}
 \def\Container{\small\sffamily}
 \def\Date{\small\sffamily}
 \def\RowFooterFixed{\footnotesize\tt}
 \def\Caption{\normalsize\bf\sffamily}
 \def\WarnBanner{\small\sffamily}
 \def\Frame{\small\sffamily}
 \def\HeaderStrongFixed{\footnotesize\bf\tt}
 \def\IndexTitle{\small\it\sffamily}
 \def\NoteContentFixed{\footnotesize\tt}
 \def\DataEmphasisFixed{\footnotesize\it\tt}
 \def\Note{\small\sffamily}
 \def\Byline{\normalsize\bf\sffamily}
 \def\FatalBanner{\small\sffamily}
 \def\ProcTitleFixed{\footnotesize\bf\tt}
 \def\ByContentFolder{\small\sffamily}
 \def\PagesProcLabel{\small\sffamily}
 \def\RowHeaderFixed{\footnotesize\tt}
 \def\RowFooterEmphasisFixed{\footnotesize\it\tt}
 \def\WarnContent{\small\sffamily}
 \def\DataEmpty{\small\sffamily}
 \def\Cell{\small\sffamily}
 \def\Header{\normalsize\bf\sffamily}
 \def\PageNo{\normalsize\bf\sffamily}
 \def\ContentProcLabel{\small\sffamily}
 \def\HeaderFixed{\footnotesize\tt}
 \def\PagesTitle{\small\it\sffamily}
 \def\RowHeaderEmpty{\normalsize\bf\sffamily}
 \def\PagesProcName{\small\sffamily}
 \def\Batch{\footnotesize\tt}
 \def\ContentItem{\small\sffamily}
 \def\Body{\small\sffamily}
 \def\PagesDate{\small\sffamily}
 \def\Index{\small\sffamily}
 \def\HeaderEmpty{\normalsize\bf\sffamily}
 \def\FooterStrong{\footnotesize\bf\sffamily}
 \def\FooterEmphasis{\footnotesize\it\sffamily}
 \def\ErrorContent{\small\sffamily}
 \def\DataFixed{\footnotesize\tt}
 \def\HeaderStrong{\normalsize\bf\sffamily}
 \def\GraphBackground{}
 \def\DataEmphasis{\small\it\sffamily}
 \def\TitleAndNoteContainer{\small\sffamily}
 \def\RowFooter{\footnotesize\bf\sffamily}
 \def\IndexItem{\small\sffamily}
 \def\BylineContainer{\small\sffamily}
 \def\FatalContentFixed{\footnotesize\tt}
 \def\BodyDate{\small\sffamily}
 \def\RowFooterStrong{\footnotesize\bf\sffamily}
 \def\UserText{\small\sffamily}
 \def\HeadersAndFooters{\footnotesize\bf\sffamily}
 \def\RowHeaderEmphasisFixed{\footnotesize\it\tt}
 \def\ErrorBanner{\small\sffamily}
 \def\ContentProcName{\small\sffamily}
 \def\RowHeaderStrong{\normalsize\bf\sffamily}
 \def\FooterEmphasisFixed{\footnotesize\it\tt}
 \def\Contents{\small\sffamily}
 \def\FooterStrongFixed{\footnotesize\bf\tt}
 \def\PagesItem{\small\sffamily}
 \def\RowHeader{\normalsize\bf\sffamily}
 \def\AfterCaption{\normalsize\bf\sffamily}
 \def\RowHeaderStrongFixed{\footnotesize\bf\tt}
 \def\RowHeaderEmphasis{\small\it\sffamily}
 \def\DataStrongFixed{\footnotesize\bf\tt}
 \def\Footer{\footnotesize\bf\sffamily}
 \def\FolderAction{\small\sffamily}
 \def\HeaderEmphasisFixed{\footnotesize\it\tt}
 \def\SystemTitle{\large\bf\it\sffamily}
 \def\ErrorContentFixed{\footnotesize\tt}
 \def\SystemFooter{\footnotesize\it\sffamily}
% Set cell padding 
\renewcommand{\arraystretch}{1.3}
% Headings
\newcommand{\heading}[2]{\csname#1\endcsname #2}
\newcommand{\proctitle}[2]{\csname#1\endcsname #2}
% Declare new column type
\newcolumntype{S}[2]{>{\csname#1\endcsname}#2}
% Set warning box style
\newcommand{\msg}[2]{\fbox{%
   \begin{minipage}{\textwidth}#2\end{minipage}}%
}

\begin{center}\heading{ProcTitle}{The CONTENTS Procedure}\end{center}
\begin{center}\begin{longtable}
{llll}\hline % colspecs
   \multicolumn{1}{S{RowHeader}{l}}{Data Set Name:} & 
   \multicolumn{1}{S{Data}{l}}{STATE.MD{\textunderscore}SIC{\textunderscore}DIVISION{\textunderscore}V23{\textunderscore}FUZZED} & 
   \multicolumn{1}{S{RowHeader}{l}}{Observations:} & 
   \multicolumn{1}{S{Data}{l}}{12636}
\\
   \multicolumn{1}{S{RowHeader}{l}}{Member Type:} & 
   \multicolumn{1}{S{Data}{l}}{DATA} & 
   \multicolumn{1}{S{RowHeader}{l}}{Variables:} & 
   \multicolumn{1}{S{Data}{l}}{60}
\\
   \multicolumn{1}{S{RowHeader}{l}}{Engine:} & 
   \multicolumn{1}{S{Data}{l}}{V8} & 
   \multicolumn{1}{S{RowHeader}{l}}{Indexes:} & 
   \multicolumn{1}{S{Data}{l}}{0}
\\
   \multicolumn{1}{S{RowHeader}{l}}{Created:} & 
   \multicolumn{1}{S{Data}{l}}{18:41 Thursday, May 16, 2002} & 
   \multicolumn{1}{S{RowHeader}{l}}{Observation Length:} & 
   \multicolumn{1}{S{Data}{l}}{288}
\\
   \multicolumn{1}{S{RowHeader}{l}}{Last Modified:} & 
   \multicolumn{1}{S{Data}{l}}{18:41 Thursday, May 16, 2002} & 
   \multicolumn{1}{S{RowHeader}{l}}{Deleted Observations:} & 
   \multicolumn{1}{S{Data}{l}}{0}
\\
   \multicolumn{1}{S{RowHeader}{l}}{Protection:} & 
   \multicolumn{1}{S{Data}{l}}{ } & 
   \multicolumn{1}{S{RowHeader}{l}}{Compressed:} & 
   \multicolumn{1}{S{Data}{l}}{NO}
\\
   \multicolumn{1}{S{RowHeader}{l}}{Data Set Type:} & 
   \multicolumn{1}{S{Data}{l}}{ } & 
   \multicolumn{1}{S{RowHeader}{l}}{Sorted:} & 
   \multicolumn{1}{S{Data}{l}}{NO}
\\
   \multicolumn{1}{S{RowHeader}{l}}{Label:} & 
   \multicolumn{1}{S{Data}{l}}{ } & 
   \multicolumn{1}{S{RowHeader}{l}}{ } & 
   \multicolumn{1}{S{Data}{l}}{ }
\\
\end{longtable}
\end{center}
\begin{center}\begin{longtable}
{rllrrl}\hline % colspecs
% table_head start
   \multicolumn{6}{S{Header}{c}}{-----Variables Ordered by Position-----}
\\
   \multicolumn{1}{S{Header}{r}}{\#} & 
   \multicolumn{1}{S{Header}{l}}{Variable} & 
   \multicolumn{1}{S{Header}{l}}{Type} & 
   \multicolumn{1}{S{Header}{r}}{Len} & 
   \multicolumn{1}{S{Header}{r}}{Pos} & 
   \multicolumn{1}{S{Header}{l}}{Label}
\\
\hline 
\endhead % table_head end
\hline 
\multicolumn{1}{r}{(cont.)}\\
\endfoot 
\hline 
\endlastfoot % table_head end
   \multicolumn{1}{S{RowHeader}{r}}{1} & 
   \multicolumn{1}{S{Data}{l}}{state} & 
   \multicolumn{1}{S{Data}{l}}{Char} & 
   \multicolumn{1}{S{Data}{r}}{2} & 
   \multicolumn{1}{S{Data}{r}}{216} & 
   \multicolumn{1}{S{Data}{l}}{FIPS State}
\\
   \multicolumn{1}{S{RowHeader}{r}}{2} & 
   \multicolumn{1}{S{Data}{l}}{year} & 
   \multicolumn{1}{S{Data}{l}}{Num} & 
   \multicolumn{1}{S{Data}{r}}{3} & 
   \multicolumn{1}{S{Data}{r}}{273} & 
   \multicolumn{1}{S{Data}{l}}{Year}
\\
   \multicolumn{1}{S{RowHeader}{r}}{3} & 
   \multicolumn{1}{S{Data}{l}}{quarter} & 
   \multicolumn{1}{S{Data}{l}}{Num} & 
   \multicolumn{1}{S{Data}{r}}{3} & 
   \multicolumn{1}{S{Data}{r}}{276} & 
   \multicolumn{1}{S{Data}{l}}{Quarter}
\\
   \multicolumn{1}{S{RowHeader}{r}}{4} & 
   \multicolumn{1}{S{Data}{l}}{sic{\textunderscore}division} & 
   \multicolumn{1}{S{Data}{l}}{Char} & 
   \multicolumn{1}{S{Data}{r}}{1} & 
   \multicolumn{1}{S{Data}{r}}{218} & 
   \multicolumn{1}{S{Data}{l}}{SIC Division}
\\
   \multicolumn{1}{S{RowHeader}{r}}{5} & 
   \multicolumn{1}{S{Data}{l}}{sex} & 
   \multicolumn{1}{S{Data}{l}}{Num} & 
   \multicolumn{1}{S{Data}{r}}{3} & 
   \multicolumn{1}{S{Data}{r}}{279} & 
   \multicolumn{1}{S{Data}{l}}{Sex}
\\
   \multicolumn{1}{S{RowHeader}{r}}{6} & 
   \multicolumn{1}{S{Data}{l}}{agegroup} & 
   \multicolumn{1}{S{Data}{l}}{Num} & 
   \multicolumn{1}{S{Data}{r}}{3} & 
   \multicolumn{1}{S{Data}{r}}{282} & 
   \multicolumn{1}{S{Data}{l}}{Age group}
\\
   \multicolumn{1}{S{RowHeader}{r}}{7} & 
   \multicolumn{1}{S{Data}{l}}{A} & 
   \multicolumn{1}{S{Data}{l}}{Num} & 
   \multicolumn{1}{S{Data}{r}}{8} & 
   \multicolumn{1}{S{Data}{r}}{0} & 
   \multicolumn{1}{S{Data}{l}}{Accessions}
\\
   \multicolumn{1}{S{RowHeader}{r}}{8} & 
   \multicolumn{1}{S{Data}{l}}{B} & 
   \multicolumn{1}{S{Data}{l}}{Num} & 
   \multicolumn{1}{S{Data}{r}}{8} & 
   \multicolumn{1}{S{Data}{r}}{8} & 
   \multicolumn{1}{S{Data}{l}}{Beginning-of-period employment}
\\
   \multicolumn{1}{S{RowHeader}{r}}{9} & 
   \multicolumn{1}{S{Data}{l}}{E} & 
   \multicolumn{1}{S{Data}{l}}{Num} & 
   \multicolumn{1}{S{Data}{r}}{8} & 
   \multicolumn{1}{S{Data}{r}}{16} & 
   \multicolumn{1}{S{Data}{l}}{End-of-period employment}
\\
   \multicolumn{1}{S{RowHeader}{r}}{10} & 
   \multicolumn{1}{S{Data}{l}}{F} & 
   \multicolumn{1}{S{Data}{l}}{Num} & 
   \multicolumn{1}{S{Data}{r}}{8} & 
   \multicolumn{1}{S{Data}{r}}{24} & 
   \multicolumn{1}{S{Data}{l}}{Full-quarter employment}
\\
   \multicolumn{1}{S{RowHeader}{r}}{11} & 
   \multicolumn{1}{S{Data}{l}}{FA} & 
   \multicolumn{1}{S{Data}{l}}{Num} & 
   \multicolumn{1}{S{Data}{r}}{8} & 
   \multicolumn{1}{S{Data}{r}}{32} & 
   \multicolumn{1}{S{Data}{l}}{Flow into full-quarter employment}
\\
   \multicolumn{1}{S{RowHeader}{r}}{12} & 
   \multicolumn{1}{S{Data}{l}}{FJC} & 
   \multicolumn{1}{S{Data}{l}}{Num} & 
   \multicolumn{1}{S{Data}{r}}{8} & 
   \multicolumn{1}{S{Data}{r}}{40} & 
   \multicolumn{1}{S{Data}{l}}{Full-quarter job creation}
\\
   \multicolumn{1}{S{RowHeader}{r}}{13} & 
   \multicolumn{1}{S{Data}{l}}{FJD} & 
   \multicolumn{1}{S{Data}{l}}{Num} & 
   \multicolumn{1}{S{Data}{r}}{8} & 
   \multicolumn{1}{S{Data}{r}}{48} & 
   \multicolumn{1}{S{Data}{l}}{Full-quarter job destruction}
\\
   \multicolumn{1}{S{RowHeader}{r}}{14} & 
   \multicolumn{1}{S{Data}{l}}{FJF} & 
   \multicolumn{1}{S{Data}{l}}{Num} & 
   \multicolumn{1}{S{Data}{r}}{8} & 
   \multicolumn{1}{S{Data}{r}}{56} & 
   \multicolumn{1}{S{Data}{l}}{Net change in full-quarter employment}
\\
   \multicolumn{1}{S{RowHeader}{r}}{15} & 
   \multicolumn{1}{S{Data}{l}}{FS} & 
   \multicolumn{1}{S{Data}{l}}{Num} & 
   \multicolumn{1}{S{Data}{r}}{8} & 
   \multicolumn{1}{S{Data}{r}}{64} & 
   \multicolumn{1}{S{Data}{l}}{Flow out of full-quarter employment}
\\
   \multicolumn{1}{S{RowHeader}{r}}{16} & 
   \multicolumn{1}{S{Data}{l}}{H} & 
   \multicolumn{1}{S{Data}{l}}{Num} & 
   \multicolumn{1}{S{Data}{r}}{8} & 
   \multicolumn{1}{S{Data}{r}}{72} & 
   \multicolumn{1}{S{Data}{l}}{New hires}
\\
   \multicolumn{1}{S{RowHeader}{r}}{17} & 
   \multicolumn{1}{S{Data}{l}}{H3} & 
   \multicolumn{1}{S{Data}{l}}{Num} & 
   \multicolumn{1}{S{Data}{r}}{8} & 
   \multicolumn{1}{S{Data}{r}}{80} & 
   \multicolumn{1}{S{Data}{l}}{Full-quarter new hires}
\\
   \multicolumn{1}{S{RowHeader}{r}}{18} & 
   \multicolumn{1}{S{Data}{l}}{JC} & 
   \multicolumn{1}{S{Data}{l}}{Num} & 
   \multicolumn{1}{S{Data}{r}}{8} & 
   \multicolumn{1}{S{Data}{r}}{88} & 
   \multicolumn{1}{S{Data}{l}}{Job creation}
\\
   \multicolumn{1}{S{RowHeader}{r}}{19} & 
   \multicolumn{1}{S{Data}{l}}{JD} & 
   \multicolumn{1}{S{Data}{l}}{Num} & 
   \multicolumn{1}{S{Data}{r}}{8} & 
   \multicolumn{1}{S{Data}{r}}{96} & 
   \multicolumn{1}{S{Data}{l}}{Job destruction}
\\
   \multicolumn{1}{S{RowHeader}{r}}{20} & 
   \multicolumn{1}{S{Data}{l}}{JF} & 
   \multicolumn{1}{S{Data}{l}}{Num} & 
   \multicolumn{1}{S{Data}{r}}{8} & 
   \multicolumn{1}{S{Data}{r}}{104} & 
   \multicolumn{1}{S{Data}{l}}{Net job flows}
\\
   \multicolumn{1}{S{RowHeader}{r}}{21} & 
   \multicolumn{1}{S{Data}{l}}{R} & 
   \multicolumn{1}{S{Data}{l}}{Num} & 
   \multicolumn{1}{S{Data}{r}}{8} & 
   \multicolumn{1}{S{Data}{r}}{112} & 
   \multicolumn{1}{S{Data}{l}}{Recalls}
\\
   \multicolumn{1}{S{RowHeader}{r}}{22} & 
   \multicolumn{1}{S{Data}{l}}{S} & 
   \multicolumn{1}{S{Data}{l}}{Num} & 
   \multicolumn{1}{S{Data}{r}}{8} & 
   \multicolumn{1}{S{Data}{r}}{120} & 
   \multicolumn{1}{S{Data}{l}}{Separations}
\\
   \multicolumn{1}{S{RowHeader}{r}}{23} & 
   \multicolumn{1}{S{Data}{l}}{Z{\textunderscore}NA} & 
   \multicolumn{1}{S{Data}{l}}{Num} & 
   \multicolumn{1}{S{Data}{r}}{8} & 
   \multicolumn{1}{S{Data}{r}}{128} & 
   \multicolumn{1}{S{Data}{l}}{Average periods of non-employment for accessions}
\\
   \multicolumn{1}{S{RowHeader}{r}}{24} & 
   \multicolumn{1}{S{Data}{l}}{Z{\textunderscore}NH} & 
   \multicolumn{1}{S{Data}{l}}{Num} & 
   \multicolumn{1}{S{Data}{r}}{8} & 
   \multicolumn{1}{S{Data}{r}}{136} & 
   \multicolumn{1}{S{Data}{l}}{Average periods of non-employment for new hires}
\\
   \multicolumn{1}{S{RowHeader}{r}}{25} & 
   \multicolumn{1}{S{Data}{l}}{Z{\textunderscore}NR} & 
   \multicolumn{1}{S{Data}{l}}{Num} & 
   \multicolumn{1}{S{Data}{r}}{8} & 
   \multicolumn{1}{S{Data}{r}}{144} & 
   \multicolumn{1}{S{Data}{l}}{Average periods of non-employment for recalls}
\\
   \multicolumn{1}{S{RowHeader}{r}}{26} & 
   \multicolumn{1}{S{Data}{l}}{Z{\textunderscore}NS} & 
   \multicolumn{1}{S{Data}{l}}{Num} & 
   \multicolumn{1}{S{Data}{r}}{8} & 
   \multicolumn{1}{S{Data}{r}}{152} & 
   \multicolumn{1}{S{Data}{l}}{Average periods of non-employment for separations}
\\
   \multicolumn{1}{S{RowHeader}{r}}{27} & 
   \multicolumn{1}{S{Data}{l}}{Z{\textunderscore}W2} & 
   \multicolumn{1}{S{Data}{l}}{Num} & 
   \multicolumn{1}{S{Data}{r}}{8} & 
   \multicolumn{1}{S{Data}{r}}{160} & 
   \multicolumn{1}{S{Data}{l}}{Average earnings of end-of-period employees}
\\
   \multicolumn{1}{S{RowHeader}{r}}{28} & 
   \multicolumn{1}{S{Data}{l}}{Z{\textunderscore}W3} & 
   \multicolumn{1}{S{Data}{l}}{Num} & 
   \multicolumn{1}{S{Data}{r}}{8} & 
   \multicolumn{1}{S{Data}{r}}{168} & 
   \multicolumn{1}{S{Data}{l}}{Average earnings of full-quarter employees}
\\
   \multicolumn{1}{S{RowHeader}{r}}{29} & 
   \multicolumn{1}{S{Data}{l}}{Z{\textunderscore}WFA} & 
   \multicolumn{1}{S{Data}{l}}{Num} & 
   \multicolumn{1}{S{Data}{r}}{8} & 
   \multicolumn{1}{S{Data}{r}}{176} & 
   \multicolumn{1}{S{Data}{l}}{Average earnings of transits to full-quarter status}
\\
   \multicolumn{1}{S{RowHeader}{r}}{30} & 
   \multicolumn{1}{S{Data}{l}}{Z{\textunderscore}WFS} & 
   \multicolumn{1}{S{Data}{l}}{Num} & 
   \multicolumn{1}{S{Data}{r}}{8} & 
   \multicolumn{1}{S{Data}{r}}{184} & 
   \multicolumn{1}{S{Data}{l}}{Average earnings of separations from full-quarter status}
\\
   \multicolumn{1}{S{RowHeader}{r}}{31} & 
   \multicolumn{1}{S{Data}{l}}{Z{\textunderscore}WH3} & 
   \multicolumn{1}{S{Data}{l}}{Num} & 
   \multicolumn{1}{S{Data}{r}}{8} & 
   \multicolumn{1}{S{Data}{r}}{192} & 
   \multicolumn{1}{S{Data}{l}}{Average earnings of full-quarter new hires}
\\
   \multicolumn{1}{S{RowHeader}{r}}{32} & 
   \multicolumn{1}{S{Data}{l}}{Z{\textunderscore}dWA} & 
   \multicolumn{1}{S{Data}{l}}{Num} & 
   \multicolumn{1}{S{Data}{r}}{8} & 
   \multicolumn{1}{S{Data}{r}}{200} & 
   \multicolumn{1}{S{Data}{l}}{Average change in total earnings for accessions}
\\
   \multicolumn{1}{S{RowHeader}{r}}{33} & 
   \multicolumn{1}{S{Data}{l}}{Z{\textunderscore}dWS} & 
   \multicolumn{1}{S{Data}{l}}{Num} & 
   \multicolumn{1}{S{Data}{r}}{8} & 
   \multicolumn{1}{S{Data}{r}}{208} & 
   \multicolumn{1}{S{Data}{l}}{Average change in total earnings for separations}
\\
   \multicolumn{1}{S{RowHeader}{r}}{34} & 
   \multicolumn{1}{S{Data}{l}}{A{\textunderscore}status} & 
   \multicolumn{1}{S{Data}{l}}{Char} & 
   \multicolumn{1}{S{Data}{r}}{2} & 
   \multicolumn{1}{S{Data}{r}}{219} & 
   \multicolumn{1}{S{Data}{l}}{Status: accessions}
\\
   \multicolumn{1}{S{RowHeader}{r}}{35} & 
   \multicolumn{1}{S{Data}{l}}{B{\textunderscore}status} & 
   \multicolumn{1}{S{Data}{l}}{Char} & 
   \multicolumn{1}{S{Data}{r}}{2} & 
   \multicolumn{1}{S{Data}{r}}{221} & 
   \multicolumn{1}{S{Data}{l}}{Status: beginning-of-period employment}
\\
   \multicolumn{1}{S{RowHeader}{r}}{36} & 
   \multicolumn{1}{S{Data}{l}}{E{\textunderscore}status} & 
   \multicolumn{1}{S{Data}{l}}{Char} & 
   \multicolumn{1}{S{Data}{r}}{2} & 
   \multicolumn{1}{S{Data}{r}}{223} & 
   \multicolumn{1}{S{Data}{l}}{Status: end-of-period employment}
\\
   \multicolumn{1}{S{RowHeader}{r}}{37} & 
   \multicolumn{1}{S{Data}{l}}{F{\textunderscore}status} & 
   \multicolumn{1}{S{Data}{l}}{Char} & 
   \multicolumn{1}{S{Data}{r}}{2} & 
   \multicolumn{1}{S{Data}{r}}{225} & 
   \multicolumn{1}{S{Data}{l}}{Status: full-quarter employment}
\\
   \multicolumn{1}{S{RowHeader}{r}}{38} & 
   \multicolumn{1}{S{Data}{l}}{FA{\textunderscore}status} & 
   \multicolumn{1}{S{Data}{l}}{Char} & 
   \multicolumn{1}{S{Data}{r}}{2} & 
   \multicolumn{1}{S{Data}{r}}{227} & 
   \multicolumn{1}{S{Data}{l}}{Status: flow into full-quarter employment}
\\
   \multicolumn{1}{S{RowHeader}{r}}{39} & 
   \multicolumn{1}{S{Data}{l}}{FJC{\textunderscore}status} & 
   \multicolumn{1}{S{Data}{l}}{Char} & 
   \multicolumn{1}{S{Data}{r}}{2} & 
   \multicolumn{1}{S{Data}{r}}{229} & 
   \multicolumn{1}{S{Data}{l}}{Status: full-quarter job creation}
\\
   \multicolumn{1}{S{RowHeader}{r}}{40} & 
   \multicolumn{1}{S{Data}{l}}{FJD{\textunderscore}status} & 
   \multicolumn{1}{S{Data}{l}}{Char} & 
   \multicolumn{1}{S{Data}{r}}{2} & 
   \multicolumn{1}{S{Data}{r}}{231} & 
   \multicolumn{1}{S{Data}{l}}{Status: full-quarter job destruction}
\\
   \multicolumn{1}{S{RowHeader}{r}}{41} & 
   \multicolumn{1}{S{Data}{l}}{FJF{\textunderscore}status} & 
   \multicolumn{1}{S{Data}{l}}{Char} & 
   \multicolumn{1}{S{Data}{r}}{2} & 
   \multicolumn{1}{S{Data}{r}}{233} & 
   \multicolumn{1}{S{Data}{l}}{Status: net change in full-quarter employment}
\\
   \multicolumn{1}{S{RowHeader}{r}}{42} & 
   \multicolumn{1}{S{Data}{l}}{FS{\textunderscore}status} & 
   \multicolumn{1}{S{Data}{l}}{Char} & 
   \multicolumn{1}{S{Data}{r}}{2} & 
   \multicolumn{1}{S{Data}{r}}{235} & 
   \multicolumn{1}{S{Data}{l}}{Status: flow out of full-quarter employment}
\\
   \multicolumn{1}{S{RowHeader}{r}}{43} & 
   \multicolumn{1}{S{Data}{l}}{H{\textunderscore}status} & 
   \multicolumn{1}{S{Data}{l}}{Char} & 
   \multicolumn{1}{S{Data}{r}}{2} & 
   \multicolumn{1}{S{Data}{r}}{237} & 
   \multicolumn{1}{S{Data}{l}}{Status: new hires}
\\
   \multicolumn{1}{S{RowHeader}{r}}{44} & 
   \multicolumn{1}{S{Data}{l}}{H3{\textunderscore}status} & 
   \multicolumn{1}{S{Data}{l}}{Char} & 
   \multicolumn{1}{S{Data}{r}}{2} & 
   \multicolumn{1}{S{Data}{r}}{239} & 
   \multicolumn{1}{S{Data}{l}}{Status: full-quarter new hires}
\\
   \multicolumn{1}{S{RowHeader}{r}}{45} & 
   \multicolumn{1}{S{Data}{l}}{JC{\textunderscore}status} & 
   \multicolumn{1}{S{Data}{l}}{Char} & 
   \multicolumn{1}{S{Data}{r}}{2} & 
   \multicolumn{1}{S{Data}{r}}{241} & 
   \multicolumn{1}{S{Data}{l}}{Status: job creation}
\\
   \multicolumn{1}{S{RowHeader}{r}}{46} & 
   \multicolumn{1}{S{Data}{l}}{JD{\textunderscore}status} & 
   \multicolumn{1}{S{Data}{l}}{Char} & 
   \multicolumn{1}{S{Data}{r}}{2} & 
   \multicolumn{1}{S{Data}{r}}{243} & 
   \multicolumn{1}{S{Data}{l}}{Status: job destruction}
\\
   \multicolumn{1}{S{RowHeader}{r}}{47} & 
   \multicolumn{1}{S{Data}{l}}{JF{\textunderscore}status} & 
   \multicolumn{1}{S{Data}{l}}{Char} & 
   \multicolumn{1}{S{Data}{r}}{2} & 
   \multicolumn{1}{S{Data}{r}}{245} & 
   \multicolumn{1}{S{Data}{l}}{Status: net job flows}
\\
   \multicolumn{1}{S{RowHeader}{r}}{48} & 
   \multicolumn{1}{S{Data}{l}}{R{\textunderscore}status} & 
   \multicolumn{1}{S{Data}{l}}{Char} & 
   \multicolumn{1}{S{Data}{r}}{2} & 
   \multicolumn{1}{S{Data}{r}}{247} & 
   \multicolumn{1}{S{Data}{l}}{Status: recalls}
\\
   \multicolumn{1}{S{RowHeader}{r}}{49} & 
   \multicolumn{1}{S{Data}{l}}{S{\textunderscore}status} & 
   \multicolumn{1}{S{Data}{l}}{Char} & 
   \multicolumn{1}{S{Data}{r}}{2} & 
   \multicolumn{1}{S{Data}{r}}{249} & 
   \multicolumn{1}{S{Data}{l}}{Status: separations}
\\
   \multicolumn{1}{S{RowHeader}{r}}{50} & 
   \multicolumn{1}{S{Data}{l}}{Z{\textunderscore}NA{\textunderscore}status} & 
   \multicolumn{1}{S{Data}{l}}{Char} & 
   \multicolumn{1}{S{Data}{r}}{2} & 
   \multicolumn{1}{S{Data}{r}}{251} & 
   \multicolumn{1}{S{Data}{l}}{Status: average periods of non-employment for accessions}
\\
   \multicolumn{1}{S{RowHeader}{r}}{51} & 
   \multicolumn{1}{S{Data}{l}}{Z{\textunderscore}NH{\textunderscore}status} & 
   \multicolumn{1}{S{Data}{l}}{Char} & 
   \multicolumn{1}{S{Data}{r}}{2} & 
   \multicolumn{1}{S{Data}{r}}{253} & 
   \multicolumn{1}{S{Data}{l}}{Status: average periods of non-employment for new hires}
\\
   \multicolumn{1}{S{RowHeader}{r}}{52} & 
   \multicolumn{1}{S{Data}{l}}{Z{\textunderscore}NR{\textunderscore}status} & 
   \multicolumn{1}{S{Data}{l}}{Char} & 
   \multicolumn{1}{S{Data}{r}}{2} & 
   \multicolumn{1}{S{Data}{r}}{255} & 
   \multicolumn{1}{S{Data}{l}}{Status: average periods of non-employment for recalls}
\\
   \multicolumn{1}{S{RowHeader}{r}}{53} & 
   \multicolumn{1}{S{Data}{l}}{Z{\textunderscore}NS{\textunderscore}status} & 
   \multicolumn{1}{S{Data}{l}}{Char} & 
   \multicolumn{1}{S{Data}{r}}{2} & 
   \multicolumn{1}{S{Data}{r}}{257} & 
   \multicolumn{1}{S{Data}{l}}{Status: average periods of non-employment for separations}
\\
   \multicolumn{1}{S{RowHeader}{r}}{54} & 
   \multicolumn{1}{S{Data}{l}}{Z{\textunderscore}W2{\textunderscore}status} & 
   \multicolumn{1}{S{Data}{l}}{Char} & 
   \multicolumn{1}{S{Data}{r}}{2} & 
   \multicolumn{1}{S{Data}{r}}{259} & 
   \multicolumn{1}{S{Data}{l}}{Status: average earnings of end-of-period employees}
\\
   \multicolumn{1}{S{RowHeader}{r}}{55} & 
   \multicolumn{1}{S{Data}{l}}{Z{\textunderscore}W3{\textunderscore}status} & 
   \multicolumn{1}{S{Data}{l}}{Char} & 
   \multicolumn{1}{S{Data}{r}}{2} & 
   \multicolumn{1}{S{Data}{r}}{261} & 
   \multicolumn{1}{S{Data}{l}}{Status: average earnings of full-quarter employees}
\\
   \multicolumn{1}{S{RowHeader}{r}}{56} & 
   \multicolumn{1}{S{Data}{l}}{Z{\textunderscore}WFA{\textunderscore}status} & 
   \multicolumn{1}{S{Data}{l}}{Char} & 
   \multicolumn{1}{S{Data}{r}}{2} & 
   \multicolumn{1}{S{Data}{r}}{263} & 
   \multicolumn{1}{S{Data}{l}}{Status: average earnings of transits to full-quarter status}
\\
   \multicolumn{1}{S{RowHeader}{r}}{57} & 
   \multicolumn{1}{S{Data}{l}}{Z{\textunderscore}WFS{\textunderscore}status} & 
   \multicolumn{1}{S{Data}{l}}{Char} & 
   \multicolumn{1}{S{Data}{r}}{2} & 
   \multicolumn{1}{S{Data}{r}}{265} & 
   \multicolumn{1}{S{Data}{l}}{Status: average earnings of separations from full-quarter status}
\\
   \multicolumn{1}{S{RowHeader}{r}}{58} & 
   \multicolumn{1}{S{Data}{l}}{Z{\textunderscore}WH3{\textunderscore}status} & 
   \multicolumn{1}{S{Data}{l}}{Char} & 
   \multicolumn{1}{S{Data}{r}}{2} & 
   \multicolumn{1}{S{Data}{r}}{267} & 
   \multicolumn{1}{S{Data}{l}}{Status: average earnings of full-quarter new hires}
\\
   \multicolumn{1}{S{RowHeader}{r}}{59} & 
   \multicolumn{1}{S{Data}{l}}{Z{\textunderscore}dWA{\textunderscore}status} & 
   \multicolumn{1}{S{Data}{l}}{Char} & 
   \multicolumn{1}{S{Data}{r}}{2} & 
   \multicolumn{1}{S{Data}{r}}{269} & 
   \multicolumn{1}{S{Data}{l}}{Status: average change in total earnings for accessions}
\\
   \multicolumn{1}{S{RowHeader}{r}}{60} & 
   \multicolumn{1}{S{Data}{l}}{Z{\textunderscore}dWS{\textunderscore}status} & 
   \multicolumn{1}{S{Data}{l}}{Char} & 
   \multicolumn{1}{S{Data}{r}}{2} & 
   \multicolumn{1}{S{Data}{r}}{271} & 
   \multicolumn{1}{S{Data}{l}}{Status: average change in total earnings for separations}
\\
\end{longtable}
\end{center}

% ====================end of output====================
 \newpage %    Generated by SAS
%    http://www.sas.com
% created by=vilhu001
% sasversion=8.2
% date=2002-05-23
% time=00:50:04
% encoding=iso-8859-1
% ====================begin of output====================
% \begin{document}

% An external file needs to be included, as specified% in latexlong.sas. This can be called sas.sty,
% in which case you want to include a line like
% \usepackage{sas}
% or it can be a simple (La)TeX file, which you 
% include by typing 
% %%
%% This is file `sas.sty',
%% generated with the docstrip utility.
%%
%% 
\NeedsTeXFormat{LaTeX2e}
\ProvidesPackage{sas}
        [2002/01/18 LEHD version 0.1
    provides definition for tables generated by SAS%
                   ]
\@ifundefined{array@processline}{\RequirePackage{array}}{}
\@ifundefined{longtable@processline}{\RequirePackage{longtable}}{}
 \def\ContentTitle{\small\it\sffamily}
 \def\Output{\small\sffamily}
 \def\HeaderEmphasis{\small\it\sffamily}
 \def\NoteContent{\small\sffamily}
 \def\FatalContent{\small\sffamily}
 \def\Graph{\small\sffamily}
 \def\WarnContentFixed{\footnotesize\tt}
 \def\NoteBanner{\small\sffamily}
 \def\DataStrong{\normalsize\bf\sffamily}
 \def\Document{\small\sffamily}
 \def\BeforeCaption{\normalsize\bf\sffamily}
 \def\ContentsDate{\small\sffamily}
 \def\Pages{\small\sffamily}
 \def\TitlesAndFooters{\footnotesize\bf\it\sffamily}
 \def\IndexProcName{\small\sffamily}
 \def\ProcTitle{\normalsize\bf\it\sffamily}
 \def\IndexAction{\small\sffamily}
 \def\Data{\small\sffamily}
 \def\Table{\small\sffamily}
 \def\FooterEmpty{\footnotesize\bf\sffamily}
 \def\SysTitleAndFooterContainer{\footnotesize\sffamily}
 \def\RowFooterEmpty{\footnotesize\bf\sffamily}
 \def\ExtendedPage{\small\it\sffamily}
 \def\FooterFixed{\footnotesize\tt}
 \def\RowFooterStrongFixed{\footnotesize\bf\tt}
 \def\RowFooterEmphasis{\footnote\it\sffamily}
 \def\ContentFolder{\small\sffamily}
 \def\Container{\small\sffamily}
 \def\Date{\small\sffamily}
 \def\RowFooterFixed{\footnotesize\tt}
 \def\Caption{\normalsize\bf\sffamily}
 \def\WarnBanner{\small\sffamily}
 \def\Frame{\small\sffamily}
 \def\HeaderStrongFixed{\footnotesize\bf\tt}
 \def\IndexTitle{\small\it\sffamily}
 \def\NoteContentFixed{\footnotesize\tt}
 \def\DataEmphasisFixed{\footnotesize\it\tt}
 \def\Note{\small\sffamily}
 \def\Byline{\normalsize\bf\sffamily}
 \def\FatalBanner{\small\sffamily}
 \def\ProcTitleFixed{\footnotesize\bf\tt}
 \def\ByContentFolder{\small\sffamily}
 \def\PagesProcLabel{\small\sffamily}
 \def\RowHeaderFixed{\footnotesize\tt}
 \def\RowFooterEmphasisFixed{\footnotesize\it\tt}
 \def\WarnContent{\small\sffamily}
 \def\DataEmpty{\small\sffamily}
 \def\Cell{\small\sffamily}
 \def\Header{\normalsize\bf\sffamily}
 \def\PageNo{\normalsize\bf\sffamily}
 \def\ContentProcLabel{\small\sffamily}
 \def\HeaderFixed{\footnotesize\tt}
 \def\PagesTitle{\small\it\sffamily}
 \def\RowHeaderEmpty{\normalsize\bf\sffamily}
 \def\PagesProcName{\small\sffamily}
 \def\Batch{\footnotesize\tt}
 \def\ContentItem{\small\sffamily}
 \def\Body{\small\sffamily}
 \def\PagesDate{\small\sffamily}
 \def\Index{\small\sffamily}
 \def\HeaderEmpty{\normalsize\bf\sffamily}
 \def\FooterStrong{\footnotesize\bf\sffamily}
 \def\FooterEmphasis{\footnotesize\it\sffamily}
 \def\ErrorContent{\small\sffamily}
 \def\DataFixed{\footnotesize\tt}
 \def\HeaderStrong{\normalsize\bf\sffamily}
 \def\GraphBackground{}
 \def\DataEmphasis{\small\it\sffamily}
 \def\TitleAndNoteContainer{\small\sffamily}
 \def\RowFooter{\footnotesize\bf\sffamily}
 \def\IndexItem{\small\sffamily}
 \def\BylineContainer{\small\sffamily}
 \def\FatalContentFixed{\footnotesize\tt}
 \def\BodyDate{\small\sffamily}
 \def\RowFooterStrong{\footnotesize\bf\sffamily}
 \def\UserText{\small\sffamily}
 \def\HeadersAndFooters{\footnotesize\bf\sffamily}
 \def\RowHeaderEmphasisFixed{\footnotesize\it\tt}
 \def\ErrorBanner{\small\sffamily}
 \def\ContentProcName{\small\sffamily}
 \def\RowHeaderStrong{\normalsize\bf\sffamily}
 \def\FooterEmphasisFixed{\footnotesize\it\tt}
 \def\Contents{\small\sffamily}
 \def\FooterStrongFixed{\footnotesize\bf\tt}
 \def\PagesItem{\small\sffamily}
 \def\RowHeader{\normalsize\bf\sffamily}
 \def\AfterCaption{\normalsize\bf\sffamily}
 \def\RowHeaderStrongFixed{\footnotesize\bf\tt}
 \def\RowHeaderEmphasis{\small\it\sffamily}
 \def\DataStrongFixed{\footnotesize\bf\tt}
 \def\Footer{\footnotesize\bf\sffamily}
 \def\FolderAction{\small\sffamily}
 \def\HeaderEmphasisFixed{\footnotesize\it\tt}
 \def\SystemTitle{\large\bf\it\sffamily}
 \def\ErrorContentFixed{\footnotesize\tt}
 \def\SystemFooter{\footnotesize\it\sffamily}
% Set cell padding 
\renewcommand{\arraystretch}{1.3}
% Headings
\newcommand{\heading}[2]{\csname#1\endcsname #2}
\newcommand{\proctitle}[2]{\csname#1\endcsname #2}
% Declare new column type
\newcolumntype{S}[2]{>{\csname#1\endcsname}#2}
% Set warning box style
\newcommand{\msg}[2]{\fbox{%
   \begin{minipage}{\textwidth}#2\end{minipage}}%
}

\begin{center}\heading{SystemTitle}{Maryland            }\end{center}
\begin{center}\heading{ProcTitle}{The FREQ Procedure}\end{center}
\begin{center}\begin{longtable}
{lrrrr}\hline % colspecs
% table_head start
   \multicolumn{5}{S{Header}{c}}{FIPS State}
\\
   \multicolumn{1}{S{Header}{l}}{state} & 
   \multicolumn{1}{S{Header}{r}}{Frequency} & 
   \multicolumn{1}{S{Header}{r}}{ Percent} & 
   \multicolumn{1}{S{Header}{r}}{Cumulative\linebreak  Frequency} & 
   \multicolumn{1}{S{Header}{r}}{Cumulative\linebreak   Percent}
\\
\hline 
\endhead % table_head end
\hline 
\multicolumn{1}{r}{(cont.)}\\
\endfoot 
\hline 
\endlastfoot % table_head end
   \multicolumn{1}{S{RowHeader}{l}}{24 MARYLAND} & 
   \multicolumn{1}{S{Data}{r}}{12636} & 
   \multicolumn{1}{S{Data}{r}}{100.00} & 
   \multicolumn{1}{S{Data}{r}}{12636} & 
   \multicolumn{1}{S{Data}{r}}{100.00}
\\
\end{longtable}
\end{center}
\begin{center}\begin{longtable}
{lrrrr}\hline % colspecs
% table_head start
   \multicolumn{5}{S{Header}{c}}{SIC Division}
\\
   \multicolumn{1}{S{Header}{l}}{sic{\textunderscore}division} & 
   \multicolumn{1}{S{Header}{r}}{Frequency} & 
   \multicolumn{1}{S{Header}{r}}{ Percent} & 
   \multicolumn{1}{S{Header}{r}}{Cumulative\linebreak  Frequency} & 
   \multicolumn{1}{S{Header}{r}}{Cumulative\linebreak   Percent}
\\
\hline 
\endhead % table_head end
\hline 
\multicolumn{1}{r}{(cont.)}\\
\endfoot 
\hline 
\endlastfoot % table_head end
   \multicolumn{1}{S{RowHeader}{l}}{A Agriculture etc.} & 
   \multicolumn{1}{S{Data}{r}}{1053} & 
   \multicolumn{1}{S{Data}{r}}{9.09} & 
   \multicolumn{1}{S{Data}{r}}{1053} & 
   \multicolumn{1}{S{Data}{r}}{9.09}
\\
   \multicolumn{1}{S{RowHeader}{l}}{B Mining} & 
   \multicolumn{1}{S{Data}{r}}{1053} & 
   \multicolumn{1}{S{Data}{r}}{9.09} & 
   \multicolumn{1}{S{Data}{r}}{2106} & 
   \multicolumn{1}{S{Data}{r}}{18.18}
\\
   \multicolumn{1}{S{RowHeader}{l}}{C Construction} & 
   \multicolumn{1}{S{Data}{r}}{1053} & 
   \multicolumn{1}{S{Data}{r}}{9.09} & 
   \multicolumn{1}{S{Data}{r}}{3159} & 
   \multicolumn{1}{S{Data}{r}}{27.27}
\\
   \multicolumn{1}{S{RowHeader}{l}}{D Manufacturing} & 
   \multicolumn{1}{S{Data}{r}}{1053} & 
   \multicolumn{1}{S{Data}{r}}{9.09} & 
   \multicolumn{1}{S{Data}{r}}{4212} & 
   \multicolumn{1}{S{Data}{r}}{36.36}
\\
   \multicolumn{1}{S{RowHeader}{l}}{E Trans. \& Utilities} & 
   \multicolumn{1}{S{Data}{r}}{1053} & 
   \multicolumn{1}{S{Data}{r}}{9.09} & 
   \multicolumn{1}{S{Data}{r}}{5265} & 
   \multicolumn{1}{S{Data}{r}}{45.45}
\\
   \multicolumn{1}{S{RowHeader}{l}}{F Wholesale trade} & 
   \multicolumn{1}{S{Data}{r}}{1053} & 
   \multicolumn{1}{S{Data}{r}}{9.09} & 
   \multicolumn{1}{S{Data}{r}}{6318} & 
   \multicolumn{1}{S{Data}{r}}{54.55}
\\
   \multicolumn{1}{S{RowHeader}{l}}{G Retail Trade} & 
   \multicolumn{1}{S{Data}{r}}{1053} & 
   \multicolumn{1}{S{Data}{r}}{9.09} & 
   \multicolumn{1}{S{Data}{r}}{7371} & 
   \multicolumn{1}{S{Data}{r}}{63.64}
\\
   \multicolumn{1}{S{RowHeader}{l}}{H FIRE} & 
   \multicolumn{1}{S{Data}{r}}{1053} & 
   \multicolumn{1}{S{Data}{r}}{9.09} & 
   \multicolumn{1}{S{Data}{r}}{8424} & 
   \multicolumn{1}{S{Data}{r}}{72.73}
\\
   \multicolumn{1}{S{RowHeader}{l}}{I Services} & 
   \multicolumn{1}{S{Data}{r}}{1053} & 
   \multicolumn{1}{S{Data}{r}}{9.09} & 
   \multicolumn{1}{S{Data}{r}}{9477} & 
   \multicolumn{1}{S{Data}{r}}{81.82}
\\
   \multicolumn{1}{S{RowHeader}{l}}{J Public Admin.} & 
   \multicolumn{1}{S{Data}{r}}{1053} & 
   \multicolumn{1}{S{Data}{r}}{9.09} & 
   \multicolumn{1}{S{Data}{r}}{10530} & 
   \multicolumn{1}{S{Data}{r}}{90.91}
\\
   \multicolumn{1}{S{RowHeader}{l}}{Other} & 
   \multicolumn{1}{S{Data}{r}}{1053} & 
   \multicolumn{1}{S{Data}{r}}{9.09} & 
   \multicolumn{1}{S{Data}{r}}{11583} & 
   \multicolumn{1}{S{Data}{r}}{100.00}
\\
\end{longtable}
\end{center}
\begin{center}\heading{ProcTitle}{Frequency Missing = 1053}\end{center}
\begin{center}\begin{longtable}
{rrrrr}\hline % colspecs
% table_head start
   \multicolumn{5}{S{Header}{c}}{Sex}
\\
   \multicolumn{1}{S{Header}{r}}{sex} & 
   \multicolumn{1}{S{Header}{r}}{Frequency} & 
   \multicolumn{1}{S{Header}{r}}{ Percent} & 
   \multicolumn{1}{S{Header}{r}}{Cumulative\linebreak  Frequency} & 
   \multicolumn{1}{S{Header}{r}}{Cumulative\linebreak   Percent}
\\
\hline 
\endhead % table_head end
\hline 
\multicolumn{1}{r}{(cont.)}\\
\endfoot 
\hline 
\endlastfoot % table_head end
   \multicolumn{1}{S{RowHeader}{r}}{0 : All} & 
   \multicolumn{1}{S{Data}{r}}{4212} & 
   \multicolumn{1}{S{Data}{r}}{33.33} & 
   \multicolumn{1}{S{Data}{r}}{4212} & 
   \multicolumn{1}{S{Data}{r}}{33.33}
\\
   \multicolumn{1}{S{RowHeader}{r}}{1 : Men} & 
   \multicolumn{1}{S{Data}{r}}{4212} & 
   \multicolumn{1}{S{Data}{r}}{33.33} & 
   \multicolumn{1}{S{Data}{r}}{8424} & 
   \multicolumn{1}{S{Data}{r}}{66.67}
\\
   \multicolumn{1}{S{RowHeader}{r}}{2 : Women} & 
   \multicolumn{1}{S{Data}{r}}{4212} & 
   \multicolumn{1}{S{Data}{r}}{33.33} & 
   \multicolumn{1}{S{Data}{r}}{12636} & 
   \multicolumn{1}{S{Data}{r}}{100.00}
\\
\end{longtable}
\end{center}
\begin{center}\begin{longtable}
{rrrrr}\hline % colspecs
% table_head start
   \multicolumn{5}{S{Header}{c}}{Age group}
\\
   \multicolumn{1}{S{Header}{r}}{agegroup} & 
   \multicolumn{1}{S{Header}{r}}{Frequency} & 
   \multicolumn{1}{S{Header}{r}}{ Percent} & 
   \multicolumn{1}{S{Header}{r}}{Cumulative\linebreak  Frequency} & 
   \multicolumn{1}{S{Header}{r}}{Cumulative\linebreak   Percent}
\\
\hline 
\endhead % table_head end
\hline 
\multicolumn{1}{r}{(cont.)}\\
\endfoot 
\hline 
\endlastfoot % table_head end
   \multicolumn{1}{S{RowHeader}{r}}{0 : All Ages} & 
   \multicolumn{1}{S{Data}{r}}{1404} & 
   \multicolumn{1}{S{Data}{r}}{11.11} & 
   \multicolumn{1}{S{Data}{r}}{1404} & 
   \multicolumn{1}{S{Data}{r}}{11.11}
\\
   \multicolumn{1}{S{RowHeader}{r}}{1 : 14-18} & 
   \multicolumn{1}{S{Data}{r}}{1404} & 
   \multicolumn{1}{S{Data}{r}}{11.11} & 
   \multicolumn{1}{S{Data}{r}}{2808} & 
   \multicolumn{1}{S{Data}{r}}{22.22}
\\
   \multicolumn{1}{S{RowHeader}{r}}{2 : 19-21} & 
   \multicolumn{1}{S{Data}{r}}{1404} & 
   \multicolumn{1}{S{Data}{r}}{11.11} & 
   \multicolumn{1}{S{Data}{r}}{4212} & 
   \multicolumn{1}{S{Data}{r}}{33.33}
\\
   \multicolumn{1}{S{RowHeader}{r}}{3 : 22-24} & 
   \multicolumn{1}{S{Data}{r}}{1404} & 
   \multicolumn{1}{S{Data}{r}}{11.11} & 
   \multicolumn{1}{S{Data}{r}}{5616} & 
   \multicolumn{1}{S{Data}{r}}{44.44}
\\
   \multicolumn{1}{S{RowHeader}{r}}{4 : 25-34} & 
   \multicolumn{1}{S{Data}{r}}{1404} & 
   \multicolumn{1}{S{Data}{r}}{11.11} & 
   \multicolumn{1}{S{Data}{r}}{7020} & 
   \multicolumn{1}{S{Data}{r}}{55.56}
\\
   \multicolumn{1}{S{RowHeader}{r}}{5 : 35-44} & 
   \multicolumn{1}{S{Data}{r}}{1404} & 
   \multicolumn{1}{S{Data}{r}}{11.11} & 
   \multicolumn{1}{S{Data}{r}}{8424} & 
   \multicolumn{1}{S{Data}{r}}{66.67}
\\
   \multicolumn{1}{S{RowHeader}{r}}{6 : 45-54} & 
   \multicolumn{1}{S{Data}{r}}{1404} & 
   \multicolumn{1}{S{Data}{r}}{11.11} & 
   \multicolumn{1}{S{Data}{r}}{9828} & 
   \multicolumn{1}{S{Data}{r}}{77.78}
\\
   \multicolumn{1}{S{RowHeader}{r}}{7 : 55-64} & 
   \multicolumn{1}{S{Data}{r}}{1404} & 
   \multicolumn{1}{S{Data}{r}}{11.11} & 
   \multicolumn{1}{S{Data}{r}}{11232} & 
   \multicolumn{1}{S{Data}{r}}{88.89}
\\
   \multicolumn{1}{S{RowHeader}{r}}{8 : 65+} & 
   \multicolumn{1}{S{Data}{r}}{1404} & 
   \multicolumn{1}{S{Data}{r}}{11.11} & 
   \multicolumn{1}{S{Data}{r}}{12636} & 
   \multicolumn{1}{S{Data}{r}}{100.00}
\\
\end{longtable}
\end{center}
\begin{center}\begin{longtable}
{llllll}\hline % colspecs
% table_head start
   \multicolumn{6}{S{Header}{c}}{Table of year by quarter}
\\
   \multicolumn{1}{S{Header}{c}}{year(Year)} & 
   \multicolumn{4}{S{Header}{c}}{quarter(Quarter)} & 
   \multicolumn{1}{S{Header}{r}}{Total}
\\
   \multicolumn{1}{l}{~} & 
   \multicolumn{1}{S{Header}{r}}{      1     } & 
   \multicolumn{1}{S{Header}{r}}{      2     } & 
   \multicolumn{1}{S{Header}{r}}{      3     } & 
   \multicolumn{1}{S{Header}{r}}{      4     } & 
   \multicolumn{1}{l}{~}
\\
\hline 
\endhead % table_head end
\hline 
\multicolumn{1}{r}{(cont.)}\\
\endfoot 
\hline 
\endlastfoot % table_head end
   \multicolumn{1}{S{Header}{r}}{1990        } & 
   \multicolumn{1}{S{Data}{r}}{   324} & 
   \multicolumn{1}{S{Data}{r}}{   324} & 
   \multicolumn{1}{S{Data}{r}}{   324} & 
   \multicolumn{1}{S{Data}{r}}{   324} & 
   \multicolumn{1}{S{Data}{r}}{  1296}
\\
   \multicolumn{1}{S{Header}{r}}{1991        } & 
   \multicolumn{1}{S{Data}{r}}{   324} & 
   \multicolumn{1}{S{Data}{r}}{   324} & 
   \multicolumn{1}{S{Data}{r}}{   324} & 
   \multicolumn{1}{S{Data}{r}}{   324} & 
   \multicolumn{1}{S{Data}{r}}{  1296}
\\
   \multicolumn{1}{S{Header}{r}}{1992        } & 
   \multicolumn{1}{S{Data}{r}}{   324} & 
   \multicolumn{1}{S{Data}{r}}{   324} & 
   \multicolumn{1}{S{Data}{r}}{   324} & 
   \multicolumn{1}{S{Data}{r}}{   324} & 
   \multicolumn{1}{S{Data}{r}}{  1296}
\\
   \multicolumn{1}{S{Header}{r}}{1993        } & 
   \multicolumn{1}{S{Data}{r}}{   324} & 
   \multicolumn{1}{S{Data}{r}}{   324} & 
   \multicolumn{1}{S{Data}{r}}{   324} & 
   \multicolumn{1}{S{Data}{r}}{   324} & 
   \multicolumn{1}{S{Data}{r}}{  1296}
\\
   \multicolumn{1}{S{Header}{r}}{1994        } & 
   \multicolumn{1}{S{Data}{r}}{   324} & 
   \multicolumn{1}{S{Data}{r}}{   324} & 
   \multicolumn{1}{S{Data}{r}}{   324} & 
   \multicolumn{1}{S{Data}{r}}{   324} & 
   \multicolumn{1}{S{Data}{r}}{  1296}
\\
   \multicolumn{1}{S{Header}{r}}{1995        } & 
   \multicolumn{1}{S{Data}{r}}{   324} & 
   \multicolumn{1}{S{Data}{r}}{   324} & 
   \multicolumn{1}{S{Data}{r}}{   324} & 
   \multicolumn{1}{S{Data}{r}}{   324} & 
   \multicolumn{1}{S{Data}{r}}{  1296}
\\
   \multicolumn{1}{S{Header}{r}}{1996        } & 
   \multicolumn{1}{S{Data}{r}}{   324} & 
   \multicolumn{1}{S{Data}{r}}{   324} & 
   \multicolumn{1}{S{Data}{r}}{   324} & 
   \multicolumn{1}{S{Data}{r}}{   324} & 
   \multicolumn{1}{S{Data}{r}}{  1296}
\\
   \multicolumn{1}{S{Header}{r}}{1997        } & 
   \multicolumn{1}{S{Data}{r}}{   324} & 
   \multicolumn{1}{S{Data}{r}}{   324} & 
   \multicolumn{1}{S{Data}{r}}{   324} & 
   \multicolumn{1}{S{Data}{r}}{   324} & 
   \multicolumn{1}{S{Data}{r}}{  1296}
\\
   \multicolumn{1}{S{Header}{r}}{1998        } & 
   \multicolumn{1}{S{Data}{r}}{   324} & 
   \multicolumn{1}{S{Data}{r}}{   324} & 
   \multicolumn{1}{S{Data}{r}}{   324} & 
   \multicolumn{1}{S{Data}{r}}{   324} & 
   \multicolumn{1}{S{Data}{r}}{  1296}
\\
   \multicolumn{1}{S{Header}{r}}{1999        } & 
   \multicolumn{1}{S{Data}{r}}{   324} & 
   \multicolumn{1}{S{Data}{r}}{   324} & 
   \multicolumn{1}{S{Data}{r}}{   324} & 
   \multicolumn{1}{S{Data}{r}}{     0} & 
   \multicolumn{1}{S{Data}{r}}{   972}
\\
   \multicolumn{1}{S{Header}{l}}{Total           } & 
   \multicolumn{1}{S{Data}{r}}{   3240} & 
   \multicolumn{1}{S{Data}{r}}{   3240} & 
   \multicolumn{1}{S{Data}{r}}{   3240} & 
   \multicolumn{1}{S{Data}{r}}{   2916} & 
   \multicolumn{1}{S{Data}{r}}{  12636}
\\
\end{longtable}
\end{center}

% ====================end of output====================

% 
% \newpage
% 
% \subsection{Minnesota}
% 
% %    Generated by SAS
%    http://www.sas.com
% created by=vilhu001
% sasversion=8.2
% date=2002-05-23
% time=00:39:34
% encoding=iso-8859-1
% ====================begin of output====================
% \begin{document}

% An external file needs to be included, as specified% in latexlong.sas. This can be called sas.sty,
% in which case you want to include a line like
% \usepackage{sas}
% or it can be a simple (La)TeX file, which you 
% include by typing 
% %%
%% This is file `sas.sty',
%% generated with the docstrip utility.
%%
%% 
\NeedsTeXFormat{LaTeX2e}
\ProvidesPackage{sas}
        [2002/01/18 LEHD version 0.1
    provides definition for tables generated by SAS%
                   ]
\@ifundefined{array@processline}{\RequirePackage{array}}{}
\@ifundefined{longtable@processline}{\RequirePackage{longtable}}{}
 \def\ContentTitle{\small\it\sffamily}
 \def\Output{\small\sffamily}
 \def\HeaderEmphasis{\small\it\sffamily}
 \def\NoteContent{\small\sffamily}
 \def\FatalContent{\small\sffamily}
 \def\Graph{\small\sffamily}
 \def\WarnContentFixed{\footnotesize\tt}
 \def\NoteBanner{\small\sffamily}
 \def\DataStrong{\normalsize\bf\sffamily}
 \def\Document{\small\sffamily}
 \def\BeforeCaption{\normalsize\bf\sffamily}
 \def\ContentsDate{\small\sffamily}
 \def\Pages{\small\sffamily}
 \def\TitlesAndFooters{\footnotesize\bf\it\sffamily}
 \def\IndexProcName{\small\sffamily}
 \def\ProcTitle{\normalsize\bf\it\sffamily}
 \def\IndexAction{\small\sffamily}
 \def\Data{\small\sffamily}
 \def\Table{\small\sffamily}
 \def\FooterEmpty{\footnotesize\bf\sffamily}
 \def\SysTitleAndFooterContainer{\footnotesize\sffamily}
 \def\RowFooterEmpty{\footnotesize\bf\sffamily}
 \def\ExtendedPage{\small\it\sffamily}
 \def\FooterFixed{\footnotesize\tt}
 \def\RowFooterStrongFixed{\footnotesize\bf\tt}
 \def\RowFooterEmphasis{\footnote\it\sffamily}
 \def\ContentFolder{\small\sffamily}
 \def\Container{\small\sffamily}
 \def\Date{\small\sffamily}
 \def\RowFooterFixed{\footnotesize\tt}
 \def\Caption{\normalsize\bf\sffamily}
 \def\WarnBanner{\small\sffamily}
 \def\Frame{\small\sffamily}
 \def\HeaderStrongFixed{\footnotesize\bf\tt}
 \def\IndexTitle{\small\it\sffamily}
 \def\NoteContentFixed{\footnotesize\tt}
 \def\DataEmphasisFixed{\footnotesize\it\tt}
 \def\Note{\small\sffamily}
 \def\Byline{\normalsize\bf\sffamily}
 \def\FatalBanner{\small\sffamily}
 \def\ProcTitleFixed{\footnotesize\bf\tt}
 \def\ByContentFolder{\small\sffamily}
 \def\PagesProcLabel{\small\sffamily}
 \def\RowHeaderFixed{\footnotesize\tt}
 \def\RowFooterEmphasisFixed{\footnotesize\it\tt}
 \def\WarnContent{\small\sffamily}
 \def\DataEmpty{\small\sffamily}
 \def\Cell{\small\sffamily}
 \def\Header{\normalsize\bf\sffamily}
 \def\PageNo{\normalsize\bf\sffamily}
 \def\ContentProcLabel{\small\sffamily}
 \def\HeaderFixed{\footnotesize\tt}
 \def\PagesTitle{\small\it\sffamily}
 \def\RowHeaderEmpty{\normalsize\bf\sffamily}
 \def\PagesProcName{\small\sffamily}
 \def\Batch{\footnotesize\tt}
 \def\ContentItem{\small\sffamily}
 \def\Body{\small\sffamily}
 \def\PagesDate{\small\sffamily}
 \def\Index{\small\sffamily}
 \def\HeaderEmpty{\normalsize\bf\sffamily}
 \def\FooterStrong{\footnotesize\bf\sffamily}
 \def\FooterEmphasis{\footnotesize\it\sffamily}
 \def\ErrorContent{\small\sffamily}
 \def\DataFixed{\footnotesize\tt}
 \def\HeaderStrong{\normalsize\bf\sffamily}
 \def\GraphBackground{}
 \def\DataEmphasis{\small\it\sffamily}
 \def\TitleAndNoteContainer{\small\sffamily}
 \def\RowFooter{\footnotesize\bf\sffamily}
 \def\IndexItem{\small\sffamily}
 \def\BylineContainer{\small\sffamily}
 \def\FatalContentFixed{\footnotesize\tt}
 \def\BodyDate{\small\sffamily}
 \def\RowFooterStrong{\footnotesize\bf\sffamily}
 \def\UserText{\small\sffamily}
 \def\HeadersAndFooters{\footnotesize\bf\sffamily}
 \def\RowHeaderEmphasisFixed{\footnotesize\it\tt}
 \def\ErrorBanner{\small\sffamily}
 \def\ContentProcName{\small\sffamily}
 \def\RowHeaderStrong{\normalsize\bf\sffamily}
 \def\FooterEmphasisFixed{\footnotesize\it\tt}
 \def\Contents{\small\sffamily}
 \def\FooterStrongFixed{\footnotesize\bf\tt}
 \def\PagesItem{\small\sffamily}
 \def\RowHeader{\normalsize\bf\sffamily}
 \def\AfterCaption{\normalsize\bf\sffamily}
 \def\RowHeaderStrongFixed{\footnotesize\bf\tt}
 \def\RowHeaderEmphasis{\small\it\sffamily}
 \def\DataStrongFixed{\footnotesize\bf\tt}
 \def\Footer{\footnotesize\bf\sffamily}
 \def\FolderAction{\small\sffamily}
 \def\HeaderEmphasisFixed{\footnotesize\it\tt}
 \def\SystemTitle{\large\bf\it\sffamily}
 \def\ErrorContentFixed{\footnotesize\tt}
 \def\SystemFooter{\footnotesize\it\sffamily}
% Set cell padding 
\renewcommand{\arraystretch}{1.3}
% Headings
\newcommand{\heading}[2]{\csname#1\endcsname #2}
\newcommand{\proctitle}[2]{\csname#1\endcsname #2}
% Declare new column type
\newcolumntype{S}[2]{>{\csname#1\endcsname}#2}
% Set warning box style
\newcommand{\msg}[2]{\fbox{%
   \begin{minipage}{\textwidth}#2\end{minipage}}%
}

\begin{center}\heading{ProcTitle}{The CONTENTS Procedure}\end{center}
\begin{center}\begin{longtable}
{llll}\hline % colspecs
   \multicolumn{1}{S{RowHeader}{l}}{Data Set Name:} & 
   \multicolumn{1}{S{Data}{l}}{STATE.MN{\textunderscore}COUNTY{\textunderscore}V23{\textunderscore}FUZZED} & 
   \multicolumn{1}{S{RowHeader}{l}}{Observations:} & 
   \multicolumn{1}{S{Data}{l}}{61452}
\\
   \multicolumn{1}{S{RowHeader}{l}}{Member Type:} & 
   \multicolumn{1}{S{Data}{l}}{DATA} & 
   \multicolumn{1}{S{RowHeader}{l}}{Variables:} & 
   \multicolumn{1}{S{Data}{l}}{60}
\\
   \multicolumn{1}{S{RowHeader}{l}}{Engine:} & 
   \multicolumn{1}{S{Data}{l}}{V8} & 
   \multicolumn{1}{S{RowHeader}{l}}{Indexes:} & 
   \multicolumn{1}{S{Data}{l}}{0}
\\
   \multicolumn{1}{S{RowHeader}{l}}{Created:} & 
   \multicolumn{1}{S{Data}{l}}{18:42 Thursday, May 16, 2002} & 
   \multicolumn{1}{S{RowHeader}{l}}{Observation Length:} & 
   \multicolumn{1}{S{Data}{l}}{288}
\\
   \multicolumn{1}{S{RowHeader}{l}}{Last Modified:} & 
   \multicolumn{1}{S{Data}{l}}{18:42 Thursday, May 16, 2002} & 
   \multicolumn{1}{S{RowHeader}{l}}{Deleted Observations:} & 
   \multicolumn{1}{S{Data}{l}}{0}
\\
   \multicolumn{1}{S{RowHeader}{l}}{Protection:} & 
   \multicolumn{1}{S{Data}{l}}{ } & 
   \multicolumn{1}{S{RowHeader}{l}}{Compressed:} & 
   \multicolumn{1}{S{Data}{l}}{NO}
\\
   \multicolumn{1}{S{RowHeader}{l}}{Data Set Type:} & 
   \multicolumn{1}{S{Data}{l}}{ } & 
   \multicolumn{1}{S{RowHeader}{l}}{Sorted:} & 
   \multicolumn{1}{S{Data}{l}}{NO}
\\
   \multicolumn{1}{S{RowHeader}{l}}{Label:} & 
   \multicolumn{1}{S{Data}{l}}{ } & 
   \multicolumn{1}{S{RowHeader}{l}}{ } & 
   \multicolumn{1}{S{Data}{l}}{ }
\\
\end{longtable}
\end{center}
\begin{center}\begin{longtable}
{rllrrl}\hline % colspecs
% table_head start
   \multicolumn{6}{S{Header}{c}}{-----Variables Ordered by Position-----}
\\
   \multicolumn{1}{S{Header}{r}}{\#} & 
   \multicolumn{1}{S{Header}{l}}{Variable} & 
   \multicolumn{1}{S{Header}{l}}{Type} & 
   \multicolumn{1}{S{Header}{r}}{Len} & 
   \multicolumn{1}{S{Header}{r}}{Pos} & 
   \multicolumn{1}{S{Header}{l}}{Label}
\\
\hline 
\endhead % table_head end
\hline 
\multicolumn{1}{r}{(cont.)}\\
\endfoot 
\hline 
\endlastfoot % table_head end
   \multicolumn{1}{S{RowHeader}{r}}{1} & 
   \multicolumn{1}{S{Data}{l}}{state} & 
   \multicolumn{1}{S{Data}{l}}{Char} & 
   \multicolumn{1}{S{Data}{r}}{2} & 
   \multicolumn{1}{S{Data}{r}}{216} & 
   \multicolumn{1}{S{Data}{l}}{FIPS State}
\\
   \multicolumn{1}{S{RowHeader}{r}}{2} & 
   \multicolumn{1}{S{Data}{l}}{year} & 
   \multicolumn{1}{S{Data}{l}}{Num} & 
   \multicolumn{1}{S{Data}{r}}{3} & 
   \multicolumn{1}{S{Data}{r}}{275} & 
   \multicolumn{1}{S{Data}{l}}{Year}
\\
   \multicolumn{1}{S{RowHeader}{r}}{3} & 
   \multicolumn{1}{S{Data}{l}}{quarter} & 
   \multicolumn{1}{S{Data}{l}}{Num} & 
   \multicolumn{1}{S{Data}{r}}{3} & 
   \multicolumn{1}{S{Data}{r}}{278} & 
   \multicolumn{1}{S{Data}{l}}{Quarter}
\\
   \multicolumn{1}{S{RowHeader}{r}}{4} & 
   \multicolumn{1}{S{Data}{l}}{county} & 
   \multicolumn{1}{S{Data}{l}}{Char} & 
   \multicolumn{1}{S{Data}{r}}{3} & 
   \multicolumn{1}{S{Data}{r}}{218} & 
   \multicolumn{1}{S{Data}{l}}{FIPS county}
\\
   \multicolumn{1}{S{RowHeader}{r}}{5} & 
   \multicolumn{1}{S{Data}{l}}{sex} & 
   \multicolumn{1}{S{Data}{l}}{Num} & 
   \multicolumn{1}{S{Data}{r}}{3} & 
   \multicolumn{1}{S{Data}{r}}{281} & 
   \multicolumn{1}{S{Data}{l}}{Sex}
\\
   \multicolumn{1}{S{RowHeader}{r}}{6} & 
   \multicolumn{1}{S{Data}{l}}{agegroup} & 
   \multicolumn{1}{S{Data}{l}}{Num} & 
   \multicolumn{1}{S{Data}{r}}{3} & 
   \multicolumn{1}{S{Data}{r}}{284} & 
   \multicolumn{1}{S{Data}{l}}{Age group}
\\
   \multicolumn{1}{S{RowHeader}{r}}{7} & 
   \multicolumn{1}{S{Data}{l}}{A} & 
   \multicolumn{1}{S{Data}{l}}{Num} & 
   \multicolumn{1}{S{Data}{r}}{8} & 
   \multicolumn{1}{S{Data}{r}}{0} & 
   \multicolumn{1}{S{Data}{l}}{Accessions}
\\
   \multicolumn{1}{S{RowHeader}{r}}{8} & 
   \multicolumn{1}{S{Data}{l}}{B} & 
   \multicolumn{1}{S{Data}{l}}{Num} & 
   \multicolumn{1}{S{Data}{r}}{8} & 
   \multicolumn{1}{S{Data}{r}}{8} & 
   \multicolumn{1}{S{Data}{l}}{Beginning-of-period employment}
\\
   \multicolumn{1}{S{RowHeader}{r}}{9} & 
   \multicolumn{1}{S{Data}{l}}{E} & 
   \multicolumn{1}{S{Data}{l}}{Num} & 
   \multicolumn{1}{S{Data}{r}}{8} & 
   \multicolumn{1}{S{Data}{r}}{16} & 
   \multicolumn{1}{S{Data}{l}}{End-of-period employment}
\\
   \multicolumn{1}{S{RowHeader}{r}}{10} & 
   \multicolumn{1}{S{Data}{l}}{F} & 
   \multicolumn{1}{S{Data}{l}}{Num} & 
   \multicolumn{1}{S{Data}{r}}{8} & 
   \multicolumn{1}{S{Data}{r}}{24} & 
   \multicolumn{1}{S{Data}{l}}{Full-quarter employment}
\\
   \multicolumn{1}{S{RowHeader}{r}}{11} & 
   \multicolumn{1}{S{Data}{l}}{FA} & 
   \multicolumn{1}{S{Data}{l}}{Num} & 
   \multicolumn{1}{S{Data}{r}}{8} & 
   \multicolumn{1}{S{Data}{r}}{32} & 
   \multicolumn{1}{S{Data}{l}}{Flow into full-quarter employment}
\\
   \multicolumn{1}{S{RowHeader}{r}}{12} & 
   \multicolumn{1}{S{Data}{l}}{FJC} & 
   \multicolumn{1}{S{Data}{l}}{Num} & 
   \multicolumn{1}{S{Data}{r}}{8} & 
   \multicolumn{1}{S{Data}{r}}{40} & 
   \multicolumn{1}{S{Data}{l}}{Full-quarter job creation}
\\
   \multicolumn{1}{S{RowHeader}{r}}{13} & 
   \multicolumn{1}{S{Data}{l}}{FJD} & 
   \multicolumn{1}{S{Data}{l}}{Num} & 
   \multicolumn{1}{S{Data}{r}}{8} & 
   \multicolumn{1}{S{Data}{r}}{48} & 
   \multicolumn{1}{S{Data}{l}}{Full-quarter job destruction}
\\
   \multicolumn{1}{S{RowHeader}{r}}{14} & 
   \multicolumn{1}{S{Data}{l}}{FJF} & 
   \multicolumn{1}{S{Data}{l}}{Num} & 
   \multicolumn{1}{S{Data}{r}}{8} & 
   \multicolumn{1}{S{Data}{r}}{56} & 
   \multicolumn{1}{S{Data}{l}}{Net change in full-quarter employment}
\\
   \multicolumn{1}{S{RowHeader}{r}}{15} & 
   \multicolumn{1}{S{Data}{l}}{FS} & 
   \multicolumn{1}{S{Data}{l}}{Num} & 
   \multicolumn{1}{S{Data}{r}}{8} & 
   \multicolumn{1}{S{Data}{r}}{64} & 
   \multicolumn{1}{S{Data}{l}}{Flow out of full-quarter employment}
\\
   \multicolumn{1}{S{RowHeader}{r}}{16} & 
   \multicolumn{1}{S{Data}{l}}{H} & 
   \multicolumn{1}{S{Data}{l}}{Num} & 
   \multicolumn{1}{S{Data}{r}}{8} & 
   \multicolumn{1}{S{Data}{r}}{72} & 
   \multicolumn{1}{S{Data}{l}}{New hires}
\\
   \multicolumn{1}{S{RowHeader}{r}}{17} & 
   \multicolumn{1}{S{Data}{l}}{H3} & 
   \multicolumn{1}{S{Data}{l}}{Num} & 
   \multicolumn{1}{S{Data}{r}}{8} & 
   \multicolumn{1}{S{Data}{r}}{80} & 
   \multicolumn{1}{S{Data}{l}}{Full-quarter new hires}
\\
   \multicolumn{1}{S{RowHeader}{r}}{18} & 
   \multicolumn{1}{S{Data}{l}}{JC} & 
   \multicolumn{1}{S{Data}{l}}{Num} & 
   \multicolumn{1}{S{Data}{r}}{8} & 
   \multicolumn{1}{S{Data}{r}}{88} & 
   \multicolumn{1}{S{Data}{l}}{Job creation}
\\
   \multicolumn{1}{S{RowHeader}{r}}{19} & 
   \multicolumn{1}{S{Data}{l}}{JD} & 
   \multicolumn{1}{S{Data}{l}}{Num} & 
   \multicolumn{1}{S{Data}{r}}{8} & 
   \multicolumn{1}{S{Data}{r}}{96} & 
   \multicolumn{1}{S{Data}{l}}{Job destruction}
\\
   \multicolumn{1}{S{RowHeader}{r}}{20} & 
   \multicolumn{1}{S{Data}{l}}{JF} & 
   \multicolumn{1}{S{Data}{l}}{Num} & 
   \multicolumn{1}{S{Data}{r}}{8} & 
   \multicolumn{1}{S{Data}{r}}{104} & 
   \multicolumn{1}{S{Data}{l}}{Net job flows}
\\
   \multicolumn{1}{S{RowHeader}{r}}{21} & 
   \multicolumn{1}{S{Data}{l}}{R} & 
   \multicolumn{1}{S{Data}{l}}{Num} & 
   \multicolumn{1}{S{Data}{r}}{8} & 
   \multicolumn{1}{S{Data}{r}}{112} & 
   \multicolumn{1}{S{Data}{l}}{Recalls}
\\
   \multicolumn{1}{S{RowHeader}{r}}{22} & 
   \multicolumn{1}{S{Data}{l}}{S} & 
   \multicolumn{1}{S{Data}{l}}{Num} & 
   \multicolumn{1}{S{Data}{r}}{8} & 
   \multicolumn{1}{S{Data}{r}}{120} & 
   \multicolumn{1}{S{Data}{l}}{Separations}
\\
   \multicolumn{1}{S{RowHeader}{r}}{23} & 
   \multicolumn{1}{S{Data}{l}}{Z{\textunderscore}NA} & 
   \multicolumn{1}{S{Data}{l}}{Num} & 
   \multicolumn{1}{S{Data}{r}}{8} & 
   \multicolumn{1}{S{Data}{r}}{128} & 
   \multicolumn{1}{S{Data}{l}}{Average periods of non-employment for accessions}
\\
   \multicolumn{1}{S{RowHeader}{r}}{24} & 
   \multicolumn{1}{S{Data}{l}}{Z{\textunderscore}NH} & 
   \multicolumn{1}{S{Data}{l}}{Num} & 
   \multicolumn{1}{S{Data}{r}}{8} & 
   \multicolumn{1}{S{Data}{r}}{136} & 
   \multicolumn{1}{S{Data}{l}}{Average periods of non-employment for new hires}
\\
   \multicolumn{1}{S{RowHeader}{r}}{25} & 
   \multicolumn{1}{S{Data}{l}}{Z{\textunderscore}NR} & 
   \multicolumn{1}{S{Data}{l}}{Num} & 
   \multicolumn{1}{S{Data}{r}}{8} & 
   \multicolumn{1}{S{Data}{r}}{144} & 
   \multicolumn{1}{S{Data}{l}}{Average periods of non-employment for recalls}
\\
   \multicolumn{1}{S{RowHeader}{r}}{26} & 
   \multicolumn{1}{S{Data}{l}}{Z{\textunderscore}NS} & 
   \multicolumn{1}{S{Data}{l}}{Num} & 
   \multicolumn{1}{S{Data}{r}}{8} & 
   \multicolumn{1}{S{Data}{r}}{152} & 
   \multicolumn{1}{S{Data}{l}}{Average periods of non-employment for separations}
\\
   \multicolumn{1}{S{RowHeader}{r}}{27} & 
   \multicolumn{1}{S{Data}{l}}{Z{\textunderscore}W2} & 
   \multicolumn{1}{S{Data}{l}}{Num} & 
   \multicolumn{1}{S{Data}{r}}{8} & 
   \multicolumn{1}{S{Data}{r}}{160} & 
   \multicolumn{1}{S{Data}{l}}{Average earnings of end-of-period employees}
\\
   \multicolumn{1}{S{RowHeader}{r}}{28} & 
   \multicolumn{1}{S{Data}{l}}{Z{\textunderscore}W3} & 
   \multicolumn{1}{S{Data}{l}}{Num} & 
   \multicolumn{1}{S{Data}{r}}{8} & 
   \multicolumn{1}{S{Data}{r}}{168} & 
   \multicolumn{1}{S{Data}{l}}{Average earnings of full-quarter employees}
\\
   \multicolumn{1}{S{RowHeader}{r}}{29} & 
   \multicolumn{1}{S{Data}{l}}{Z{\textunderscore}WFA} & 
   \multicolumn{1}{S{Data}{l}}{Num} & 
   \multicolumn{1}{S{Data}{r}}{8} & 
   \multicolumn{1}{S{Data}{r}}{176} & 
   \multicolumn{1}{S{Data}{l}}{Average earnings of transits to full-quarter status}
\\
   \multicolumn{1}{S{RowHeader}{r}}{30} & 
   \multicolumn{1}{S{Data}{l}}{Z{\textunderscore}WFS} & 
   \multicolumn{1}{S{Data}{l}}{Num} & 
   \multicolumn{1}{S{Data}{r}}{8} & 
   \multicolumn{1}{S{Data}{r}}{184} & 
   \multicolumn{1}{S{Data}{l}}{Average earnings of separations from full-quarter status}
\\
   \multicolumn{1}{S{RowHeader}{r}}{31} & 
   \multicolumn{1}{S{Data}{l}}{Z{\textunderscore}WH3} & 
   \multicolumn{1}{S{Data}{l}}{Num} & 
   \multicolumn{1}{S{Data}{r}}{8} & 
   \multicolumn{1}{S{Data}{r}}{192} & 
   \multicolumn{1}{S{Data}{l}}{Average earnings of full-quarter new hires}
\\
   \multicolumn{1}{S{RowHeader}{r}}{32} & 
   \multicolumn{1}{S{Data}{l}}{Z{\textunderscore}dWA} & 
   \multicolumn{1}{S{Data}{l}}{Num} & 
   \multicolumn{1}{S{Data}{r}}{8} & 
   \multicolumn{1}{S{Data}{r}}{200} & 
   \multicolumn{1}{S{Data}{l}}{Average change in total earnings for accessions}
\\
   \multicolumn{1}{S{RowHeader}{r}}{33} & 
   \multicolumn{1}{S{Data}{l}}{Z{\textunderscore}dWS} & 
   \multicolumn{1}{S{Data}{l}}{Num} & 
   \multicolumn{1}{S{Data}{r}}{8} & 
   \multicolumn{1}{S{Data}{r}}{208} & 
   \multicolumn{1}{S{Data}{l}}{Average change in total earnings for separations}
\\
   \multicolumn{1}{S{RowHeader}{r}}{34} & 
   \multicolumn{1}{S{Data}{l}}{A{\textunderscore}status} & 
   \multicolumn{1}{S{Data}{l}}{Char} & 
   \multicolumn{1}{S{Data}{r}}{2} & 
   \multicolumn{1}{S{Data}{r}}{221} & 
   \multicolumn{1}{S{Data}{l}}{Status: accessions}
\\
   \multicolumn{1}{S{RowHeader}{r}}{35} & 
   \multicolumn{1}{S{Data}{l}}{B{\textunderscore}status} & 
   \multicolumn{1}{S{Data}{l}}{Char} & 
   \multicolumn{1}{S{Data}{r}}{2} & 
   \multicolumn{1}{S{Data}{r}}{223} & 
   \multicolumn{1}{S{Data}{l}}{Status: beginning-of-period employment}
\\
   \multicolumn{1}{S{RowHeader}{r}}{36} & 
   \multicolumn{1}{S{Data}{l}}{E{\textunderscore}status} & 
   \multicolumn{1}{S{Data}{l}}{Char} & 
   \multicolumn{1}{S{Data}{r}}{2} & 
   \multicolumn{1}{S{Data}{r}}{225} & 
   \multicolumn{1}{S{Data}{l}}{Status: end-of-period employment}
\\
   \multicolumn{1}{S{RowHeader}{r}}{37} & 
   \multicolumn{1}{S{Data}{l}}{F{\textunderscore}status} & 
   \multicolumn{1}{S{Data}{l}}{Char} & 
   \multicolumn{1}{S{Data}{r}}{2} & 
   \multicolumn{1}{S{Data}{r}}{227} & 
   \multicolumn{1}{S{Data}{l}}{Status: full-quarter employment}
\\
   \multicolumn{1}{S{RowHeader}{r}}{38} & 
   \multicolumn{1}{S{Data}{l}}{FA{\textunderscore}status} & 
   \multicolumn{1}{S{Data}{l}}{Char} & 
   \multicolumn{1}{S{Data}{r}}{2} & 
   \multicolumn{1}{S{Data}{r}}{229} & 
   \multicolumn{1}{S{Data}{l}}{Status: flow into full-quarter employment}
\\
   \multicolumn{1}{S{RowHeader}{r}}{39} & 
   \multicolumn{1}{S{Data}{l}}{FJC{\textunderscore}status} & 
   \multicolumn{1}{S{Data}{l}}{Char} & 
   \multicolumn{1}{S{Data}{r}}{2} & 
   \multicolumn{1}{S{Data}{r}}{231} & 
   \multicolumn{1}{S{Data}{l}}{Status: full-quarter job creation}
\\
   \multicolumn{1}{S{RowHeader}{r}}{40} & 
   \multicolumn{1}{S{Data}{l}}{FJD{\textunderscore}status} & 
   \multicolumn{1}{S{Data}{l}}{Char} & 
   \multicolumn{1}{S{Data}{r}}{2} & 
   \multicolumn{1}{S{Data}{r}}{233} & 
   \multicolumn{1}{S{Data}{l}}{Status: full-quarter job destruction}
\\
   \multicolumn{1}{S{RowHeader}{r}}{41} & 
   \multicolumn{1}{S{Data}{l}}{FJF{\textunderscore}status} & 
   \multicolumn{1}{S{Data}{l}}{Char} & 
   \multicolumn{1}{S{Data}{r}}{2} & 
   \multicolumn{1}{S{Data}{r}}{235} & 
   \multicolumn{1}{S{Data}{l}}{Status: net change in full-quarter employment}
\\
   \multicolumn{1}{S{RowHeader}{r}}{42} & 
   \multicolumn{1}{S{Data}{l}}{FS{\textunderscore}status} & 
   \multicolumn{1}{S{Data}{l}}{Char} & 
   \multicolumn{1}{S{Data}{r}}{2} & 
   \multicolumn{1}{S{Data}{r}}{237} & 
   \multicolumn{1}{S{Data}{l}}{Status: flow out of full-quarter employment}
\\
   \multicolumn{1}{S{RowHeader}{r}}{43} & 
   \multicolumn{1}{S{Data}{l}}{H{\textunderscore}status} & 
   \multicolumn{1}{S{Data}{l}}{Char} & 
   \multicolumn{1}{S{Data}{r}}{2} & 
   \multicolumn{1}{S{Data}{r}}{239} & 
   \multicolumn{1}{S{Data}{l}}{Status: new hires}
\\
   \multicolumn{1}{S{RowHeader}{r}}{44} & 
   \multicolumn{1}{S{Data}{l}}{H3{\textunderscore}status} & 
   \multicolumn{1}{S{Data}{l}}{Char} & 
   \multicolumn{1}{S{Data}{r}}{2} & 
   \multicolumn{1}{S{Data}{r}}{241} & 
   \multicolumn{1}{S{Data}{l}}{Status: full-quarter new hires}
\\
   \multicolumn{1}{S{RowHeader}{r}}{45} & 
   \multicolumn{1}{S{Data}{l}}{JC{\textunderscore}status} & 
   \multicolumn{1}{S{Data}{l}}{Char} & 
   \multicolumn{1}{S{Data}{r}}{2} & 
   \multicolumn{1}{S{Data}{r}}{243} & 
   \multicolumn{1}{S{Data}{l}}{Status: job creation}
\\
   \multicolumn{1}{S{RowHeader}{r}}{46} & 
   \multicolumn{1}{S{Data}{l}}{JD{\textunderscore}status} & 
   \multicolumn{1}{S{Data}{l}}{Char} & 
   \multicolumn{1}{S{Data}{r}}{2} & 
   \multicolumn{1}{S{Data}{r}}{245} & 
   \multicolumn{1}{S{Data}{l}}{Status: job destruction}
\\
   \multicolumn{1}{S{RowHeader}{r}}{47} & 
   \multicolumn{1}{S{Data}{l}}{JF{\textunderscore}status} & 
   \multicolumn{1}{S{Data}{l}}{Char} & 
   \multicolumn{1}{S{Data}{r}}{2} & 
   \multicolumn{1}{S{Data}{r}}{247} & 
   \multicolumn{1}{S{Data}{l}}{Status: net job flows}
\\
   \multicolumn{1}{S{RowHeader}{r}}{48} & 
   \multicolumn{1}{S{Data}{l}}{R{\textunderscore}status} & 
   \multicolumn{1}{S{Data}{l}}{Char} & 
   \multicolumn{1}{S{Data}{r}}{2} & 
   \multicolumn{1}{S{Data}{r}}{249} & 
   \multicolumn{1}{S{Data}{l}}{Status: recalls}
\\
   \multicolumn{1}{S{RowHeader}{r}}{49} & 
   \multicolumn{1}{S{Data}{l}}{S{\textunderscore}status} & 
   \multicolumn{1}{S{Data}{l}}{Char} & 
   \multicolumn{1}{S{Data}{r}}{2} & 
   \multicolumn{1}{S{Data}{r}}{251} & 
   \multicolumn{1}{S{Data}{l}}{Status: separations}
\\
   \multicolumn{1}{S{RowHeader}{r}}{50} & 
   \multicolumn{1}{S{Data}{l}}{Z{\textunderscore}NA{\textunderscore}status} & 
   \multicolumn{1}{S{Data}{l}}{Char} & 
   \multicolumn{1}{S{Data}{r}}{2} & 
   \multicolumn{1}{S{Data}{r}}{253} & 
   \multicolumn{1}{S{Data}{l}}{Status: average periods of non-employment for accessions}
\\
   \multicolumn{1}{S{RowHeader}{r}}{51} & 
   \multicolumn{1}{S{Data}{l}}{Z{\textunderscore}NH{\textunderscore}status} & 
   \multicolumn{1}{S{Data}{l}}{Char} & 
   \multicolumn{1}{S{Data}{r}}{2} & 
   \multicolumn{1}{S{Data}{r}}{255} & 
   \multicolumn{1}{S{Data}{l}}{Status: average periods of non-employment for new hires}
\\
   \multicolumn{1}{S{RowHeader}{r}}{52} & 
   \multicolumn{1}{S{Data}{l}}{Z{\textunderscore}NR{\textunderscore}status} & 
   \multicolumn{1}{S{Data}{l}}{Char} & 
   \multicolumn{1}{S{Data}{r}}{2} & 
   \multicolumn{1}{S{Data}{r}}{257} & 
   \multicolumn{1}{S{Data}{l}}{Status: average periods of non-employment for recalls}
\\
   \multicolumn{1}{S{RowHeader}{r}}{53} & 
   \multicolumn{1}{S{Data}{l}}{Z{\textunderscore}NS{\textunderscore}status} & 
   \multicolumn{1}{S{Data}{l}}{Char} & 
   \multicolumn{1}{S{Data}{r}}{2} & 
   \multicolumn{1}{S{Data}{r}}{259} & 
   \multicolumn{1}{S{Data}{l}}{Status: average periods of non-employment for separations}
\\
   \multicolumn{1}{S{RowHeader}{r}}{54} & 
   \multicolumn{1}{S{Data}{l}}{Z{\textunderscore}W2{\textunderscore}status} & 
   \multicolumn{1}{S{Data}{l}}{Char} & 
   \multicolumn{1}{S{Data}{r}}{2} & 
   \multicolumn{1}{S{Data}{r}}{261} & 
   \multicolumn{1}{S{Data}{l}}{Status: average earnings of end-of-period employees}
\\
   \multicolumn{1}{S{RowHeader}{r}}{55} & 
   \multicolumn{1}{S{Data}{l}}{Z{\textunderscore}W3{\textunderscore}status} & 
   \multicolumn{1}{S{Data}{l}}{Char} & 
   \multicolumn{1}{S{Data}{r}}{2} & 
   \multicolumn{1}{S{Data}{r}}{263} & 
   \multicolumn{1}{S{Data}{l}}{Status: average earnings of full-quarter employees}
\\
   \multicolumn{1}{S{RowHeader}{r}}{56} & 
   \multicolumn{1}{S{Data}{l}}{Z{\textunderscore}WFA{\textunderscore}status} & 
   \multicolumn{1}{S{Data}{l}}{Char} & 
   \multicolumn{1}{S{Data}{r}}{2} & 
   \multicolumn{1}{S{Data}{r}}{265} & 
   \multicolumn{1}{S{Data}{l}}{Status: average earnings of transits to full-quarter status}
\\
   \multicolumn{1}{S{RowHeader}{r}}{57} & 
   \multicolumn{1}{S{Data}{l}}{Z{\textunderscore}WFS{\textunderscore}status} & 
   \multicolumn{1}{S{Data}{l}}{Char} & 
   \multicolumn{1}{S{Data}{r}}{2} & 
   \multicolumn{1}{S{Data}{r}}{267} & 
   \multicolumn{1}{S{Data}{l}}{Status: average earnings of separations from full-quarter status}
\\
   \multicolumn{1}{S{RowHeader}{r}}{58} & 
   \multicolumn{1}{S{Data}{l}}{Z{\textunderscore}WH3{\textunderscore}status} & 
   \multicolumn{1}{S{Data}{l}}{Char} & 
   \multicolumn{1}{S{Data}{r}}{2} & 
   \multicolumn{1}{S{Data}{r}}{269} & 
   \multicolumn{1}{S{Data}{l}}{Status: average earnings of full-quarter new hires}
\\
   \multicolumn{1}{S{RowHeader}{r}}{59} & 
   \multicolumn{1}{S{Data}{l}}{Z{\textunderscore}dWA{\textunderscore}status} & 
   \multicolumn{1}{S{Data}{l}}{Char} & 
   \multicolumn{1}{S{Data}{r}}{2} & 
   \multicolumn{1}{S{Data}{r}}{271} & 
   \multicolumn{1}{S{Data}{l}}{Status: average change in total earnings for accessions}
\\
   \multicolumn{1}{S{RowHeader}{r}}{60} & 
   \multicolumn{1}{S{Data}{l}}{Z{\textunderscore}dWS{\textunderscore}status} & 
   \multicolumn{1}{S{Data}{l}}{Char} & 
   \multicolumn{1}{S{Data}{r}}{2} & 
   \multicolumn{1}{S{Data}{r}}{273} & 
   \multicolumn{1}{S{Data}{l}}{Status: average change in total earnings for separations}
\\
\end{longtable}
\end{center}

% ====================end of output====================
 \newpage %    Generated by SAS
%    http://www.sas.com
% created by=vilhu001
% sasversion=8.2
% date=2002-05-23
% time=00:50:05
% encoding=iso-8859-1
% ====================begin of output====================
% \begin{document}

% An external file needs to be included, as specified% in latexlong.sas. This can be called sas.sty,
% in which case you want to include a line like
% \usepackage{sas}
% or it can be a simple (La)TeX file, which you 
% include by typing 
% %%
%% This is file `sas.sty',
%% generated with the docstrip utility.
%%
%% 
\NeedsTeXFormat{LaTeX2e}
\ProvidesPackage{sas}
        [2002/01/18 LEHD version 0.1
    provides definition for tables generated by SAS%
                   ]
\@ifundefined{array@processline}{\RequirePackage{array}}{}
\@ifundefined{longtable@processline}{\RequirePackage{longtable}}{}
 \def\ContentTitle{\small\it\sffamily}
 \def\Output{\small\sffamily}
 \def\HeaderEmphasis{\small\it\sffamily}
 \def\NoteContent{\small\sffamily}
 \def\FatalContent{\small\sffamily}
 \def\Graph{\small\sffamily}
 \def\WarnContentFixed{\footnotesize\tt}
 \def\NoteBanner{\small\sffamily}
 \def\DataStrong{\normalsize\bf\sffamily}
 \def\Document{\small\sffamily}
 \def\BeforeCaption{\normalsize\bf\sffamily}
 \def\ContentsDate{\small\sffamily}
 \def\Pages{\small\sffamily}
 \def\TitlesAndFooters{\footnotesize\bf\it\sffamily}
 \def\IndexProcName{\small\sffamily}
 \def\ProcTitle{\normalsize\bf\it\sffamily}
 \def\IndexAction{\small\sffamily}
 \def\Data{\small\sffamily}
 \def\Table{\small\sffamily}
 \def\FooterEmpty{\footnotesize\bf\sffamily}
 \def\SysTitleAndFooterContainer{\footnotesize\sffamily}
 \def\RowFooterEmpty{\footnotesize\bf\sffamily}
 \def\ExtendedPage{\small\it\sffamily}
 \def\FooterFixed{\footnotesize\tt}
 \def\RowFooterStrongFixed{\footnotesize\bf\tt}
 \def\RowFooterEmphasis{\footnote\it\sffamily}
 \def\ContentFolder{\small\sffamily}
 \def\Container{\small\sffamily}
 \def\Date{\small\sffamily}
 \def\RowFooterFixed{\footnotesize\tt}
 \def\Caption{\normalsize\bf\sffamily}
 \def\WarnBanner{\small\sffamily}
 \def\Frame{\small\sffamily}
 \def\HeaderStrongFixed{\footnotesize\bf\tt}
 \def\IndexTitle{\small\it\sffamily}
 \def\NoteContentFixed{\footnotesize\tt}
 \def\DataEmphasisFixed{\footnotesize\it\tt}
 \def\Note{\small\sffamily}
 \def\Byline{\normalsize\bf\sffamily}
 \def\FatalBanner{\small\sffamily}
 \def\ProcTitleFixed{\footnotesize\bf\tt}
 \def\ByContentFolder{\small\sffamily}
 \def\PagesProcLabel{\small\sffamily}
 \def\RowHeaderFixed{\footnotesize\tt}
 \def\RowFooterEmphasisFixed{\footnotesize\it\tt}
 \def\WarnContent{\small\sffamily}
 \def\DataEmpty{\small\sffamily}
 \def\Cell{\small\sffamily}
 \def\Header{\normalsize\bf\sffamily}
 \def\PageNo{\normalsize\bf\sffamily}
 \def\ContentProcLabel{\small\sffamily}
 \def\HeaderFixed{\footnotesize\tt}
 \def\PagesTitle{\small\it\sffamily}
 \def\RowHeaderEmpty{\normalsize\bf\sffamily}
 \def\PagesProcName{\small\sffamily}
 \def\Batch{\footnotesize\tt}
 \def\ContentItem{\small\sffamily}
 \def\Body{\small\sffamily}
 \def\PagesDate{\small\sffamily}
 \def\Index{\small\sffamily}
 \def\HeaderEmpty{\normalsize\bf\sffamily}
 \def\FooterStrong{\footnotesize\bf\sffamily}
 \def\FooterEmphasis{\footnotesize\it\sffamily}
 \def\ErrorContent{\small\sffamily}
 \def\DataFixed{\footnotesize\tt}
 \def\HeaderStrong{\normalsize\bf\sffamily}
 \def\GraphBackground{}
 \def\DataEmphasis{\small\it\sffamily}
 \def\TitleAndNoteContainer{\small\sffamily}
 \def\RowFooter{\footnotesize\bf\sffamily}
 \def\IndexItem{\small\sffamily}
 \def\BylineContainer{\small\sffamily}
 \def\FatalContentFixed{\footnotesize\tt}
 \def\BodyDate{\small\sffamily}
 \def\RowFooterStrong{\footnotesize\bf\sffamily}
 \def\UserText{\small\sffamily}
 \def\HeadersAndFooters{\footnotesize\bf\sffamily}
 \def\RowHeaderEmphasisFixed{\footnotesize\it\tt}
 \def\ErrorBanner{\small\sffamily}
 \def\ContentProcName{\small\sffamily}
 \def\RowHeaderStrong{\normalsize\bf\sffamily}
 \def\FooterEmphasisFixed{\footnotesize\it\tt}
 \def\Contents{\small\sffamily}
 \def\FooterStrongFixed{\footnotesize\bf\tt}
 \def\PagesItem{\small\sffamily}
 \def\RowHeader{\normalsize\bf\sffamily}
 \def\AfterCaption{\normalsize\bf\sffamily}
 \def\RowHeaderStrongFixed{\footnotesize\bf\tt}
 \def\RowHeaderEmphasis{\small\it\sffamily}
 \def\DataStrongFixed{\footnotesize\bf\tt}
 \def\Footer{\footnotesize\bf\sffamily}
 \def\FolderAction{\small\sffamily}
 \def\HeaderEmphasisFixed{\footnotesize\it\tt}
 \def\SystemTitle{\large\bf\it\sffamily}
 \def\ErrorContentFixed{\footnotesize\tt}
 \def\SystemFooter{\footnotesize\it\sffamily}
% Set cell padding 
\renewcommand{\arraystretch}{1.3}
% Headings
\newcommand{\heading}[2]{\csname#1\endcsname #2}
\newcommand{\proctitle}[2]{\csname#1\endcsname #2}
% Declare new column type
\newcolumntype{S}[2]{>{\csname#1\endcsname}#2}
% Set warning box style
\newcommand{\msg}[2]{\fbox{%
   \begin{minipage}{\textwidth}#2\end{minipage}}%
}

\begin{center}\heading{SystemTitle}{Minnesota           }\end{center}
\begin{center}\heading{ProcTitle}{The FREQ Procedure}\end{center}
\begin{center}\begin{longtable}
{lrrrr}\hline % colspecs
% table_head start
   \multicolumn{5}{S{Header}{c}}{FIPS State}
\\
   \multicolumn{1}{S{Header}{l}}{state} & 
   \multicolumn{1}{S{Header}{r}}{Frequency} & 
   \multicolumn{1}{S{Header}{r}}{ Percent} & 
   \multicolumn{1}{S{Header}{r}}{Cumulative\linebreak  Frequency} & 
   \multicolumn{1}{S{Header}{r}}{Cumulative\linebreak   Percent}
\\
\hline 
\endhead % table_head end
\hline 
\multicolumn{1}{r}{(cont.)}\\
\endfoot 
\hline 
\endlastfoot % table_head end
   \multicolumn{1}{S{RowHeader}{l}}{27 MINNESOTA} & 
   \multicolumn{1}{S{Data}{r}}{61452} & 
   \multicolumn{1}{S{Data}{r}}{100.00} & 
   \multicolumn{1}{S{Data}{r}}{61452} & 
   \multicolumn{1}{S{Data}{r}}{100.00}
\\
\end{longtable}
\end{center}
\begin{center}\begin{longtable}
{lrrrr}\hline % colspecs
% table_head start
   \multicolumn{5}{S{Header}{c}}{FIPS county}
\\
   \multicolumn{1}{S{Header}{l}}{county} & 
   \multicolumn{1}{S{Header}{r}}{Frequency} & 
   \multicolumn{1}{S{Header}{r}}{ Percent} & 
   \multicolumn{1}{S{Header}{r}}{Cumulative\linebreak  Frequency} & 
   \multicolumn{1}{S{Header}{r}}{Cumulative\linebreak   Percent}
\\
\hline 
\endhead % table_head end
\hline 
\multicolumn{1}{r}{(cont.)}\\
\endfoot 
\hline 
\endlastfoot % table_head end
   \multicolumn{1}{S{RowHeader}{l}}{000 MINNESOTA} & 
   \multicolumn{1}{S{Data}{r}}{702} & 
   \multicolumn{1}{S{Data}{r}}{1.14} & 
   \multicolumn{1}{S{Data}{r}}{702} & 
   \multicolumn{1}{S{Data}{r}}{1.14}
\\
   \multicolumn{1}{S{RowHeader}{l}}{001 AITKIN} & 
   \multicolumn{1}{S{Data}{r}}{702} & 
   \multicolumn{1}{S{Data}{r}}{1.14} & 
   \multicolumn{1}{S{Data}{r}}{1404} & 
   \multicolumn{1}{S{Data}{r}}{2.28}
\\
   \multicolumn{1}{S{RowHeader}{l}}{003 ANOKA} & 
   \multicolumn{1}{S{Data}{r}}{702} & 
   \multicolumn{1}{S{Data}{r}}{1.14} & 
   \multicolumn{1}{S{Data}{r}}{2106} & 
   \multicolumn{1}{S{Data}{r}}{3.43}
\\
   \multicolumn{1}{S{RowHeader}{l}}{005 BECKER} & 
   \multicolumn{1}{S{Data}{r}}{702} & 
   \multicolumn{1}{S{Data}{r}}{1.14} & 
   \multicolumn{1}{S{Data}{r}}{2808} & 
   \multicolumn{1}{S{Data}{r}}{4.57}
\\
   \multicolumn{1}{S{RowHeader}{l}}{007 BELTRAMI} & 
   \multicolumn{1}{S{Data}{r}}{702} & 
   \multicolumn{1}{S{Data}{r}}{1.14} & 
   \multicolumn{1}{S{Data}{r}}{3510} & 
   \multicolumn{1}{S{Data}{r}}{5.71}
\\
   \multicolumn{1}{S{RowHeader}{l}}{009 BENTON} & 
   \multicolumn{1}{S{Data}{r}}{702} & 
   \multicolumn{1}{S{Data}{r}}{1.14} & 
   \multicolumn{1}{S{Data}{r}}{4212} & 
   \multicolumn{1}{S{Data}{r}}{6.85}
\\
   \multicolumn{1}{S{RowHeader}{l}}{011 BIG STONE} & 
   \multicolumn{1}{S{Data}{r}}{702} & 
   \multicolumn{1}{S{Data}{r}}{1.14} & 
   \multicolumn{1}{S{Data}{r}}{4914} & 
   \multicolumn{1}{S{Data}{r}}{8.00}
\\
   \multicolumn{1}{S{RowHeader}{l}}{013 BLUE EARTH} & 
   \multicolumn{1}{S{Data}{r}}{702} & 
   \multicolumn{1}{S{Data}{r}}{1.14} & 
   \multicolumn{1}{S{Data}{r}}{5616} & 
   \multicolumn{1}{S{Data}{r}}{9.14}
\\
   \multicolumn{1}{S{RowHeader}{l}}{015 BROWN} & 
   \multicolumn{1}{S{Data}{r}}{702} & 
   \multicolumn{1}{S{Data}{r}}{1.14} & 
   \multicolumn{1}{S{Data}{r}}{6318} & 
   \multicolumn{1}{S{Data}{r}}{10.28}
\\
   \multicolumn{1}{S{RowHeader}{l}}{017 CARLTON} & 
   \multicolumn{1}{S{Data}{r}}{702} & 
   \multicolumn{1}{S{Data}{r}}{1.14} & 
   \multicolumn{1}{S{Data}{r}}{7020} & 
   \multicolumn{1}{S{Data}{r}}{11.42}
\\
   \multicolumn{1}{S{RowHeader}{l}}{019 CARVER} & 
   \multicolumn{1}{S{Data}{r}}{702} & 
   \multicolumn{1}{S{Data}{r}}{1.14} & 
   \multicolumn{1}{S{Data}{r}}{7722} & 
   \multicolumn{1}{S{Data}{r}}{12.57}
\\
   \multicolumn{1}{S{RowHeader}{l}}{021 CASS} & 
   \multicolumn{1}{S{Data}{r}}{702} & 
   \multicolumn{1}{S{Data}{r}}{1.14} & 
   \multicolumn{1}{S{Data}{r}}{8424} & 
   \multicolumn{1}{S{Data}{r}}{13.71}
\\
   \multicolumn{1}{S{RowHeader}{l}}{023 CHIPPEWA} & 
   \multicolumn{1}{S{Data}{r}}{702} & 
   \multicolumn{1}{S{Data}{r}}{1.14} & 
   \multicolumn{1}{S{Data}{r}}{9126} & 
   \multicolumn{1}{S{Data}{r}}{14.85}
\\
   \multicolumn{1}{S{RowHeader}{l}}{025 CHISAGO} & 
   \multicolumn{1}{S{Data}{r}}{702} & 
   \multicolumn{1}{S{Data}{r}}{1.14} & 
   \multicolumn{1}{S{Data}{r}}{9828} & 
   \multicolumn{1}{S{Data}{r}}{15.99}
\\
   \multicolumn{1}{S{RowHeader}{l}}{027 CLAY} & 
   \multicolumn{1}{S{Data}{r}}{702} & 
   \multicolumn{1}{S{Data}{r}}{1.14} & 
   \multicolumn{1}{S{Data}{r}}{10530} & 
   \multicolumn{1}{S{Data}{r}}{17.14}
\\
   \multicolumn{1}{S{RowHeader}{l}}{029 CLEARWATER} & 
   \multicolumn{1}{S{Data}{r}}{702} & 
   \multicolumn{1}{S{Data}{r}}{1.14} & 
   \multicolumn{1}{S{Data}{r}}{11232} & 
   \multicolumn{1}{S{Data}{r}}{18.28}
\\
   \multicolumn{1}{S{RowHeader}{l}}{031 COOK} & 
   \multicolumn{1}{S{Data}{r}}{702} & 
   \multicolumn{1}{S{Data}{r}}{1.14} & 
   \multicolumn{1}{S{Data}{r}}{11934} & 
   \multicolumn{1}{S{Data}{r}}{19.42}
\\
   \multicolumn{1}{S{RowHeader}{l}}{033 COTTONWOOD} & 
   \multicolumn{1}{S{Data}{r}}{702} & 
   \multicolumn{1}{S{Data}{r}}{1.14} & 
   \multicolumn{1}{S{Data}{r}}{12636} & 
   \multicolumn{1}{S{Data}{r}}{20.56}
\\
   \multicolumn{1}{S{RowHeader}{l}}{035 CROW WING} & 
   \multicolumn{1}{S{Data}{r}}{702} & 
   \multicolumn{1}{S{Data}{r}}{1.14} & 
   \multicolumn{1}{S{Data}{r}}{13338} & 
   \multicolumn{1}{S{Data}{r}}{21.70}
\\
   \multicolumn{1}{S{RowHeader}{l}}{037 DAKOTA} & 
   \multicolumn{1}{S{Data}{r}}{702} & 
   \multicolumn{1}{S{Data}{r}}{1.14} & 
   \multicolumn{1}{S{Data}{r}}{14040} & 
   \multicolumn{1}{S{Data}{r}}{22.85}
\\
   \multicolumn{1}{S{RowHeader}{l}}{039 DODGE} & 
   \multicolumn{1}{S{Data}{r}}{702} & 
   \multicolumn{1}{S{Data}{r}}{1.14} & 
   \multicolumn{1}{S{Data}{r}}{14742} & 
   \multicolumn{1}{S{Data}{r}}{23.99}
\\
   \multicolumn{1}{S{RowHeader}{l}}{041 DOUGLAS} & 
   \multicolumn{1}{S{Data}{r}}{702} & 
   \multicolumn{1}{S{Data}{r}}{1.14} & 
   \multicolumn{1}{S{Data}{r}}{15444} & 
   \multicolumn{1}{S{Data}{r}}{25.13}
\\
   \multicolumn{1}{S{RowHeader}{l}}{043 FARIBAULT} & 
   \multicolumn{1}{S{Data}{r}}{702} & 
   \multicolumn{1}{S{Data}{r}}{1.14} & 
   \multicolumn{1}{S{Data}{r}}{16146} & 
   \multicolumn{1}{S{Data}{r}}{26.27}
\\
   \multicolumn{1}{S{RowHeader}{l}}{045 FILLMORE} & 
   \multicolumn{1}{S{Data}{r}}{702} & 
   \multicolumn{1}{S{Data}{r}}{1.14} & 
   \multicolumn{1}{S{Data}{r}}{16848} & 
   \multicolumn{1}{S{Data}{r}}{27.42}
\\
   \multicolumn{1}{S{RowHeader}{l}}{047 FREEBORN} & 
   \multicolumn{1}{S{Data}{r}}{702} & 
   \multicolumn{1}{S{Data}{r}}{1.14} & 
   \multicolumn{1}{S{Data}{r}}{17550} & 
   \multicolumn{1}{S{Data}{r}}{28.56}
\\
   \multicolumn{1}{S{RowHeader}{l}}{049 GOODHUE} & 
   \multicolumn{1}{S{Data}{r}}{702} & 
   \multicolumn{1}{S{Data}{r}}{1.14} & 
   \multicolumn{1}{S{Data}{r}}{18252} & 
   \multicolumn{1}{S{Data}{r}}{29.70}
\\
   \multicolumn{1}{S{RowHeader}{l}}{051 GRANT} & 
   \multicolumn{1}{S{Data}{r}}{702} & 
   \multicolumn{1}{S{Data}{r}}{1.14} & 
   \multicolumn{1}{S{Data}{r}}{18954} & 
   \multicolumn{1}{S{Data}{r}}{30.84}
\\
   \multicolumn{1}{S{RowHeader}{l}}{053 HENNEPIN} & 
   \multicolumn{1}{S{Data}{r}}{702} & 
   \multicolumn{1}{S{Data}{r}}{1.14} & 
   \multicolumn{1}{S{Data}{r}}{19656} & 
   \multicolumn{1}{S{Data}{r}}{31.99}
\\
   \multicolumn{1}{S{RowHeader}{l}}{055 HOUSTON} & 
   \multicolumn{1}{S{Data}{r}}{702} & 
   \multicolumn{1}{S{Data}{r}}{1.14} & 
   \multicolumn{1}{S{Data}{r}}{20358} & 
   \multicolumn{1}{S{Data}{r}}{33.13}
\\
   \multicolumn{1}{S{RowHeader}{l}}{057 HUBBARD} & 
   \multicolumn{1}{S{Data}{r}}{702} & 
   \multicolumn{1}{S{Data}{r}}{1.14} & 
   \multicolumn{1}{S{Data}{r}}{21060} & 
   \multicolumn{1}{S{Data}{r}}{34.27}
\\
   \multicolumn{1}{S{RowHeader}{l}}{059 ISANTI} & 
   \multicolumn{1}{S{Data}{r}}{702} & 
   \multicolumn{1}{S{Data}{r}}{1.14} & 
   \multicolumn{1}{S{Data}{r}}{21762} & 
   \multicolumn{1}{S{Data}{r}}{35.41}
\\
   \multicolumn{1}{S{RowHeader}{l}}{061 ITASCA} & 
   \multicolumn{1}{S{Data}{r}}{702} & 
   \multicolumn{1}{S{Data}{r}}{1.14} & 
   \multicolumn{1}{S{Data}{r}}{22464} & 
   \multicolumn{1}{S{Data}{r}}{36.56}
\\
   \multicolumn{1}{S{RowHeader}{l}}{063 JACKSON} & 
   \multicolumn{1}{S{Data}{r}}{702} & 
   \multicolumn{1}{S{Data}{r}}{1.14} & 
   \multicolumn{1}{S{Data}{r}}{23166} & 
   \multicolumn{1}{S{Data}{r}}{37.70}
\\
   \multicolumn{1}{S{RowHeader}{l}}{065 KANABEC} & 
   \multicolumn{1}{S{Data}{r}}{702} & 
   \multicolumn{1}{S{Data}{r}}{1.14} & 
   \multicolumn{1}{S{Data}{r}}{23868} & 
   \multicolumn{1}{S{Data}{r}}{38.84}
\\
   \multicolumn{1}{S{RowHeader}{l}}{067 KANDIYOHI} & 
   \multicolumn{1}{S{Data}{r}}{702} & 
   \multicolumn{1}{S{Data}{r}}{1.14} & 
   \multicolumn{1}{S{Data}{r}}{24570} & 
   \multicolumn{1}{S{Data}{r}}{39.98}
\\
   \multicolumn{1}{S{RowHeader}{l}}{069 KITTSON} & 
   \multicolumn{1}{S{Data}{r}}{702} & 
   \multicolumn{1}{S{Data}{r}}{1.14} & 
   \multicolumn{1}{S{Data}{r}}{25272} & 
   \multicolumn{1}{S{Data}{r}}{41.12}
\\
   \multicolumn{1}{S{RowHeader}{l}}{071 KOOCHICHING} & 
   \multicolumn{1}{S{Data}{r}}{702} & 
   \multicolumn{1}{S{Data}{r}}{1.14} & 
   \multicolumn{1}{S{Data}{r}}{25974} & 
   \multicolumn{1}{S{Data}{r}}{42.27}
\\
   \multicolumn{1}{S{RowHeader}{l}}{073 LAC QUI PARLE} & 
   \multicolumn{1}{S{Data}{r}}{702} & 
   \multicolumn{1}{S{Data}{r}}{1.14} & 
   \multicolumn{1}{S{Data}{r}}{26676} & 
   \multicolumn{1}{S{Data}{r}}{43.41}
\\
   \multicolumn{1}{S{RowHeader}{l}}{075 LAKE} & 
   \multicolumn{1}{S{Data}{r}}{702} & 
   \multicolumn{1}{S{Data}{r}}{1.14} & 
   \multicolumn{1}{S{Data}{r}}{27378} & 
   \multicolumn{1}{S{Data}{r}}{44.55}
\\
   \multicolumn{1}{S{RowHeader}{l}}{077 LAKE OF THE WOODS} & 
   \multicolumn{1}{S{Data}{r}}{702} & 
   \multicolumn{1}{S{Data}{r}}{1.14} & 
   \multicolumn{1}{S{Data}{r}}{28080} & 
   \multicolumn{1}{S{Data}{r}}{45.69}
\\
   \multicolumn{1}{S{RowHeader}{l}}{079 LE SUEUR} & 
   \multicolumn{1}{S{Data}{r}}{702} & 
   \multicolumn{1}{S{Data}{r}}{1.14} & 
   \multicolumn{1}{S{Data}{r}}{28782} & 
   \multicolumn{1}{S{Data}{r}}{46.84}
\\
   \multicolumn{1}{S{RowHeader}{l}}{081 LINCOLN} & 
   \multicolumn{1}{S{Data}{r}}{702} & 
   \multicolumn{1}{S{Data}{r}}{1.14} & 
   \multicolumn{1}{S{Data}{r}}{29484} & 
   \multicolumn{1}{S{Data}{r}}{47.98}
\\
   \multicolumn{1}{S{RowHeader}{l}}{083 LYON} & 
   \multicolumn{1}{S{Data}{r}}{702} & 
   \multicolumn{1}{S{Data}{r}}{1.14} & 
   \multicolumn{1}{S{Data}{r}}{30186} & 
   \multicolumn{1}{S{Data}{r}}{49.12}
\\
   \multicolumn{1}{S{RowHeader}{l}}{085 MCLEOD} & 
   \multicolumn{1}{S{Data}{r}}{702} & 
   \multicolumn{1}{S{Data}{r}}{1.14} & 
   \multicolumn{1}{S{Data}{r}}{30888} & 
   \multicolumn{1}{S{Data}{r}}{50.26}
\\
   \multicolumn{1}{S{RowHeader}{l}}{087 MAHNOMEN} & 
   \multicolumn{1}{S{Data}{r}}{702} & 
   \multicolumn{1}{S{Data}{r}}{1.14} & 
   \multicolumn{1}{S{Data}{r}}{31590} & 
   \multicolumn{1}{S{Data}{r}}{51.41}
\\
   \multicolumn{1}{S{RowHeader}{l}}{089 MARSHALL} & 
   \multicolumn{1}{S{Data}{r}}{702} & 
   \multicolumn{1}{S{Data}{r}}{1.14} & 
   \multicolumn{1}{S{Data}{r}}{32292} & 
   \multicolumn{1}{S{Data}{r}}{52.55}
\\
   \multicolumn{1}{S{RowHeader}{l}}{091 MARTIN} & 
   \multicolumn{1}{S{Data}{r}}{702} & 
   \multicolumn{1}{S{Data}{r}}{1.14} & 
   \multicolumn{1}{S{Data}{r}}{32994} & 
   \multicolumn{1}{S{Data}{r}}{53.69}
\\
   \multicolumn{1}{S{RowHeader}{l}}{093 MEEKER} & 
   \multicolumn{1}{S{Data}{r}}{702} & 
   \multicolumn{1}{S{Data}{r}}{1.14} & 
   \multicolumn{1}{S{Data}{r}}{33696} & 
   \multicolumn{1}{S{Data}{r}}{54.83}
\\
   \multicolumn{1}{S{RowHeader}{l}}{095 MILLE LACS} & 
   \multicolumn{1}{S{Data}{r}}{702} & 
   \multicolumn{1}{S{Data}{r}}{1.14} & 
   \multicolumn{1}{S{Data}{r}}{34398} & 
   \multicolumn{1}{S{Data}{r}}{55.98}
\\
   \multicolumn{1}{S{RowHeader}{l}}{097 MORRISON} & 
   \multicolumn{1}{S{Data}{r}}{702} & 
   \multicolumn{1}{S{Data}{r}}{1.14} & 
   \multicolumn{1}{S{Data}{r}}{35100} & 
   \multicolumn{1}{S{Data}{r}}{57.12}
\\
   \multicolumn{1}{S{RowHeader}{l}}{099 MOWER} & 
   \multicolumn{1}{S{Data}{r}}{702} & 
   \multicolumn{1}{S{Data}{r}}{1.14} & 
   \multicolumn{1}{S{Data}{r}}{35802} & 
   \multicolumn{1}{S{Data}{r}}{58.26}
\\
   \multicolumn{1}{S{RowHeader}{l}}{101 MURRAY} & 
   \multicolumn{1}{S{Data}{r}}{702} & 
   \multicolumn{1}{S{Data}{r}}{1.14} & 
   \multicolumn{1}{S{Data}{r}}{36504} & 
   \multicolumn{1}{S{Data}{r}}{59.40}
\\
   \multicolumn{1}{S{RowHeader}{l}}{103 NICOLLET} & 
   \multicolumn{1}{S{Data}{r}}{702} & 
   \multicolumn{1}{S{Data}{r}}{1.14} & 
   \multicolumn{1}{S{Data}{r}}{37206} & 
   \multicolumn{1}{S{Data}{r}}{60.54}
\\
   \multicolumn{1}{S{RowHeader}{l}}{105 NOBLES} & 
   \multicolumn{1}{S{Data}{r}}{702} & 
   \multicolumn{1}{S{Data}{r}}{1.14} & 
   \multicolumn{1}{S{Data}{r}}{37908} & 
   \multicolumn{1}{S{Data}{r}}{61.69}
\\
   \multicolumn{1}{S{RowHeader}{l}}{107 NORMAN} & 
   \multicolumn{1}{S{Data}{r}}{702} & 
   \multicolumn{1}{S{Data}{r}}{1.14} & 
   \multicolumn{1}{S{Data}{r}}{38610} & 
   \multicolumn{1}{S{Data}{r}}{62.83}
\\
   \multicolumn{1}{S{RowHeader}{l}}{109 OLMSTED} & 
   \multicolumn{1}{S{Data}{r}}{702} & 
   \multicolumn{1}{S{Data}{r}}{1.14} & 
   \multicolumn{1}{S{Data}{r}}{39312} & 
   \multicolumn{1}{S{Data}{r}}{63.97}
\\
   \multicolumn{1}{S{RowHeader}{l}}{111 OTTER TAIL} & 
   \multicolumn{1}{S{Data}{r}}{702} & 
   \multicolumn{1}{S{Data}{r}}{1.14} & 
   \multicolumn{1}{S{Data}{r}}{40014} & 
   \multicolumn{1}{S{Data}{r}}{65.11}
\\
   \multicolumn{1}{S{RowHeader}{l}}{113 PENNINGTON} & 
   \multicolumn{1}{S{Data}{r}}{702} & 
   \multicolumn{1}{S{Data}{r}}{1.14} & 
   \multicolumn{1}{S{Data}{r}}{40716} & 
   \multicolumn{1}{S{Data}{r}}{66.26}
\\
   \multicolumn{1}{S{RowHeader}{l}}{115 PINE} & 
   \multicolumn{1}{S{Data}{r}}{702} & 
   \multicolumn{1}{S{Data}{r}}{1.14} & 
   \multicolumn{1}{S{Data}{r}}{41418} & 
   \multicolumn{1}{S{Data}{r}}{67.40}
\\
   \multicolumn{1}{S{RowHeader}{l}}{117 PIPESTONE} & 
   \multicolumn{1}{S{Data}{r}}{702} & 
   \multicolumn{1}{S{Data}{r}}{1.14} & 
   \multicolumn{1}{S{Data}{r}}{42120} & 
   \multicolumn{1}{S{Data}{r}}{68.54}
\\
   \multicolumn{1}{S{RowHeader}{l}}{119 POLK} & 
   \multicolumn{1}{S{Data}{r}}{702} & 
   \multicolumn{1}{S{Data}{r}}{1.14} & 
   \multicolumn{1}{S{Data}{r}}{42822} & 
   \multicolumn{1}{S{Data}{r}}{69.68}
\\
   \multicolumn{1}{S{RowHeader}{l}}{121 POPE} & 
   \multicolumn{1}{S{Data}{r}}{702} & 
   \multicolumn{1}{S{Data}{r}}{1.14} & 
   \multicolumn{1}{S{Data}{r}}{43524} & 
   \multicolumn{1}{S{Data}{r}}{70.83}
\\
   \multicolumn{1}{S{RowHeader}{l}}{123 RAMSEY} & 
   \multicolumn{1}{S{Data}{r}}{702} & 
   \multicolumn{1}{S{Data}{r}}{1.14} & 
   \multicolumn{1}{S{Data}{r}}{44226} & 
   \multicolumn{1}{S{Data}{r}}{71.97}
\\
   \multicolumn{1}{S{RowHeader}{l}}{125 RED LAKE} & 
   \multicolumn{1}{S{Data}{r}}{540} & 
   \multicolumn{1}{S{Data}{r}}{0.88} & 
   \multicolumn{1}{S{Data}{r}}{44766} & 
   \multicolumn{1}{S{Data}{r}}{72.85}
\\
   \multicolumn{1}{S{RowHeader}{l}}{127 REDWOOD} & 
   \multicolumn{1}{S{Data}{r}}{702} & 
   \multicolumn{1}{S{Data}{r}}{1.14} & 
   \multicolumn{1}{S{Data}{r}}{45468} & 
   \multicolumn{1}{S{Data}{r}}{73.99}
\\
   \multicolumn{1}{S{RowHeader}{l}}{129 RENVILLE} & 
   \multicolumn{1}{S{Data}{r}}{702} & 
   \multicolumn{1}{S{Data}{r}}{1.14} & 
   \multicolumn{1}{S{Data}{r}}{46170} & 
   \multicolumn{1}{S{Data}{r}}{75.13}
\\
   \multicolumn{1}{S{RowHeader}{l}}{131 RICE} & 
   \multicolumn{1}{S{Data}{r}}{702} & 
   \multicolumn{1}{S{Data}{r}}{1.14} & 
   \multicolumn{1}{S{Data}{r}}{46872} & 
   \multicolumn{1}{S{Data}{r}}{76.27}
\\
   \multicolumn{1}{S{RowHeader}{l}}{133 ROCK} & 
   \multicolumn{1}{S{Data}{r}}{702} & 
   \multicolumn{1}{S{Data}{r}}{1.14} & 
   \multicolumn{1}{S{Data}{r}}{47574} & 
   \multicolumn{1}{S{Data}{r}}{77.42}
\\
   \multicolumn{1}{S{RowHeader}{l}}{135 ROSEAU} & 
   \multicolumn{1}{S{Data}{r}}{702} & 
   \multicolumn{1}{S{Data}{r}}{1.14} & 
   \multicolumn{1}{S{Data}{r}}{48276} & 
   \multicolumn{1}{S{Data}{r}}{78.56}
\\
   \multicolumn{1}{S{RowHeader}{l}}{137 ST. LOUIS} & 
   \multicolumn{1}{S{Data}{r}}{702} & 
   \multicolumn{1}{S{Data}{r}}{1.14} & 
   \multicolumn{1}{S{Data}{r}}{48978} & 
   \multicolumn{1}{S{Data}{r}}{79.70}
\\
   \multicolumn{1}{S{RowHeader}{l}}{139 SCOTT} & 
   \multicolumn{1}{S{Data}{r}}{702} & 
   \multicolumn{1}{S{Data}{r}}{1.14} & 
   \multicolumn{1}{S{Data}{r}}{49680} & 
   \multicolumn{1}{S{Data}{r}}{80.84}
\\
   \multicolumn{1}{S{RowHeader}{l}}{141 SHERBURNE} & 
   \multicolumn{1}{S{Data}{r}}{702} & 
   \multicolumn{1}{S{Data}{r}}{1.14} & 
   \multicolumn{1}{S{Data}{r}}{50382} & 
   \multicolumn{1}{S{Data}{r}}{81.99}
\\
   \multicolumn{1}{S{RowHeader}{l}}{143 SIBLEY} & 
   \multicolumn{1}{S{Data}{r}}{702} & 
   \multicolumn{1}{S{Data}{r}}{1.14} & 
   \multicolumn{1}{S{Data}{r}}{51084} & 
   \multicolumn{1}{S{Data}{r}}{83.13}
\\
   \multicolumn{1}{S{RowHeader}{l}}{145 STEARNS} & 
   \multicolumn{1}{S{Data}{r}}{702} & 
   \multicolumn{1}{S{Data}{r}}{1.14} & 
   \multicolumn{1}{S{Data}{r}}{51786} & 
   \multicolumn{1}{S{Data}{r}}{84.27}
\\
   \multicolumn{1}{S{RowHeader}{l}}{147 STEELE} & 
   \multicolumn{1}{S{Data}{r}}{702} & 
   \multicolumn{1}{S{Data}{r}}{1.14} & 
   \multicolumn{1}{S{Data}{r}}{52488} & 
   \multicolumn{1}{S{Data}{r}}{85.41}
\\
   \multicolumn{1}{S{RowHeader}{l}}{149 STEVENS} & 
   \multicolumn{1}{S{Data}{r}}{702} & 
   \multicolumn{1}{S{Data}{r}}{1.14} & 
   \multicolumn{1}{S{Data}{r}}{53190} & 
   \multicolumn{1}{S{Data}{r}}{86.56}
\\
   \multicolumn{1}{S{RowHeader}{l}}{151 SWIFT} & 
   \multicolumn{1}{S{Data}{r}}{702} & 
   \multicolumn{1}{S{Data}{r}}{1.14} & 
   \multicolumn{1}{S{Data}{r}}{53892} & 
   \multicolumn{1}{S{Data}{r}}{87.70}
\\
   \multicolumn{1}{S{RowHeader}{l}}{153 TODD} & 
   \multicolumn{1}{S{Data}{r}}{702} & 
   \multicolumn{1}{S{Data}{r}}{1.14} & 
   \multicolumn{1}{S{Data}{r}}{54594} & 
   \multicolumn{1}{S{Data}{r}}{88.84}
\\
   \multicolumn{1}{S{RowHeader}{l}}{155 TRAVERSE} & 
   \multicolumn{1}{S{Data}{r}}{540} & 
   \multicolumn{1}{S{Data}{r}}{0.88} & 
   \multicolumn{1}{S{Data}{r}}{55134} & 
   \multicolumn{1}{S{Data}{r}}{89.72}
\\
   \multicolumn{1}{S{RowHeader}{l}}{157 WABASHA} & 
   \multicolumn{1}{S{Data}{r}}{702} & 
   \multicolumn{1}{S{Data}{r}}{1.14} & 
   \multicolumn{1}{S{Data}{r}}{55836} & 
   \multicolumn{1}{S{Data}{r}}{90.86}
\\
   \multicolumn{1}{S{RowHeader}{l}}{159 WADENA} & 
   \multicolumn{1}{S{Data}{r}}{702} & 
   \multicolumn{1}{S{Data}{r}}{1.14} & 
   \multicolumn{1}{S{Data}{r}}{56538} & 
   \multicolumn{1}{S{Data}{r}}{92.00}
\\
   \multicolumn{1}{S{RowHeader}{l}}{161 WASECA} & 
   \multicolumn{1}{S{Data}{r}}{702} & 
   \multicolumn{1}{S{Data}{r}}{1.14} & 
   \multicolumn{1}{S{Data}{r}}{57240} & 
   \multicolumn{1}{S{Data}{r}}{93.15}
\\
   \multicolumn{1}{S{RowHeader}{l}}{163 WASHINGTON} & 
   \multicolumn{1}{S{Data}{r}}{702} & 
   \multicolumn{1}{S{Data}{r}}{1.14} & 
   \multicolumn{1}{S{Data}{r}}{57942} & 
   \multicolumn{1}{S{Data}{r}}{94.29}
\\
   \multicolumn{1}{S{RowHeader}{l}}{165 WATONWAN} & 
   \multicolumn{1}{S{Data}{r}}{702} & 
   \multicolumn{1}{S{Data}{r}}{1.14} & 
   \multicolumn{1}{S{Data}{r}}{58644} & 
   \multicolumn{1}{S{Data}{r}}{95.43}
\\
   \multicolumn{1}{S{RowHeader}{l}}{167 WILKIN} & 
   \multicolumn{1}{S{Data}{r}}{702} & 
   \multicolumn{1}{S{Data}{r}}{1.14} & 
   \multicolumn{1}{S{Data}{r}}{59346} & 
   \multicolumn{1}{S{Data}{r}}{96.57}
\\
   \multicolumn{1}{S{RowHeader}{l}}{169 WINONA} & 
   \multicolumn{1}{S{Data}{r}}{702} & 
   \multicolumn{1}{S{Data}{r}}{1.14} & 
   \multicolumn{1}{S{Data}{r}}{60048} & 
   \multicolumn{1}{S{Data}{r}}{97.72}
\\
   \multicolumn{1}{S{RowHeader}{l}}{171 WRIGHT} & 
   \multicolumn{1}{S{Data}{r}}{702} & 
   \multicolumn{1}{S{Data}{r}}{1.14} & 
   \multicolumn{1}{S{Data}{r}}{60750} & 
   \multicolumn{1}{S{Data}{r}}{98.86}
\\
   \multicolumn{1}{S{RowHeader}{l}}{173 YELLOW MEDICINE} & 
   \multicolumn{1}{S{Data}{r}}{702} & 
   \multicolumn{1}{S{Data}{r}}{1.14} & 
   \multicolumn{1}{S{Data}{r}}{61452} & 
   \multicolumn{1}{S{Data}{r}}{100.00}
\\
\end{longtable}
\end{center}
\begin{center}\begin{longtable}
{rrrrr}\hline % colspecs
% table_head start
   \multicolumn{5}{S{Header}{c}}{Sex}
\\
   \multicolumn{1}{S{Header}{r}}{sex} & 
   \multicolumn{1}{S{Header}{r}}{Frequency} & 
   \multicolumn{1}{S{Header}{r}}{ Percent} & 
   \multicolumn{1}{S{Header}{r}}{Cumulative\linebreak  Frequency} & 
   \multicolumn{1}{S{Header}{r}}{Cumulative\linebreak   Percent}
\\
\hline 
\endhead % table_head end
\hline 
\multicolumn{1}{r}{(cont.)}\\
\endfoot 
\hline 
\endlastfoot % table_head end
   \multicolumn{1}{S{RowHeader}{r}}{0 : All} & 
   \multicolumn{1}{S{Data}{r}}{20484} & 
   \multicolumn{1}{S{Data}{r}}{33.33} & 
   \multicolumn{1}{S{Data}{r}}{20484} & 
   \multicolumn{1}{S{Data}{r}}{33.33}
\\
   \multicolumn{1}{S{RowHeader}{r}}{1 : Men} & 
   \multicolumn{1}{S{Data}{r}}{20484} & 
   \multicolumn{1}{S{Data}{r}}{33.33} & 
   \multicolumn{1}{S{Data}{r}}{40968} & 
   \multicolumn{1}{S{Data}{r}}{66.67}
\\
   \multicolumn{1}{S{RowHeader}{r}}{2 : Women} & 
   \multicolumn{1}{S{Data}{r}}{20484} & 
   \multicolumn{1}{S{Data}{r}}{33.33} & 
   \multicolumn{1}{S{Data}{r}}{61452} & 
   \multicolumn{1}{S{Data}{r}}{100.00}
\\
\end{longtable}
\end{center}
\begin{center}\begin{longtable}
{rrrrr}\hline % colspecs
% table_head start
   \multicolumn{5}{S{Header}{c}}{Age group}
\\
   \multicolumn{1}{S{Header}{r}}{agegroup} & 
   \multicolumn{1}{S{Header}{r}}{Frequency} & 
   \multicolumn{1}{S{Header}{r}}{ Percent} & 
   \multicolumn{1}{S{Header}{r}}{Cumulative\linebreak  Frequency} & 
   \multicolumn{1}{S{Header}{r}}{Cumulative\linebreak   Percent}
\\
\hline 
\endhead % table_head end
\hline 
\multicolumn{1}{r}{(cont.)}\\
\endfoot 
\hline 
\endlastfoot % table_head end
   \multicolumn{1}{S{RowHeader}{r}}{0 : All Ages} & 
   \multicolumn{1}{S{Data}{r}}{6828} & 
   \multicolumn{1}{S{Data}{r}}{11.11} & 
   \multicolumn{1}{S{Data}{r}}{6828} & 
   \multicolumn{1}{S{Data}{r}}{11.11}
\\
   \multicolumn{1}{S{RowHeader}{r}}{1 : 14-18} & 
   \multicolumn{1}{S{Data}{r}}{6828} & 
   \multicolumn{1}{S{Data}{r}}{11.11} & 
   \multicolumn{1}{S{Data}{r}}{13656} & 
   \multicolumn{1}{S{Data}{r}}{22.22}
\\
   \multicolumn{1}{S{RowHeader}{r}}{2 : 19-21} & 
   \multicolumn{1}{S{Data}{r}}{6828} & 
   \multicolumn{1}{S{Data}{r}}{11.11} & 
   \multicolumn{1}{S{Data}{r}}{20484} & 
   \multicolumn{1}{S{Data}{r}}{33.33}
\\
   \multicolumn{1}{S{RowHeader}{r}}{3 : 22-24} & 
   \multicolumn{1}{S{Data}{r}}{6828} & 
   \multicolumn{1}{S{Data}{r}}{11.11} & 
   \multicolumn{1}{S{Data}{r}}{27312} & 
   \multicolumn{1}{S{Data}{r}}{44.44}
\\
   \multicolumn{1}{S{RowHeader}{r}}{4 : 25-34} & 
   \multicolumn{1}{S{Data}{r}}{6828} & 
   \multicolumn{1}{S{Data}{r}}{11.11} & 
   \multicolumn{1}{S{Data}{r}}{34140} & 
   \multicolumn{1}{S{Data}{r}}{55.56}
\\
   \multicolumn{1}{S{RowHeader}{r}}{5 : 35-44} & 
   \multicolumn{1}{S{Data}{r}}{6828} & 
   \multicolumn{1}{S{Data}{r}}{11.11} & 
   \multicolumn{1}{S{Data}{r}}{40968} & 
   \multicolumn{1}{S{Data}{r}}{66.67}
\\
   \multicolumn{1}{S{RowHeader}{r}}{6 : 45-54} & 
   \multicolumn{1}{S{Data}{r}}{6828} & 
   \multicolumn{1}{S{Data}{r}}{11.11} & 
   \multicolumn{1}{S{Data}{r}}{47796} & 
   \multicolumn{1}{S{Data}{r}}{77.78}
\\
   \multicolumn{1}{S{RowHeader}{r}}{7 : 55-64} & 
   \multicolumn{1}{S{Data}{r}}{6828} & 
   \multicolumn{1}{S{Data}{r}}{11.11} & 
   \multicolumn{1}{S{Data}{r}}{54624} & 
   \multicolumn{1}{S{Data}{r}}{88.89}
\\
   \multicolumn{1}{S{RowHeader}{r}}{8 : 65+} & 
   \multicolumn{1}{S{Data}{r}}{6828} & 
   \multicolumn{1}{S{Data}{r}}{11.11} & 
   \multicolumn{1}{S{Data}{r}}{61452} & 
   \multicolumn{1}{S{Data}{r}}{100.00}
\\
\end{longtable}
\end{center}
\begin{center}\begin{longtable}
{llllll}\hline % colspecs
% table_head start
   \multicolumn{6}{S{Header}{c}}{Table of year by quarter}
\\
   \multicolumn{1}{S{Header}{c}}{year(Year)} & 
   \multicolumn{4}{S{Header}{c}}{quarter(Quarter)} & 
   \multicolumn{1}{S{Header}{r}}{Total}
\\
   \multicolumn{1}{l}{~} & 
   \multicolumn{1}{S{Header}{r}}{      1     } & 
   \multicolumn{1}{S{Header}{r}}{      2     } & 
   \multicolumn{1}{S{Header}{r}}{      3     } & 
   \multicolumn{1}{S{Header}{r}}{      4     } & 
   \multicolumn{1}{l}{~}
\\
\hline 
\endhead % table_head end
\hline 
\multicolumn{1}{r}{(cont.)}\\
\endfoot 
\hline 
\endlastfoot % table_head end
   \multicolumn{1}{S{Header}{r}}{1994        } & 
   \multicolumn{1}{S{Data}{r}}{     0} & 
   \multicolumn{1}{S{Data}{r}}{     0} & 
   \multicolumn{1}{S{Data}{r}}{  2376} & 
   \multicolumn{1}{S{Data}{r}}{  2376} & 
   \multicolumn{1}{S{Data}{r}}{  4752}
\\
   \multicolumn{1}{S{Header}{r}}{1995        } & 
   \multicolumn{1}{S{Data}{r}}{  2376} & 
   \multicolumn{1}{S{Data}{r}}{  2376} & 
   \multicolumn{1}{S{Data}{r}}{  2376} & 
   \multicolumn{1}{S{Data}{r}}{  2376} & 
   \multicolumn{1}{S{Data}{r}}{  9504}
\\
   \multicolumn{1}{S{Header}{r}}{1996        } & 
   \multicolumn{1}{S{Data}{r}}{  2376} & 
   \multicolumn{1}{S{Data}{r}}{  2376} & 
   \multicolumn{1}{S{Data}{r}}{  2376} & 
   \multicolumn{1}{S{Data}{r}}{  2376} & 
   \multicolumn{1}{S{Data}{r}}{  9504}
\\
   \multicolumn{1}{S{Header}{r}}{1997        } & 
   \multicolumn{1}{S{Data}{r}}{  2376} & 
   \multicolumn{1}{S{Data}{r}}{  2376} & 
   \multicolumn{1}{S{Data}{r}}{  2376} & 
   \multicolumn{1}{S{Data}{r}}{  2376} & 
   \multicolumn{1}{S{Data}{r}}{  9504}
\\
   \multicolumn{1}{S{Header}{r}}{1998        } & 
   \multicolumn{1}{S{Data}{r}}{  2349} & 
   \multicolumn{1}{S{Data}{r}}{  2349} & 
   \multicolumn{1}{S{Data}{r}}{  2349} & 
   \multicolumn{1}{S{Data}{r}}{  2376} & 
   \multicolumn{1}{S{Data}{r}}{  9423}
\\
   \multicolumn{1}{S{Header}{r}}{1999        } & 
   \multicolumn{1}{S{Data}{r}}{  2376} & 
   \multicolumn{1}{S{Data}{r}}{  2349} & 
   \multicolumn{1}{S{Data}{r}}{  2322} & 
   \multicolumn{1}{S{Data}{r}}{  2322} & 
   \multicolumn{1}{S{Data}{r}}{  9369}
\\
   \multicolumn{1}{S{Header}{r}}{2000        } & 
   \multicolumn{1}{S{Data}{r}}{  2349} & 
   \multicolumn{1}{S{Data}{r}}{  2349} & 
   \multicolumn{1}{S{Data}{r}}{  2349} & 
   \multicolumn{1}{S{Data}{r}}{  2349} & 
   \multicolumn{1}{S{Data}{r}}{  9396}
\\
   \multicolumn{1}{S{Header}{l}}{Total           } & 
   \multicolumn{1}{S{Data}{r}}{  14202} & 
   \multicolumn{1}{S{Data}{r}}{  14175} & 
   \multicolumn{1}{S{Data}{r}}{  16524} & 
   \multicolumn{1}{S{Data}{r}}{  16551} & 
   \multicolumn{1}{S{Data}{r}}{  61452}
\\
\end{longtable}
\end{center}

% ====================end of output====================

% \newpage %    Generated by SAS
%    http://www.sas.com
% created by=vilhu001
% sasversion=8.2
% date=2002-05-23
% time=00:39:34
% encoding=iso-8859-1
% ====================begin of output====================
% \begin{document}

% An external file needs to be included, as specified% in latexlong.sas. This can be called sas.sty,
% in which case you want to include a line like
% \usepackage{sas}
% or it can be a simple (La)TeX file, which you 
% include by typing 
% %%
%% This is file `sas.sty',
%% generated with the docstrip utility.
%%
%% 
\NeedsTeXFormat{LaTeX2e}
\ProvidesPackage{sas}
        [2002/01/18 LEHD version 0.1
    provides definition for tables generated by SAS%
                   ]
\@ifundefined{array@processline}{\RequirePackage{array}}{}
\@ifundefined{longtable@processline}{\RequirePackage{longtable}}{}
 \def\ContentTitle{\small\it\sffamily}
 \def\Output{\small\sffamily}
 \def\HeaderEmphasis{\small\it\sffamily}
 \def\NoteContent{\small\sffamily}
 \def\FatalContent{\small\sffamily}
 \def\Graph{\small\sffamily}
 \def\WarnContentFixed{\footnotesize\tt}
 \def\NoteBanner{\small\sffamily}
 \def\DataStrong{\normalsize\bf\sffamily}
 \def\Document{\small\sffamily}
 \def\BeforeCaption{\normalsize\bf\sffamily}
 \def\ContentsDate{\small\sffamily}
 \def\Pages{\small\sffamily}
 \def\TitlesAndFooters{\footnotesize\bf\it\sffamily}
 \def\IndexProcName{\small\sffamily}
 \def\ProcTitle{\normalsize\bf\it\sffamily}
 \def\IndexAction{\small\sffamily}
 \def\Data{\small\sffamily}
 \def\Table{\small\sffamily}
 \def\FooterEmpty{\footnotesize\bf\sffamily}
 \def\SysTitleAndFooterContainer{\footnotesize\sffamily}
 \def\RowFooterEmpty{\footnotesize\bf\sffamily}
 \def\ExtendedPage{\small\it\sffamily}
 \def\FooterFixed{\footnotesize\tt}
 \def\RowFooterStrongFixed{\footnotesize\bf\tt}
 \def\RowFooterEmphasis{\footnote\it\sffamily}
 \def\ContentFolder{\small\sffamily}
 \def\Container{\small\sffamily}
 \def\Date{\small\sffamily}
 \def\RowFooterFixed{\footnotesize\tt}
 \def\Caption{\normalsize\bf\sffamily}
 \def\WarnBanner{\small\sffamily}
 \def\Frame{\small\sffamily}
 \def\HeaderStrongFixed{\footnotesize\bf\tt}
 \def\IndexTitle{\small\it\sffamily}
 \def\NoteContentFixed{\footnotesize\tt}
 \def\DataEmphasisFixed{\footnotesize\it\tt}
 \def\Note{\small\sffamily}
 \def\Byline{\normalsize\bf\sffamily}
 \def\FatalBanner{\small\sffamily}
 \def\ProcTitleFixed{\footnotesize\bf\tt}
 \def\ByContentFolder{\small\sffamily}
 \def\PagesProcLabel{\small\sffamily}
 \def\RowHeaderFixed{\footnotesize\tt}
 \def\RowFooterEmphasisFixed{\footnotesize\it\tt}
 \def\WarnContent{\small\sffamily}
 \def\DataEmpty{\small\sffamily}
 \def\Cell{\small\sffamily}
 \def\Header{\normalsize\bf\sffamily}
 \def\PageNo{\normalsize\bf\sffamily}
 \def\ContentProcLabel{\small\sffamily}
 \def\HeaderFixed{\footnotesize\tt}
 \def\PagesTitle{\small\it\sffamily}
 \def\RowHeaderEmpty{\normalsize\bf\sffamily}
 \def\PagesProcName{\small\sffamily}
 \def\Batch{\footnotesize\tt}
 \def\ContentItem{\small\sffamily}
 \def\Body{\small\sffamily}
 \def\PagesDate{\small\sffamily}
 \def\Index{\small\sffamily}
 \def\HeaderEmpty{\normalsize\bf\sffamily}
 \def\FooterStrong{\footnotesize\bf\sffamily}
 \def\FooterEmphasis{\footnotesize\it\sffamily}
 \def\ErrorContent{\small\sffamily}
 \def\DataFixed{\footnotesize\tt}
 \def\HeaderStrong{\normalsize\bf\sffamily}
 \def\GraphBackground{}
 \def\DataEmphasis{\small\it\sffamily}
 \def\TitleAndNoteContainer{\small\sffamily}
 \def\RowFooter{\footnotesize\bf\sffamily}
 \def\IndexItem{\small\sffamily}
 \def\BylineContainer{\small\sffamily}
 \def\FatalContentFixed{\footnotesize\tt}
 \def\BodyDate{\small\sffamily}
 \def\RowFooterStrong{\footnotesize\bf\sffamily}
 \def\UserText{\small\sffamily}
 \def\HeadersAndFooters{\footnotesize\bf\sffamily}
 \def\RowHeaderEmphasisFixed{\footnotesize\it\tt}
 \def\ErrorBanner{\small\sffamily}
 \def\ContentProcName{\small\sffamily}
 \def\RowHeaderStrong{\normalsize\bf\sffamily}
 \def\FooterEmphasisFixed{\footnotesize\it\tt}
 \def\Contents{\small\sffamily}
 \def\FooterStrongFixed{\footnotesize\bf\tt}
 \def\PagesItem{\small\sffamily}
 \def\RowHeader{\normalsize\bf\sffamily}
 \def\AfterCaption{\normalsize\bf\sffamily}
 \def\RowHeaderStrongFixed{\footnotesize\bf\tt}
 \def\RowHeaderEmphasis{\small\it\sffamily}
 \def\DataStrongFixed{\footnotesize\bf\tt}
 \def\Footer{\footnotesize\bf\sffamily}
 \def\FolderAction{\small\sffamily}
 \def\HeaderEmphasisFixed{\footnotesize\it\tt}
 \def\SystemTitle{\large\bf\it\sffamily}
 \def\ErrorContentFixed{\footnotesize\tt}
 \def\SystemFooter{\footnotesize\it\sffamily}
% Set cell padding 
\renewcommand{\arraystretch}{1.3}
% Headings
\newcommand{\heading}[2]{\csname#1\endcsname #2}
\newcommand{\proctitle}[2]{\csname#1\endcsname #2}
% Declare new column type
\newcolumntype{S}[2]{>{\csname#1\endcsname}#2}
% Set warning box style
\newcommand{\msg}[2]{\fbox{%
   \begin{minipage}{\textwidth}#2\end{minipage}}%
}

\begin{center}\heading{ProcTitle}{The CONTENTS Procedure}\end{center}
\begin{center}\begin{longtable}
{llll}\hline % colspecs
   \multicolumn{1}{S{RowHeader}{l}}{Data Set Name:} & 
   \multicolumn{1}{S{Data}{l}}{STATE.MN{\textunderscore}SIC{\textunderscore}DIVISION{\textunderscore}V23{\textunderscore}FUZZED} & 
   \multicolumn{1}{S{RowHeader}{l}}{Observations:} & 
   \multicolumn{1}{S{Data}{l}}{8424}
\\
   \multicolumn{1}{S{RowHeader}{l}}{Member Type:} & 
   \multicolumn{1}{S{Data}{l}}{DATA} & 
   \multicolumn{1}{S{RowHeader}{l}}{Variables:} & 
   \multicolumn{1}{S{Data}{l}}{60}
\\
   \multicolumn{1}{S{RowHeader}{l}}{Engine:} & 
   \multicolumn{1}{S{Data}{l}}{V8} & 
   \multicolumn{1}{S{RowHeader}{l}}{Indexes:} & 
   \multicolumn{1}{S{Data}{l}}{0}
\\
   \multicolumn{1}{S{RowHeader}{l}}{Created:} & 
   \multicolumn{1}{S{Data}{l}}{18:42 Thursday, May 16, 2002} & 
   \multicolumn{1}{S{RowHeader}{l}}{Observation Length:} & 
   \multicolumn{1}{S{Data}{l}}{288}
\\
   \multicolumn{1}{S{RowHeader}{l}}{Last Modified:} & 
   \multicolumn{1}{S{Data}{l}}{18:42 Thursday, May 16, 2002} & 
   \multicolumn{1}{S{RowHeader}{l}}{Deleted Observations:} & 
   \multicolumn{1}{S{Data}{l}}{0}
\\
   \multicolumn{1}{S{RowHeader}{l}}{Protection:} & 
   \multicolumn{1}{S{Data}{l}}{ } & 
   \multicolumn{1}{S{RowHeader}{l}}{Compressed:} & 
   \multicolumn{1}{S{Data}{l}}{NO}
\\
   \multicolumn{1}{S{RowHeader}{l}}{Data Set Type:} & 
   \multicolumn{1}{S{Data}{l}}{ } & 
   \multicolumn{1}{S{RowHeader}{l}}{Sorted:} & 
   \multicolumn{1}{S{Data}{l}}{NO}
\\
   \multicolumn{1}{S{RowHeader}{l}}{Label:} & 
   \multicolumn{1}{S{Data}{l}}{ } & 
   \multicolumn{1}{S{RowHeader}{l}}{ } & 
   \multicolumn{1}{S{Data}{l}}{ }
\\
\end{longtable}
\end{center}
\begin{center}\begin{longtable}
{rllrrl}\hline % colspecs
% table_head start
   \multicolumn{6}{S{Header}{c}}{-----Variables Ordered by Position-----}
\\
   \multicolumn{1}{S{Header}{r}}{\#} & 
   \multicolumn{1}{S{Header}{l}}{Variable} & 
   \multicolumn{1}{S{Header}{l}}{Type} & 
   \multicolumn{1}{S{Header}{r}}{Len} & 
   \multicolumn{1}{S{Header}{r}}{Pos} & 
   \multicolumn{1}{S{Header}{l}}{Label}
\\
\hline 
\endhead % table_head end
\hline 
\multicolumn{1}{r}{(cont.)}\\
\endfoot 
\hline 
\endlastfoot % table_head end
   \multicolumn{1}{S{RowHeader}{r}}{1} & 
   \multicolumn{1}{S{Data}{l}}{state} & 
   \multicolumn{1}{S{Data}{l}}{Char} & 
   \multicolumn{1}{S{Data}{r}}{2} & 
   \multicolumn{1}{S{Data}{r}}{216} & 
   \multicolumn{1}{S{Data}{l}}{FIPS State}
\\
   \multicolumn{1}{S{RowHeader}{r}}{2} & 
   \multicolumn{1}{S{Data}{l}}{year} & 
   \multicolumn{1}{S{Data}{l}}{Num} & 
   \multicolumn{1}{S{Data}{r}}{3} & 
   \multicolumn{1}{S{Data}{r}}{273} & 
   \multicolumn{1}{S{Data}{l}}{Year}
\\
   \multicolumn{1}{S{RowHeader}{r}}{3} & 
   \multicolumn{1}{S{Data}{l}}{quarter} & 
   \multicolumn{1}{S{Data}{l}}{Num} & 
   \multicolumn{1}{S{Data}{r}}{3} & 
   \multicolumn{1}{S{Data}{r}}{276} & 
   \multicolumn{1}{S{Data}{l}}{Quarter}
\\
   \multicolumn{1}{S{RowHeader}{r}}{4} & 
   \multicolumn{1}{S{Data}{l}}{sic{\textunderscore}division} & 
   \multicolumn{1}{S{Data}{l}}{Char} & 
   \multicolumn{1}{S{Data}{r}}{1} & 
   \multicolumn{1}{S{Data}{r}}{218} & 
   \multicolumn{1}{S{Data}{l}}{SIC Division}
\\
   \multicolumn{1}{S{RowHeader}{r}}{5} & 
   \multicolumn{1}{S{Data}{l}}{sex} & 
   \multicolumn{1}{S{Data}{l}}{Num} & 
   \multicolumn{1}{S{Data}{r}}{3} & 
   \multicolumn{1}{S{Data}{r}}{279} & 
   \multicolumn{1}{S{Data}{l}}{Sex}
\\
   \multicolumn{1}{S{RowHeader}{r}}{6} & 
   \multicolumn{1}{S{Data}{l}}{agegroup} & 
   \multicolumn{1}{S{Data}{l}}{Num} & 
   \multicolumn{1}{S{Data}{r}}{3} & 
   \multicolumn{1}{S{Data}{r}}{282} & 
   \multicolumn{1}{S{Data}{l}}{Age group}
\\
   \multicolumn{1}{S{RowHeader}{r}}{7} & 
   \multicolumn{1}{S{Data}{l}}{A} & 
   \multicolumn{1}{S{Data}{l}}{Num} & 
   \multicolumn{1}{S{Data}{r}}{8} & 
   \multicolumn{1}{S{Data}{r}}{0} & 
   \multicolumn{1}{S{Data}{l}}{Accessions}
\\
   \multicolumn{1}{S{RowHeader}{r}}{8} & 
   \multicolumn{1}{S{Data}{l}}{B} & 
   \multicolumn{1}{S{Data}{l}}{Num} & 
   \multicolumn{1}{S{Data}{r}}{8} & 
   \multicolumn{1}{S{Data}{r}}{8} & 
   \multicolumn{1}{S{Data}{l}}{Beginning-of-period employment}
\\
   \multicolumn{1}{S{RowHeader}{r}}{9} & 
   \multicolumn{1}{S{Data}{l}}{E} & 
   \multicolumn{1}{S{Data}{l}}{Num} & 
   \multicolumn{1}{S{Data}{r}}{8} & 
   \multicolumn{1}{S{Data}{r}}{16} & 
   \multicolumn{1}{S{Data}{l}}{End-of-period employment}
\\
   \multicolumn{1}{S{RowHeader}{r}}{10} & 
   \multicolumn{1}{S{Data}{l}}{F} & 
   \multicolumn{1}{S{Data}{l}}{Num} & 
   \multicolumn{1}{S{Data}{r}}{8} & 
   \multicolumn{1}{S{Data}{r}}{24} & 
   \multicolumn{1}{S{Data}{l}}{Full-quarter employment}
\\
   \multicolumn{1}{S{RowHeader}{r}}{11} & 
   \multicolumn{1}{S{Data}{l}}{FA} & 
   \multicolumn{1}{S{Data}{l}}{Num} & 
   \multicolumn{1}{S{Data}{r}}{8} & 
   \multicolumn{1}{S{Data}{r}}{32} & 
   \multicolumn{1}{S{Data}{l}}{Flow into full-quarter employment}
\\
   \multicolumn{1}{S{RowHeader}{r}}{12} & 
   \multicolumn{1}{S{Data}{l}}{FJC} & 
   \multicolumn{1}{S{Data}{l}}{Num} & 
   \multicolumn{1}{S{Data}{r}}{8} & 
   \multicolumn{1}{S{Data}{r}}{40} & 
   \multicolumn{1}{S{Data}{l}}{Full-quarter job creation}
\\
   \multicolumn{1}{S{RowHeader}{r}}{13} & 
   \multicolumn{1}{S{Data}{l}}{FJD} & 
   \multicolumn{1}{S{Data}{l}}{Num} & 
   \multicolumn{1}{S{Data}{r}}{8} & 
   \multicolumn{1}{S{Data}{r}}{48} & 
   \multicolumn{1}{S{Data}{l}}{Full-quarter job destruction}
\\
   \multicolumn{1}{S{RowHeader}{r}}{14} & 
   \multicolumn{1}{S{Data}{l}}{FJF} & 
   \multicolumn{1}{S{Data}{l}}{Num} & 
   \multicolumn{1}{S{Data}{r}}{8} & 
   \multicolumn{1}{S{Data}{r}}{56} & 
   \multicolumn{1}{S{Data}{l}}{Net change in full-quarter employment}
\\
   \multicolumn{1}{S{RowHeader}{r}}{15} & 
   \multicolumn{1}{S{Data}{l}}{FS} & 
   \multicolumn{1}{S{Data}{l}}{Num} & 
   \multicolumn{1}{S{Data}{r}}{8} & 
   \multicolumn{1}{S{Data}{r}}{64} & 
   \multicolumn{1}{S{Data}{l}}{Flow out of full-quarter employment}
\\
   \multicolumn{1}{S{RowHeader}{r}}{16} & 
   \multicolumn{1}{S{Data}{l}}{H} & 
   \multicolumn{1}{S{Data}{l}}{Num} & 
   \multicolumn{1}{S{Data}{r}}{8} & 
   \multicolumn{1}{S{Data}{r}}{72} & 
   \multicolumn{1}{S{Data}{l}}{New hires}
\\
   \multicolumn{1}{S{RowHeader}{r}}{17} & 
   \multicolumn{1}{S{Data}{l}}{H3} & 
   \multicolumn{1}{S{Data}{l}}{Num} & 
   \multicolumn{1}{S{Data}{r}}{8} & 
   \multicolumn{1}{S{Data}{r}}{80} & 
   \multicolumn{1}{S{Data}{l}}{Full-quarter new hires}
\\
   \multicolumn{1}{S{RowHeader}{r}}{18} & 
   \multicolumn{1}{S{Data}{l}}{JC} & 
   \multicolumn{1}{S{Data}{l}}{Num} & 
   \multicolumn{1}{S{Data}{r}}{8} & 
   \multicolumn{1}{S{Data}{r}}{88} & 
   \multicolumn{1}{S{Data}{l}}{Job creation}
\\
   \multicolumn{1}{S{RowHeader}{r}}{19} & 
   \multicolumn{1}{S{Data}{l}}{JD} & 
   \multicolumn{1}{S{Data}{l}}{Num} & 
   \multicolumn{1}{S{Data}{r}}{8} & 
   \multicolumn{1}{S{Data}{r}}{96} & 
   \multicolumn{1}{S{Data}{l}}{Job destruction}
\\
   \multicolumn{1}{S{RowHeader}{r}}{20} & 
   \multicolumn{1}{S{Data}{l}}{JF} & 
   \multicolumn{1}{S{Data}{l}}{Num} & 
   \multicolumn{1}{S{Data}{r}}{8} & 
   \multicolumn{1}{S{Data}{r}}{104} & 
   \multicolumn{1}{S{Data}{l}}{Net job flows}
\\
   \multicolumn{1}{S{RowHeader}{r}}{21} & 
   \multicolumn{1}{S{Data}{l}}{R} & 
   \multicolumn{1}{S{Data}{l}}{Num} & 
   \multicolumn{1}{S{Data}{r}}{8} & 
   \multicolumn{1}{S{Data}{r}}{112} & 
   \multicolumn{1}{S{Data}{l}}{Recalls}
\\
   \multicolumn{1}{S{RowHeader}{r}}{22} & 
   \multicolumn{1}{S{Data}{l}}{S} & 
   \multicolumn{1}{S{Data}{l}}{Num} & 
   \multicolumn{1}{S{Data}{r}}{8} & 
   \multicolumn{1}{S{Data}{r}}{120} & 
   \multicolumn{1}{S{Data}{l}}{Separations}
\\
   \multicolumn{1}{S{RowHeader}{r}}{23} & 
   \multicolumn{1}{S{Data}{l}}{Z{\textunderscore}NA} & 
   \multicolumn{1}{S{Data}{l}}{Num} & 
   \multicolumn{1}{S{Data}{r}}{8} & 
   \multicolumn{1}{S{Data}{r}}{128} & 
   \multicolumn{1}{S{Data}{l}}{Average periods of non-employment for accessions}
\\
   \multicolumn{1}{S{RowHeader}{r}}{24} & 
   \multicolumn{1}{S{Data}{l}}{Z{\textunderscore}NH} & 
   \multicolumn{1}{S{Data}{l}}{Num} & 
   \multicolumn{1}{S{Data}{r}}{8} & 
   \multicolumn{1}{S{Data}{r}}{136} & 
   \multicolumn{1}{S{Data}{l}}{Average periods of non-employment for new hires}
\\
   \multicolumn{1}{S{RowHeader}{r}}{25} & 
   \multicolumn{1}{S{Data}{l}}{Z{\textunderscore}NR} & 
   \multicolumn{1}{S{Data}{l}}{Num} & 
   \multicolumn{1}{S{Data}{r}}{8} & 
   \multicolumn{1}{S{Data}{r}}{144} & 
   \multicolumn{1}{S{Data}{l}}{Average periods of non-employment for recalls}
\\
   \multicolumn{1}{S{RowHeader}{r}}{26} & 
   \multicolumn{1}{S{Data}{l}}{Z{\textunderscore}NS} & 
   \multicolumn{1}{S{Data}{l}}{Num} & 
   \multicolumn{1}{S{Data}{r}}{8} & 
   \multicolumn{1}{S{Data}{r}}{152} & 
   \multicolumn{1}{S{Data}{l}}{Average periods of non-employment for separations}
\\
   \multicolumn{1}{S{RowHeader}{r}}{27} & 
   \multicolumn{1}{S{Data}{l}}{Z{\textunderscore}W2} & 
   \multicolumn{1}{S{Data}{l}}{Num} & 
   \multicolumn{1}{S{Data}{r}}{8} & 
   \multicolumn{1}{S{Data}{r}}{160} & 
   \multicolumn{1}{S{Data}{l}}{Average earnings of end-of-period employees}
\\
   \multicolumn{1}{S{RowHeader}{r}}{28} & 
   \multicolumn{1}{S{Data}{l}}{Z{\textunderscore}W3} & 
   \multicolumn{1}{S{Data}{l}}{Num} & 
   \multicolumn{1}{S{Data}{r}}{8} & 
   \multicolumn{1}{S{Data}{r}}{168} & 
   \multicolumn{1}{S{Data}{l}}{Average earnings of full-quarter employees}
\\
   \multicolumn{1}{S{RowHeader}{r}}{29} & 
   \multicolumn{1}{S{Data}{l}}{Z{\textunderscore}WFA} & 
   \multicolumn{1}{S{Data}{l}}{Num} & 
   \multicolumn{1}{S{Data}{r}}{8} & 
   \multicolumn{1}{S{Data}{r}}{176} & 
   \multicolumn{1}{S{Data}{l}}{Average earnings of transits to full-quarter status}
\\
   \multicolumn{1}{S{RowHeader}{r}}{30} & 
   \multicolumn{1}{S{Data}{l}}{Z{\textunderscore}WFS} & 
   \multicolumn{1}{S{Data}{l}}{Num} & 
   \multicolumn{1}{S{Data}{r}}{8} & 
   \multicolumn{1}{S{Data}{r}}{184} & 
   \multicolumn{1}{S{Data}{l}}{Average earnings of separations from full-quarter status}
\\
   \multicolumn{1}{S{RowHeader}{r}}{31} & 
   \multicolumn{1}{S{Data}{l}}{Z{\textunderscore}WH3} & 
   \multicolumn{1}{S{Data}{l}}{Num} & 
   \multicolumn{1}{S{Data}{r}}{8} & 
   \multicolumn{1}{S{Data}{r}}{192} & 
   \multicolumn{1}{S{Data}{l}}{Average earnings of full-quarter new hires}
\\
   \multicolumn{1}{S{RowHeader}{r}}{32} & 
   \multicolumn{1}{S{Data}{l}}{Z{\textunderscore}dWA} & 
   \multicolumn{1}{S{Data}{l}}{Num} & 
   \multicolumn{1}{S{Data}{r}}{8} & 
   \multicolumn{1}{S{Data}{r}}{200} & 
   \multicolumn{1}{S{Data}{l}}{Average change in total earnings for accessions}
\\
   \multicolumn{1}{S{RowHeader}{r}}{33} & 
   \multicolumn{1}{S{Data}{l}}{Z{\textunderscore}dWS} & 
   \multicolumn{1}{S{Data}{l}}{Num} & 
   \multicolumn{1}{S{Data}{r}}{8} & 
   \multicolumn{1}{S{Data}{r}}{208} & 
   \multicolumn{1}{S{Data}{l}}{Average change in total earnings for separations}
\\
   \multicolumn{1}{S{RowHeader}{r}}{34} & 
   \multicolumn{1}{S{Data}{l}}{A{\textunderscore}status} & 
   \multicolumn{1}{S{Data}{l}}{Char} & 
   \multicolumn{1}{S{Data}{r}}{2} & 
   \multicolumn{1}{S{Data}{r}}{219} & 
   \multicolumn{1}{S{Data}{l}}{Status: accessions}
\\
   \multicolumn{1}{S{RowHeader}{r}}{35} & 
   \multicolumn{1}{S{Data}{l}}{B{\textunderscore}status} & 
   \multicolumn{1}{S{Data}{l}}{Char} & 
   \multicolumn{1}{S{Data}{r}}{2} & 
   \multicolumn{1}{S{Data}{r}}{221} & 
   \multicolumn{1}{S{Data}{l}}{Status: beginning-of-period employment}
\\
   \multicolumn{1}{S{RowHeader}{r}}{36} & 
   \multicolumn{1}{S{Data}{l}}{E{\textunderscore}status} & 
   \multicolumn{1}{S{Data}{l}}{Char} & 
   \multicolumn{1}{S{Data}{r}}{2} & 
   \multicolumn{1}{S{Data}{r}}{223} & 
   \multicolumn{1}{S{Data}{l}}{Status: end-of-period employment}
\\
   \multicolumn{1}{S{RowHeader}{r}}{37} & 
   \multicolumn{1}{S{Data}{l}}{F{\textunderscore}status} & 
   \multicolumn{1}{S{Data}{l}}{Char} & 
   \multicolumn{1}{S{Data}{r}}{2} & 
   \multicolumn{1}{S{Data}{r}}{225} & 
   \multicolumn{1}{S{Data}{l}}{Status: full-quarter employment}
\\
   \multicolumn{1}{S{RowHeader}{r}}{38} & 
   \multicolumn{1}{S{Data}{l}}{FA{\textunderscore}status} & 
   \multicolumn{1}{S{Data}{l}}{Char} & 
   \multicolumn{1}{S{Data}{r}}{2} & 
   \multicolumn{1}{S{Data}{r}}{227} & 
   \multicolumn{1}{S{Data}{l}}{Status: flow into full-quarter employment}
\\
   \multicolumn{1}{S{RowHeader}{r}}{39} & 
   \multicolumn{1}{S{Data}{l}}{FJC{\textunderscore}status} & 
   \multicolumn{1}{S{Data}{l}}{Char} & 
   \multicolumn{1}{S{Data}{r}}{2} & 
   \multicolumn{1}{S{Data}{r}}{229} & 
   \multicolumn{1}{S{Data}{l}}{Status: full-quarter job creation}
\\
   \multicolumn{1}{S{RowHeader}{r}}{40} & 
   \multicolumn{1}{S{Data}{l}}{FJD{\textunderscore}status} & 
   \multicolumn{1}{S{Data}{l}}{Char} & 
   \multicolumn{1}{S{Data}{r}}{2} & 
   \multicolumn{1}{S{Data}{r}}{231} & 
   \multicolumn{1}{S{Data}{l}}{Status: full-quarter job destruction}
\\
   \multicolumn{1}{S{RowHeader}{r}}{41} & 
   \multicolumn{1}{S{Data}{l}}{FJF{\textunderscore}status} & 
   \multicolumn{1}{S{Data}{l}}{Char} & 
   \multicolumn{1}{S{Data}{r}}{2} & 
   \multicolumn{1}{S{Data}{r}}{233} & 
   \multicolumn{1}{S{Data}{l}}{Status: net change in full-quarter employment}
\\
   \multicolumn{1}{S{RowHeader}{r}}{42} & 
   \multicolumn{1}{S{Data}{l}}{FS{\textunderscore}status} & 
   \multicolumn{1}{S{Data}{l}}{Char} & 
   \multicolumn{1}{S{Data}{r}}{2} & 
   \multicolumn{1}{S{Data}{r}}{235} & 
   \multicolumn{1}{S{Data}{l}}{Status: flow out of full-quarter employment}
\\
   \multicolumn{1}{S{RowHeader}{r}}{43} & 
   \multicolumn{1}{S{Data}{l}}{H{\textunderscore}status} & 
   \multicolumn{1}{S{Data}{l}}{Char} & 
   \multicolumn{1}{S{Data}{r}}{2} & 
   \multicolumn{1}{S{Data}{r}}{237} & 
   \multicolumn{1}{S{Data}{l}}{Status: new hires}
\\
   \multicolumn{1}{S{RowHeader}{r}}{44} & 
   \multicolumn{1}{S{Data}{l}}{H3{\textunderscore}status} & 
   \multicolumn{1}{S{Data}{l}}{Char} & 
   \multicolumn{1}{S{Data}{r}}{2} & 
   \multicolumn{1}{S{Data}{r}}{239} & 
   \multicolumn{1}{S{Data}{l}}{Status: full-quarter new hires}
\\
   \multicolumn{1}{S{RowHeader}{r}}{45} & 
   \multicolumn{1}{S{Data}{l}}{JC{\textunderscore}status} & 
   \multicolumn{1}{S{Data}{l}}{Char} & 
   \multicolumn{1}{S{Data}{r}}{2} & 
   \multicolumn{1}{S{Data}{r}}{241} & 
   \multicolumn{1}{S{Data}{l}}{Status: job creation}
\\
   \multicolumn{1}{S{RowHeader}{r}}{46} & 
   \multicolumn{1}{S{Data}{l}}{JD{\textunderscore}status} & 
   \multicolumn{1}{S{Data}{l}}{Char} & 
   \multicolumn{1}{S{Data}{r}}{2} & 
   \multicolumn{1}{S{Data}{r}}{243} & 
   \multicolumn{1}{S{Data}{l}}{Status: job destruction}
\\
   \multicolumn{1}{S{RowHeader}{r}}{47} & 
   \multicolumn{1}{S{Data}{l}}{JF{\textunderscore}status} & 
   \multicolumn{1}{S{Data}{l}}{Char} & 
   \multicolumn{1}{S{Data}{r}}{2} & 
   \multicolumn{1}{S{Data}{r}}{245} & 
   \multicolumn{1}{S{Data}{l}}{Status: net job flows}
\\
   \multicolumn{1}{S{RowHeader}{r}}{48} & 
   \multicolumn{1}{S{Data}{l}}{R{\textunderscore}status} & 
   \multicolumn{1}{S{Data}{l}}{Char} & 
   \multicolumn{1}{S{Data}{r}}{2} & 
   \multicolumn{1}{S{Data}{r}}{247} & 
   \multicolumn{1}{S{Data}{l}}{Status: recalls}
\\
   \multicolumn{1}{S{RowHeader}{r}}{49} & 
   \multicolumn{1}{S{Data}{l}}{S{\textunderscore}status} & 
   \multicolumn{1}{S{Data}{l}}{Char} & 
   \multicolumn{1}{S{Data}{r}}{2} & 
   \multicolumn{1}{S{Data}{r}}{249} & 
   \multicolumn{1}{S{Data}{l}}{Status: separations}
\\
   \multicolumn{1}{S{RowHeader}{r}}{50} & 
   \multicolumn{1}{S{Data}{l}}{Z{\textunderscore}NA{\textunderscore}status} & 
   \multicolumn{1}{S{Data}{l}}{Char} & 
   \multicolumn{1}{S{Data}{r}}{2} & 
   \multicolumn{1}{S{Data}{r}}{251} & 
   \multicolumn{1}{S{Data}{l}}{Status: average periods of non-employment for accessions}
\\
   \multicolumn{1}{S{RowHeader}{r}}{51} & 
   \multicolumn{1}{S{Data}{l}}{Z{\textunderscore}NH{\textunderscore}status} & 
   \multicolumn{1}{S{Data}{l}}{Char} & 
   \multicolumn{1}{S{Data}{r}}{2} & 
   \multicolumn{1}{S{Data}{r}}{253} & 
   \multicolumn{1}{S{Data}{l}}{Status: average periods of non-employment for new hires}
\\
   \multicolumn{1}{S{RowHeader}{r}}{52} & 
   \multicolumn{1}{S{Data}{l}}{Z{\textunderscore}NR{\textunderscore}status} & 
   \multicolumn{1}{S{Data}{l}}{Char} & 
   \multicolumn{1}{S{Data}{r}}{2} & 
   \multicolumn{1}{S{Data}{r}}{255} & 
   \multicolumn{1}{S{Data}{l}}{Status: average periods of non-employment for recalls}
\\
   \multicolumn{1}{S{RowHeader}{r}}{53} & 
   \multicolumn{1}{S{Data}{l}}{Z{\textunderscore}NS{\textunderscore}status} & 
   \multicolumn{1}{S{Data}{l}}{Char} & 
   \multicolumn{1}{S{Data}{r}}{2} & 
   \multicolumn{1}{S{Data}{r}}{257} & 
   \multicolumn{1}{S{Data}{l}}{Status: average periods of non-employment for separations}
\\
   \multicolumn{1}{S{RowHeader}{r}}{54} & 
   \multicolumn{1}{S{Data}{l}}{Z{\textunderscore}W2{\textunderscore}status} & 
   \multicolumn{1}{S{Data}{l}}{Char} & 
   \multicolumn{1}{S{Data}{r}}{2} & 
   \multicolumn{1}{S{Data}{r}}{259} & 
   \multicolumn{1}{S{Data}{l}}{Status: average earnings of end-of-period employees}
\\
   \multicolumn{1}{S{RowHeader}{r}}{55} & 
   \multicolumn{1}{S{Data}{l}}{Z{\textunderscore}W3{\textunderscore}status} & 
   \multicolumn{1}{S{Data}{l}}{Char} & 
   \multicolumn{1}{S{Data}{r}}{2} & 
   \multicolumn{1}{S{Data}{r}}{261} & 
   \multicolumn{1}{S{Data}{l}}{Status: average earnings of full-quarter employees}
\\
   \multicolumn{1}{S{RowHeader}{r}}{56} & 
   \multicolumn{1}{S{Data}{l}}{Z{\textunderscore}WFA{\textunderscore}status} & 
   \multicolumn{1}{S{Data}{l}}{Char} & 
   \multicolumn{1}{S{Data}{r}}{2} & 
   \multicolumn{1}{S{Data}{r}}{263} & 
   \multicolumn{1}{S{Data}{l}}{Status: average earnings of transits to full-quarter status}
\\
   \multicolumn{1}{S{RowHeader}{r}}{57} & 
   \multicolumn{1}{S{Data}{l}}{Z{\textunderscore}WFS{\textunderscore}status} & 
   \multicolumn{1}{S{Data}{l}}{Char} & 
   \multicolumn{1}{S{Data}{r}}{2} & 
   \multicolumn{1}{S{Data}{r}}{265} & 
   \multicolumn{1}{S{Data}{l}}{Status: average earnings of separations from full-quarter status}
\\
   \multicolumn{1}{S{RowHeader}{r}}{58} & 
   \multicolumn{1}{S{Data}{l}}{Z{\textunderscore}WH3{\textunderscore}status} & 
   \multicolumn{1}{S{Data}{l}}{Char} & 
   \multicolumn{1}{S{Data}{r}}{2} & 
   \multicolumn{1}{S{Data}{r}}{267} & 
   \multicolumn{1}{S{Data}{l}}{Status: average earnings of full-quarter new hires}
\\
   \multicolumn{1}{S{RowHeader}{r}}{59} & 
   \multicolumn{1}{S{Data}{l}}{Z{\textunderscore}dWA{\textunderscore}status} & 
   \multicolumn{1}{S{Data}{l}}{Char} & 
   \multicolumn{1}{S{Data}{r}}{2} & 
   \multicolumn{1}{S{Data}{r}}{269} & 
   \multicolumn{1}{S{Data}{l}}{Status: average change in total earnings for accessions}
\\
   \multicolumn{1}{S{RowHeader}{r}}{60} & 
   \multicolumn{1}{S{Data}{l}}{Z{\textunderscore}dWS{\textunderscore}status} & 
   \multicolumn{1}{S{Data}{l}}{Char} & 
   \multicolumn{1}{S{Data}{r}}{2} & 
   \multicolumn{1}{S{Data}{r}}{271} & 
   \multicolumn{1}{S{Data}{l}}{Status: average change in total earnings for separations}
\\
\end{longtable}
\end{center}

% ====================end of output====================
 \newpage %    Generated by SAS
%    http://www.sas.com
% created by=vilhu001
% sasversion=8.2
% date=2002-05-23
% time=00:50:06
% encoding=iso-8859-1
% ====================begin of output====================
% \begin{document}

% An external file needs to be included, as specified% in latexlong.sas. This can be called sas.sty,
% in which case you want to include a line like
% \usepackage{sas}
% or it can be a simple (La)TeX file, which you 
% include by typing 
% %%
%% This is file `sas.sty',
%% generated with the docstrip utility.
%%
%% 
\NeedsTeXFormat{LaTeX2e}
\ProvidesPackage{sas}
        [2002/01/18 LEHD version 0.1
    provides definition for tables generated by SAS%
                   ]
\@ifundefined{array@processline}{\RequirePackage{array}}{}
\@ifundefined{longtable@processline}{\RequirePackage{longtable}}{}
 \def\ContentTitle{\small\it\sffamily}
 \def\Output{\small\sffamily}
 \def\HeaderEmphasis{\small\it\sffamily}
 \def\NoteContent{\small\sffamily}
 \def\FatalContent{\small\sffamily}
 \def\Graph{\small\sffamily}
 \def\WarnContentFixed{\footnotesize\tt}
 \def\NoteBanner{\small\sffamily}
 \def\DataStrong{\normalsize\bf\sffamily}
 \def\Document{\small\sffamily}
 \def\BeforeCaption{\normalsize\bf\sffamily}
 \def\ContentsDate{\small\sffamily}
 \def\Pages{\small\sffamily}
 \def\TitlesAndFooters{\footnotesize\bf\it\sffamily}
 \def\IndexProcName{\small\sffamily}
 \def\ProcTitle{\normalsize\bf\it\sffamily}
 \def\IndexAction{\small\sffamily}
 \def\Data{\small\sffamily}
 \def\Table{\small\sffamily}
 \def\FooterEmpty{\footnotesize\bf\sffamily}
 \def\SysTitleAndFooterContainer{\footnotesize\sffamily}
 \def\RowFooterEmpty{\footnotesize\bf\sffamily}
 \def\ExtendedPage{\small\it\sffamily}
 \def\FooterFixed{\footnotesize\tt}
 \def\RowFooterStrongFixed{\footnotesize\bf\tt}
 \def\RowFooterEmphasis{\footnote\it\sffamily}
 \def\ContentFolder{\small\sffamily}
 \def\Container{\small\sffamily}
 \def\Date{\small\sffamily}
 \def\RowFooterFixed{\footnotesize\tt}
 \def\Caption{\normalsize\bf\sffamily}
 \def\WarnBanner{\small\sffamily}
 \def\Frame{\small\sffamily}
 \def\HeaderStrongFixed{\footnotesize\bf\tt}
 \def\IndexTitle{\small\it\sffamily}
 \def\NoteContentFixed{\footnotesize\tt}
 \def\DataEmphasisFixed{\footnotesize\it\tt}
 \def\Note{\small\sffamily}
 \def\Byline{\normalsize\bf\sffamily}
 \def\FatalBanner{\small\sffamily}
 \def\ProcTitleFixed{\footnotesize\bf\tt}
 \def\ByContentFolder{\small\sffamily}
 \def\PagesProcLabel{\small\sffamily}
 \def\RowHeaderFixed{\footnotesize\tt}
 \def\RowFooterEmphasisFixed{\footnotesize\it\tt}
 \def\WarnContent{\small\sffamily}
 \def\DataEmpty{\small\sffamily}
 \def\Cell{\small\sffamily}
 \def\Header{\normalsize\bf\sffamily}
 \def\PageNo{\normalsize\bf\sffamily}
 \def\ContentProcLabel{\small\sffamily}
 \def\HeaderFixed{\footnotesize\tt}
 \def\PagesTitle{\small\it\sffamily}
 \def\RowHeaderEmpty{\normalsize\bf\sffamily}
 \def\PagesProcName{\small\sffamily}
 \def\Batch{\footnotesize\tt}
 \def\ContentItem{\small\sffamily}
 \def\Body{\small\sffamily}
 \def\PagesDate{\small\sffamily}
 \def\Index{\small\sffamily}
 \def\HeaderEmpty{\normalsize\bf\sffamily}
 \def\FooterStrong{\footnotesize\bf\sffamily}
 \def\FooterEmphasis{\footnotesize\it\sffamily}
 \def\ErrorContent{\small\sffamily}
 \def\DataFixed{\footnotesize\tt}
 \def\HeaderStrong{\normalsize\bf\sffamily}
 \def\GraphBackground{}
 \def\DataEmphasis{\small\it\sffamily}
 \def\TitleAndNoteContainer{\small\sffamily}
 \def\RowFooter{\footnotesize\bf\sffamily}
 \def\IndexItem{\small\sffamily}
 \def\BylineContainer{\small\sffamily}
 \def\FatalContentFixed{\footnotesize\tt}
 \def\BodyDate{\small\sffamily}
 \def\RowFooterStrong{\footnotesize\bf\sffamily}
 \def\UserText{\small\sffamily}
 \def\HeadersAndFooters{\footnotesize\bf\sffamily}
 \def\RowHeaderEmphasisFixed{\footnotesize\it\tt}
 \def\ErrorBanner{\small\sffamily}
 \def\ContentProcName{\small\sffamily}
 \def\RowHeaderStrong{\normalsize\bf\sffamily}
 \def\FooterEmphasisFixed{\footnotesize\it\tt}
 \def\Contents{\small\sffamily}
 \def\FooterStrongFixed{\footnotesize\bf\tt}
 \def\PagesItem{\small\sffamily}
 \def\RowHeader{\normalsize\bf\sffamily}
 \def\AfterCaption{\normalsize\bf\sffamily}
 \def\RowHeaderStrongFixed{\footnotesize\bf\tt}
 \def\RowHeaderEmphasis{\small\it\sffamily}
 \def\DataStrongFixed{\footnotesize\bf\tt}
 \def\Footer{\footnotesize\bf\sffamily}
 \def\FolderAction{\small\sffamily}
 \def\HeaderEmphasisFixed{\footnotesize\it\tt}
 \def\SystemTitle{\large\bf\it\sffamily}
 \def\ErrorContentFixed{\footnotesize\tt}
 \def\SystemFooter{\footnotesize\it\sffamily}
% Set cell padding 
\renewcommand{\arraystretch}{1.3}
% Headings
\newcommand{\heading}[2]{\csname#1\endcsname #2}
\newcommand{\proctitle}[2]{\csname#1\endcsname #2}
% Declare new column type
\newcolumntype{S}[2]{>{\csname#1\endcsname}#2}
% Set warning box style
\newcommand{\msg}[2]{\fbox{%
   \begin{minipage}{\textwidth}#2\end{minipage}}%
}

\begin{center}\heading{SystemTitle}{Minnesota           }\end{center}
\begin{center}\heading{ProcTitle}{The FREQ Procedure}\end{center}
\begin{center}\begin{longtable}
{lrrrr}\hline % colspecs
% table_head start
   \multicolumn{5}{S{Header}{c}}{FIPS State}
\\
   \multicolumn{1}{S{Header}{l}}{state} & 
   \multicolumn{1}{S{Header}{r}}{Frequency} & 
   \multicolumn{1}{S{Header}{r}}{ Percent} & 
   \multicolumn{1}{S{Header}{r}}{Cumulative\linebreak  Frequency} & 
   \multicolumn{1}{S{Header}{r}}{Cumulative\linebreak   Percent}
\\
\hline 
\endhead % table_head end
\hline 
\multicolumn{1}{r}{(cont.)}\\
\endfoot 
\hline 
\endlastfoot % table_head end
   \multicolumn{1}{S{RowHeader}{l}}{27 MINNESOTA} & 
   \multicolumn{1}{S{Data}{r}}{8424} & 
   \multicolumn{1}{S{Data}{r}}{100.00} & 
   \multicolumn{1}{S{Data}{r}}{8424} & 
   \multicolumn{1}{S{Data}{r}}{100.00}
\\
\end{longtable}
\end{center}
\begin{center}\begin{longtable}
{lrrrr}\hline % colspecs
% table_head start
   \multicolumn{5}{S{Header}{c}}{SIC Division}
\\
   \multicolumn{1}{S{Header}{l}}{sic{\textunderscore}division} & 
   \multicolumn{1}{S{Header}{r}}{Frequency} & 
   \multicolumn{1}{S{Header}{r}}{ Percent} & 
   \multicolumn{1}{S{Header}{r}}{Cumulative\linebreak  Frequency} & 
   \multicolumn{1}{S{Header}{r}}{Cumulative\linebreak   Percent}
\\
\hline 
\endhead % table_head end
\hline 
\multicolumn{1}{r}{(cont.)}\\
\endfoot 
\hline 
\endlastfoot % table_head end
   \multicolumn{1}{S{RowHeader}{l}}{A Agriculture etc.} & 
   \multicolumn{1}{S{Data}{r}}{702} & 
   \multicolumn{1}{S{Data}{r}}{9.09} & 
   \multicolumn{1}{S{Data}{r}}{702} & 
   \multicolumn{1}{S{Data}{r}}{9.09}
\\
   \multicolumn{1}{S{RowHeader}{l}}{B Mining} & 
   \multicolumn{1}{S{Data}{r}}{702} & 
   \multicolumn{1}{S{Data}{r}}{9.09} & 
   \multicolumn{1}{S{Data}{r}}{1404} & 
   \multicolumn{1}{S{Data}{r}}{18.18}
\\
   \multicolumn{1}{S{RowHeader}{l}}{C Construction} & 
   \multicolumn{1}{S{Data}{r}}{702} & 
   \multicolumn{1}{S{Data}{r}}{9.09} & 
   \multicolumn{1}{S{Data}{r}}{2106} & 
   \multicolumn{1}{S{Data}{r}}{27.27}
\\
   \multicolumn{1}{S{RowHeader}{l}}{D Manufacturing} & 
   \multicolumn{1}{S{Data}{r}}{702} & 
   \multicolumn{1}{S{Data}{r}}{9.09} & 
   \multicolumn{1}{S{Data}{r}}{2808} & 
   \multicolumn{1}{S{Data}{r}}{36.36}
\\
   \multicolumn{1}{S{RowHeader}{l}}{E Trans. \& Utilities} & 
   \multicolumn{1}{S{Data}{r}}{702} & 
   \multicolumn{1}{S{Data}{r}}{9.09} & 
   \multicolumn{1}{S{Data}{r}}{3510} & 
   \multicolumn{1}{S{Data}{r}}{45.45}
\\
   \multicolumn{1}{S{RowHeader}{l}}{F Wholesale trade} & 
   \multicolumn{1}{S{Data}{r}}{702} & 
   \multicolumn{1}{S{Data}{r}}{9.09} & 
   \multicolumn{1}{S{Data}{r}}{4212} & 
   \multicolumn{1}{S{Data}{r}}{54.55}
\\
   \multicolumn{1}{S{RowHeader}{l}}{G Retail Trade} & 
   \multicolumn{1}{S{Data}{r}}{702} & 
   \multicolumn{1}{S{Data}{r}}{9.09} & 
   \multicolumn{1}{S{Data}{r}}{4914} & 
   \multicolumn{1}{S{Data}{r}}{63.64}
\\
   \multicolumn{1}{S{RowHeader}{l}}{H FIRE} & 
   \multicolumn{1}{S{Data}{r}}{702} & 
   \multicolumn{1}{S{Data}{r}}{9.09} & 
   \multicolumn{1}{S{Data}{r}}{5616} & 
   \multicolumn{1}{S{Data}{r}}{72.73}
\\
   \multicolumn{1}{S{RowHeader}{l}}{I Services} & 
   \multicolumn{1}{S{Data}{r}}{702} & 
   \multicolumn{1}{S{Data}{r}}{9.09} & 
   \multicolumn{1}{S{Data}{r}}{6318} & 
   \multicolumn{1}{S{Data}{r}}{81.82}
\\
   \multicolumn{1}{S{RowHeader}{l}}{J Public Admin.} & 
   \multicolumn{1}{S{Data}{r}}{702} & 
   \multicolumn{1}{S{Data}{r}}{9.09} & 
   \multicolumn{1}{S{Data}{r}}{7020} & 
   \multicolumn{1}{S{Data}{r}}{90.91}
\\
   \multicolumn{1}{S{RowHeader}{l}}{Other} & 
   \multicolumn{1}{S{Data}{r}}{702} & 
   \multicolumn{1}{S{Data}{r}}{9.09} & 
   \multicolumn{1}{S{Data}{r}}{7722} & 
   \multicolumn{1}{S{Data}{r}}{100.00}
\\
\end{longtable}
\end{center}
\begin{center}\heading{ProcTitle}{Frequency Missing = 702}\end{center}
\begin{center}\begin{longtable}
{rrrrr}\hline % colspecs
% table_head start
   \multicolumn{5}{S{Header}{c}}{Sex}
\\
   \multicolumn{1}{S{Header}{r}}{sex} & 
   \multicolumn{1}{S{Header}{r}}{Frequency} & 
   \multicolumn{1}{S{Header}{r}}{ Percent} & 
   \multicolumn{1}{S{Header}{r}}{Cumulative\linebreak  Frequency} & 
   \multicolumn{1}{S{Header}{r}}{Cumulative\linebreak   Percent}
\\
\hline 
\endhead % table_head end
\hline 
\multicolumn{1}{r}{(cont.)}\\
\endfoot 
\hline 
\endlastfoot % table_head end
   \multicolumn{1}{S{RowHeader}{r}}{0 : All} & 
   \multicolumn{1}{S{Data}{r}}{2808} & 
   \multicolumn{1}{S{Data}{r}}{33.33} & 
   \multicolumn{1}{S{Data}{r}}{2808} & 
   \multicolumn{1}{S{Data}{r}}{33.33}
\\
   \multicolumn{1}{S{RowHeader}{r}}{1 : Men} & 
   \multicolumn{1}{S{Data}{r}}{2808} & 
   \multicolumn{1}{S{Data}{r}}{33.33} & 
   \multicolumn{1}{S{Data}{r}}{5616} & 
   \multicolumn{1}{S{Data}{r}}{66.67}
\\
   \multicolumn{1}{S{RowHeader}{r}}{2 : Women} & 
   \multicolumn{1}{S{Data}{r}}{2808} & 
   \multicolumn{1}{S{Data}{r}}{33.33} & 
   \multicolumn{1}{S{Data}{r}}{8424} & 
   \multicolumn{1}{S{Data}{r}}{100.00}
\\
\end{longtable}
\end{center}
\begin{center}\begin{longtable}
{rrrrr}\hline % colspecs
% table_head start
   \multicolumn{5}{S{Header}{c}}{Age group}
\\
   \multicolumn{1}{S{Header}{r}}{agegroup} & 
   \multicolumn{1}{S{Header}{r}}{Frequency} & 
   \multicolumn{1}{S{Header}{r}}{ Percent} & 
   \multicolumn{1}{S{Header}{r}}{Cumulative\linebreak  Frequency} & 
   \multicolumn{1}{S{Header}{r}}{Cumulative\linebreak   Percent}
\\
\hline 
\endhead % table_head end
\hline 
\multicolumn{1}{r}{(cont.)}\\
\endfoot 
\hline 
\endlastfoot % table_head end
   \multicolumn{1}{S{RowHeader}{r}}{0 : All Ages} & 
   \multicolumn{1}{S{Data}{r}}{936} & 
   \multicolumn{1}{S{Data}{r}}{11.11} & 
   \multicolumn{1}{S{Data}{r}}{936} & 
   \multicolumn{1}{S{Data}{r}}{11.11}
\\
   \multicolumn{1}{S{RowHeader}{r}}{1 : 14-18} & 
   \multicolumn{1}{S{Data}{r}}{936} & 
   \multicolumn{1}{S{Data}{r}}{11.11} & 
   \multicolumn{1}{S{Data}{r}}{1872} & 
   \multicolumn{1}{S{Data}{r}}{22.22}
\\
   \multicolumn{1}{S{RowHeader}{r}}{2 : 19-21} & 
   \multicolumn{1}{S{Data}{r}}{936} & 
   \multicolumn{1}{S{Data}{r}}{11.11} & 
   \multicolumn{1}{S{Data}{r}}{2808} & 
   \multicolumn{1}{S{Data}{r}}{33.33}
\\
   \multicolumn{1}{S{RowHeader}{r}}{3 : 22-24} & 
   \multicolumn{1}{S{Data}{r}}{936} & 
   \multicolumn{1}{S{Data}{r}}{11.11} & 
   \multicolumn{1}{S{Data}{r}}{3744} & 
   \multicolumn{1}{S{Data}{r}}{44.44}
\\
   \multicolumn{1}{S{RowHeader}{r}}{4 : 25-34} & 
   \multicolumn{1}{S{Data}{r}}{936} & 
   \multicolumn{1}{S{Data}{r}}{11.11} & 
   \multicolumn{1}{S{Data}{r}}{4680} & 
   \multicolumn{1}{S{Data}{r}}{55.56}
\\
   \multicolumn{1}{S{RowHeader}{r}}{5 : 35-44} & 
   \multicolumn{1}{S{Data}{r}}{936} & 
   \multicolumn{1}{S{Data}{r}}{11.11} & 
   \multicolumn{1}{S{Data}{r}}{5616} & 
   \multicolumn{1}{S{Data}{r}}{66.67}
\\
   \multicolumn{1}{S{RowHeader}{r}}{6 : 45-54} & 
   \multicolumn{1}{S{Data}{r}}{936} & 
   \multicolumn{1}{S{Data}{r}}{11.11} & 
   \multicolumn{1}{S{Data}{r}}{6552} & 
   \multicolumn{1}{S{Data}{r}}{77.78}
\\
   \multicolumn{1}{S{RowHeader}{r}}{7 : 55-64} & 
   \multicolumn{1}{S{Data}{r}}{936} & 
   \multicolumn{1}{S{Data}{r}}{11.11} & 
   \multicolumn{1}{S{Data}{r}}{7488} & 
   \multicolumn{1}{S{Data}{r}}{88.89}
\\
   \multicolumn{1}{S{RowHeader}{r}}{8 : 65+} & 
   \multicolumn{1}{S{Data}{r}}{936} & 
   \multicolumn{1}{S{Data}{r}}{11.11} & 
   \multicolumn{1}{S{Data}{r}}{8424} & 
   \multicolumn{1}{S{Data}{r}}{100.00}
\\
\end{longtable}
\end{center}
\begin{center}\begin{longtable}
{llllll}\hline % colspecs
% table_head start
   \multicolumn{6}{S{Header}{c}}{Table of year by quarter}
\\
   \multicolumn{1}{S{Header}{c}}{year(Year)} & 
   \multicolumn{4}{S{Header}{c}}{quarter(Quarter)} & 
   \multicolumn{1}{S{Header}{r}}{Total}
\\
   \multicolumn{1}{l}{~} & 
   \multicolumn{1}{S{Header}{r}}{      1     } & 
   \multicolumn{1}{S{Header}{r}}{      2     } & 
   \multicolumn{1}{S{Header}{r}}{      3     } & 
   \multicolumn{1}{S{Header}{r}}{      4     } & 
   \multicolumn{1}{l}{~}
\\
\hline 
\endhead % table_head end
\hline 
\multicolumn{1}{r}{(cont.)}\\
\endfoot 
\hline 
\endlastfoot % table_head end
   \multicolumn{1}{S{Header}{r}}{1994        } & 
   \multicolumn{1}{S{Data}{r}}{     0} & 
   \multicolumn{1}{S{Data}{r}}{     0} & 
   \multicolumn{1}{S{Data}{r}}{   324} & 
   \multicolumn{1}{S{Data}{r}}{   324} & 
   \multicolumn{1}{S{Data}{r}}{   648}
\\
   \multicolumn{1}{S{Header}{r}}{1995        } & 
   \multicolumn{1}{S{Data}{r}}{   324} & 
   \multicolumn{1}{S{Data}{r}}{   324} & 
   \multicolumn{1}{S{Data}{r}}{   324} & 
   \multicolumn{1}{S{Data}{r}}{   324} & 
   \multicolumn{1}{S{Data}{r}}{  1296}
\\
   \multicolumn{1}{S{Header}{r}}{1996        } & 
   \multicolumn{1}{S{Data}{r}}{   324} & 
   \multicolumn{1}{S{Data}{r}}{   324} & 
   \multicolumn{1}{S{Data}{r}}{   324} & 
   \multicolumn{1}{S{Data}{r}}{   324} & 
   \multicolumn{1}{S{Data}{r}}{  1296}
\\
   \multicolumn{1}{S{Header}{r}}{1997        } & 
   \multicolumn{1}{S{Data}{r}}{   324} & 
   \multicolumn{1}{S{Data}{r}}{   324} & 
   \multicolumn{1}{S{Data}{r}}{   324} & 
   \multicolumn{1}{S{Data}{r}}{   324} & 
   \multicolumn{1}{S{Data}{r}}{  1296}
\\
   \multicolumn{1}{S{Header}{r}}{1998        } & 
   \multicolumn{1}{S{Data}{r}}{   324} & 
   \multicolumn{1}{S{Data}{r}}{   324} & 
   \multicolumn{1}{S{Data}{r}}{   324} & 
   \multicolumn{1}{S{Data}{r}}{   324} & 
   \multicolumn{1}{S{Data}{r}}{  1296}
\\
   \multicolumn{1}{S{Header}{r}}{1999        } & 
   \multicolumn{1}{S{Data}{r}}{   324} & 
   \multicolumn{1}{S{Data}{r}}{   324} & 
   \multicolumn{1}{S{Data}{r}}{   324} & 
   \multicolumn{1}{S{Data}{r}}{   324} & 
   \multicolumn{1}{S{Data}{r}}{  1296}
\\
   \multicolumn{1}{S{Header}{r}}{2000        } & 
   \multicolumn{1}{S{Data}{r}}{   324} & 
   \multicolumn{1}{S{Data}{r}}{   324} & 
   \multicolumn{1}{S{Data}{r}}{   324} & 
   \multicolumn{1}{S{Data}{r}}{   324} & 
   \multicolumn{1}{S{Data}{r}}{  1296}
\\
   \multicolumn{1}{S{Header}{l}}{Total           } & 
   \multicolumn{1}{S{Data}{r}}{   1944} & 
   \multicolumn{1}{S{Data}{r}}{   1944} & 
   \multicolumn{1}{S{Data}{r}}{   2268} & 
   \multicolumn{1}{S{Data}{r}}{   2268} & 
   \multicolumn{1}{S{Data}{r}}{   8424}
\\
\end{longtable}
\end{center}

% ====================end of output====================

% 
% \newpage
% 
% \subsection{North Carolina}
% 
% \input{\mypath/nc_county_v23_fuzzed.tex} \newpage \input{\mypath/nc_county_v23_fuzzed.freq.tex}
% \newpage \input{\mypath/nc_sic_division_v23_fuzzed.tex} \newpage \input{%
% \mypath/nc_sic_division_v23_fuzzed.freq.tex}
% 
% \subsection{Texas}
% 
% \input{\mypath/tx_county_v23_fuzzed.tex} \newpage \input{\mypath/tx_county_v23_fuzzed.freq.tex}
% \newpage \input{\mypath/tx_sic_division_v23_fuzzed.tex} \newpage \input{%
% \mypath/tx_sic_division_v23_fuzzed.freq.tex}


\subsection{Example statistics}
\label{sec:example_statistics}

Table~\Vref{tab:table_example} shows the extreme case of statistics for a
small county and for one gender only.%
%
%\footnote{The exact county and gender remain anonymous for data
%  confidentiality reasons.} 
%
Two types of confidentiality measures are
implemented within the same table. Ten cells have been distorted through
the injection of noise described earlier. Furthermore, one cell has been
suppressed because the statistics in that cell is based on too few
individuals.


  % this is from July Tables for CA, table9.html

\begin{table}[htbp]
\small
\begin{center}
  \caption{Example table: ALEXANDER, IL, Men only, by age group}
%  \caption{Anonymous county, one gender only, by age group}
  \label{tab:table_example}
  \begin{tabular}{rrrrrrrrrrrrr}
\hline
\\[-.3cm]
        &          & &        & &           & &     & & Average & &Average & \\
        &Beginning-& &        & &           & &     & &earnings & &earnings& \\
        &of-period & &  Job   & &    Job    & & New & &of full- & &of full-& \\
        &employment& &creation& &destruction& &hires& & quarter & &quarter & \\
        &          & &        & &           & &     & &employees& &  new   & \\
        &          & &        & &           & &     & &         & & hires  & \\
\hline                                                                        
All Ages&     1,266& &      67& &         76& &  192& &    6,764& &   3,945& \\
14-18   &        21&*&      10&*&          1&*&   22& &    1,794&*&     921&*\\
19-21   &        44& &       9& &          5& &   21& &    3,835& &   3,645&*\\
22-24   &        74& &      12& &         14& &   19& &    4,020& &   3,425& \\
25-34   &       286& &      26& &         32& &   58& &    6,041& &   5,128& \\
35-44   &       356& &      22& &         27& &   38& &    6,477& &   4,827& \\
45-54   &       278& &      12& &         12& &   25& &    8,644& &   1,861&*\\
55-64   &       158& &       2& &          9& &    6& &    8,592& &   4,207&*\\
65+     &        50& &       1& &          4& &     &d&    3,379& &   1,216&*\\
\hline                                                                        
\multicolumn{13}{l}{\footnotesize  * indicates significant distortion   is necessary to preserve confidentiality}\\[-.15cm]
\multicolumn{13}{l}{\footnotesize  d indicates an estimate is based on less
  than 3 employees in the at-risk group}\\[-.15cm]
\multicolumn{13}{l}{\footnotesize  n indicates an estimate is not defined because no employees are in the relevant category}\\[-.15cm]
\end{tabular}
\end{center}
\end{table}

%%% Local Variables: 
%%% mode: latex
%%% TeX-master: "qwi-overview"
%%% End: 

  
  Figure~\Vref{fig:figure_example} presents the data in a different way.
  The graph shows net job creation rates by county, for both genders, and
  for the youngest workers. It shows large dispersion across the state, and
  highlights the value added from joining demographic and firm-level data
  to form new statistics.

  % for qwi-overview.tex
% Example graphics
\begin{figure}[htbp]
  \begin{center}
\centerline{\includegraphics[width=\textwidth]{\mypath/il_gmap_Ages_19-21_All_net.eps}} 
    
    \caption{Example graph}
    \label{fig:figure_example}
  \end{center}
\end{figure}
%%% Local Variables: 
%%% mode: latex
%%% TeX-master: "qwi-overview"
%%% End: 

  

%%% Local Variables: 
%%% mode: latex
%%% TeX-master: "qwi-overview"
%%% End: 

%TCIDATA{Version=4.00.0.2321}
%TCIDATA{LaTeXparent=0,0,sw-edit.tex}
                      
%TCIDATA{ChildDefaults=chapter:1,page:1}
%
% Time-stamp: <05/03/09 22:01:22 vilhuber>

%\section{Forming Aggregated Estimates}
%\label{sec:aggregate}


\subsection{What are the QWI statistics?}
\label{sec:qwi:what}

[ describe variables / statistics here ]


\subsection{Computing the statistics}
\label{sec:qwi:compute}


Aggregating the QWI data is a four step process, which can be summarized as
follows:

\begin{itemize}
\item[\ ] 

\begin{enumerate}
\item The basic variables, as discussed above, are created for each
employment history (PIK\index{PIK}-SEIN\index{SEIN}-SEINUNIT\index{SEINUNIT} pair) and for every quarter
that the pair exits. Note that each job history stems from an implicate of
the U2W, and
is weighted accordingly. 


\item The QWI system sums for each employer's workplace the following variables:
beginning-of-period employment, end-of-period employment, accessions, new
hires, recalls, separations, full-quarter employment, full-quarter
accessions, full-quarter new hires, total earnings of full-quarter
employees, total earnings of full-quarter accessions, and total earnings of
full-quarter new hires. Job creations, job destructions, and net job flows
are estimated for each workplace using the beginning and end of quarter
employment estimates for that workplace. The first-layer of
disclosure-proofing is also applied at this step.

\item The workplace-level variables in the list above are summed over the
relevant aggregating unit (county or SIC) for each quarter. Average
earnings of full-quarter employees, full-quarter accessions, and
full-quarter new hires are estimated by taking the ratio of total earnings
of the relevant category to the total number of individuals in that
category. \ For example, avearge earnings of full-quarter men ages 55-64 for
a given year, quarter and county is the ratio of total earnings of
full-quarter men ages 55-64 to the number of full-quarter men ages 55-64 in
that year, quarter, and county.

\item The beginning-of-quarter employment for each county or SIC division is
controlled (raked) to the BLS estimate of total county employment in month
one of that quarter from the Covered Employment and Wages series. At this
point the other estimates and the demographic groups are also raked to
preserve the underlying relations among the variables.
\end{enumerate}
\end{itemize}

%\chapter{Worker and Job Flow Analysis\label{cha:jobflow}}
%
%%TCIDATA{LaTeXparent=0,0,sw-edit.tex}

% -*- latex -*- 
%
% Time-stamp: <02/07/26 13:43:51 vilhuber> 
%              Automatically adjusted if using Xemacs
%              Please adjust manually if using other editors
%
% jobflow_analysis.tex
% Responsible: John/Paul/Bryce
% Part of QWI_methods.tex

%\section{Overview}
%\index{employment status}
%\index{point-in-time}
%\index{full-quarter}
%\index{accession}
%\index{separation}
%\index{job flows}
%\index{new hires}
%\index{job creation}
%\index{job destruction}
%\index{earnings}
%\index{earnings!change}

The detailed description that follows makes explicit the links to the sets
of files described earlier.
The statistics calculated in this section are based on definitions  summarized in \Cite{AbowdCorbelKramarz99} and
\Cite{DavisHaltiwanger99}. 
As mentioned before, employment is measured at two points in
time (beginning and end of  
quarter) and according to two concepts (any employment status and 
full-quarter employment status). Worker flows are captured by accessions and 
separations with respect to both employment status concepts. Job flows are 
captured by gross job creation and destruction at the firm level, again 
according to both employment concepts. Accessions are further separated into 
new hires and recalls. Earnings and earnings change statistics are 
calculated for each of the worker flow categories as well as for both 
employment statuses.

%\subsection{Calculation of statistics}

The worker and employment flow statistics reported at the county and 
SIC division level are calculated through a multi-step 
\index{county}\index{SIC}
process.%
%
%\footnote{Details on the program sequence used to create the job 
%low statistics are available in Appendix~\Vref{app:jobflow_technical}.}
%
The EHF\index{EHF} (see Subsection~\ref{cha:ehf}), which contains individual
work and earnings histories, is combined with information from the
ICF\index{ICF} (see Subsection~\ref{cha:icf}) to incorporate demographic
characteristics of workers such as age and sex. For each worker in each
year and quarter, an array of jobs at various SEINs\index{SEIN} is stored.
The statistics listed in Subsection~\Vref{cha:definitions} are computed, when
appropriate, for each individual/job/quarter combination. The statistics
are then aggregated to the SEIN level by age and sex to create a file of
totals for each SEIN/year/quarter/agegroup/sexgroup combination. Both the
Workforce Investment Act (WIA)\index{WIA} and CPS\index{CPS} age groups are
used. The totals are stored by age/sex group as well as further aggregated
within SEIN over age and sex group to produce the overall total for the
SEIN as well as marginal totals for sex and age (for example, the total for
females of all ages).  All totals are then aggregated twice more: once to
the industry level and once to the county level. At this point the
statistics are in their final form except for the handling of disclosure
issues, as discussed below.

\subsection{Examples}

The following tables provide an example of how the flow statistics are
computed for four hypothetical individuals who work at three hypothetical
employers over a two year sample period. All individuals and firms in this
example are fictitious. Table~\Vref{table1} summarizes the earnings history
\index{earnings history} of each individual as it would appear in the
employment history file\index{EHF}.  Table~\Vref{table2} presents the
individual level employment flow statistics that can be computed from the
individual work histories. Note that individual 1 leaves employer
\textit{X} at some point during the second quarter of 1995, and that she
begins working for employer \textit{Y} during the same quarter. In
Table~\ref{table2}, employment flow statistics as defined in
Subsection~\ref{cha:definitions} have been computed for every quarter of every
job worked by Person~\textit{1}. Person~\textit{1} is considered to be
employed at employer \textit{X} from 1994:1 -- 1995:2.  Hence, \textit{e=1}
from 1994:1 through 1995:1 since she is still employed at \textit{X} at the
end of each of these quarters. Similarly, \textit{b=1} from 1994:2 through
1995:2 since she is employed at \textit{X} from the very beginning of these
quarters. Note that \textit{b} is missing in 1994:1. The first quarter of
the analysis is out-of-scope for \textit{b}, since it depends on employment
information from the previous quarter. Also note that for in-scope periods,
end-of-quarter employment\index{employment!point-in-time} at time
\textit{t} is equal to beginning-of quarter employment at time \textit{t +
  1}. In Subsection~\ref{cha:definitions}, this identity (Identity
\ref{identity:1}) is defined for aggregates, but as shown in the example it
holds at the individual level as well.

\input{\mypath/jobflow_analysis.table1.tex}

Moving on, \textit{f=1} for Person \textit{1}/Employer \textit{X} from
1994:2-1995:1, but \textit{f} is missing during 1994:1, which is
out-of-scope, and \textit{f=0} during 1995:2 because she is no longer
employed at \textit{X} in 1995:3. In 1995:2 \textit{s=1} and \textit{fs=1}
for Person~\textit{1}/ Employer~\textit{X} because she separates from
Employer~\textit{X} sometime during this quarter and appears to have been
in this job for the entire preceding quarter (1995:1).  In 1995:2,
\textit{a=1} for Person~\textit{1} and Employer~\textit{Y} because she
enters a relationship with Employer~\textit{Y} sometime during this
quarter, and \textit{fa=1} in 1995:3 because this is her first full quarter
\index{employment!full-quarter} at Employer~\textit{Y}. New hires\index{New
  hires}, \textit{h}, is also 1 because she has no previous relationship
with Employer~\textit{Y} in the last four quarters, and recalls\index{recall}
\textit{r=0} for the same reason. A variety of wage measures are also
calculated for each individual: \textit{w1} is simply the wage earned at
each job each quarter, while measures such as \textit{w2}, \textit{w3},
\textit{wa} are calculated as an individual's wage if he or she meets a
certain criteria (\textit{e=1} for \textit{w2}, \textit{f=1} for
\textit{w3}, \textit{etc}.).

\input{\mypath/jobflow_analysis.table2.tex}

In Table~\Vref{table3}, the individual statistics are aggregated to the
employer level by summing individual statistics by SEIN. \textit{E} for
Employer~\textit{X} in 95:1, then, is the sum of \textit{e} over all
individuals working at \textit{X} in 1995:1 (in this case individuals 1 and
4). Since \textit{e=1} for Individual~\textit{1}, who remains with
Employer~\textit{X} the next quarter, and \textit{e=0} for
Individual~\textit{4}, who has no wage record with Employer~\textit{X} the
next quarter, \textit{E=1}. Similarly, since \textit{a=0} for both
individuals this quarter (both worked at \textit{X} last quarter also),
\textit{A=0}. The job flow\mindex{job flow} at Employer~\textit{X}, defined
as the net increase in employment over that quarter, is calculated as the
difference between the number of end-of-quarter jobs held and the number of
beginning-of quarter jobs held. Thus, $JF = E -- B$, or 1 -- 2 = -1 in this
case. Because there was a negative net job flow of 1 this quarter, job
creation\index{job creation} \textit{JC= 0} and job destruction\index{job
  destruction} \textit{JD=1}. Total payroll \textit{W1} is also computed
for each employer; for Employer~\textit{X} in 1995:1 it is simply the sum
of the wages paid to individuals 1 and 4: {\$}5000 + {\$}4000 = {\$}9000.
Individual~1 also had end-of-quarter wages \textit{w2=5000} because she was
end-of-quarter employed at \textit{X} this period. For
Individual~\textit{4}, \textit{w2=0} because \textit{e=0} at \textit{X} in
1995:1. Total end of quarter wages \textit{W2} for Employer~\textit{X} in
1995:1 is then calculated as the sum of wages at all end-of-quarter jobs.
In this case, it is simply {\$}5000 since Individual~1 has the only
end-of-quarter job at \textit{X} in 1995:1.

\input{\mypath/jobflow_analysis.table3.tex}

%\marginpar{\tiny This paragraph (and chapter) might need to be tagged for the index!!}
Several identities from Subsection~\ref{cha:definitions} are illustrated in
Table~\ref{table3}.  Once again Identity~\ref{identity:1} ($B_{jt} = E_{jt
  - 1} )$ is noticeable just from glancing at the columns of numbers B and
E.  Identity~\ref{identity:3}, $E_{jt} = B_{jt} + A_{jt} - S_{jt} $ also
holds whenever all four variables are in-scope. For example, for
Employer~\textit{X} in 1995:1, \textit{E = 1 = 2 + 0 -- 1}. For this
employer in 1995:2, \textit{E = 0 = 1 + 0 -- 1}. Identity~\ref{identity:4},
$JF_{jt} = JC_{jt} - JD_{jt}$ is also true: for \textit{X} in 1995: 1
\textit{JF = -1 = 0 -- 1}. Identity~\ref{identity:5}, $E_{jt} = B_{jt} +
JC_{jt} - JD_{jt} $ , (\textit{X} in 1995:1 : \textit{E = 1 = 2 + 0 -- 1})
and Identity~\ref{identity:6}, $A_{jt} - S_{jt} = JC_{jt} - JD_{jt} $,
(\textit{X} in 1995: 1 : \textit{A -- S = 0 -- 1 = 0 -- 1 = JC -- JD}).
Finally, Identity~\ref{identity:15}, the total payroll identity ($W_{1jt} =
W_{2jt} + WS_{jt} )$ is met in all cases. For example, for SEIN \textit{X}
in 1995: 2, \textit{W1 = {\$}9000 = 5000 + 4000 = W2 + WS}. When \textit{WS
  is missing}, as in most cases, \textit{W1} and \textit{W2} are simply
equal because every wage is an end-of-quarter wage.

\input{\mypath/jobflow_analysis.table4.tex}

In Table~\Vref{table4}, the SEIN-level statistics are aggregated in a
similar way to create total flows and average wages. These can be thought
of as county totals if the hypothetical universe includes just a single
county. The total flows are computed exactly as the employer level flows in
Table~\ref{table3}. For 1995:1, total jobs at the end of quarter, total
\textit{E}, is just the sum of \textit{E} for \textit{X}, \textit{Y}, and
\textit{Z}: 1 + 0 + 1 = 2. Note that this is the same as the sum of all
individual \textit{e} in Table\ref{table2} for 1995:1. Total accessions are computed
similarly (\textit{A = 0 + 0 + 0}) as are total wages (\textit{W1 = 9000 +
  2000 + 6000 = 17000}).  Average wages (for example, \textit{Z\_W2},
\textit{Z\_WA}) are computed by summing total wages for \textit{X},
\textit{Y}, and \textit{Z}, and dividing by the total number of individuals
used to calculate the particular wage measure. For example, \textit{Z\_W2}
for 1995: 1 is computed as the sum of \textit{W2} for \textit{X},
\textit{Y}, and \textit{Z} where defined (5000 + 6000) divided by the total
number of end-of-quarter positions (\textit{E = 2}) for an average
end-of-quarter wage of \textit{Z\_W2} = {\$}5500. \textit{Z\_WA} is
undefined for this quarter because there are no accessions this quarter. In
1995:2 they are computed as the sum of \textit{WA} for \textit{X},
\textit{Y}, and \textit{Z} where defined (2000, since \textit{WA} is only defined
for \textit{Y} this quarter) divided by the total number of accessions this
quarter (1) so the average wage to accessions in 1995:2 is simply {\$}2000.




%%% Local Variables: 
%%% mode: latex
%%% TeX-master: "qwi-overview"
%%% End: 

%%TCIDATA{LaTeXparent=0,0,sw-edit.tex}

% -*- latex -*- 
%
% Time-stamp: <02/07/26 09:44:20 vilhuber> 
%              Automatically adjusted if using Xemacs
%              Please adjust manually if using other editors
%
% jobflow_analysis_data_consistency.tex



\section{Data Consistency}

Figure~\Vref{figure2}  shows the consistency of
beginning-of-quarter employment totals based on the LEHD processing with
the BLS\index{BLS} data from the Covered Worker series (CEW)\index{CEW},
for the state of Illinois.
The difference in overall levels reflects systematic differences in the
assumptions of the QWI system and the BLS CEW system. Single-quarter
discrepancies reflect data difficulties with the unemployment insurance
wage records in the historical databases. 


% \begin{figure}[htbp]
% \begin{center}
% \caption{Data consistency: California\label{figure2}}
% \centerline{\includegraphics[width=6.45in]{\mypath/GraphTable/gplotca}}
% \end{center}
% \end{figure}

%\begin{figure}[htbp]
%\begin{center}
%\caption{Data consistency: Florida\label{figure2fl}}
%\centerline{\includegraphics[width=6.45in]{\mypath/GraphTable/gplotfl}}
%\end{center}
%\end{figure}
%
\begin{figure}[htbp]
\begin{center}
\caption{Data consistency: Illinois\label{figure2}}
\centerline{\includegraphics[width=6.45in]{\mypath/GraphTable/gplotil}}
\end{center}
\end{figure}
%
%\begin{figure}[htbp]
%\begin{center}
%\caption{Data consistency: Maryland\label{figure2md}}
%\centerline{\includegraphics[width=6.45in]{\mypath/GraphTable/gplotmd}}
%\end{center}
%\end{figure}
%
%\begin{figure}[htbp]
%\begin{center}
%\caption{Data consistency: Minnesota\label{figure2mn}}
%\centerline{\includegraphics[width=6.45in]{\mypath/GraphTable/gplotmn}}
%\end{center}
%\end{figure}
%
%\begin{figure}[htbp]
%\begin{center}
%\caption{Data consistency: Texas\label{figure2tx}}
%\centerline{\includegraphics[width=6.45in]{\mypath/GraphTable/gplottx}}
%\end{center}
%\end{figure}

%%% Local Variables: 
%%% mode: latex
%%% TeX-master: "qwi-overview"
%%% End: 



\section{The Final product}
\label{sec:final}

\subsection{Summary Variable Definitions}

\subsubsection{Timing and Category Variables}

\index{Timing Variables|see{Variables, timing}} 
\index{Category Variables|see{Variables, category}} \mindex{Variables!timing}
\mindex{Variables!category}

Timing and categorical variables are used to describe the population and
time period that the content variables cover. The first such variable is 
\textsf{STATE}\mindex{STATE}, which is the two-digit FIPS%
\index{FIPS} code for the state upon which the employment dynamics estimates
are based. The next two variables (\textsf{YEAR}\mindex{YEAR} and \textsf{%
QUARTER}\mindex{QUARTER}) refer to the calendar year and quarter covered by
the content variables. The \textsf{COUNTY}\mindex{COUNTY} variable
(county-level data file) is the three-digit FIPS%
\index{FIPS} code for the county (within the state). The \textsf{%
SIC\_DIVISION}\mindex{SIC\_DIVISION} variable (sic-division-level data file)
is the one-character SIC\ (1987) major industry group. The \textsf{SEX}%
\mindex{SEX} variable indicates whether the data cover men or women. The 
\textsf{AGEGROUP}\mindex{AGEGROUP} variable indicates which of the eight age
categories the data cover.

\subsubsection{Content Variables}

\index{Content Variables|see{Variables, content}} \mindex{Variables!content}

The quarterly employment estimates for beginning of quarter employment are
contained the variable \textsf{B} and the estimates for end of quarter
employment are found in the variable \textsf{E}. Accessions are reported in
the variable \textsf{A}. New hires are in \textsf{H} and recalls are
reported in \textsf{R}. Separations are reported in the variable \textsf{S}.

Because of the confidentiality protection system used for the Employment
Dynamics Estimates, the estimate of beginning-of-quarter employment for both
sexes (\textsf{SEX}=0) and all age groups (\textsf{AGEGROUP}=0) is exactly
equal to the BLS-published Covered Employment and Wages estimate of
employment on the 12th day of the first month of the quarter for the
relevant geographic and industrial category. For example, in California the
QWI estimate for beginning-of-quarter employment in the entire state in
1999:3 is 14,440,000 (\textsf{B}=14,440,000 for \textsf{STATE}=``06'', 
\textsf{YEAR}=1999, \textsf{QUARTER}=3, \textsf{COUNTY}=``000'' (or \textsf{%
SIC\_DIVISION}=(blank)),\textsf{SEX}=0, \textsf{AGEGROUP}=0), which exactly
equals the BLS CEW estimate for month 1 in 1999:3 for the entire state,
combining all establishment sizes and all ownership categories. \ Similarly,
the QWI estimate of end-of-quarter employment is controlled for the category
both sexes (\textsf{SEX}=0) and all age groups (\textsf{AGEGROUP}=0) to
equal the BLS-published CEW estimate of employment on the 12th day of the
first month of the succeeding quarter. Again, considering California, the
QWI estimate for end-of-quarter employment in 1999:3 is 14,660,000, which
exactly equals the BLS\ CEW estimate for month 1 in 1999:4 for the entire
state (\textsf{E}=14,660,000 for \textsf{STATE}=``06'', \textsf{YEAR}=1999, 
\textsf{QUARTER}=3, \textsf{COUNTY}=``000'' (or \textsf{SIC\_DIVISION}%
=(blank)), \textsf{SEX}=0, \textsf{AGEGROUP}=0). 
%See the tables in Appendix %
%\ref{app:raking_data} for a complete list of the BLS\ series used in this
%control.

Quarterly employment estimates are also provided on a full-quarter basis.
These estimates are reported in the variable \textsf{F}. Full-quarter
accessions are reported in \textsf{FA}. Full-quarter separations are
reported in \textsf{FS}. Full-quarter new hires are in \textsf{H3. }The
raking step of the QWI confidentiality protection system used to disclosure
proof the variables \textsf{B} and \textsf{E} (and related variables) does
not affect the estimates of full-quarter employment and related flows.

Job creations and destructions are reported in the variables \textsf{JC} and 
\textsf{JD}, respectively. Net job flows are reported in the variable 
\textsf{JF}. Full-quarter job creations and destructions are reported in 
\textsf{FJC} and \textsf{FJD}, respectively. Full-quarter net job flows are
in \textsf{FJF}.

Average earnings of full-quarter employees can be found in \textsf{Z\_W3}.
Average earnings of full-quarter new hires are reported in \textsf{Z\_WH3}.

%See the table of contents at the end of this primer for a list of other
%variables and definitions in the QWI data files.

\subsubsection{Status Flag Variables}

\index{Status Flag Variables|see{Variables, status flag}} %
\mindex{Variables!status flag}

Every variable in QWI data files has an associated status flag. These
variables are called \textsf{[varname]\_status}. The status flag variables
are also shown in the contents tables at the end of this primer. The status
flag has three distinct values:

\begin{itemize}
\item[$*$] indicates significant distortion is necessary to preserve
confidentiality

\item[$d$] indicates an estimate is based on $<3$ employees in the at-risk group.

\item[$n$] indicates an estimate is not defined because no employees are in
the relevant category
\end{itemize}

%\newpage

% \subsection{Data structure\label{primer_tables}}
% 
% Appendix~\Vref{app:data_structure} describes the contents of a typical
% output statistics file, in this case for the state of Texas, and aggregated
% to the county level. A similar file exists at the industry aggregation
% level, and the same pair of files is constructed for every available state.
% 
%\input{\mypath/ca_county_v23_fuzzed.tex} \newpage \input{\mypath/ca_county_v23_fuzzed.freq.tex}
%\newpage \input{\mypath/ca_sic_division_v23_fuzzed.tex} \newpage \input{%
%\mypath/ca_sic_division_v23_fuzzed.freq.tex}

% \newpage
% 
% \subsection{Florida}
% 
% \input{\mypath/fl_county_v23_fuzzed.tex} \newpage \input{\mypath/fl_county_v23_fuzzed.freq.tex}
% \newpage \input{\mypath/fl_sic_division_v23_fuzzed.tex} \newpage \input{%
% \mypath/fl_sic_division_v23_fuzzed.freq.tex}
% 
% \newpage
% 
% \subsection{Illinois}
% 
% \input{\mypath/il_county_v23_fuzzed.tex} \newpage \input{\mypath/il_county_v23_fuzzed.freq.tex}
% \newpage \input{\mypath/il_sic_division_v23_fuzzed.tex} \newpage \input{%
% \mypath/il_sic_division_v23_fuzzed.freq.tex}
% 
% \newpage
% 
% \subsection{Maryland}
% 
% \input{\mypath/md_county_v23_fuzzed.tex} \newpage %    Generated by SAS
%    http://www.sas.com
% created by=vilhu001
% sasversion=8.2
% date=2002-05-23
% time=00:50:04
% encoding=iso-8859-1
% ====================begin of output====================
% \begin{document}

% An external file needs to be included, as specified% in latexlong.sas. This can be called sas.sty,
% in which case you want to include a line like
% \usepackage{sas}
% or it can be a simple (La)TeX file, which you 
% include by typing 
% %%
%% This is file `sas.sty',
%% generated with the docstrip utility.
%%
%% 
\NeedsTeXFormat{LaTeX2e}
\ProvidesPackage{sas}
        [2002/01/18 LEHD version 0.1
    provides definition for tables generated by SAS%
                   ]
\@ifundefined{array@processline}{\RequirePackage{array}}{}
\@ifundefined{longtable@processline}{\RequirePackage{longtable}}{}
 \def\ContentTitle{\small\it\sffamily}
 \def\Output{\small\sffamily}
 \def\HeaderEmphasis{\small\it\sffamily}
 \def\NoteContent{\small\sffamily}
 \def\FatalContent{\small\sffamily}
 \def\Graph{\small\sffamily}
 \def\WarnContentFixed{\footnotesize\tt}
 \def\NoteBanner{\small\sffamily}
 \def\DataStrong{\normalsize\bf\sffamily}
 \def\Document{\small\sffamily}
 \def\BeforeCaption{\normalsize\bf\sffamily}
 \def\ContentsDate{\small\sffamily}
 \def\Pages{\small\sffamily}
 \def\TitlesAndFooters{\footnotesize\bf\it\sffamily}
 \def\IndexProcName{\small\sffamily}
 \def\ProcTitle{\normalsize\bf\it\sffamily}
 \def\IndexAction{\small\sffamily}
 \def\Data{\small\sffamily}
 \def\Table{\small\sffamily}
 \def\FooterEmpty{\footnotesize\bf\sffamily}
 \def\SysTitleAndFooterContainer{\footnotesize\sffamily}
 \def\RowFooterEmpty{\footnotesize\bf\sffamily}
 \def\ExtendedPage{\small\it\sffamily}
 \def\FooterFixed{\footnotesize\tt}
 \def\RowFooterStrongFixed{\footnotesize\bf\tt}
 \def\RowFooterEmphasis{\footnote\it\sffamily}
 \def\ContentFolder{\small\sffamily}
 \def\Container{\small\sffamily}
 \def\Date{\small\sffamily}
 \def\RowFooterFixed{\footnotesize\tt}
 \def\Caption{\normalsize\bf\sffamily}
 \def\WarnBanner{\small\sffamily}
 \def\Frame{\small\sffamily}
 \def\HeaderStrongFixed{\footnotesize\bf\tt}
 \def\IndexTitle{\small\it\sffamily}
 \def\NoteContentFixed{\footnotesize\tt}
 \def\DataEmphasisFixed{\footnotesize\it\tt}
 \def\Note{\small\sffamily}
 \def\Byline{\normalsize\bf\sffamily}
 \def\FatalBanner{\small\sffamily}
 \def\ProcTitleFixed{\footnotesize\bf\tt}
 \def\ByContentFolder{\small\sffamily}
 \def\PagesProcLabel{\small\sffamily}
 \def\RowHeaderFixed{\footnotesize\tt}
 \def\RowFooterEmphasisFixed{\footnotesize\it\tt}
 \def\WarnContent{\small\sffamily}
 \def\DataEmpty{\small\sffamily}
 \def\Cell{\small\sffamily}
 \def\Header{\normalsize\bf\sffamily}
 \def\PageNo{\normalsize\bf\sffamily}
 \def\ContentProcLabel{\small\sffamily}
 \def\HeaderFixed{\footnotesize\tt}
 \def\PagesTitle{\small\it\sffamily}
 \def\RowHeaderEmpty{\normalsize\bf\sffamily}
 \def\PagesProcName{\small\sffamily}
 \def\Batch{\footnotesize\tt}
 \def\ContentItem{\small\sffamily}
 \def\Body{\small\sffamily}
 \def\PagesDate{\small\sffamily}
 \def\Index{\small\sffamily}
 \def\HeaderEmpty{\normalsize\bf\sffamily}
 \def\FooterStrong{\footnotesize\bf\sffamily}
 \def\FooterEmphasis{\footnotesize\it\sffamily}
 \def\ErrorContent{\small\sffamily}
 \def\DataFixed{\footnotesize\tt}
 \def\HeaderStrong{\normalsize\bf\sffamily}
 \def\GraphBackground{}
 \def\DataEmphasis{\small\it\sffamily}
 \def\TitleAndNoteContainer{\small\sffamily}
 \def\RowFooter{\footnotesize\bf\sffamily}
 \def\IndexItem{\small\sffamily}
 \def\BylineContainer{\small\sffamily}
 \def\FatalContentFixed{\footnotesize\tt}
 \def\BodyDate{\small\sffamily}
 \def\RowFooterStrong{\footnotesize\bf\sffamily}
 \def\UserText{\small\sffamily}
 \def\HeadersAndFooters{\footnotesize\bf\sffamily}
 \def\RowHeaderEmphasisFixed{\footnotesize\it\tt}
 \def\ErrorBanner{\small\sffamily}
 \def\ContentProcName{\small\sffamily}
 \def\RowHeaderStrong{\normalsize\bf\sffamily}
 \def\FooterEmphasisFixed{\footnotesize\it\tt}
 \def\Contents{\small\sffamily}
 \def\FooterStrongFixed{\footnotesize\bf\tt}
 \def\PagesItem{\small\sffamily}
 \def\RowHeader{\normalsize\bf\sffamily}
 \def\AfterCaption{\normalsize\bf\sffamily}
 \def\RowHeaderStrongFixed{\footnotesize\bf\tt}
 \def\RowHeaderEmphasis{\small\it\sffamily}
 \def\DataStrongFixed{\footnotesize\bf\tt}
 \def\Footer{\footnotesize\bf\sffamily}
 \def\FolderAction{\small\sffamily}
 \def\HeaderEmphasisFixed{\footnotesize\it\tt}
 \def\SystemTitle{\large\bf\it\sffamily}
 \def\ErrorContentFixed{\footnotesize\tt}
 \def\SystemFooter{\footnotesize\it\sffamily}
% Set cell padding 
\renewcommand{\arraystretch}{1.3}
% Headings
\newcommand{\heading}[2]{\csname#1\endcsname #2}
\newcommand{\proctitle}[2]{\csname#1\endcsname #2}
% Declare new column type
\newcolumntype{S}[2]{>{\csname#1\endcsname}#2}
% Set warning box style
\newcommand{\msg}[2]{\fbox{%
   \begin{minipage}{\textwidth}#2\end{minipage}}%
}

\begin{center}\heading{SystemTitle}{Maryland            }\end{center}
\begin{center}\heading{ProcTitle}{The FREQ Procedure}\end{center}
\begin{center}\begin{longtable}
{lrrrr}\hline % colspecs
% table_head start
   \multicolumn{5}{S{Header}{c}}{FIPS State}
\\
   \multicolumn{1}{S{Header}{l}}{state} & 
   \multicolumn{1}{S{Header}{r}}{Frequency} & 
   \multicolumn{1}{S{Header}{r}}{ Percent} & 
   \multicolumn{1}{S{Header}{r}}{Cumulative\linebreak  Frequency} & 
   \multicolumn{1}{S{Header}{r}}{Cumulative\linebreak   Percent}
\\
\hline 
\endhead % table_head end
\hline 
\multicolumn{1}{r}{(cont.)}\\
\endfoot 
\hline 
\endlastfoot % table_head end
   \multicolumn{1}{S{RowHeader}{l}}{24 MARYLAND} & 
   \multicolumn{1}{S{Data}{r}}{26271} & 
   \multicolumn{1}{S{Data}{r}}{100.00} & 
   \multicolumn{1}{S{Data}{r}}{26271} & 
   \multicolumn{1}{S{Data}{r}}{100.00}
\\
\end{longtable}
\end{center}
\begin{center}\begin{longtable}
{lrrrr}\hline % colspecs
% table_head start
   \multicolumn{5}{S{Header}{c}}{FIPS county}
\\
   \multicolumn{1}{S{Header}{l}}{county} & 
   \multicolumn{1}{S{Header}{r}}{Frequency} & 
   \multicolumn{1}{S{Header}{r}}{ Percent} & 
   \multicolumn{1}{S{Header}{r}}{Cumulative\linebreak  Frequency} & 
   \multicolumn{1}{S{Header}{r}}{Cumulative\linebreak   Percent}
\\
\hline 
\endhead % table_head end
\hline 
\multicolumn{1}{r}{(cont.)}\\
\endfoot 
\hline 
\endlastfoot % table_head end
   \multicolumn{1}{S{RowHeader}{l}}{000 MARYLAND} & 
   \multicolumn{1}{S{Data}{r}}{1053} & 
   \multicolumn{1}{S{Data}{r}}{4.01} & 
   \multicolumn{1}{S{Data}{r}}{1053} & 
   \multicolumn{1}{S{Data}{r}}{4.01}
\\
   \multicolumn{1}{S{RowHeader}{l}}{001 ALLEGANY} & 
   \multicolumn{1}{S{Data}{r}}{1053} & 
   \multicolumn{1}{S{Data}{r}}{4.01} & 
   \multicolumn{1}{S{Data}{r}}{2106} & 
   \multicolumn{1}{S{Data}{r}}{8.02}
\\
   \multicolumn{1}{S{RowHeader}{l}}{003 ANNE ARUNDEL} & 
   \multicolumn{1}{S{Data}{r}}{1053} & 
   \multicolumn{1}{S{Data}{r}}{4.01} & 
   \multicolumn{1}{S{Data}{r}}{3159} & 
   \multicolumn{1}{S{Data}{r}}{12.02}
\\
   \multicolumn{1}{S{RowHeader}{l}}{005 BALTIMORE} & 
   \multicolumn{1}{S{Data}{r}}{1053} & 
   \multicolumn{1}{S{Data}{r}}{4.01} & 
   \multicolumn{1}{S{Data}{r}}{4212} & 
   \multicolumn{1}{S{Data}{r}}{16.03}
\\
   \multicolumn{1}{S{RowHeader}{l}}{009 CALVERT} & 
   \multicolumn{1}{S{Data}{r}}{1053} & 
   \multicolumn{1}{S{Data}{r}}{4.01} & 
   \multicolumn{1}{S{Data}{r}}{5265} & 
   \multicolumn{1}{S{Data}{r}}{20.04}
\\
   \multicolumn{1}{S{RowHeader}{l}}{011 CAROLINE} & 
   \multicolumn{1}{S{Data}{r}}{1053} & 
   \multicolumn{1}{S{Data}{r}}{4.01} & 
   \multicolumn{1}{S{Data}{r}}{6318} & 
   \multicolumn{1}{S{Data}{r}}{24.05}
\\
   \multicolumn{1}{S{RowHeader}{l}}{013 CARROLL} & 
   \multicolumn{1}{S{Data}{r}}{1053} & 
   \multicolumn{1}{S{Data}{r}}{4.01} & 
   \multicolumn{1}{S{Data}{r}}{7371} & 
   \multicolumn{1}{S{Data}{r}}{28.06}
\\
   \multicolumn{1}{S{RowHeader}{l}}{015 CECIL} & 
   \multicolumn{1}{S{Data}{r}}{1053} & 
   \multicolumn{1}{S{Data}{r}}{4.01} & 
   \multicolumn{1}{S{Data}{r}}{8424} & 
   \multicolumn{1}{S{Data}{r}}{32.07}
\\
   \multicolumn{1}{S{RowHeader}{l}}{017 CHARLES} & 
   \multicolumn{1}{S{Data}{r}}{1053} & 
   \multicolumn{1}{S{Data}{r}}{4.01} & 
   \multicolumn{1}{S{Data}{r}}{9477} & 
   \multicolumn{1}{S{Data}{r}}{36.07}
\\
   \multicolumn{1}{S{RowHeader}{l}}{019 DORCHESTER} & 
   \multicolumn{1}{S{Data}{r}}{1053} & 
   \multicolumn{1}{S{Data}{r}}{4.01} & 
   \multicolumn{1}{S{Data}{r}}{10530} & 
   \multicolumn{1}{S{Data}{r}}{40.08}
\\
   \multicolumn{1}{S{RowHeader}{l}}{021 FRQWIRICK} & 
   \multicolumn{1}{S{Data}{r}}{1053} & 
   \multicolumn{1}{S{Data}{r}}{4.01} & 
   \multicolumn{1}{S{Data}{r}}{11583} & 
   \multicolumn{1}{S{Data}{r}}{44.09}
\\
   \multicolumn{1}{S{RowHeader}{l}}{023 GARRETT} & 
   \multicolumn{1}{S{Data}{r}}{1053} & 
   \multicolumn{1}{S{Data}{r}}{4.01} & 
   \multicolumn{1}{S{Data}{r}}{12636} & 
   \multicolumn{1}{S{Data}{r}}{48.10}
\\
   \multicolumn{1}{S{RowHeader}{l}}{025 HARFORD} & 
   \multicolumn{1}{S{Data}{r}}{1053} & 
   \multicolumn{1}{S{Data}{r}}{4.01} & 
   \multicolumn{1}{S{Data}{r}}{13689} & 
   \multicolumn{1}{S{Data}{r}}{52.11}
\\
   \multicolumn{1}{S{RowHeader}{l}}{027 HOWARD} & 
   \multicolumn{1}{S{Data}{r}}{1053} & 
   \multicolumn{1}{S{Data}{r}}{4.01} & 
   \multicolumn{1}{S{Data}{r}}{14742} & 
   \multicolumn{1}{S{Data}{r}}{56.12}
\\
   \multicolumn{1}{S{RowHeader}{l}}{029 KENT} & 
   \multicolumn{1}{S{Data}{r}}{1053} & 
   \multicolumn{1}{S{Data}{r}}{4.01} & 
   \multicolumn{1}{S{Data}{r}}{15795} & 
   \multicolumn{1}{S{Data}{r}}{60.12}
\\
   \multicolumn{1}{S{RowHeader}{l}}{031 MONTGOMERY} & 
   \multicolumn{1}{S{Data}{r}}{1053} & 
   \multicolumn{1}{S{Data}{r}}{4.01} & 
   \multicolumn{1}{S{Data}{r}}{16848} & 
   \multicolumn{1}{S{Data}{r}}{64.13}
\\
   \multicolumn{1}{S{RowHeader}{l}}{033 PRINCE GEORGE'S} & 
   \multicolumn{1}{S{Data}{r}}{1053} & 
   \multicolumn{1}{S{Data}{r}}{4.01} & 
   \multicolumn{1}{S{Data}{r}}{17901} & 
   \multicolumn{1}{S{Data}{r}}{68.14}
\\
   \multicolumn{1}{S{RowHeader}{l}}{035 QUEEN ANNE'S} & 
   \multicolumn{1}{S{Data}{r}}{1053} & 
   \multicolumn{1}{S{Data}{r}}{4.01} & 
   \multicolumn{1}{S{Data}{r}}{18954} & 
   \multicolumn{1}{S{Data}{r}}{72.15}
\\
   \multicolumn{1}{S{RowHeader}{l}}{037 SOMERSET} & 
   \multicolumn{1}{S{Data}{r}}{1053} & 
   \multicolumn{1}{S{Data}{r}}{4.01} & 
   \multicolumn{1}{S{Data}{r}}{20007} & 
   \multicolumn{1}{S{Data}{r}}{76.16}
\\
   \multicolumn{1}{S{RowHeader}{l}}{039 ST. MARY'S} & 
   \multicolumn{1}{S{Data}{r}}{999} & 
   \multicolumn{1}{S{Data}{r}}{3.80} & 
   \multicolumn{1}{S{Data}{r}}{21006} & 
   \multicolumn{1}{S{Data}{r}}{79.96}
\\
   \multicolumn{1}{S{RowHeader}{l}}{041 TALBOT} & 
   \multicolumn{1}{S{Data}{r}}{1053} & 
   \multicolumn{1}{S{Data}{r}}{4.01} & 
   \multicolumn{1}{S{Data}{r}}{22059} & 
   \multicolumn{1}{S{Data}{r}}{83.97}
\\
   \multicolumn{1}{S{RowHeader}{l}}{043 WASHINGTON} & 
   \multicolumn{1}{S{Data}{r}}{1053} & 
   \multicolumn{1}{S{Data}{r}}{4.01} & 
   \multicolumn{1}{S{Data}{r}}{23112} & 
   \multicolumn{1}{S{Data}{r}}{87.98}
\\
   \multicolumn{1}{S{RowHeader}{l}}{045 WICOMICO} & 
   \multicolumn{1}{S{Data}{r}}{1053} & 
   \multicolumn{1}{S{Data}{r}}{4.01} & 
   \multicolumn{1}{S{Data}{r}}{24165} & 
   \multicolumn{1}{S{Data}{r}}{91.98}
\\
   \multicolumn{1}{S{RowHeader}{l}}{047 WORCESTER} & 
   \multicolumn{1}{S{Data}{r}}{1053} & 
   \multicolumn{1}{S{Data}{r}}{4.01} & 
   \multicolumn{1}{S{Data}{r}}{25218} & 
   \multicolumn{1}{S{Data}{r}}{95.99}
\\
   \multicolumn{1}{S{RowHeader}{l}}{510 BALTIMORE CITY} & 
   \multicolumn{1}{S{Data}{r}}{1053} & 
   \multicolumn{1}{S{Data}{r}}{4.01} & 
   \multicolumn{1}{S{Data}{r}}{26271} & 
   \multicolumn{1}{S{Data}{r}}{100.00}
\\
\end{longtable}
\end{center}
\begin{center}\begin{longtable}
{rrrrr}\hline % colspecs
% table_head start
   \multicolumn{5}{S{Header}{c}}{Sex}
\\
   \multicolumn{1}{S{Header}{r}}{sex} & 
   \multicolumn{1}{S{Header}{r}}{Frequency} & 
   \multicolumn{1}{S{Header}{r}}{ Percent} & 
   \multicolumn{1}{S{Header}{r}}{Cumulative\linebreak  Frequency} & 
   \multicolumn{1}{S{Header}{r}}{Cumulative\linebreak   Percent}
\\
\hline 
\endhead % table_head end
\hline 
\multicolumn{1}{r}{(cont.)}\\
\endfoot 
\hline 
\endlastfoot % table_head end
   \multicolumn{1}{S{RowHeader}{r}}{0 : All} & 
   \multicolumn{1}{S{Data}{r}}{8757} & 
   \multicolumn{1}{S{Data}{r}}{33.33} & 
   \multicolumn{1}{S{Data}{r}}{8757} & 
   \multicolumn{1}{S{Data}{r}}{33.33}
\\
   \multicolumn{1}{S{RowHeader}{r}}{1 : Men} & 
   \multicolumn{1}{S{Data}{r}}{8757} & 
   \multicolumn{1}{S{Data}{r}}{33.33} & 
   \multicolumn{1}{S{Data}{r}}{17514} & 
   \multicolumn{1}{S{Data}{r}}{66.67}
\\
   \multicolumn{1}{S{RowHeader}{r}}{2 : Women} & 
   \multicolumn{1}{S{Data}{r}}{8757} & 
   \multicolumn{1}{S{Data}{r}}{33.33} & 
   \multicolumn{1}{S{Data}{r}}{26271} & 
   \multicolumn{1}{S{Data}{r}}{100.00}
\\
\end{longtable}
\end{center}
\begin{center}\begin{longtable}
{rrrrr}\hline % colspecs
% table_head start
   \multicolumn{5}{S{Header}{c}}{Age group}
\\
   \multicolumn{1}{S{Header}{r}}{agegroup} & 
   \multicolumn{1}{S{Header}{r}}{Frequency} & 
   \multicolumn{1}{S{Header}{r}}{ Percent} & 
   \multicolumn{1}{S{Header}{r}}{Cumulative\linebreak  Frequency} & 
   \multicolumn{1}{S{Header}{r}}{Cumulative\linebreak   Percent}
\\
\hline 
\endhead % table_head end
\hline 
\multicolumn{1}{r}{(cont.)}\\
\endfoot 
\hline 
\endlastfoot % table_head end
   \multicolumn{1}{S{RowHeader}{r}}{0 : All Ages} & 
   \multicolumn{1}{S{Data}{r}}{2919} & 
   \multicolumn{1}{S{Data}{r}}{11.11} & 
   \multicolumn{1}{S{Data}{r}}{2919} & 
   \multicolumn{1}{S{Data}{r}}{11.11}
\\
   \multicolumn{1}{S{RowHeader}{r}}{1 : 14-18} & 
   \multicolumn{1}{S{Data}{r}}{2919} & 
   \multicolumn{1}{S{Data}{r}}{11.11} & 
   \multicolumn{1}{S{Data}{r}}{5838} & 
   \multicolumn{1}{S{Data}{r}}{22.22}
\\
   \multicolumn{1}{S{RowHeader}{r}}{2 : 19-21} & 
   \multicolumn{1}{S{Data}{r}}{2919} & 
   \multicolumn{1}{S{Data}{r}}{11.11} & 
   \multicolumn{1}{S{Data}{r}}{8757} & 
   \multicolumn{1}{S{Data}{r}}{33.33}
\\
   \multicolumn{1}{S{RowHeader}{r}}{3 : 22-24} & 
   \multicolumn{1}{S{Data}{r}}{2919} & 
   \multicolumn{1}{S{Data}{r}}{11.11} & 
   \multicolumn{1}{S{Data}{r}}{11676} & 
   \multicolumn{1}{S{Data}{r}}{44.44}
\\
   \multicolumn{1}{S{RowHeader}{r}}{4 : 25-34} & 
   \multicolumn{1}{S{Data}{r}}{2919} & 
   \multicolumn{1}{S{Data}{r}}{11.11} & 
   \multicolumn{1}{S{Data}{r}}{14595} & 
   \multicolumn{1}{S{Data}{r}}{55.56}
\\
   \multicolumn{1}{S{RowHeader}{r}}{5 : 35-44} & 
   \multicolumn{1}{S{Data}{r}}{2919} & 
   \multicolumn{1}{S{Data}{r}}{11.11} & 
   \multicolumn{1}{S{Data}{r}}{17514} & 
   \multicolumn{1}{S{Data}{r}}{66.67}
\\
   \multicolumn{1}{S{RowHeader}{r}}{6 : 45-54} & 
   \multicolumn{1}{S{Data}{r}}{2919} & 
   \multicolumn{1}{S{Data}{r}}{11.11} & 
   \multicolumn{1}{S{Data}{r}}{20433} & 
   \multicolumn{1}{S{Data}{r}}{77.78}
\\
   \multicolumn{1}{S{RowHeader}{r}}{7 : 55-64} & 
   \multicolumn{1}{S{Data}{r}}{2919} & 
   \multicolumn{1}{S{Data}{r}}{11.11} & 
   \multicolumn{1}{S{Data}{r}}{23352} & 
   \multicolumn{1}{S{Data}{r}}{88.89}
\\
   \multicolumn{1}{S{RowHeader}{r}}{8 : 65+} & 
   \multicolumn{1}{S{Data}{r}}{2919} & 
   \multicolumn{1}{S{Data}{r}}{11.11} & 
   \multicolumn{1}{S{Data}{r}}{26271} & 
   \multicolumn{1}{S{Data}{r}}{100.00}
\\
\end{longtable}
\end{center}
\begin{center}\begin{longtable}
{llllll}\hline % colspecs
% table_head start
   \multicolumn{6}{S{Header}{c}}{Table of year by quarter}
\\
   \multicolumn{1}{S{Header}{c}}{year(Year)} & 
   \multicolumn{4}{S{Header}{c}}{quarter(Quarter)} & 
   \multicolumn{1}{S{Header}{r}}{Total}
\\
   \multicolumn{1}{l}{~} & 
   \multicolumn{1}{S{Header}{r}}{      1     } & 
   \multicolumn{1}{S{Header}{r}}{      2     } & 
   \multicolumn{1}{S{Header}{r}}{      3     } & 
   \multicolumn{1}{S{Header}{r}}{      4     } & 
   \multicolumn{1}{l}{~}
\\
\hline 
\endhead % table_head end
\hline 
\multicolumn{1}{r}{(cont.)}\\
\endfoot 
\hline 
\endlastfoot % table_head end
   \multicolumn{1}{S{Header}{r}}{1990        } & 
   \multicolumn{1}{S{Data}{r}}{   675} & 
   \multicolumn{1}{S{Data}{r}}{   675} & 
   \multicolumn{1}{S{Data}{r}}{   675} & 
   \multicolumn{1}{S{Data}{r}}{   675} & 
   \multicolumn{1}{S{Data}{r}}{  2700}
\\
   \multicolumn{1}{S{Header}{r}}{1991        } & 
   \multicolumn{1}{S{Data}{r}}{   675} & 
   \multicolumn{1}{S{Data}{r}}{   675} & 
   \multicolumn{1}{S{Data}{r}}{   675} & 
   \multicolumn{1}{S{Data}{r}}{   648} & 
   \multicolumn{1}{S{Data}{r}}{  2673}
\\
   \multicolumn{1}{S{Header}{r}}{1992        } & 
   \multicolumn{1}{S{Data}{r}}{   648} & 
   \multicolumn{1}{S{Data}{r}}{   675} & 
   \multicolumn{1}{S{Data}{r}}{   675} & 
   \multicolumn{1}{S{Data}{r}}{   675} & 
   \multicolumn{1}{S{Data}{r}}{  2673}
\\
   \multicolumn{1}{S{Header}{r}}{1993        } & 
   \multicolumn{1}{S{Data}{r}}{   675} & 
   \multicolumn{1}{S{Data}{r}}{   675} & 
   \multicolumn{1}{S{Data}{r}}{   675} & 
   \multicolumn{1}{S{Data}{r}}{   675} & 
   \multicolumn{1}{S{Data}{r}}{  2700}
\\
   \multicolumn{1}{S{Header}{r}}{1994        } & 
   \multicolumn{1}{S{Data}{r}}{   675} & 
   \multicolumn{1}{S{Data}{r}}{   675} & 
   \multicolumn{1}{S{Data}{r}}{   675} & 
   \multicolumn{1}{S{Data}{r}}{   675} & 
   \multicolumn{1}{S{Data}{r}}{  2700}
\\
   \multicolumn{1}{S{Header}{r}}{1995        } & 
   \multicolumn{1}{S{Data}{r}}{   675} & 
   \multicolumn{1}{S{Data}{r}}{   675} & 
   \multicolumn{1}{S{Data}{r}}{   675} & 
   \multicolumn{1}{S{Data}{r}}{   675} & 
   \multicolumn{1}{S{Data}{r}}{  2700}
\\
   \multicolumn{1}{S{Header}{r}}{1996        } & 
   \multicolumn{1}{S{Data}{r}}{   675} & 
   \multicolumn{1}{S{Data}{r}}{   675} & 
   \multicolumn{1}{S{Data}{r}}{   675} & 
   \multicolumn{1}{S{Data}{r}}{   675} & 
   \multicolumn{1}{S{Data}{r}}{  2700}
\\
   \multicolumn{1}{S{Header}{r}}{1997        } & 
   \multicolumn{1}{S{Data}{r}}{   675} & 
   \multicolumn{1}{S{Data}{r}}{   675} & 
   \multicolumn{1}{S{Data}{r}}{   675} & 
   \multicolumn{1}{S{Data}{r}}{   675} & 
   \multicolumn{1}{S{Data}{r}}{  2700}
\\
   \multicolumn{1}{S{Header}{r}}{1998        } & 
   \multicolumn{1}{S{Data}{r}}{   675} & 
   \multicolumn{1}{S{Data}{r}}{   675} & 
   \multicolumn{1}{S{Data}{r}}{   675} & 
   \multicolumn{1}{S{Data}{r}}{   675} & 
   \multicolumn{1}{S{Data}{r}}{  2700}
\\
   \multicolumn{1}{S{Header}{r}}{1999        } & 
   \multicolumn{1}{S{Data}{r}}{   675} & 
   \multicolumn{1}{S{Data}{r}}{   675} & 
   \multicolumn{1}{S{Data}{r}}{   675} & 
   \multicolumn{1}{S{Data}{r}}{     0} & 
   \multicolumn{1}{S{Data}{r}}{  2025}
\\
   \multicolumn{1}{S{Header}{l}}{Total           } & 
   \multicolumn{1}{S{Data}{r}}{   6723} & 
   \multicolumn{1}{S{Data}{r}}{   6750} & 
   \multicolumn{1}{S{Data}{r}}{   6750} & 
   \multicolumn{1}{S{Data}{r}}{   6048} & 
   \multicolumn{1}{S{Data}{r}}{  26271}
\\
\end{longtable}
\end{center}

% ====================end of output====================

% \newpage \input{\mypath/md_sic_division_v23_fuzzed.tex} \newpage %    Generated by SAS
%    http://www.sas.com
% created by=vilhu001
% sasversion=8.2
% date=2002-05-23
% time=00:50:04
% encoding=iso-8859-1
% ====================begin of output====================
% \begin{document}

% An external file needs to be included, as specified% in latexlong.sas. This can be called sas.sty,
% in which case you want to include a line like
% \usepackage{sas}
% or it can be a simple (La)TeX file, which you 
% include by typing 
% %%
%% This is file `sas.sty',
%% generated with the docstrip utility.
%%
%% 
\NeedsTeXFormat{LaTeX2e}
\ProvidesPackage{sas}
        [2002/01/18 LEHD version 0.1
    provides definition for tables generated by SAS%
                   ]
\@ifundefined{array@processline}{\RequirePackage{array}}{}
\@ifundefined{longtable@processline}{\RequirePackage{longtable}}{}
 \def\ContentTitle{\small\it\sffamily}
 \def\Output{\small\sffamily}
 \def\HeaderEmphasis{\small\it\sffamily}
 \def\NoteContent{\small\sffamily}
 \def\FatalContent{\small\sffamily}
 \def\Graph{\small\sffamily}
 \def\WarnContentFixed{\footnotesize\tt}
 \def\NoteBanner{\small\sffamily}
 \def\DataStrong{\normalsize\bf\sffamily}
 \def\Document{\small\sffamily}
 \def\BeforeCaption{\normalsize\bf\sffamily}
 \def\ContentsDate{\small\sffamily}
 \def\Pages{\small\sffamily}
 \def\TitlesAndFooters{\footnotesize\bf\it\sffamily}
 \def\IndexProcName{\small\sffamily}
 \def\ProcTitle{\normalsize\bf\it\sffamily}
 \def\IndexAction{\small\sffamily}
 \def\Data{\small\sffamily}
 \def\Table{\small\sffamily}
 \def\FooterEmpty{\footnotesize\bf\sffamily}
 \def\SysTitleAndFooterContainer{\footnotesize\sffamily}
 \def\RowFooterEmpty{\footnotesize\bf\sffamily}
 \def\ExtendedPage{\small\it\sffamily}
 \def\FooterFixed{\footnotesize\tt}
 \def\RowFooterStrongFixed{\footnotesize\bf\tt}
 \def\RowFooterEmphasis{\footnote\it\sffamily}
 \def\ContentFolder{\small\sffamily}
 \def\Container{\small\sffamily}
 \def\Date{\small\sffamily}
 \def\RowFooterFixed{\footnotesize\tt}
 \def\Caption{\normalsize\bf\sffamily}
 \def\WarnBanner{\small\sffamily}
 \def\Frame{\small\sffamily}
 \def\HeaderStrongFixed{\footnotesize\bf\tt}
 \def\IndexTitle{\small\it\sffamily}
 \def\NoteContentFixed{\footnotesize\tt}
 \def\DataEmphasisFixed{\footnotesize\it\tt}
 \def\Note{\small\sffamily}
 \def\Byline{\normalsize\bf\sffamily}
 \def\FatalBanner{\small\sffamily}
 \def\ProcTitleFixed{\footnotesize\bf\tt}
 \def\ByContentFolder{\small\sffamily}
 \def\PagesProcLabel{\small\sffamily}
 \def\RowHeaderFixed{\footnotesize\tt}
 \def\RowFooterEmphasisFixed{\footnotesize\it\tt}
 \def\WarnContent{\small\sffamily}
 \def\DataEmpty{\small\sffamily}
 \def\Cell{\small\sffamily}
 \def\Header{\normalsize\bf\sffamily}
 \def\PageNo{\normalsize\bf\sffamily}
 \def\ContentProcLabel{\small\sffamily}
 \def\HeaderFixed{\footnotesize\tt}
 \def\PagesTitle{\small\it\sffamily}
 \def\RowHeaderEmpty{\normalsize\bf\sffamily}
 \def\PagesProcName{\small\sffamily}
 \def\Batch{\footnotesize\tt}
 \def\ContentItem{\small\sffamily}
 \def\Body{\small\sffamily}
 \def\PagesDate{\small\sffamily}
 \def\Index{\small\sffamily}
 \def\HeaderEmpty{\normalsize\bf\sffamily}
 \def\FooterStrong{\footnotesize\bf\sffamily}
 \def\FooterEmphasis{\footnotesize\it\sffamily}
 \def\ErrorContent{\small\sffamily}
 \def\DataFixed{\footnotesize\tt}
 \def\HeaderStrong{\normalsize\bf\sffamily}
 \def\GraphBackground{}
 \def\DataEmphasis{\small\it\sffamily}
 \def\TitleAndNoteContainer{\small\sffamily}
 \def\RowFooter{\footnotesize\bf\sffamily}
 \def\IndexItem{\small\sffamily}
 \def\BylineContainer{\small\sffamily}
 \def\FatalContentFixed{\footnotesize\tt}
 \def\BodyDate{\small\sffamily}
 \def\RowFooterStrong{\footnotesize\bf\sffamily}
 \def\UserText{\small\sffamily}
 \def\HeadersAndFooters{\footnotesize\bf\sffamily}
 \def\RowHeaderEmphasisFixed{\footnotesize\it\tt}
 \def\ErrorBanner{\small\sffamily}
 \def\ContentProcName{\small\sffamily}
 \def\RowHeaderStrong{\normalsize\bf\sffamily}
 \def\FooterEmphasisFixed{\footnotesize\it\tt}
 \def\Contents{\small\sffamily}
 \def\FooterStrongFixed{\footnotesize\bf\tt}
 \def\PagesItem{\small\sffamily}
 \def\RowHeader{\normalsize\bf\sffamily}
 \def\AfterCaption{\normalsize\bf\sffamily}
 \def\RowHeaderStrongFixed{\footnotesize\bf\tt}
 \def\RowHeaderEmphasis{\small\it\sffamily}
 \def\DataStrongFixed{\footnotesize\bf\tt}
 \def\Footer{\footnotesize\bf\sffamily}
 \def\FolderAction{\small\sffamily}
 \def\HeaderEmphasisFixed{\footnotesize\it\tt}
 \def\SystemTitle{\large\bf\it\sffamily}
 \def\ErrorContentFixed{\footnotesize\tt}
 \def\SystemFooter{\footnotesize\it\sffamily}
% Set cell padding 
\renewcommand{\arraystretch}{1.3}
% Headings
\newcommand{\heading}[2]{\csname#1\endcsname #2}
\newcommand{\proctitle}[2]{\csname#1\endcsname #2}
% Declare new column type
\newcolumntype{S}[2]{>{\csname#1\endcsname}#2}
% Set warning box style
\newcommand{\msg}[2]{\fbox{%
   \begin{minipage}{\textwidth}#2\end{minipage}}%
}

\begin{center}\heading{SystemTitle}{Maryland            }\end{center}
\begin{center}\heading{ProcTitle}{The FREQ Procedure}\end{center}
\begin{center}\begin{longtable}
{lrrrr}\hline % colspecs
% table_head start
   \multicolumn{5}{S{Header}{c}}{FIPS State}
\\
   \multicolumn{1}{S{Header}{l}}{state} & 
   \multicolumn{1}{S{Header}{r}}{Frequency} & 
   \multicolumn{1}{S{Header}{r}}{ Percent} & 
   \multicolumn{1}{S{Header}{r}}{Cumulative\linebreak  Frequency} & 
   \multicolumn{1}{S{Header}{r}}{Cumulative\linebreak   Percent}
\\
\hline 
\endhead % table_head end
\hline 
\multicolumn{1}{r}{(cont.)}\\
\endfoot 
\hline 
\endlastfoot % table_head end
   \multicolumn{1}{S{RowHeader}{l}}{24 MARYLAND} & 
   \multicolumn{1}{S{Data}{r}}{12636} & 
   \multicolumn{1}{S{Data}{r}}{100.00} & 
   \multicolumn{1}{S{Data}{r}}{12636} & 
   \multicolumn{1}{S{Data}{r}}{100.00}
\\
\end{longtable}
\end{center}
\begin{center}\begin{longtable}
{lrrrr}\hline % colspecs
% table_head start
   \multicolumn{5}{S{Header}{c}}{SIC Division}
\\
   \multicolumn{1}{S{Header}{l}}{sic{\textunderscore}division} & 
   \multicolumn{1}{S{Header}{r}}{Frequency} & 
   \multicolumn{1}{S{Header}{r}}{ Percent} & 
   \multicolumn{1}{S{Header}{r}}{Cumulative\linebreak  Frequency} & 
   \multicolumn{1}{S{Header}{r}}{Cumulative\linebreak   Percent}
\\
\hline 
\endhead % table_head end
\hline 
\multicolumn{1}{r}{(cont.)}\\
\endfoot 
\hline 
\endlastfoot % table_head end
   \multicolumn{1}{S{RowHeader}{l}}{A Agriculture etc.} & 
   \multicolumn{1}{S{Data}{r}}{1053} & 
   \multicolumn{1}{S{Data}{r}}{9.09} & 
   \multicolumn{1}{S{Data}{r}}{1053} & 
   \multicolumn{1}{S{Data}{r}}{9.09}
\\
   \multicolumn{1}{S{RowHeader}{l}}{B Mining} & 
   \multicolumn{1}{S{Data}{r}}{1053} & 
   \multicolumn{1}{S{Data}{r}}{9.09} & 
   \multicolumn{1}{S{Data}{r}}{2106} & 
   \multicolumn{1}{S{Data}{r}}{18.18}
\\
   \multicolumn{1}{S{RowHeader}{l}}{C Construction} & 
   \multicolumn{1}{S{Data}{r}}{1053} & 
   \multicolumn{1}{S{Data}{r}}{9.09} & 
   \multicolumn{1}{S{Data}{r}}{3159} & 
   \multicolumn{1}{S{Data}{r}}{27.27}
\\
   \multicolumn{1}{S{RowHeader}{l}}{D Manufacturing} & 
   \multicolumn{1}{S{Data}{r}}{1053} & 
   \multicolumn{1}{S{Data}{r}}{9.09} & 
   \multicolumn{1}{S{Data}{r}}{4212} & 
   \multicolumn{1}{S{Data}{r}}{36.36}
\\
   \multicolumn{1}{S{RowHeader}{l}}{E Trans. \& Utilities} & 
   \multicolumn{1}{S{Data}{r}}{1053} & 
   \multicolumn{1}{S{Data}{r}}{9.09} & 
   \multicolumn{1}{S{Data}{r}}{5265} & 
   \multicolumn{1}{S{Data}{r}}{45.45}
\\
   \multicolumn{1}{S{RowHeader}{l}}{F Wholesale trade} & 
   \multicolumn{1}{S{Data}{r}}{1053} & 
   \multicolumn{1}{S{Data}{r}}{9.09} & 
   \multicolumn{1}{S{Data}{r}}{6318} & 
   \multicolumn{1}{S{Data}{r}}{54.55}
\\
   \multicolumn{1}{S{RowHeader}{l}}{G Retail Trade} & 
   \multicolumn{1}{S{Data}{r}}{1053} & 
   \multicolumn{1}{S{Data}{r}}{9.09} & 
   \multicolumn{1}{S{Data}{r}}{7371} & 
   \multicolumn{1}{S{Data}{r}}{63.64}
\\
   \multicolumn{1}{S{RowHeader}{l}}{H FIRE} & 
   \multicolumn{1}{S{Data}{r}}{1053} & 
   \multicolumn{1}{S{Data}{r}}{9.09} & 
   \multicolumn{1}{S{Data}{r}}{8424} & 
   \multicolumn{1}{S{Data}{r}}{72.73}
\\
   \multicolumn{1}{S{RowHeader}{l}}{I Services} & 
   \multicolumn{1}{S{Data}{r}}{1053} & 
   \multicolumn{1}{S{Data}{r}}{9.09} & 
   \multicolumn{1}{S{Data}{r}}{9477} & 
   \multicolumn{1}{S{Data}{r}}{81.82}
\\
   \multicolumn{1}{S{RowHeader}{l}}{J Public Admin.} & 
   \multicolumn{1}{S{Data}{r}}{1053} & 
   \multicolumn{1}{S{Data}{r}}{9.09} & 
   \multicolumn{1}{S{Data}{r}}{10530} & 
   \multicolumn{1}{S{Data}{r}}{90.91}
\\
   \multicolumn{1}{S{RowHeader}{l}}{Other} & 
   \multicolumn{1}{S{Data}{r}}{1053} & 
   \multicolumn{1}{S{Data}{r}}{9.09} & 
   \multicolumn{1}{S{Data}{r}}{11583} & 
   \multicolumn{1}{S{Data}{r}}{100.00}
\\
\end{longtable}
\end{center}
\begin{center}\heading{ProcTitle}{Frequency Missing = 1053}\end{center}
\begin{center}\begin{longtable}
{rrrrr}\hline % colspecs
% table_head start
   \multicolumn{5}{S{Header}{c}}{Sex}
\\
   \multicolumn{1}{S{Header}{r}}{sex} & 
   \multicolumn{1}{S{Header}{r}}{Frequency} & 
   \multicolumn{1}{S{Header}{r}}{ Percent} & 
   \multicolumn{1}{S{Header}{r}}{Cumulative\linebreak  Frequency} & 
   \multicolumn{1}{S{Header}{r}}{Cumulative\linebreak   Percent}
\\
\hline 
\endhead % table_head end
\hline 
\multicolumn{1}{r}{(cont.)}\\
\endfoot 
\hline 
\endlastfoot % table_head end
   \multicolumn{1}{S{RowHeader}{r}}{0 : All} & 
   \multicolumn{1}{S{Data}{r}}{4212} & 
   \multicolumn{1}{S{Data}{r}}{33.33} & 
   \multicolumn{1}{S{Data}{r}}{4212} & 
   \multicolumn{1}{S{Data}{r}}{33.33}
\\
   \multicolumn{1}{S{RowHeader}{r}}{1 : Men} & 
   \multicolumn{1}{S{Data}{r}}{4212} & 
   \multicolumn{1}{S{Data}{r}}{33.33} & 
   \multicolumn{1}{S{Data}{r}}{8424} & 
   \multicolumn{1}{S{Data}{r}}{66.67}
\\
   \multicolumn{1}{S{RowHeader}{r}}{2 : Women} & 
   \multicolumn{1}{S{Data}{r}}{4212} & 
   \multicolumn{1}{S{Data}{r}}{33.33} & 
   \multicolumn{1}{S{Data}{r}}{12636} & 
   \multicolumn{1}{S{Data}{r}}{100.00}
\\
\end{longtable}
\end{center}
\begin{center}\begin{longtable}
{rrrrr}\hline % colspecs
% table_head start
   \multicolumn{5}{S{Header}{c}}{Age group}
\\
   \multicolumn{1}{S{Header}{r}}{agegroup} & 
   \multicolumn{1}{S{Header}{r}}{Frequency} & 
   \multicolumn{1}{S{Header}{r}}{ Percent} & 
   \multicolumn{1}{S{Header}{r}}{Cumulative\linebreak  Frequency} & 
   \multicolumn{1}{S{Header}{r}}{Cumulative\linebreak   Percent}
\\
\hline 
\endhead % table_head end
\hline 
\multicolumn{1}{r}{(cont.)}\\
\endfoot 
\hline 
\endlastfoot % table_head end
   \multicolumn{1}{S{RowHeader}{r}}{0 : All Ages} & 
   \multicolumn{1}{S{Data}{r}}{1404} & 
   \multicolumn{1}{S{Data}{r}}{11.11} & 
   \multicolumn{1}{S{Data}{r}}{1404} & 
   \multicolumn{1}{S{Data}{r}}{11.11}
\\
   \multicolumn{1}{S{RowHeader}{r}}{1 : 14-18} & 
   \multicolumn{1}{S{Data}{r}}{1404} & 
   \multicolumn{1}{S{Data}{r}}{11.11} & 
   \multicolumn{1}{S{Data}{r}}{2808} & 
   \multicolumn{1}{S{Data}{r}}{22.22}
\\
   \multicolumn{1}{S{RowHeader}{r}}{2 : 19-21} & 
   \multicolumn{1}{S{Data}{r}}{1404} & 
   \multicolumn{1}{S{Data}{r}}{11.11} & 
   \multicolumn{1}{S{Data}{r}}{4212} & 
   \multicolumn{1}{S{Data}{r}}{33.33}
\\
   \multicolumn{1}{S{RowHeader}{r}}{3 : 22-24} & 
   \multicolumn{1}{S{Data}{r}}{1404} & 
   \multicolumn{1}{S{Data}{r}}{11.11} & 
   \multicolumn{1}{S{Data}{r}}{5616} & 
   \multicolumn{1}{S{Data}{r}}{44.44}
\\
   \multicolumn{1}{S{RowHeader}{r}}{4 : 25-34} & 
   \multicolumn{1}{S{Data}{r}}{1404} & 
   \multicolumn{1}{S{Data}{r}}{11.11} & 
   \multicolumn{1}{S{Data}{r}}{7020} & 
   \multicolumn{1}{S{Data}{r}}{55.56}
\\
   \multicolumn{1}{S{RowHeader}{r}}{5 : 35-44} & 
   \multicolumn{1}{S{Data}{r}}{1404} & 
   \multicolumn{1}{S{Data}{r}}{11.11} & 
   \multicolumn{1}{S{Data}{r}}{8424} & 
   \multicolumn{1}{S{Data}{r}}{66.67}
\\
   \multicolumn{1}{S{RowHeader}{r}}{6 : 45-54} & 
   \multicolumn{1}{S{Data}{r}}{1404} & 
   \multicolumn{1}{S{Data}{r}}{11.11} & 
   \multicolumn{1}{S{Data}{r}}{9828} & 
   \multicolumn{1}{S{Data}{r}}{77.78}
\\
   \multicolumn{1}{S{RowHeader}{r}}{7 : 55-64} & 
   \multicolumn{1}{S{Data}{r}}{1404} & 
   \multicolumn{1}{S{Data}{r}}{11.11} & 
   \multicolumn{1}{S{Data}{r}}{11232} & 
   \multicolumn{1}{S{Data}{r}}{88.89}
\\
   \multicolumn{1}{S{RowHeader}{r}}{8 : 65+} & 
   \multicolumn{1}{S{Data}{r}}{1404} & 
   \multicolumn{1}{S{Data}{r}}{11.11} & 
   \multicolumn{1}{S{Data}{r}}{12636} & 
   \multicolumn{1}{S{Data}{r}}{100.00}
\\
\end{longtable}
\end{center}
\begin{center}\begin{longtable}
{llllll}\hline % colspecs
% table_head start
   \multicolumn{6}{S{Header}{c}}{Table of year by quarter}
\\
   \multicolumn{1}{S{Header}{c}}{year(Year)} & 
   \multicolumn{4}{S{Header}{c}}{quarter(Quarter)} & 
   \multicolumn{1}{S{Header}{r}}{Total}
\\
   \multicolumn{1}{l}{~} & 
   \multicolumn{1}{S{Header}{r}}{      1     } & 
   \multicolumn{1}{S{Header}{r}}{      2     } & 
   \multicolumn{1}{S{Header}{r}}{      3     } & 
   \multicolumn{1}{S{Header}{r}}{      4     } & 
   \multicolumn{1}{l}{~}
\\
\hline 
\endhead % table_head end
\hline 
\multicolumn{1}{r}{(cont.)}\\
\endfoot 
\hline 
\endlastfoot % table_head end
   \multicolumn{1}{S{Header}{r}}{1990        } & 
   \multicolumn{1}{S{Data}{r}}{   324} & 
   \multicolumn{1}{S{Data}{r}}{   324} & 
   \multicolumn{1}{S{Data}{r}}{   324} & 
   \multicolumn{1}{S{Data}{r}}{   324} & 
   \multicolumn{1}{S{Data}{r}}{  1296}
\\
   \multicolumn{1}{S{Header}{r}}{1991        } & 
   \multicolumn{1}{S{Data}{r}}{   324} & 
   \multicolumn{1}{S{Data}{r}}{   324} & 
   \multicolumn{1}{S{Data}{r}}{   324} & 
   \multicolumn{1}{S{Data}{r}}{   324} & 
   \multicolumn{1}{S{Data}{r}}{  1296}
\\
   \multicolumn{1}{S{Header}{r}}{1992        } & 
   \multicolumn{1}{S{Data}{r}}{   324} & 
   \multicolumn{1}{S{Data}{r}}{   324} & 
   \multicolumn{1}{S{Data}{r}}{   324} & 
   \multicolumn{1}{S{Data}{r}}{   324} & 
   \multicolumn{1}{S{Data}{r}}{  1296}
\\
   \multicolumn{1}{S{Header}{r}}{1993        } & 
   \multicolumn{1}{S{Data}{r}}{   324} & 
   \multicolumn{1}{S{Data}{r}}{   324} & 
   \multicolumn{1}{S{Data}{r}}{   324} & 
   \multicolumn{1}{S{Data}{r}}{   324} & 
   \multicolumn{1}{S{Data}{r}}{  1296}
\\
   \multicolumn{1}{S{Header}{r}}{1994        } & 
   \multicolumn{1}{S{Data}{r}}{   324} & 
   \multicolumn{1}{S{Data}{r}}{   324} & 
   \multicolumn{1}{S{Data}{r}}{   324} & 
   \multicolumn{1}{S{Data}{r}}{   324} & 
   \multicolumn{1}{S{Data}{r}}{  1296}
\\
   \multicolumn{1}{S{Header}{r}}{1995        } & 
   \multicolumn{1}{S{Data}{r}}{   324} & 
   \multicolumn{1}{S{Data}{r}}{   324} & 
   \multicolumn{1}{S{Data}{r}}{   324} & 
   \multicolumn{1}{S{Data}{r}}{   324} & 
   \multicolumn{1}{S{Data}{r}}{  1296}
\\
   \multicolumn{1}{S{Header}{r}}{1996        } & 
   \multicolumn{1}{S{Data}{r}}{   324} & 
   \multicolumn{1}{S{Data}{r}}{   324} & 
   \multicolumn{1}{S{Data}{r}}{   324} & 
   \multicolumn{1}{S{Data}{r}}{   324} & 
   \multicolumn{1}{S{Data}{r}}{  1296}
\\
   \multicolumn{1}{S{Header}{r}}{1997        } & 
   \multicolumn{1}{S{Data}{r}}{   324} & 
   \multicolumn{1}{S{Data}{r}}{   324} & 
   \multicolumn{1}{S{Data}{r}}{   324} & 
   \multicolumn{1}{S{Data}{r}}{   324} & 
   \multicolumn{1}{S{Data}{r}}{  1296}
\\
   \multicolumn{1}{S{Header}{r}}{1998        } & 
   \multicolumn{1}{S{Data}{r}}{   324} & 
   \multicolumn{1}{S{Data}{r}}{   324} & 
   \multicolumn{1}{S{Data}{r}}{   324} & 
   \multicolumn{1}{S{Data}{r}}{   324} & 
   \multicolumn{1}{S{Data}{r}}{  1296}
\\
   \multicolumn{1}{S{Header}{r}}{1999        } & 
   \multicolumn{1}{S{Data}{r}}{   324} & 
   \multicolumn{1}{S{Data}{r}}{   324} & 
   \multicolumn{1}{S{Data}{r}}{   324} & 
   \multicolumn{1}{S{Data}{r}}{     0} & 
   \multicolumn{1}{S{Data}{r}}{   972}
\\
   \multicolumn{1}{S{Header}{l}}{Total           } & 
   \multicolumn{1}{S{Data}{r}}{   3240} & 
   \multicolumn{1}{S{Data}{r}}{   3240} & 
   \multicolumn{1}{S{Data}{r}}{   3240} & 
   \multicolumn{1}{S{Data}{r}}{   2916} & 
   \multicolumn{1}{S{Data}{r}}{  12636}
\\
\end{longtable}
\end{center}

% ====================end of output====================

% 
% \newpage
% 
% \subsection{Minnesota}
% 
% \input{\mypath/mn_county_v23_fuzzed.tex} \newpage %    Generated by SAS
%    http://www.sas.com
% created by=vilhu001
% sasversion=8.2
% date=2002-05-23
% time=00:50:05
% encoding=iso-8859-1
% ====================begin of output====================
% \begin{document}

% An external file needs to be included, as specified% in latexlong.sas. This can be called sas.sty,
% in which case you want to include a line like
% \usepackage{sas}
% or it can be a simple (La)TeX file, which you 
% include by typing 
% %%
%% This is file `sas.sty',
%% generated with the docstrip utility.
%%
%% 
\NeedsTeXFormat{LaTeX2e}
\ProvidesPackage{sas}
        [2002/01/18 LEHD version 0.1
    provides definition for tables generated by SAS%
                   ]
\@ifundefined{array@processline}{\RequirePackage{array}}{}
\@ifundefined{longtable@processline}{\RequirePackage{longtable}}{}
 \def\ContentTitle{\small\it\sffamily}
 \def\Output{\small\sffamily}
 \def\HeaderEmphasis{\small\it\sffamily}
 \def\NoteContent{\small\sffamily}
 \def\FatalContent{\small\sffamily}
 \def\Graph{\small\sffamily}
 \def\WarnContentFixed{\footnotesize\tt}
 \def\NoteBanner{\small\sffamily}
 \def\DataStrong{\normalsize\bf\sffamily}
 \def\Document{\small\sffamily}
 \def\BeforeCaption{\normalsize\bf\sffamily}
 \def\ContentsDate{\small\sffamily}
 \def\Pages{\small\sffamily}
 \def\TitlesAndFooters{\footnotesize\bf\it\sffamily}
 \def\IndexProcName{\small\sffamily}
 \def\ProcTitle{\normalsize\bf\it\sffamily}
 \def\IndexAction{\small\sffamily}
 \def\Data{\small\sffamily}
 \def\Table{\small\sffamily}
 \def\FooterEmpty{\footnotesize\bf\sffamily}
 \def\SysTitleAndFooterContainer{\footnotesize\sffamily}
 \def\RowFooterEmpty{\footnotesize\bf\sffamily}
 \def\ExtendedPage{\small\it\sffamily}
 \def\FooterFixed{\footnotesize\tt}
 \def\RowFooterStrongFixed{\footnotesize\bf\tt}
 \def\RowFooterEmphasis{\footnote\it\sffamily}
 \def\ContentFolder{\small\sffamily}
 \def\Container{\small\sffamily}
 \def\Date{\small\sffamily}
 \def\RowFooterFixed{\footnotesize\tt}
 \def\Caption{\normalsize\bf\sffamily}
 \def\WarnBanner{\small\sffamily}
 \def\Frame{\small\sffamily}
 \def\HeaderStrongFixed{\footnotesize\bf\tt}
 \def\IndexTitle{\small\it\sffamily}
 \def\NoteContentFixed{\footnotesize\tt}
 \def\DataEmphasisFixed{\footnotesize\it\tt}
 \def\Note{\small\sffamily}
 \def\Byline{\normalsize\bf\sffamily}
 \def\FatalBanner{\small\sffamily}
 \def\ProcTitleFixed{\footnotesize\bf\tt}
 \def\ByContentFolder{\small\sffamily}
 \def\PagesProcLabel{\small\sffamily}
 \def\RowHeaderFixed{\footnotesize\tt}
 \def\RowFooterEmphasisFixed{\footnotesize\it\tt}
 \def\WarnContent{\small\sffamily}
 \def\DataEmpty{\small\sffamily}
 \def\Cell{\small\sffamily}
 \def\Header{\normalsize\bf\sffamily}
 \def\PageNo{\normalsize\bf\sffamily}
 \def\ContentProcLabel{\small\sffamily}
 \def\HeaderFixed{\footnotesize\tt}
 \def\PagesTitle{\small\it\sffamily}
 \def\RowHeaderEmpty{\normalsize\bf\sffamily}
 \def\PagesProcName{\small\sffamily}
 \def\Batch{\footnotesize\tt}
 \def\ContentItem{\small\sffamily}
 \def\Body{\small\sffamily}
 \def\PagesDate{\small\sffamily}
 \def\Index{\small\sffamily}
 \def\HeaderEmpty{\normalsize\bf\sffamily}
 \def\FooterStrong{\footnotesize\bf\sffamily}
 \def\FooterEmphasis{\footnotesize\it\sffamily}
 \def\ErrorContent{\small\sffamily}
 \def\DataFixed{\footnotesize\tt}
 \def\HeaderStrong{\normalsize\bf\sffamily}
 \def\GraphBackground{}
 \def\DataEmphasis{\small\it\sffamily}
 \def\TitleAndNoteContainer{\small\sffamily}
 \def\RowFooter{\footnotesize\bf\sffamily}
 \def\IndexItem{\small\sffamily}
 \def\BylineContainer{\small\sffamily}
 \def\FatalContentFixed{\footnotesize\tt}
 \def\BodyDate{\small\sffamily}
 \def\RowFooterStrong{\footnotesize\bf\sffamily}
 \def\UserText{\small\sffamily}
 \def\HeadersAndFooters{\footnotesize\bf\sffamily}
 \def\RowHeaderEmphasisFixed{\footnotesize\it\tt}
 \def\ErrorBanner{\small\sffamily}
 \def\ContentProcName{\small\sffamily}
 \def\RowHeaderStrong{\normalsize\bf\sffamily}
 \def\FooterEmphasisFixed{\footnotesize\it\tt}
 \def\Contents{\small\sffamily}
 \def\FooterStrongFixed{\footnotesize\bf\tt}
 \def\PagesItem{\small\sffamily}
 \def\RowHeader{\normalsize\bf\sffamily}
 \def\AfterCaption{\normalsize\bf\sffamily}
 \def\RowHeaderStrongFixed{\footnotesize\bf\tt}
 \def\RowHeaderEmphasis{\small\it\sffamily}
 \def\DataStrongFixed{\footnotesize\bf\tt}
 \def\Footer{\footnotesize\bf\sffamily}
 \def\FolderAction{\small\sffamily}
 \def\HeaderEmphasisFixed{\footnotesize\it\tt}
 \def\SystemTitle{\large\bf\it\sffamily}
 \def\ErrorContentFixed{\footnotesize\tt}
 \def\SystemFooter{\footnotesize\it\sffamily}
% Set cell padding 
\renewcommand{\arraystretch}{1.3}
% Headings
\newcommand{\heading}[2]{\csname#1\endcsname #2}
\newcommand{\proctitle}[2]{\csname#1\endcsname #2}
% Declare new column type
\newcolumntype{S}[2]{>{\csname#1\endcsname}#2}
% Set warning box style
\newcommand{\msg}[2]{\fbox{%
   \begin{minipage}{\textwidth}#2\end{minipage}}%
}

\begin{center}\heading{SystemTitle}{Minnesota           }\end{center}
\begin{center}\heading{ProcTitle}{The FREQ Procedure}\end{center}
\begin{center}\begin{longtable}
{lrrrr}\hline % colspecs
% table_head start
   \multicolumn{5}{S{Header}{c}}{FIPS State}
\\
   \multicolumn{1}{S{Header}{l}}{state} & 
   \multicolumn{1}{S{Header}{r}}{Frequency} & 
   \multicolumn{1}{S{Header}{r}}{ Percent} & 
   \multicolumn{1}{S{Header}{r}}{Cumulative\linebreak  Frequency} & 
   \multicolumn{1}{S{Header}{r}}{Cumulative\linebreak   Percent}
\\
\hline 
\endhead % table_head end
\hline 
\multicolumn{1}{r}{(cont.)}\\
\endfoot 
\hline 
\endlastfoot % table_head end
   \multicolumn{1}{S{RowHeader}{l}}{27 MINNESOTA} & 
   \multicolumn{1}{S{Data}{r}}{61452} & 
   \multicolumn{1}{S{Data}{r}}{100.00} & 
   \multicolumn{1}{S{Data}{r}}{61452} & 
   \multicolumn{1}{S{Data}{r}}{100.00}
\\
\end{longtable}
\end{center}
\begin{center}\begin{longtable}
{lrrrr}\hline % colspecs
% table_head start
   \multicolumn{5}{S{Header}{c}}{FIPS county}
\\
   \multicolumn{1}{S{Header}{l}}{county} & 
   \multicolumn{1}{S{Header}{r}}{Frequency} & 
   \multicolumn{1}{S{Header}{r}}{ Percent} & 
   \multicolumn{1}{S{Header}{r}}{Cumulative\linebreak  Frequency} & 
   \multicolumn{1}{S{Header}{r}}{Cumulative\linebreak   Percent}
\\
\hline 
\endhead % table_head end
\hline 
\multicolumn{1}{r}{(cont.)}\\
\endfoot 
\hline 
\endlastfoot % table_head end
   \multicolumn{1}{S{RowHeader}{l}}{000 MINNESOTA} & 
   \multicolumn{1}{S{Data}{r}}{702} & 
   \multicolumn{1}{S{Data}{r}}{1.14} & 
   \multicolumn{1}{S{Data}{r}}{702} & 
   \multicolumn{1}{S{Data}{r}}{1.14}
\\
   \multicolumn{1}{S{RowHeader}{l}}{001 AITKIN} & 
   \multicolumn{1}{S{Data}{r}}{702} & 
   \multicolumn{1}{S{Data}{r}}{1.14} & 
   \multicolumn{1}{S{Data}{r}}{1404} & 
   \multicolumn{1}{S{Data}{r}}{2.28}
\\
   \multicolumn{1}{S{RowHeader}{l}}{003 ANOKA} & 
   \multicolumn{1}{S{Data}{r}}{702} & 
   \multicolumn{1}{S{Data}{r}}{1.14} & 
   \multicolumn{1}{S{Data}{r}}{2106} & 
   \multicolumn{1}{S{Data}{r}}{3.43}
\\
   \multicolumn{1}{S{RowHeader}{l}}{005 BECKER} & 
   \multicolumn{1}{S{Data}{r}}{702} & 
   \multicolumn{1}{S{Data}{r}}{1.14} & 
   \multicolumn{1}{S{Data}{r}}{2808} & 
   \multicolumn{1}{S{Data}{r}}{4.57}
\\
   \multicolumn{1}{S{RowHeader}{l}}{007 BELTRAMI} & 
   \multicolumn{1}{S{Data}{r}}{702} & 
   \multicolumn{1}{S{Data}{r}}{1.14} & 
   \multicolumn{1}{S{Data}{r}}{3510} & 
   \multicolumn{1}{S{Data}{r}}{5.71}
\\
   \multicolumn{1}{S{RowHeader}{l}}{009 BENTON} & 
   \multicolumn{1}{S{Data}{r}}{702} & 
   \multicolumn{1}{S{Data}{r}}{1.14} & 
   \multicolumn{1}{S{Data}{r}}{4212} & 
   \multicolumn{1}{S{Data}{r}}{6.85}
\\
   \multicolumn{1}{S{RowHeader}{l}}{011 BIG STONE} & 
   \multicolumn{1}{S{Data}{r}}{702} & 
   \multicolumn{1}{S{Data}{r}}{1.14} & 
   \multicolumn{1}{S{Data}{r}}{4914} & 
   \multicolumn{1}{S{Data}{r}}{8.00}
\\
   \multicolumn{1}{S{RowHeader}{l}}{013 BLUE EARTH} & 
   \multicolumn{1}{S{Data}{r}}{702} & 
   \multicolumn{1}{S{Data}{r}}{1.14} & 
   \multicolumn{1}{S{Data}{r}}{5616} & 
   \multicolumn{1}{S{Data}{r}}{9.14}
\\
   \multicolumn{1}{S{RowHeader}{l}}{015 BROWN} & 
   \multicolumn{1}{S{Data}{r}}{702} & 
   \multicolumn{1}{S{Data}{r}}{1.14} & 
   \multicolumn{1}{S{Data}{r}}{6318} & 
   \multicolumn{1}{S{Data}{r}}{10.28}
\\
   \multicolumn{1}{S{RowHeader}{l}}{017 CARLTON} & 
   \multicolumn{1}{S{Data}{r}}{702} & 
   \multicolumn{1}{S{Data}{r}}{1.14} & 
   \multicolumn{1}{S{Data}{r}}{7020} & 
   \multicolumn{1}{S{Data}{r}}{11.42}
\\
   \multicolumn{1}{S{RowHeader}{l}}{019 CARVER} & 
   \multicolumn{1}{S{Data}{r}}{702} & 
   \multicolumn{1}{S{Data}{r}}{1.14} & 
   \multicolumn{1}{S{Data}{r}}{7722} & 
   \multicolumn{1}{S{Data}{r}}{12.57}
\\
   \multicolumn{1}{S{RowHeader}{l}}{021 CASS} & 
   \multicolumn{1}{S{Data}{r}}{702} & 
   \multicolumn{1}{S{Data}{r}}{1.14} & 
   \multicolumn{1}{S{Data}{r}}{8424} & 
   \multicolumn{1}{S{Data}{r}}{13.71}
\\
   \multicolumn{1}{S{RowHeader}{l}}{023 CHIPPEWA} & 
   \multicolumn{1}{S{Data}{r}}{702} & 
   \multicolumn{1}{S{Data}{r}}{1.14} & 
   \multicolumn{1}{S{Data}{r}}{9126} & 
   \multicolumn{1}{S{Data}{r}}{14.85}
\\
   \multicolumn{1}{S{RowHeader}{l}}{025 CHISAGO} & 
   \multicolumn{1}{S{Data}{r}}{702} & 
   \multicolumn{1}{S{Data}{r}}{1.14} & 
   \multicolumn{1}{S{Data}{r}}{9828} & 
   \multicolumn{1}{S{Data}{r}}{15.99}
\\
   \multicolumn{1}{S{RowHeader}{l}}{027 CLAY} & 
   \multicolumn{1}{S{Data}{r}}{702} & 
   \multicolumn{1}{S{Data}{r}}{1.14} & 
   \multicolumn{1}{S{Data}{r}}{10530} & 
   \multicolumn{1}{S{Data}{r}}{17.14}
\\
   \multicolumn{1}{S{RowHeader}{l}}{029 CLEARWATER} & 
   \multicolumn{1}{S{Data}{r}}{702} & 
   \multicolumn{1}{S{Data}{r}}{1.14} & 
   \multicolumn{1}{S{Data}{r}}{11232} & 
   \multicolumn{1}{S{Data}{r}}{18.28}
\\
   \multicolumn{1}{S{RowHeader}{l}}{031 COOK} & 
   \multicolumn{1}{S{Data}{r}}{702} & 
   \multicolumn{1}{S{Data}{r}}{1.14} & 
   \multicolumn{1}{S{Data}{r}}{11934} & 
   \multicolumn{1}{S{Data}{r}}{19.42}
\\
   \multicolumn{1}{S{RowHeader}{l}}{033 COTTONWOOD} & 
   \multicolumn{1}{S{Data}{r}}{702} & 
   \multicolumn{1}{S{Data}{r}}{1.14} & 
   \multicolumn{1}{S{Data}{r}}{12636} & 
   \multicolumn{1}{S{Data}{r}}{20.56}
\\
   \multicolumn{1}{S{RowHeader}{l}}{035 CROW WING} & 
   \multicolumn{1}{S{Data}{r}}{702} & 
   \multicolumn{1}{S{Data}{r}}{1.14} & 
   \multicolumn{1}{S{Data}{r}}{13338} & 
   \multicolumn{1}{S{Data}{r}}{21.70}
\\
   \multicolumn{1}{S{RowHeader}{l}}{037 DAKOTA} & 
   \multicolumn{1}{S{Data}{r}}{702} & 
   \multicolumn{1}{S{Data}{r}}{1.14} & 
   \multicolumn{1}{S{Data}{r}}{14040} & 
   \multicolumn{1}{S{Data}{r}}{22.85}
\\
   \multicolumn{1}{S{RowHeader}{l}}{039 DODGE} & 
   \multicolumn{1}{S{Data}{r}}{702} & 
   \multicolumn{1}{S{Data}{r}}{1.14} & 
   \multicolumn{1}{S{Data}{r}}{14742} & 
   \multicolumn{1}{S{Data}{r}}{23.99}
\\
   \multicolumn{1}{S{RowHeader}{l}}{041 DOUGLAS} & 
   \multicolumn{1}{S{Data}{r}}{702} & 
   \multicolumn{1}{S{Data}{r}}{1.14} & 
   \multicolumn{1}{S{Data}{r}}{15444} & 
   \multicolumn{1}{S{Data}{r}}{25.13}
\\
   \multicolumn{1}{S{RowHeader}{l}}{043 FARIBAULT} & 
   \multicolumn{1}{S{Data}{r}}{702} & 
   \multicolumn{1}{S{Data}{r}}{1.14} & 
   \multicolumn{1}{S{Data}{r}}{16146} & 
   \multicolumn{1}{S{Data}{r}}{26.27}
\\
   \multicolumn{1}{S{RowHeader}{l}}{045 FILLMORE} & 
   \multicolumn{1}{S{Data}{r}}{702} & 
   \multicolumn{1}{S{Data}{r}}{1.14} & 
   \multicolumn{1}{S{Data}{r}}{16848} & 
   \multicolumn{1}{S{Data}{r}}{27.42}
\\
   \multicolumn{1}{S{RowHeader}{l}}{047 FREEBORN} & 
   \multicolumn{1}{S{Data}{r}}{702} & 
   \multicolumn{1}{S{Data}{r}}{1.14} & 
   \multicolumn{1}{S{Data}{r}}{17550} & 
   \multicolumn{1}{S{Data}{r}}{28.56}
\\
   \multicolumn{1}{S{RowHeader}{l}}{049 GOODHUE} & 
   \multicolumn{1}{S{Data}{r}}{702} & 
   \multicolumn{1}{S{Data}{r}}{1.14} & 
   \multicolumn{1}{S{Data}{r}}{18252} & 
   \multicolumn{1}{S{Data}{r}}{29.70}
\\
   \multicolumn{1}{S{RowHeader}{l}}{051 GRANT} & 
   \multicolumn{1}{S{Data}{r}}{702} & 
   \multicolumn{1}{S{Data}{r}}{1.14} & 
   \multicolumn{1}{S{Data}{r}}{18954} & 
   \multicolumn{1}{S{Data}{r}}{30.84}
\\
   \multicolumn{1}{S{RowHeader}{l}}{053 HENNEPIN} & 
   \multicolumn{1}{S{Data}{r}}{702} & 
   \multicolumn{1}{S{Data}{r}}{1.14} & 
   \multicolumn{1}{S{Data}{r}}{19656} & 
   \multicolumn{1}{S{Data}{r}}{31.99}
\\
   \multicolumn{1}{S{RowHeader}{l}}{055 HOUSTON} & 
   \multicolumn{1}{S{Data}{r}}{702} & 
   \multicolumn{1}{S{Data}{r}}{1.14} & 
   \multicolumn{1}{S{Data}{r}}{20358} & 
   \multicolumn{1}{S{Data}{r}}{33.13}
\\
   \multicolumn{1}{S{RowHeader}{l}}{057 HUBBARD} & 
   \multicolumn{1}{S{Data}{r}}{702} & 
   \multicolumn{1}{S{Data}{r}}{1.14} & 
   \multicolumn{1}{S{Data}{r}}{21060} & 
   \multicolumn{1}{S{Data}{r}}{34.27}
\\
   \multicolumn{1}{S{RowHeader}{l}}{059 ISANTI} & 
   \multicolumn{1}{S{Data}{r}}{702} & 
   \multicolumn{1}{S{Data}{r}}{1.14} & 
   \multicolumn{1}{S{Data}{r}}{21762} & 
   \multicolumn{1}{S{Data}{r}}{35.41}
\\
   \multicolumn{1}{S{RowHeader}{l}}{061 ITASCA} & 
   \multicolumn{1}{S{Data}{r}}{702} & 
   \multicolumn{1}{S{Data}{r}}{1.14} & 
   \multicolumn{1}{S{Data}{r}}{22464} & 
   \multicolumn{1}{S{Data}{r}}{36.56}
\\
   \multicolumn{1}{S{RowHeader}{l}}{063 JACKSON} & 
   \multicolumn{1}{S{Data}{r}}{702} & 
   \multicolumn{1}{S{Data}{r}}{1.14} & 
   \multicolumn{1}{S{Data}{r}}{23166} & 
   \multicolumn{1}{S{Data}{r}}{37.70}
\\
   \multicolumn{1}{S{RowHeader}{l}}{065 KANABEC} & 
   \multicolumn{1}{S{Data}{r}}{702} & 
   \multicolumn{1}{S{Data}{r}}{1.14} & 
   \multicolumn{1}{S{Data}{r}}{23868} & 
   \multicolumn{1}{S{Data}{r}}{38.84}
\\
   \multicolumn{1}{S{RowHeader}{l}}{067 KANDIYOHI} & 
   \multicolumn{1}{S{Data}{r}}{702} & 
   \multicolumn{1}{S{Data}{r}}{1.14} & 
   \multicolumn{1}{S{Data}{r}}{24570} & 
   \multicolumn{1}{S{Data}{r}}{39.98}
\\
   \multicolumn{1}{S{RowHeader}{l}}{069 KITTSON} & 
   \multicolumn{1}{S{Data}{r}}{702} & 
   \multicolumn{1}{S{Data}{r}}{1.14} & 
   \multicolumn{1}{S{Data}{r}}{25272} & 
   \multicolumn{1}{S{Data}{r}}{41.12}
\\
   \multicolumn{1}{S{RowHeader}{l}}{071 KOOCHICHING} & 
   \multicolumn{1}{S{Data}{r}}{702} & 
   \multicolumn{1}{S{Data}{r}}{1.14} & 
   \multicolumn{1}{S{Data}{r}}{25974} & 
   \multicolumn{1}{S{Data}{r}}{42.27}
\\
   \multicolumn{1}{S{RowHeader}{l}}{073 LAC QUI PARLE} & 
   \multicolumn{1}{S{Data}{r}}{702} & 
   \multicolumn{1}{S{Data}{r}}{1.14} & 
   \multicolumn{1}{S{Data}{r}}{26676} & 
   \multicolumn{1}{S{Data}{r}}{43.41}
\\
   \multicolumn{1}{S{RowHeader}{l}}{075 LAKE} & 
   \multicolumn{1}{S{Data}{r}}{702} & 
   \multicolumn{1}{S{Data}{r}}{1.14} & 
   \multicolumn{1}{S{Data}{r}}{27378} & 
   \multicolumn{1}{S{Data}{r}}{44.55}
\\
   \multicolumn{1}{S{RowHeader}{l}}{077 LAKE OF THE WOODS} & 
   \multicolumn{1}{S{Data}{r}}{702} & 
   \multicolumn{1}{S{Data}{r}}{1.14} & 
   \multicolumn{1}{S{Data}{r}}{28080} & 
   \multicolumn{1}{S{Data}{r}}{45.69}
\\
   \multicolumn{1}{S{RowHeader}{l}}{079 LE SUEUR} & 
   \multicolumn{1}{S{Data}{r}}{702} & 
   \multicolumn{1}{S{Data}{r}}{1.14} & 
   \multicolumn{1}{S{Data}{r}}{28782} & 
   \multicolumn{1}{S{Data}{r}}{46.84}
\\
   \multicolumn{1}{S{RowHeader}{l}}{081 LINCOLN} & 
   \multicolumn{1}{S{Data}{r}}{702} & 
   \multicolumn{1}{S{Data}{r}}{1.14} & 
   \multicolumn{1}{S{Data}{r}}{29484} & 
   \multicolumn{1}{S{Data}{r}}{47.98}
\\
   \multicolumn{1}{S{RowHeader}{l}}{083 LYON} & 
   \multicolumn{1}{S{Data}{r}}{702} & 
   \multicolumn{1}{S{Data}{r}}{1.14} & 
   \multicolumn{1}{S{Data}{r}}{30186} & 
   \multicolumn{1}{S{Data}{r}}{49.12}
\\
   \multicolumn{1}{S{RowHeader}{l}}{085 MCLEOD} & 
   \multicolumn{1}{S{Data}{r}}{702} & 
   \multicolumn{1}{S{Data}{r}}{1.14} & 
   \multicolumn{1}{S{Data}{r}}{30888} & 
   \multicolumn{1}{S{Data}{r}}{50.26}
\\
   \multicolumn{1}{S{RowHeader}{l}}{087 MAHNOMEN} & 
   \multicolumn{1}{S{Data}{r}}{702} & 
   \multicolumn{1}{S{Data}{r}}{1.14} & 
   \multicolumn{1}{S{Data}{r}}{31590} & 
   \multicolumn{1}{S{Data}{r}}{51.41}
\\
   \multicolumn{1}{S{RowHeader}{l}}{089 MARSHALL} & 
   \multicolumn{1}{S{Data}{r}}{702} & 
   \multicolumn{1}{S{Data}{r}}{1.14} & 
   \multicolumn{1}{S{Data}{r}}{32292} & 
   \multicolumn{1}{S{Data}{r}}{52.55}
\\
   \multicolumn{1}{S{RowHeader}{l}}{091 MARTIN} & 
   \multicolumn{1}{S{Data}{r}}{702} & 
   \multicolumn{1}{S{Data}{r}}{1.14} & 
   \multicolumn{1}{S{Data}{r}}{32994} & 
   \multicolumn{1}{S{Data}{r}}{53.69}
\\
   \multicolumn{1}{S{RowHeader}{l}}{093 MEEKER} & 
   \multicolumn{1}{S{Data}{r}}{702} & 
   \multicolumn{1}{S{Data}{r}}{1.14} & 
   \multicolumn{1}{S{Data}{r}}{33696} & 
   \multicolumn{1}{S{Data}{r}}{54.83}
\\
   \multicolumn{1}{S{RowHeader}{l}}{095 MILLE LACS} & 
   \multicolumn{1}{S{Data}{r}}{702} & 
   \multicolumn{1}{S{Data}{r}}{1.14} & 
   \multicolumn{1}{S{Data}{r}}{34398} & 
   \multicolumn{1}{S{Data}{r}}{55.98}
\\
   \multicolumn{1}{S{RowHeader}{l}}{097 MORRISON} & 
   \multicolumn{1}{S{Data}{r}}{702} & 
   \multicolumn{1}{S{Data}{r}}{1.14} & 
   \multicolumn{1}{S{Data}{r}}{35100} & 
   \multicolumn{1}{S{Data}{r}}{57.12}
\\
   \multicolumn{1}{S{RowHeader}{l}}{099 MOWER} & 
   \multicolumn{1}{S{Data}{r}}{702} & 
   \multicolumn{1}{S{Data}{r}}{1.14} & 
   \multicolumn{1}{S{Data}{r}}{35802} & 
   \multicolumn{1}{S{Data}{r}}{58.26}
\\
   \multicolumn{1}{S{RowHeader}{l}}{101 MURRAY} & 
   \multicolumn{1}{S{Data}{r}}{702} & 
   \multicolumn{1}{S{Data}{r}}{1.14} & 
   \multicolumn{1}{S{Data}{r}}{36504} & 
   \multicolumn{1}{S{Data}{r}}{59.40}
\\
   \multicolumn{1}{S{RowHeader}{l}}{103 NICOLLET} & 
   \multicolumn{1}{S{Data}{r}}{702} & 
   \multicolumn{1}{S{Data}{r}}{1.14} & 
   \multicolumn{1}{S{Data}{r}}{37206} & 
   \multicolumn{1}{S{Data}{r}}{60.54}
\\
   \multicolumn{1}{S{RowHeader}{l}}{105 NOBLES} & 
   \multicolumn{1}{S{Data}{r}}{702} & 
   \multicolumn{1}{S{Data}{r}}{1.14} & 
   \multicolumn{1}{S{Data}{r}}{37908} & 
   \multicolumn{1}{S{Data}{r}}{61.69}
\\
   \multicolumn{1}{S{RowHeader}{l}}{107 NORMAN} & 
   \multicolumn{1}{S{Data}{r}}{702} & 
   \multicolumn{1}{S{Data}{r}}{1.14} & 
   \multicolumn{1}{S{Data}{r}}{38610} & 
   \multicolumn{1}{S{Data}{r}}{62.83}
\\
   \multicolumn{1}{S{RowHeader}{l}}{109 OLMSTED} & 
   \multicolumn{1}{S{Data}{r}}{702} & 
   \multicolumn{1}{S{Data}{r}}{1.14} & 
   \multicolumn{1}{S{Data}{r}}{39312} & 
   \multicolumn{1}{S{Data}{r}}{63.97}
\\
   \multicolumn{1}{S{RowHeader}{l}}{111 OTTER TAIL} & 
   \multicolumn{1}{S{Data}{r}}{702} & 
   \multicolumn{1}{S{Data}{r}}{1.14} & 
   \multicolumn{1}{S{Data}{r}}{40014} & 
   \multicolumn{1}{S{Data}{r}}{65.11}
\\
   \multicolumn{1}{S{RowHeader}{l}}{113 PENNINGTON} & 
   \multicolumn{1}{S{Data}{r}}{702} & 
   \multicolumn{1}{S{Data}{r}}{1.14} & 
   \multicolumn{1}{S{Data}{r}}{40716} & 
   \multicolumn{1}{S{Data}{r}}{66.26}
\\
   \multicolumn{1}{S{RowHeader}{l}}{115 PINE} & 
   \multicolumn{1}{S{Data}{r}}{702} & 
   \multicolumn{1}{S{Data}{r}}{1.14} & 
   \multicolumn{1}{S{Data}{r}}{41418} & 
   \multicolumn{1}{S{Data}{r}}{67.40}
\\
   \multicolumn{1}{S{RowHeader}{l}}{117 PIPESTONE} & 
   \multicolumn{1}{S{Data}{r}}{702} & 
   \multicolumn{1}{S{Data}{r}}{1.14} & 
   \multicolumn{1}{S{Data}{r}}{42120} & 
   \multicolumn{1}{S{Data}{r}}{68.54}
\\
   \multicolumn{1}{S{RowHeader}{l}}{119 POLK} & 
   \multicolumn{1}{S{Data}{r}}{702} & 
   \multicolumn{1}{S{Data}{r}}{1.14} & 
   \multicolumn{1}{S{Data}{r}}{42822} & 
   \multicolumn{1}{S{Data}{r}}{69.68}
\\
   \multicolumn{1}{S{RowHeader}{l}}{121 POPE} & 
   \multicolumn{1}{S{Data}{r}}{702} & 
   \multicolumn{1}{S{Data}{r}}{1.14} & 
   \multicolumn{1}{S{Data}{r}}{43524} & 
   \multicolumn{1}{S{Data}{r}}{70.83}
\\
   \multicolumn{1}{S{RowHeader}{l}}{123 RAMSEY} & 
   \multicolumn{1}{S{Data}{r}}{702} & 
   \multicolumn{1}{S{Data}{r}}{1.14} & 
   \multicolumn{1}{S{Data}{r}}{44226} & 
   \multicolumn{1}{S{Data}{r}}{71.97}
\\
   \multicolumn{1}{S{RowHeader}{l}}{125 RED LAKE} & 
   \multicolumn{1}{S{Data}{r}}{540} & 
   \multicolumn{1}{S{Data}{r}}{0.88} & 
   \multicolumn{1}{S{Data}{r}}{44766} & 
   \multicolumn{1}{S{Data}{r}}{72.85}
\\
   \multicolumn{1}{S{RowHeader}{l}}{127 REDWOOD} & 
   \multicolumn{1}{S{Data}{r}}{702} & 
   \multicolumn{1}{S{Data}{r}}{1.14} & 
   \multicolumn{1}{S{Data}{r}}{45468} & 
   \multicolumn{1}{S{Data}{r}}{73.99}
\\
   \multicolumn{1}{S{RowHeader}{l}}{129 RENVILLE} & 
   \multicolumn{1}{S{Data}{r}}{702} & 
   \multicolumn{1}{S{Data}{r}}{1.14} & 
   \multicolumn{1}{S{Data}{r}}{46170} & 
   \multicolumn{1}{S{Data}{r}}{75.13}
\\
   \multicolumn{1}{S{RowHeader}{l}}{131 RICE} & 
   \multicolumn{1}{S{Data}{r}}{702} & 
   \multicolumn{1}{S{Data}{r}}{1.14} & 
   \multicolumn{1}{S{Data}{r}}{46872} & 
   \multicolumn{1}{S{Data}{r}}{76.27}
\\
   \multicolumn{1}{S{RowHeader}{l}}{133 ROCK} & 
   \multicolumn{1}{S{Data}{r}}{702} & 
   \multicolumn{1}{S{Data}{r}}{1.14} & 
   \multicolumn{1}{S{Data}{r}}{47574} & 
   \multicolumn{1}{S{Data}{r}}{77.42}
\\
   \multicolumn{1}{S{RowHeader}{l}}{135 ROSEAU} & 
   \multicolumn{1}{S{Data}{r}}{702} & 
   \multicolumn{1}{S{Data}{r}}{1.14} & 
   \multicolumn{1}{S{Data}{r}}{48276} & 
   \multicolumn{1}{S{Data}{r}}{78.56}
\\
   \multicolumn{1}{S{RowHeader}{l}}{137 ST. LOUIS} & 
   \multicolumn{1}{S{Data}{r}}{702} & 
   \multicolumn{1}{S{Data}{r}}{1.14} & 
   \multicolumn{1}{S{Data}{r}}{48978} & 
   \multicolumn{1}{S{Data}{r}}{79.70}
\\
   \multicolumn{1}{S{RowHeader}{l}}{139 SCOTT} & 
   \multicolumn{1}{S{Data}{r}}{702} & 
   \multicolumn{1}{S{Data}{r}}{1.14} & 
   \multicolumn{1}{S{Data}{r}}{49680} & 
   \multicolumn{1}{S{Data}{r}}{80.84}
\\
   \multicolumn{1}{S{RowHeader}{l}}{141 SHERBURNE} & 
   \multicolumn{1}{S{Data}{r}}{702} & 
   \multicolumn{1}{S{Data}{r}}{1.14} & 
   \multicolumn{1}{S{Data}{r}}{50382} & 
   \multicolumn{1}{S{Data}{r}}{81.99}
\\
   \multicolumn{1}{S{RowHeader}{l}}{143 SIBLEY} & 
   \multicolumn{1}{S{Data}{r}}{702} & 
   \multicolumn{1}{S{Data}{r}}{1.14} & 
   \multicolumn{1}{S{Data}{r}}{51084} & 
   \multicolumn{1}{S{Data}{r}}{83.13}
\\
   \multicolumn{1}{S{RowHeader}{l}}{145 STEARNS} & 
   \multicolumn{1}{S{Data}{r}}{702} & 
   \multicolumn{1}{S{Data}{r}}{1.14} & 
   \multicolumn{1}{S{Data}{r}}{51786} & 
   \multicolumn{1}{S{Data}{r}}{84.27}
\\
   \multicolumn{1}{S{RowHeader}{l}}{147 STEELE} & 
   \multicolumn{1}{S{Data}{r}}{702} & 
   \multicolumn{1}{S{Data}{r}}{1.14} & 
   \multicolumn{1}{S{Data}{r}}{52488} & 
   \multicolumn{1}{S{Data}{r}}{85.41}
\\
   \multicolumn{1}{S{RowHeader}{l}}{149 STEVENS} & 
   \multicolumn{1}{S{Data}{r}}{702} & 
   \multicolumn{1}{S{Data}{r}}{1.14} & 
   \multicolumn{1}{S{Data}{r}}{53190} & 
   \multicolumn{1}{S{Data}{r}}{86.56}
\\
   \multicolumn{1}{S{RowHeader}{l}}{151 SWIFT} & 
   \multicolumn{1}{S{Data}{r}}{702} & 
   \multicolumn{1}{S{Data}{r}}{1.14} & 
   \multicolumn{1}{S{Data}{r}}{53892} & 
   \multicolumn{1}{S{Data}{r}}{87.70}
\\
   \multicolumn{1}{S{RowHeader}{l}}{153 TODD} & 
   \multicolumn{1}{S{Data}{r}}{702} & 
   \multicolumn{1}{S{Data}{r}}{1.14} & 
   \multicolumn{1}{S{Data}{r}}{54594} & 
   \multicolumn{1}{S{Data}{r}}{88.84}
\\
   \multicolumn{1}{S{RowHeader}{l}}{155 TRAVERSE} & 
   \multicolumn{1}{S{Data}{r}}{540} & 
   \multicolumn{1}{S{Data}{r}}{0.88} & 
   \multicolumn{1}{S{Data}{r}}{55134} & 
   \multicolumn{1}{S{Data}{r}}{89.72}
\\
   \multicolumn{1}{S{RowHeader}{l}}{157 WABASHA} & 
   \multicolumn{1}{S{Data}{r}}{702} & 
   \multicolumn{1}{S{Data}{r}}{1.14} & 
   \multicolumn{1}{S{Data}{r}}{55836} & 
   \multicolumn{1}{S{Data}{r}}{90.86}
\\
   \multicolumn{1}{S{RowHeader}{l}}{159 WADENA} & 
   \multicolumn{1}{S{Data}{r}}{702} & 
   \multicolumn{1}{S{Data}{r}}{1.14} & 
   \multicolumn{1}{S{Data}{r}}{56538} & 
   \multicolumn{1}{S{Data}{r}}{92.00}
\\
   \multicolumn{1}{S{RowHeader}{l}}{161 WASECA} & 
   \multicolumn{1}{S{Data}{r}}{702} & 
   \multicolumn{1}{S{Data}{r}}{1.14} & 
   \multicolumn{1}{S{Data}{r}}{57240} & 
   \multicolumn{1}{S{Data}{r}}{93.15}
\\
   \multicolumn{1}{S{RowHeader}{l}}{163 WASHINGTON} & 
   \multicolumn{1}{S{Data}{r}}{702} & 
   \multicolumn{1}{S{Data}{r}}{1.14} & 
   \multicolumn{1}{S{Data}{r}}{57942} & 
   \multicolumn{1}{S{Data}{r}}{94.29}
\\
   \multicolumn{1}{S{RowHeader}{l}}{165 WATONWAN} & 
   \multicolumn{1}{S{Data}{r}}{702} & 
   \multicolumn{1}{S{Data}{r}}{1.14} & 
   \multicolumn{1}{S{Data}{r}}{58644} & 
   \multicolumn{1}{S{Data}{r}}{95.43}
\\
   \multicolumn{1}{S{RowHeader}{l}}{167 WILKIN} & 
   \multicolumn{1}{S{Data}{r}}{702} & 
   \multicolumn{1}{S{Data}{r}}{1.14} & 
   \multicolumn{1}{S{Data}{r}}{59346} & 
   \multicolumn{1}{S{Data}{r}}{96.57}
\\
   \multicolumn{1}{S{RowHeader}{l}}{169 WINONA} & 
   \multicolumn{1}{S{Data}{r}}{702} & 
   \multicolumn{1}{S{Data}{r}}{1.14} & 
   \multicolumn{1}{S{Data}{r}}{60048} & 
   \multicolumn{1}{S{Data}{r}}{97.72}
\\
   \multicolumn{1}{S{RowHeader}{l}}{171 WRIGHT} & 
   \multicolumn{1}{S{Data}{r}}{702} & 
   \multicolumn{1}{S{Data}{r}}{1.14} & 
   \multicolumn{1}{S{Data}{r}}{60750} & 
   \multicolumn{1}{S{Data}{r}}{98.86}
\\
   \multicolumn{1}{S{RowHeader}{l}}{173 YELLOW MEDICINE} & 
   \multicolumn{1}{S{Data}{r}}{702} & 
   \multicolumn{1}{S{Data}{r}}{1.14} & 
   \multicolumn{1}{S{Data}{r}}{61452} & 
   \multicolumn{1}{S{Data}{r}}{100.00}
\\
\end{longtable}
\end{center}
\begin{center}\begin{longtable}
{rrrrr}\hline % colspecs
% table_head start
   \multicolumn{5}{S{Header}{c}}{Sex}
\\
   \multicolumn{1}{S{Header}{r}}{sex} & 
   \multicolumn{1}{S{Header}{r}}{Frequency} & 
   \multicolumn{1}{S{Header}{r}}{ Percent} & 
   \multicolumn{1}{S{Header}{r}}{Cumulative\linebreak  Frequency} & 
   \multicolumn{1}{S{Header}{r}}{Cumulative\linebreak   Percent}
\\
\hline 
\endhead % table_head end
\hline 
\multicolumn{1}{r}{(cont.)}\\
\endfoot 
\hline 
\endlastfoot % table_head end
   \multicolumn{1}{S{RowHeader}{r}}{0 : All} & 
   \multicolumn{1}{S{Data}{r}}{20484} & 
   \multicolumn{1}{S{Data}{r}}{33.33} & 
   \multicolumn{1}{S{Data}{r}}{20484} & 
   \multicolumn{1}{S{Data}{r}}{33.33}
\\
   \multicolumn{1}{S{RowHeader}{r}}{1 : Men} & 
   \multicolumn{1}{S{Data}{r}}{20484} & 
   \multicolumn{1}{S{Data}{r}}{33.33} & 
   \multicolumn{1}{S{Data}{r}}{40968} & 
   \multicolumn{1}{S{Data}{r}}{66.67}
\\
   \multicolumn{1}{S{RowHeader}{r}}{2 : Women} & 
   \multicolumn{1}{S{Data}{r}}{20484} & 
   \multicolumn{1}{S{Data}{r}}{33.33} & 
   \multicolumn{1}{S{Data}{r}}{61452} & 
   \multicolumn{1}{S{Data}{r}}{100.00}
\\
\end{longtable}
\end{center}
\begin{center}\begin{longtable}
{rrrrr}\hline % colspecs
% table_head start
   \multicolumn{5}{S{Header}{c}}{Age group}
\\
   \multicolumn{1}{S{Header}{r}}{agegroup} & 
   \multicolumn{1}{S{Header}{r}}{Frequency} & 
   \multicolumn{1}{S{Header}{r}}{ Percent} & 
   \multicolumn{1}{S{Header}{r}}{Cumulative\linebreak  Frequency} & 
   \multicolumn{1}{S{Header}{r}}{Cumulative\linebreak   Percent}
\\
\hline 
\endhead % table_head end
\hline 
\multicolumn{1}{r}{(cont.)}\\
\endfoot 
\hline 
\endlastfoot % table_head end
   \multicolumn{1}{S{RowHeader}{r}}{0 : All Ages} & 
   \multicolumn{1}{S{Data}{r}}{6828} & 
   \multicolumn{1}{S{Data}{r}}{11.11} & 
   \multicolumn{1}{S{Data}{r}}{6828} & 
   \multicolumn{1}{S{Data}{r}}{11.11}
\\
   \multicolumn{1}{S{RowHeader}{r}}{1 : 14-18} & 
   \multicolumn{1}{S{Data}{r}}{6828} & 
   \multicolumn{1}{S{Data}{r}}{11.11} & 
   \multicolumn{1}{S{Data}{r}}{13656} & 
   \multicolumn{1}{S{Data}{r}}{22.22}
\\
   \multicolumn{1}{S{RowHeader}{r}}{2 : 19-21} & 
   \multicolumn{1}{S{Data}{r}}{6828} & 
   \multicolumn{1}{S{Data}{r}}{11.11} & 
   \multicolumn{1}{S{Data}{r}}{20484} & 
   \multicolumn{1}{S{Data}{r}}{33.33}
\\
   \multicolumn{1}{S{RowHeader}{r}}{3 : 22-24} & 
   \multicolumn{1}{S{Data}{r}}{6828} & 
   \multicolumn{1}{S{Data}{r}}{11.11} & 
   \multicolumn{1}{S{Data}{r}}{27312} & 
   \multicolumn{1}{S{Data}{r}}{44.44}
\\
   \multicolumn{1}{S{RowHeader}{r}}{4 : 25-34} & 
   \multicolumn{1}{S{Data}{r}}{6828} & 
   \multicolumn{1}{S{Data}{r}}{11.11} & 
   \multicolumn{1}{S{Data}{r}}{34140} & 
   \multicolumn{1}{S{Data}{r}}{55.56}
\\
   \multicolumn{1}{S{RowHeader}{r}}{5 : 35-44} & 
   \multicolumn{1}{S{Data}{r}}{6828} & 
   \multicolumn{1}{S{Data}{r}}{11.11} & 
   \multicolumn{1}{S{Data}{r}}{40968} & 
   \multicolumn{1}{S{Data}{r}}{66.67}
\\
   \multicolumn{1}{S{RowHeader}{r}}{6 : 45-54} & 
   \multicolumn{1}{S{Data}{r}}{6828} & 
   \multicolumn{1}{S{Data}{r}}{11.11} & 
   \multicolumn{1}{S{Data}{r}}{47796} & 
   \multicolumn{1}{S{Data}{r}}{77.78}
\\
   \multicolumn{1}{S{RowHeader}{r}}{7 : 55-64} & 
   \multicolumn{1}{S{Data}{r}}{6828} & 
   \multicolumn{1}{S{Data}{r}}{11.11} & 
   \multicolumn{1}{S{Data}{r}}{54624} & 
   \multicolumn{1}{S{Data}{r}}{88.89}
\\
   \multicolumn{1}{S{RowHeader}{r}}{8 : 65+} & 
   \multicolumn{1}{S{Data}{r}}{6828} & 
   \multicolumn{1}{S{Data}{r}}{11.11} & 
   \multicolumn{1}{S{Data}{r}}{61452} & 
   \multicolumn{1}{S{Data}{r}}{100.00}
\\
\end{longtable}
\end{center}
\begin{center}\begin{longtable}
{llllll}\hline % colspecs
% table_head start
   \multicolumn{6}{S{Header}{c}}{Table of year by quarter}
\\
   \multicolumn{1}{S{Header}{c}}{year(Year)} & 
   \multicolumn{4}{S{Header}{c}}{quarter(Quarter)} & 
   \multicolumn{1}{S{Header}{r}}{Total}
\\
   \multicolumn{1}{l}{~} & 
   \multicolumn{1}{S{Header}{r}}{      1     } & 
   \multicolumn{1}{S{Header}{r}}{      2     } & 
   \multicolumn{1}{S{Header}{r}}{      3     } & 
   \multicolumn{1}{S{Header}{r}}{      4     } & 
   \multicolumn{1}{l}{~}
\\
\hline 
\endhead % table_head end
\hline 
\multicolumn{1}{r}{(cont.)}\\
\endfoot 
\hline 
\endlastfoot % table_head end
   \multicolumn{1}{S{Header}{r}}{1994        } & 
   \multicolumn{1}{S{Data}{r}}{     0} & 
   \multicolumn{1}{S{Data}{r}}{     0} & 
   \multicolumn{1}{S{Data}{r}}{  2376} & 
   \multicolumn{1}{S{Data}{r}}{  2376} & 
   \multicolumn{1}{S{Data}{r}}{  4752}
\\
   \multicolumn{1}{S{Header}{r}}{1995        } & 
   \multicolumn{1}{S{Data}{r}}{  2376} & 
   \multicolumn{1}{S{Data}{r}}{  2376} & 
   \multicolumn{1}{S{Data}{r}}{  2376} & 
   \multicolumn{1}{S{Data}{r}}{  2376} & 
   \multicolumn{1}{S{Data}{r}}{  9504}
\\
   \multicolumn{1}{S{Header}{r}}{1996        } & 
   \multicolumn{1}{S{Data}{r}}{  2376} & 
   \multicolumn{1}{S{Data}{r}}{  2376} & 
   \multicolumn{1}{S{Data}{r}}{  2376} & 
   \multicolumn{1}{S{Data}{r}}{  2376} & 
   \multicolumn{1}{S{Data}{r}}{  9504}
\\
   \multicolumn{1}{S{Header}{r}}{1997        } & 
   \multicolumn{1}{S{Data}{r}}{  2376} & 
   \multicolumn{1}{S{Data}{r}}{  2376} & 
   \multicolumn{1}{S{Data}{r}}{  2376} & 
   \multicolumn{1}{S{Data}{r}}{  2376} & 
   \multicolumn{1}{S{Data}{r}}{  9504}
\\
   \multicolumn{1}{S{Header}{r}}{1998        } & 
   \multicolumn{1}{S{Data}{r}}{  2349} & 
   \multicolumn{1}{S{Data}{r}}{  2349} & 
   \multicolumn{1}{S{Data}{r}}{  2349} & 
   \multicolumn{1}{S{Data}{r}}{  2376} & 
   \multicolumn{1}{S{Data}{r}}{  9423}
\\
   \multicolumn{1}{S{Header}{r}}{1999        } & 
   \multicolumn{1}{S{Data}{r}}{  2376} & 
   \multicolumn{1}{S{Data}{r}}{  2349} & 
   \multicolumn{1}{S{Data}{r}}{  2322} & 
   \multicolumn{1}{S{Data}{r}}{  2322} & 
   \multicolumn{1}{S{Data}{r}}{  9369}
\\
   \multicolumn{1}{S{Header}{r}}{2000        } & 
   \multicolumn{1}{S{Data}{r}}{  2349} & 
   \multicolumn{1}{S{Data}{r}}{  2349} & 
   \multicolumn{1}{S{Data}{r}}{  2349} & 
   \multicolumn{1}{S{Data}{r}}{  2349} & 
   \multicolumn{1}{S{Data}{r}}{  9396}
\\
   \multicolumn{1}{S{Header}{l}}{Total           } & 
   \multicolumn{1}{S{Data}{r}}{  14202} & 
   \multicolumn{1}{S{Data}{r}}{  14175} & 
   \multicolumn{1}{S{Data}{r}}{  16524} & 
   \multicolumn{1}{S{Data}{r}}{  16551} & 
   \multicolumn{1}{S{Data}{r}}{  61452}
\\
\end{longtable}
\end{center}

% ====================end of output====================

% \newpage %    Generated by SAS
%    http://www.sas.com
% created by=vilhu001
% sasversion=8.2
% date=2002-05-23
% time=00:39:34
% encoding=iso-8859-1
% ====================begin of output====================
% \begin{document}

% An external file needs to be included, as specified% in latexlong.sas. This can be called sas.sty,
% in which case you want to include a line like
% \usepackage{sas}
% or it can be a simple (La)TeX file, which you 
% include by typing 
% %%
%% This is file `sas.sty',
%% generated with the docstrip utility.
%%
%% 
\NeedsTeXFormat{LaTeX2e}
\ProvidesPackage{sas}
        [2002/01/18 LEHD version 0.1
    provides definition for tables generated by SAS%
                   ]
\@ifundefined{array@processline}{\RequirePackage{array}}{}
\@ifundefined{longtable@processline}{\RequirePackage{longtable}}{}
 \def\ContentTitle{\small\it\sffamily}
 \def\Output{\small\sffamily}
 \def\HeaderEmphasis{\small\it\sffamily}
 \def\NoteContent{\small\sffamily}
 \def\FatalContent{\small\sffamily}
 \def\Graph{\small\sffamily}
 \def\WarnContentFixed{\footnotesize\tt}
 \def\NoteBanner{\small\sffamily}
 \def\DataStrong{\normalsize\bf\sffamily}
 \def\Document{\small\sffamily}
 \def\BeforeCaption{\normalsize\bf\sffamily}
 \def\ContentsDate{\small\sffamily}
 \def\Pages{\small\sffamily}
 \def\TitlesAndFooters{\footnotesize\bf\it\sffamily}
 \def\IndexProcName{\small\sffamily}
 \def\ProcTitle{\normalsize\bf\it\sffamily}
 \def\IndexAction{\small\sffamily}
 \def\Data{\small\sffamily}
 \def\Table{\small\sffamily}
 \def\FooterEmpty{\footnotesize\bf\sffamily}
 \def\SysTitleAndFooterContainer{\footnotesize\sffamily}
 \def\RowFooterEmpty{\footnotesize\bf\sffamily}
 \def\ExtendedPage{\small\it\sffamily}
 \def\FooterFixed{\footnotesize\tt}
 \def\RowFooterStrongFixed{\footnotesize\bf\tt}
 \def\RowFooterEmphasis{\footnote\it\sffamily}
 \def\ContentFolder{\small\sffamily}
 \def\Container{\small\sffamily}
 \def\Date{\small\sffamily}
 \def\RowFooterFixed{\footnotesize\tt}
 \def\Caption{\normalsize\bf\sffamily}
 \def\WarnBanner{\small\sffamily}
 \def\Frame{\small\sffamily}
 \def\HeaderStrongFixed{\footnotesize\bf\tt}
 \def\IndexTitle{\small\it\sffamily}
 \def\NoteContentFixed{\footnotesize\tt}
 \def\DataEmphasisFixed{\footnotesize\it\tt}
 \def\Note{\small\sffamily}
 \def\Byline{\normalsize\bf\sffamily}
 \def\FatalBanner{\small\sffamily}
 \def\ProcTitleFixed{\footnotesize\bf\tt}
 \def\ByContentFolder{\small\sffamily}
 \def\PagesProcLabel{\small\sffamily}
 \def\RowHeaderFixed{\footnotesize\tt}
 \def\RowFooterEmphasisFixed{\footnotesize\it\tt}
 \def\WarnContent{\small\sffamily}
 \def\DataEmpty{\small\sffamily}
 \def\Cell{\small\sffamily}
 \def\Header{\normalsize\bf\sffamily}
 \def\PageNo{\normalsize\bf\sffamily}
 \def\ContentProcLabel{\small\sffamily}
 \def\HeaderFixed{\footnotesize\tt}
 \def\PagesTitle{\small\it\sffamily}
 \def\RowHeaderEmpty{\normalsize\bf\sffamily}
 \def\PagesProcName{\small\sffamily}
 \def\Batch{\footnotesize\tt}
 \def\ContentItem{\small\sffamily}
 \def\Body{\small\sffamily}
 \def\PagesDate{\small\sffamily}
 \def\Index{\small\sffamily}
 \def\HeaderEmpty{\normalsize\bf\sffamily}
 \def\FooterStrong{\footnotesize\bf\sffamily}
 \def\FooterEmphasis{\footnotesize\it\sffamily}
 \def\ErrorContent{\small\sffamily}
 \def\DataFixed{\footnotesize\tt}
 \def\HeaderStrong{\normalsize\bf\sffamily}
 \def\GraphBackground{}
 \def\DataEmphasis{\small\it\sffamily}
 \def\TitleAndNoteContainer{\small\sffamily}
 \def\RowFooter{\footnotesize\bf\sffamily}
 \def\IndexItem{\small\sffamily}
 \def\BylineContainer{\small\sffamily}
 \def\FatalContentFixed{\footnotesize\tt}
 \def\BodyDate{\small\sffamily}
 \def\RowFooterStrong{\footnotesize\bf\sffamily}
 \def\UserText{\small\sffamily}
 \def\HeadersAndFooters{\footnotesize\bf\sffamily}
 \def\RowHeaderEmphasisFixed{\footnotesize\it\tt}
 \def\ErrorBanner{\small\sffamily}
 \def\ContentProcName{\small\sffamily}
 \def\RowHeaderStrong{\normalsize\bf\sffamily}
 \def\FooterEmphasisFixed{\footnotesize\it\tt}
 \def\Contents{\small\sffamily}
 \def\FooterStrongFixed{\footnotesize\bf\tt}
 \def\PagesItem{\small\sffamily}
 \def\RowHeader{\normalsize\bf\sffamily}
 \def\AfterCaption{\normalsize\bf\sffamily}
 \def\RowHeaderStrongFixed{\footnotesize\bf\tt}
 \def\RowHeaderEmphasis{\small\it\sffamily}
 \def\DataStrongFixed{\footnotesize\bf\tt}
 \def\Footer{\footnotesize\bf\sffamily}
 \def\FolderAction{\small\sffamily}
 \def\HeaderEmphasisFixed{\footnotesize\it\tt}
 \def\SystemTitle{\large\bf\it\sffamily}
 \def\ErrorContentFixed{\footnotesize\tt}
 \def\SystemFooter{\footnotesize\it\sffamily}
% Set cell padding 
\renewcommand{\arraystretch}{1.3}
% Headings
\newcommand{\heading}[2]{\csname#1\endcsname #2}
\newcommand{\proctitle}[2]{\csname#1\endcsname #2}
% Declare new column type
\newcolumntype{S}[2]{>{\csname#1\endcsname}#2}
% Set warning box style
\newcommand{\msg}[2]{\fbox{%
   \begin{minipage}{\textwidth}#2\end{minipage}}%
}

\begin{center}\heading{ProcTitle}{The CONTENTS Procedure}\end{center}
\begin{center}\begin{longtable}
{llll}\hline % colspecs
   \multicolumn{1}{S{RowHeader}{l}}{Data Set Name:} & 
   \multicolumn{1}{S{Data}{l}}{STATE.MN{\textunderscore}SIC{\textunderscore}DIVISION{\textunderscore}V23{\textunderscore}FUZZED} & 
   \multicolumn{1}{S{RowHeader}{l}}{Observations:} & 
   \multicolumn{1}{S{Data}{l}}{8424}
\\
   \multicolumn{1}{S{RowHeader}{l}}{Member Type:} & 
   \multicolumn{1}{S{Data}{l}}{DATA} & 
   \multicolumn{1}{S{RowHeader}{l}}{Variables:} & 
   \multicolumn{1}{S{Data}{l}}{60}
\\
   \multicolumn{1}{S{RowHeader}{l}}{Engine:} & 
   \multicolumn{1}{S{Data}{l}}{V8} & 
   \multicolumn{1}{S{RowHeader}{l}}{Indexes:} & 
   \multicolumn{1}{S{Data}{l}}{0}
\\
   \multicolumn{1}{S{RowHeader}{l}}{Created:} & 
   \multicolumn{1}{S{Data}{l}}{18:42 Thursday, May 16, 2002} & 
   \multicolumn{1}{S{RowHeader}{l}}{Observation Length:} & 
   \multicolumn{1}{S{Data}{l}}{288}
\\
   \multicolumn{1}{S{RowHeader}{l}}{Last Modified:} & 
   \multicolumn{1}{S{Data}{l}}{18:42 Thursday, May 16, 2002} & 
   \multicolumn{1}{S{RowHeader}{l}}{Deleted Observations:} & 
   \multicolumn{1}{S{Data}{l}}{0}
\\
   \multicolumn{1}{S{RowHeader}{l}}{Protection:} & 
   \multicolumn{1}{S{Data}{l}}{ } & 
   \multicolumn{1}{S{RowHeader}{l}}{Compressed:} & 
   \multicolumn{1}{S{Data}{l}}{NO}
\\
   \multicolumn{1}{S{RowHeader}{l}}{Data Set Type:} & 
   \multicolumn{1}{S{Data}{l}}{ } & 
   \multicolumn{1}{S{RowHeader}{l}}{Sorted:} & 
   \multicolumn{1}{S{Data}{l}}{NO}
\\
   \multicolumn{1}{S{RowHeader}{l}}{Label:} & 
   \multicolumn{1}{S{Data}{l}}{ } & 
   \multicolumn{1}{S{RowHeader}{l}}{ } & 
   \multicolumn{1}{S{Data}{l}}{ }
\\
\end{longtable}
\end{center}
\begin{center}\begin{longtable}
{rllrrl}\hline % colspecs
% table_head start
   \multicolumn{6}{S{Header}{c}}{-----Variables Ordered by Position-----}
\\
   \multicolumn{1}{S{Header}{r}}{\#} & 
   \multicolumn{1}{S{Header}{l}}{Variable} & 
   \multicolumn{1}{S{Header}{l}}{Type} & 
   \multicolumn{1}{S{Header}{r}}{Len} & 
   \multicolumn{1}{S{Header}{r}}{Pos} & 
   \multicolumn{1}{S{Header}{l}}{Label}
\\
\hline 
\endhead % table_head end
\hline 
\multicolumn{1}{r}{(cont.)}\\
\endfoot 
\hline 
\endlastfoot % table_head end
   \multicolumn{1}{S{RowHeader}{r}}{1} & 
   \multicolumn{1}{S{Data}{l}}{state} & 
   \multicolumn{1}{S{Data}{l}}{Char} & 
   \multicolumn{1}{S{Data}{r}}{2} & 
   \multicolumn{1}{S{Data}{r}}{216} & 
   \multicolumn{1}{S{Data}{l}}{FIPS State}
\\
   \multicolumn{1}{S{RowHeader}{r}}{2} & 
   \multicolumn{1}{S{Data}{l}}{year} & 
   \multicolumn{1}{S{Data}{l}}{Num} & 
   \multicolumn{1}{S{Data}{r}}{3} & 
   \multicolumn{1}{S{Data}{r}}{273} & 
   \multicolumn{1}{S{Data}{l}}{Year}
\\
   \multicolumn{1}{S{RowHeader}{r}}{3} & 
   \multicolumn{1}{S{Data}{l}}{quarter} & 
   \multicolumn{1}{S{Data}{l}}{Num} & 
   \multicolumn{1}{S{Data}{r}}{3} & 
   \multicolumn{1}{S{Data}{r}}{276} & 
   \multicolumn{1}{S{Data}{l}}{Quarter}
\\
   \multicolumn{1}{S{RowHeader}{r}}{4} & 
   \multicolumn{1}{S{Data}{l}}{sic{\textunderscore}division} & 
   \multicolumn{1}{S{Data}{l}}{Char} & 
   \multicolumn{1}{S{Data}{r}}{1} & 
   \multicolumn{1}{S{Data}{r}}{218} & 
   \multicolumn{1}{S{Data}{l}}{SIC Division}
\\
   \multicolumn{1}{S{RowHeader}{r}}{5} & 
   \multicolumn{1}{S{Data}{l}}{sex} & 
   \multicolumn{1}{S{Data}{l}}{Num} & 
   \multicolumn{1}{S{Data}{r}}{3} & 
   \multicolumn{1}{S{Data}{r}}{279} & 
   \multicolumn{1}{S{Data}{l}}{Sex}
\\
   \multicolumn{1}{S{RowHeader}{r}}{6} & 
   \multicolumn{1}{S{Data}{l}}{agegroup} & 
   \multicolumn{1}{S{Data}{l}}{Num} & 
   \multicolumn{1}{S{Data}{r}}{3} & 
   \multicolumn{1}{S{Data}{r}}{282} & 
   \multicolumn{1}{S{Data}{l}}{Age group}
\\
   \multicolumn{1}{S{RowHeader}{r}}{7} & 
   \multicolumn{1}{S{Data}{l}}{A} & 
   \multicolumn{1}{S{Data}{l}}{Num} & 
   \multicolumn{1}{S{Data}{r}}{8} & 
   \multicolumn{1}{S{Data}{r}}{0} & 
   \multicolumn{1}{S{Data}{l}}{Accessions}
\\
   \multicolumn{1}{S{RowHeader}{r}}{8} & 
   \multicolumn{1}{S{Data}{l}}{B} & 
   \multicolumn{1}{S{Data}{l}}{Num} & 
   \multicolumn{1}{S{Data}{r}}{8} & 
   \multicolumn{1}{S{Data}{r}}{8} & 
   \multicolumn{1}{S{Data}{l}}{Beginning-of-period employment}
\\
   \multicolumn{1}{S{RowHeader}{r}}{9} & 
   \multicolumn{1}{S{Data}{l}}{E} & 
   \multicolumn{1}{S{Data}{l}}{Num} & 
   \multicolumn{1}{S{Data}{r}}{8} & 
   \multicolumn{1}{S{Data}{r}}{16} & 
   \multicolumn{1}{S{Data}{l}}{End-of-period employment}
\\
   \multicolumn{1}{S{RowHeader}{r}}{10} & 
   \multicolumn{1}{S{Data}{l}}{F} & 
   \multicolumn{1}{S{Data}{l}}{Num} & 
   \multicolumn{1}{S{Data}{r}}{8} & 
   \multicolumn{1}{S{Data}{r}}{24} & 
   \multicolumn{1}{S{Data}{l}}{Full-quarter employment}
\\
   \multicolumn{1}{S{RowHeader}{r}}{11} & 
   \multicolumn{1}{S{Data}{l}}{FA} & 
   \multicolumn{1}{S{Data}{l}}{Num} & 
   \multicolumn{1}{S{Data}{r}}{8} & 
   \multicolumn{1}{S{Data}{r}}{32} & 
   \multicolumn{1}{S{Data}{l}}{Flow into full-quarter employment}
\\
   \multicolumn{1}{S{RowHeader}{r}}{12} & 
   \multicolumn{1}{S{Data}{l}}{FJC} & 
   \multicolumn{1}{S{Data}{l}}{Num} & 
   \multicolumn{1}{S{Data}{r}}{8} & 
   \multicolumn{1}{S{Data}{r}}{40} & 
   \multicolumn{1}{S{Data}{l}}{Full-quarter job creation}
\\
   \multicolumn{1}{S{RowHeader}{r}}{13} & 
   \multicolumn{1}{S{Data}{l}}{FJD} & 
   \multicolumn{1}{S{Data}{l}}{Num} & 
   \multicolumn{1}{S{Data}{r}}{8} & 
   \multicolumn{1}{S{Data}{r}}{48} & 
   \multicolumn{1}{S{Data}{l}}{Full-quarter job destruction}
\\
   \multicolumn{1}{S{RowHeader}{r}}{14} & 
   \multicolumn{1}{S{Data}{l}}{FJF} & 
   \multicolumn{1}{S{Data}{l}}{Num} & 
   \multicolumn{1}{S{Data}{r}}{8} & 
   \multicolumn{1}{S{Data}{r}}{56} & 
   \multicolumn{1}{S{Data}{l}}{Net change in full-quarter employment}
\\
   \multicolumn{1}{S{RowHeader}{r}}{15} & 
   \multicolumn{1}{S{Data}{l}}{FS} & 
   \multicolumn{1}{S{Data}{l}}{Num} & 
   \multicolumn{1}{S{Data}{r}}{8} & 
   \multicolumn{1}{S{Data}{r}}{64} & 
   \multicolumn{1}{S{Data}{l}}{Flow out of full-quarter employment}
\\
   \multicolumn{1}{S{RowHeader}{r}}{16} & 
   \multicolumn{1}{S{Data}{l}}{H} & 
   \multicolumn{1}{S{Data}{l}}{Num} & 
   \multicolumn{1}{S{Data}{r}}{8} & 
   \multicolumn{1}{S{Data}{r}}{72} & 
   \multicolumn{1}{S{Data}{l}}{New hires}
\\
   \multicolumn{1}{S{RowHeader}{r}}{17} & 
   \multicolumn{1}{S{Data}{l}}{H3} & 
   \multicolumn{1}{S{Data}{l}}{Num} & 
   \multicolumn{1}{S{Data}{r}}{8} & 
   \multicolumn{1}{S{Data}{r}}{80} & 
   \multicolumn{1}{S{Data}{l}}{Full-quarter new hires}
\\
   \multicolumn{1}{S{RowHeader}{r}}{18} & 
   \multicolumn{1}{S{Data}{l}}{JC} & 
   \multicolumn{1}{S{Data}{l}}{Num} & 
   \multicolumn{1}{S{Data}{r}}{8} & 
   \multicolumn{1}{S{Data}{r}}{88} & 
   \multicolumn{1}{S{Data}{l}}{Job creation}
\\
   \multicolumn{1}{S{RowHeader}{r}}{19} & 
   \multicolumn{1}{S{Data}{l}}{JD} & 
   \multicolumn{1}{S{Data}{l}}{Num} & 
   \multicolumn{1}{S{Data}{r}}{8} & 
   \multicolumn{1}{S{Data}{r}}{96} & 
   \multicolumn{1}{S{Data}{l}}{Job destruction}
\\
   \multicolumn{1}{S{RowHeader}{r}}{20} & 
   \multicolumn{1}{S{Data}{l}}{JF} & 
   \multicolumn{1}{S{Data}{l}}{Num} & 
   \multicolumn{1}{S{Data}{r}}{8} & 
   \multicolumn{1}{S{Data}{r}}{104} & 
   \multicolumn{1}{S{Data}{l}}{Net job flows}
\\
   \multicolumn{1}{S{RowHeader}{r}}{21} & 
   \multicolumn{1}{S{Data}{l}}{R} & 
   \multicolumn{1}{S{Data}{l}}{Num} & 
   \multicolumn{1}{S{Data}{r}}{8} & 
   \multicolumn{1}{S{Data}{r}}{112} & 
   \multicolumn{1}{S{Data}{l}}{Recalls}
\\
   \multicolumn{1}{S{RowHeader}{r}}{22} & 
   \multicolumn{1}{S{Data}{l}}{S} & 
   \multicolumn{1}{S{Data}{l}}{Num} & 
   \multicolumn{1}{S{Data}{r}}{8} & 
   \multicolumn{1}{S{Data}{r}}{120} & 
   \multicolumn{1}{S{Data}{l}}{Separations}
\\
   \multicolumn{1}{S{RowHeader}{r}}{23} & 
   \multicolumn{1}{S{Data}{l}}{Z{\textunderscore}NA} & 
   \multicolumn{1}{S{Data}{l}}{Num} & 
   \multicolumn{1}{S{Data}{r}}{8} & 
   \multicolumn{1}{S{Data}{r}}{128} & 
   \multicolumn{1}{S{Data}{l}}{Average periods of non-employment for accessions}
\\
   \multicolumn{1}{S{RowHeader}{r}}{24} & 
   \multicolumn{1}{S{Data}{l}}{Z{\textunderscore}NH} & 
   \multicolumn{1}{S{Data}{l}}{Num} & 
   \multicolumn{1}{S{Data}{r}}{8} & 
   \multicolumn{1}{S{Data}{r}}{136} & 
   \multicolumn{1}{S{Data}{l}}{Average periods of non-employment for new hires}
\\
   \multicolumn{1}{S{RowHeader}{r}}{25} & 
   \multicolumn{1}{S{Data}{l}}{Z{\textunderscore}NR} & 
   \multicolumn{1}{S{Data}{l}}{Num} & 
   \multicolumn{1}{S{Data}{r}}{8} & 
   \multicolumn{1}{S{Data}{r}}{144} & 
   \multicolumn{1}{S{Data}{l}}{Average periods of non-employment for recalls}
\\
   \multicolumn{1}{S{RowHeader}{r}}{26} & 
   \multicolumn{1}{S{Data}{l}}{Z{\textunderscore}NS} & 
   \multicolumn{1}{S{Data}{l}}{Num} & 
   \multicolumn{1}{S{Data}{r}}{8} & 
   \multicolumn{1}{S{Data}{r}}{152} & 
   \multicolumn{1}{S{Data}{l}}{Average periods of non-employment for separations}
\\
   \multicolumn{1}{S{RowHeader}{r}}{27} & 
   \multicolumn{1}{S{Data}{l}}{Z{\textunderscore}W2} & 
   \multicolumn{1}{S{Data}{l}}{Num} & 
   \multicolumn{1}{S{Data}{r}}{8} & 
   \multicolumn{1}{S{Data}{r}}{160} & 
   \multicolumn{1}{S{Data}{l}}{Average earnings of end-of-period employees}
\\
   \multicolumn{1}{S{RowHeader}{r}}{28} & 
   \multicolumn{1}{S{Data}{l}}{Z{\textunderscore}W3} & 
   \multicolumn{1}{S{Data}{l}}{Num} & 
   \multicolumn{1}{S{Data}{r}}{8} & 
   \multicolumn{1}{S{Data}{r}}{168} & 
   \multicolumn{1}{S{Data}{l}}{Average earnings of full-quarter employees}
\\
   \multicolumn{1}{S{RowHeader}{r}}{29} & 
   \multicolumn{1}{S{Data}{l}}{Z{\textunderscore}WFA} & 
   \multicolumn{1}{S{Data}{l}}{Num} & 
   \multicolumn{1}{S{Data}{r}}{8} & 
   \multicolumn{1}{S{Data}{r}}{176} & 
   \multicolumn{1}{S{Data}{l}}{Average earnings of transits to full-quarter status}
\\
   \multicolumn{1}{S{RowHeader}{r}}{30} & 
   \multicolumn{1}{S{Data}{l}}{Z{\textunderscore}WFS} & 
   \multicolumn{1}{S{Data}{l}}{Num} & 
   \multicolumn{1}{S{Data}{r}}{8} & 
   \multicolumn{1}{S{Data}{r}}{184} & 
   \multicolumn{1}{S{Data}{l}}{Average earnings of separations from full-quarter status}
\\
   \multicolumn{1}{S{RowHeader}{r}}{31} & 
   \multicolumn{1}{S{Data}{l}}{Z{\textunderscore}WH3} & 
   \multicolumn{1}{S{Data}{l}}{Num} & 
   \multicolumn{1}{S{Data}{r}}{8} & 
   \multicolumn{1}{S{Data}{r}}{192} & 
   \multicolumn{1}{S{Data}{l}}{Average earnings of full-quarter new hires}
\\
   \multicolumn{1}{S{RowHeader}{r}}{32} & 
   \multicolumn{1}{S{Data}{l}}{Z{\textunderscore}dWA} & 
   \multicolumn{1}{S{Data}{l}}{Num} & 
   \multicolumn{1}{S{Data}{r}}{8} & 
   \multicolumn{1}{S{Data}{r}}{200} & 
   \multicolumn{1}{S{Data}{l}}{Average change in total earnings for accessions}
\\
   \multicolumn{1}{S{RowHeader}{r}}{33} & 
   \multicolumn{1}{S{Data}{l}}{Z{\textunderscore}dWS} & 
   \multicolumn{1}{S{Data}{l}}{Num} & 
   \multicolumn{1}{S{Data}{r}}{8} & 
   \multicolumn{1}{S{Data}{r}}{208} & 
   \multicolumn{1}{S{Data}{l}}{Average change in total earnings for separations}
\\
   \multicolumn{1}{S{RowHeader}{r}}{34} & 
   \multicolumn{1}{S{Data}{l}}{A{\textunderscore}status} & 
   \multicolumn{1}{S{Data}{l}}{Char} & 
   \multicolumn{1}{S{Data}{r}}{2} & 
   \multicolumn{1}{S{Data}{r}}{219} & 
   \multicolumn{1}{S{Data}{l}}{Status: accessions}
\\
   \multicolumn{1}{S{RowHeader}{r}}{35} & 
   \multicolumn{1}{S{Data}{l}}{B{\textunderscore}status} & 
   \multicolumn{1}{S{Data}{l}}{Char} & 
   \multicolumn{1}{S{Data}{r}}{2} & 
   \multicolumn{1}{S{Data}{r}}{221} & 
   \multicolumn{1}{S{Data}{l}}{Status: beginning-of-period employment}
\\
   \multicolumn{1}{S{RowHeader}{r}}{36} & 
   \multicolumn{1}{S{Data}{l}}{E{\textunderscore}status} & 
   \multicolumn{1}{S{Data}{l}}{Char} & 
   \multicolumn{1}{S{Data}{r}}{2} & 
   \multicolumn{1}{S{Data}{r}}{223} & 
   \multicolumn{1}{S{Data}{l}}{Status: end-of-period employment}
\\
   \multicolumn{1}{S{RowHeader}{r}}{37} & 
   \multicolumn{1}{S{Data}{l}}{F{\textunderscore}status} & 
   \multicolumn{1}{S{Data}{l}}{Char} & 
   \multicolumn{1}{S{Data}{r}}{2} & 
   \multicolumn{1}{S{Data}{r}}{225} & 
   \multicolumn{1}{S{Data}{l}}{Status: full-quarter employment}
\\
   \multicolumn{1}{S{RowHeader}{r}}{38} & 
   \multicolumn{1}{S{Data}{l}}{FA{\textunderscore}status} & 
   \multicolumn{1}{S{Data}{l}}{Char} & 
   \multicolumn{1}{S{Data}{r}}{2} & 
   \multicolumn{1}{S{Data}{r}}{227} & 
   \multicolumn{1}{S{Data}{l}}{Status: flow into full-quarter employment}
\\
   \multicolumn{1}{S{RowHeader}{r}}{39} & 
   \multicolumn{1}{S{Data}{l}}{FJC{\textunderscore}status} & 
   \multicolumn{1}{S{Data}{l}}{Char} & 
   \multicolumn{1}{S{Data}{r}}{2} & 
   \multicolumn{1}{S{Data}{r}}{229} & 
   \multicolumn{1}{S{Data}{l}}{Status: full-quarter job creation}
\\
   \multicolumn{1}{S{RowHeader}{r}}{40} & 
   \multicolumn{1}{S{Data}{l}}{FJD{\textunderscore}status} & 
   \multicolumn{1}{S{Data}{l}}{Char} & 
   \multicolumn{1}{S{Data}{r}}{2} & 
   \multicolumn{1}{S{Data}{r}}{231} & 
   \multicolumn{1}{S{Data}{l}}{Status: full-quarter job destruction}
\\
   \multicolumn{1}{S{RowHeader}{r}}{41} & 
   \multicolumn{1}{S{Data}{l}}{FJF{\textunderscore}status} & 
   \multicolumn{1}{S{Data}{l}}{Char} & 
   \multicolumn{1}{S{Data}{r}}{2} & 
   \multicolumn{1}{S{Data}{r}}{233} & 
   \multicolumn{1}{S{Data}{l}}{Status: net change in full-quarter employment}
\\
   \multicolumn{1}{S{RowHeader}{r}}{42} & 
   \multicolumn{1}{S{Data}{l}}{FS{\textunderscore}status} & 
   \multicolumn{1}{S{Data}{l}}{Char} & 
   \multicolumn{1}{S{Data}{r}}{2} & 
   \multicolumn{1}{S{Data}{r}}{235} & 
   \multicolumn{1}{S{Data}{l}}{Status: flow out of full-quarter employment}
\\
   \multicolumn{1}{S{RowHeader}{r}}{43} & 
   \multicolumn{1}{S{Data}{l}}{H{\textunderscore}status} & 
   \multicolumn{1}{S{Data}{l}}{Char} & 
   \multicolumn{1}{S{Data}{r}}{2} & 
   \multicolumn{1}{S{Data}{r}}{237} & 
   \multicolumn{1}{S{Data}{l}}{Status: new hires}
\\
   \multicolumn{1}{S{RowHeader}{r}}{44} & 
   \multicolumn{1}{S{Data}{l}}{H3{\textunderscore}status} & 
   \multicolumn{1}{S{Data}{l}}{Char} & 
   \multicolumn{1}{S{Data}{r}}{2} & 
   \multicolumn{1}{S{Data}{r}}{239} & 
   \multicolumn{1}{S{Data}{l}}{Status: full-quarter new hires}
\\
   \multicolumn{1}{S{RowHeader}{r}}{45} & 
   \multicolumn{1}{S{Data}{l}}{JC{\textunderscore}status} & 
   \multicolumn{1}{S{Data}{l}}{Char} & 
   \multicolumn{1}{S{Data}{r}}{2} & 
   \multicolumn{1}{S{Data}{r}}{241} & 
   \multicolumn{1}{S{Data}{l}}{Status: job creation}
\\
   \multicolumn{1}{S{RowHeader}{r}}{46} & 
   \multicolumn{1}{S{Data}{l}}{JD{\textunderscore}status} & 
   \multicolumn{1}{S{Data}{l}}{Char} & 
   \multicolumn{1}{S{Data}{r}}{2} & 
   \multicolumn{1}{S{Data}{r}}{243} & 
   \multicolumn{1}{S{Data}{l}}{Status: job destruction}
\\
   \multicolumn{1}{S{RowHeader}{r}}{47} & 
   \multicolumn{1}{S{Data}{l}}{JF{\textunderscore}status} & 
   \multicolumn{1}{S{Data}{l}}{Char} & 
   \multicolumn{1}{S{Data}{r}}{2} & 
   \multicolumn{1}{S{Data}{r}}{245} & 
   \multicolumn{1}{S{Data}{l}}{Status: net job flows}
\\
   \multicolumn{1}{S{RowHeader}{r}}{48} & 
   \multicolumn{1}{S{Data}{l}}{R{\textunderscore}status} & 
   \multicolumn{1}{S{Data}{l}}{Char} & 
   \multicolumn{1}{S{Data}{r}}{2} & 
   \multicolumn{1}{S{Data}{r}}{247} & 
   \multicolumn{1}{S{Data}{l}}{Status: recalls}
\\
   \multicolumn{1}{S{RowHeader}{r}}{49} & 
   \multicolumn{1}{S{Data}{l}}{S{\textunderscore}status} & 
   \multicolumn{1}{S{Data}{l}}{Char} & 
   \multicolumn{1}{S{Data}{r}}{2} & 
   \multicolumn{1}{S{Data}{r}}{249} & 
   \multicolumn{1}{S{Data}{l}}{Status: separations}
\\
   \multicolumn{1}{S{RowHeader}{r}}{50} & 
   \multicolumn{1}{S{Data}{l}}{Z{\textunderscore}NA{\textunderscore}status} & 
   \multicolumn{1}{S{Data}{l}}{Char} & 
   \multicolumn{1}{S{Data}{r}}{2} & 
   \multicolumn{1}{S{Data}{r}}{251} & 
   \multicolumn{1}{S{Data}{l}}{Status: average periods of non-employment for accessions}
\\
   \multicolumn{1}{S{RowHeader}{r}}{51} & 
   \multicolumn{1}{S{Data}{l}}{Z{\textunderscore}NH{\textunderscore}status} & 
   \multicolumn{1}{S{Data}{l}}{Char} & 
   \multicolumn{1}{S{Data}{r}}{2} & 
   \multicolumn{1}{S{Data}{r}}{253} & 
   \multicolumn{1}{S{Data}{l}}{Status: average periods of non-employment for new hires}
\\
   \multicolumn{1}{S{RowHeader}{r}}{52} & 
   \multicolumn{1}{S{Data}{l}}{Z{\textunderscore}NR{\textunderscore}status} & 
   \multicolumn{1}{S{Data}{l}}{Char} & 
   \multicolumn{1}{S{Data}{r}}{2} & 
   \multicolumn{1}{S{Data}{r}}{255} & 
   \multicolumn{1}{S{Data}{l}}{Status: average periods of non-employment for recalls}
\\
   \multicolumn{1}{S{RowHeader}{r}}{53} & 
   \multicolumn{1}{S{Data}{l}}{Z{\textunderscore}NS{\textunderscore}status} & 
   \multicolumn{1}{S{Data}{l}}{Char} & 
   \multicolumn{1}{S{Data}{r}}{2} & 
   \multicolumn{1}{S{Data}{r}}{257} & 
   \multicolumn{1}{S{Data}{l}}{Status: average periods of non-employment for separations}
\\
   \multicolumn{1}{S{RowHeader}{r}}{54} & 
   \multicolumn{1}{S{Data}{l}}{Z{\textunderscore}W2{\textunderscore}status} & 
   \multicolumn{1}{S{Data}{l}}{Char} & 
   \multicolumn{1}{S{Data}{r}}{2} & 
   \multicolumn{1}{S{Data}{r}}{259} & 
   \multicolumn{1}{S{Data}{l}}{Status: average earnings of end-of-period employees}
\\
   \multicolumn{1}{S{RowHeader}{r}}{55} & 
   \multicolumn{1}{S{Data}{l}}{Z{\textunderscore}W3{\textunderscore}status} & 
   \multicolumn{1}{S{Data}{l}}{Char} & 
   \multicolumn{1}{S{Data}{r}}{2} & 
   \multicolumn{1}{S{Data}{r}}{261} & 
   \multicolumn{1}{S{Data}{l}}{Status: average earnings of full-quarter employees}
\\
   \multicolumn{1}{S{RowHeader}{r}}{56} & 
   \multicolumn{1}{S{Data}{l}}{Z{\textunderscore}WFA{\textunderscore}status} & 
   \multicolumn{1}{S{Data}{l}}{Char} & 
   \multicolumn{1}{S{Data}{r}}{2} & 
   \multicolumn{1}{S{Data}{r}}{263} & 
   \multicolumn{1}{S{Data}{l}}{Status: average earnings of transits to full-quarter status}
\\
   \multicolumn{1}{S{RowHeader}{r}}{57} & 
   \multicolumn{1}{S{Data}{l}}{Z{\textunderscore}WFS{\textunderscore}status} & 
   \multicolumn{1}{S{Data}{l}}{Char} & 
   \multicolumn{1}{S{Data}{r}}{2} & 
   \multicolumn{1}{S{Data}{r}}{265} & 
   \multicolumn{1}{S{Data}{l}}{Status: average earnings of separations from full-quarter status}
\\
   \multicolumn{1}{S{RowHeader}{r}}{58} & 
   \multicolumn{1}{S{Data}{l}}{Z{\textunderscore}WH3{\textunderscore}status} & 
   \multicolumn{1}{S{Data}{l}}{Char} & 
   \multicolumn{1}{S{Data}{r}}{2} & 
   \multicolumn{1}{S{Data}{r}}{267} & 
   \multicolumn{1}{S{Data}{l}}{Status: average earnings of full-quarter new hires}
\\
   \multicolumn{1}{S{RowHeader}{r}}{59} & 
   \multicolumn{1}{S{Data}{l}}{Z{\textunderscore}dWA{\textunderscore}status} & 
   \multicolumn{1}{S{Data}{l}}{Char} & 
   \multicolumn{1}{S{Data}{r}}{2} & 
   \multicolumn{1}{S{Data}{r}}{269} & 
   \multicolumn{1}{S{Data}{l}}{Status: average change in total earnings for accessions}
\\
   \multicolumn{1}{S{RowHeader}{r}}{60} & 
   \multicolumn{1}{S{Data}{l}}{Z{\textunderscore}dWS{\textunderscore}status} & 
   \multicolumn{1}{S{Data}{l}}{Char} & 
   \multicolumn{1}{S{Data}{r}}{2} & 
   \multicolumn{1}{S{Data}{r}}{271} & 
   \multicolumn{1}{S{Data}{l}}{Status: average change in total earnings for separations}
\\
\end{longtable}
\end{center}

% ====================end of output====================
 \newpage \input{%
% \mypath/mn_sic_division_v23_fuzzed.freq.tex}
% 
% \newpage
% 
% \subsection{North Carolina}
% 
% \input{\mypath/nc_county_v23_fuzzed.tex} \newpage \input{\mypath/nc_county_v23_fuzzed.freq.tex}
% \newpage \input{\mypath/nc_sic_division_v23_fuzzed.tex} \newpage \input{%
% \mypath/nc_sic_division_v23_fuzzed.freq.tex}
% 
% \subsection{Texas}
% 
% \input{\mypath/tx_county_v23_fuzzed.tex} \newpage \input{\mypath/tx_county_v23_fuzzed.freq.tex}
% \newpage \input{\mypath/tx_sic_division_v23_fuzzed.tex} \newpage \input{%
% \mypath/tx_sic_division_v23_fuzzed.freq.tex}


%\subsection{Example statistics}
%\label{sec:example_statistics}
%
%Table~\Vref{tab:table_example} shows the extreme case of statistics for a
%small county and for one gender only.%
%%
%%\footnote{The exact county and gender remain anonymous for data
%%  confidentiality reasons.} 
%%
%Two types of confidentiality measures are
%implemented within the same table. Ten cells have been distorted through
%the injection of noise described earlier. Furthermore, one cell has been
%suppressed because the statistics in that cell is based on too few
%individuals.
%
%
%  % this is from July Tables for CA, table9.html

\begin{table}[htbp]
\small
\begin{center}
  \caption{Example table: ALEXANDER, IL, Men only, by age group}
%  \caption{Anonymous county, one gender only, by age group}
  \label{tab:table_example}
  \begin{tabular}{rrrrrrrrrrrrr}
\hline
\\[-.3cm]
        &          & &        & &           & &     & & Average & &Average & \\
        &Beginning-& &        & &           & &     & &earnings & &earnings& \\
        &of-period & &  Job   & &    Job    & & New & &of full- & &of full-& \\
        &employment& &creation& &destruction& &hires& & quarter & &quarter & \\
        &          & &        & &           & &     & &employees& &  new   & \\
        &          & &        & &           & &     & &         & & hires  & \\
\hline                                                                        
All Ages&     1,266& &      67& &         76& &  192& &    6,764& &   3,945& \\
14-18   &        21&*&      10&*&          1&*&   22& &    1,794&*&     921&*\\
19-21   &        44& &       9& &          5& &   21& &    3,835& &   3,645&*\\
22-24   &        74& &      12& &         14& &   19& &    4,020& &   3,425& \\
25-34   &       286& &      26& &         32& &   58& &    6,041& &   5,128& \\
35-44   &       356& &      22& &         27& &   38& &    6,477& &   4,827& \\
45-54   &       278& &      12& &         12& &   25& &    8,644& &   1,861&*\\
55-64   &       158& &       2& &          9& &    6& &    8,592& &   4,207&*\\
65+     &        50& &       1& &          4& &     &d&    3,379& &   1,216&*\\
\hline                                                                        
\multicolumn{13}{l}{\footnotesize  * indicates significant distortion   is necessary to preserve confidentiality}\\[-.15cm]
\multicolumn{13}{l}{\footnotesize  d indicates an estimate is based on less
  than 3 employees in the at-risk group}\\[-.15cm]
\multicolumn{13}{l}{\footnotesize  n indicates an estimate is not defined because no employees are in the relevant category}\\[-.15cm]
\end{tabular}
\end{center}
\end{table}

%%% Local Variables: 
%%% mode: latex
%%% TeX-master: "qwi-overview"
%%% End: 

%  
%  Figure~\Vref{fig:figure_example} presents the data in a different way.
%  The graph shows net job creation rates by county, for both genders, and
%  for the youngest workers. It shows large dispersion across the state, and
%  highlights the value added from joining demographic and firm-level data
%  to form new statistics.
%
%  % for qwi-overview.tex
% Example graphics
\begin{figure}[htbp]
  \begin{center}
\centerline{\includegraphics[width=\textwidth]{\mypath/il_gmap_Ages_19-21_All_net.eps}} 
    
    \caption{Example graph}
    \label{fig:figure_example}
  \end{center}
\end{figure}
%%% Local Variables: 
%%% mode: latex
%%% TeX-master: "qwi-overview"
%%% End: 


%%% Local Variables: 
%%% mode: latex
%%% TeX-master: "qwi-overview"
%%% End: 

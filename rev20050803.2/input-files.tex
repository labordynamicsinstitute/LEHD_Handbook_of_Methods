%TCIDATA{Version=4.00.0.2321}
%TCIDATA{LaTeXparent=0,0,sw-edit.tex}
                      
%TCIDATA{ChildDefaults=chapter:1,page:1}
%
% Time-stamp: <05/03/15 22:49:16 vilhuber>

The underlying data are wage records extracted from UI administrative files from each participating state, as
well as from the (typically independently created) files from the Quarterly
Census of Employment and Wages (QCEW, formerly known as 'ES-202'). These
data are received by LEHD on a quarterly basis, with historical time series
extending back to the early 1990s for some states. 

% use Erika's list of caveats here?

\section{Wage records: UI}
\label{sec:input_files:ui}

Wage records correspond to the report of an individual's
UI\index{UI}-covered earnings by an employing entity, identified by a state
UI account number (called State Employer Identification Number, or
'SEIN'\index{SEIN} in the LEHD system). An individual's {UI} wage record
appears if at least one employer reports earnings of at least one dollar
for that individual during the quarter. Thus, the job
must produce at least one dollar of {UI}-covered earnings during a given
quarter to count in the LEHD system. Maximum earnings reported are defined in a
specific state's unemployment insurance system, and observed top-coding
varies across states and time.

A record is completed with information on the individual's Social Security
Number (later masked within the LEHD system), first name, last name, and
middle initial. A few states include additional information: the firm's
reporting unit, or establishment (called SEINUNIT in the LEHD system),
available for Minnesota, and a crucial component to the Unit-to-Worker
impute described later; weeks worked, available for some years in Florida;
hours worked, available for Washington state.


\section{Employer reports: ES202}
\label{sec:input_files:es202}

% from ECF intro
The employer reports are  based on information from each state's
Department of Employment Security. The data are collected as part of the
Covered Employment and Wages (CEW)\index{CEW} program, also known as the
ES-202 program, which is administered by the U.S. Bureau of Labor
Statistics (BLS)\index{BLS}. This cooperative program between the states and the federal
government collects employment, payroll, and location information from
employers covered by state unemployment insurance programs.
%
These are the same records that form the basis of the Quarterly
Census of Employment and Wages (QCEW), but are referred to in the LEHD
system by their old acronym ``ES202''.  The fundamental unit of these files
is a 'reporting unit', typically taken to be equivalent to a
'establishment'. Most firms only have one establishment ('single-units'),
but most employment is in firms with multiple establishments
('multi-units'). One report per establishment per quarter is filed%
%
\footnote{These data are also used to compile the  Business Employment
Dynamics (BED) data at the BLS.}%
%
.


The information contained in these files has increased substantially over
the years. Employers report wages subject to statutory payroll
taxes on this form, together with some other information. Common to all years, and critical to LEHD processing, are
information on the employer's identity (the SEIN), the reporting unit's
identify (SEINUNIT), ownership information, employment on the 12th of each month covered by the
quarter, and total wages paid over the course of the quarter. Additional
information pertains to industry classifications (initially SIC, and later
NAICS). Other information include the federal EIN, geography both at a high
level (county or MSA) and low level (street address). A recent expansion of
the record layout has increased the informational content substantially.


\section{Administrative demographic information: PCF}
\label{sec:input_files:pcf}

The UI and ES202 files are the core data files describing the economic
activity of individuals and firms. Although combined, these files contain a
tremendous amount of detail on the economic activity, they contain little
or no demographic information on the individuals. This information comes
from a third administrative data source, compiled by ARRS/PRED.
%\marginpar{\tiny [correct name???]} %
The Person Characteristics File (PCF) 
%is used to define a number
%of frames at the Census Bureau, in particular the frame of each decade's
%Census. It 
contains information on gender, date of birth, place of birth,
citizenship, and race, most of which is extraced in turn from the Social
Security Administration's Numident file. Other information contains place
of residence for several years culled from other administrative sources.

\section{Demographic products}
\label{cha:Demographic}

Many individuals have appeared in at least one of the eligible Census demographic 
products, and their detailed demographic information from 
%those surveys 
 the 1984, 1990-1993, and 1996 SIPP panels as well as
from March Demographic Supplement to the Current Population
Survey (CPS\index{CPS}) from 1983 onwards
%
can be linked to the extensive longitudinal data gleaned from the state
records. They are used by the ICF.


%
%

\section{Economic censuses and annual surveys}
\index{Annual surveys!manufacturing}
\index{Annual surveys!trade}
\index{Annual surveys!service}
\index{Annual surveys!transportation}
\index{Annual surveys!communication}
\index{Economic census}

%\marginpar{\tiny (2002?) economic censuses. Does the ECF use these?}
These data include the complete 1987, 1992 and 1997 economic censuses, all 
annual surveys of manufacturing, service, trade, transportation and 
communication industries and selected, approved fields from the Census 
Bureau's %establishment master file.%
%\index{Establishment master file}
Business Register.

Linkage to these data is based upon exact EIN\index{EIN} matches,
supplemented with statistical matching to recover
establishments\index{establishment}. 


\section{Identifiers and their longitudinal consistency}
\label{sec:coding}


Both the wage records and employer reports are administrative data -
comprehensive, but sometimes less than perfect. In particular, spurious
changes in the numerical entity identifers used for longitudinal matching
can have a significant impact on most economic uses of the data.

\subsection{Scope of data and identifiers}
\label{sec:scope}

In the LEHD system,  a person is identified initially by the Social
Security Number, and later by their Protected Identification Key
(PIK). This identifier is national in scope, and individuals can be tracked
across all states and time periods. Not all individuals are in-scope at all
times. To be included in the wage record database, an individual's job must
be covered by a state's unemployment insurance system. The prime exclusions
are agriculture and to some extent the public sector. Coverage varies across states and
time, although on average, 98\% of all private-sector jobs are covered.
 \citet{tp-2002-16} provides a survey of coverage for a subset of the
current participant states in the LEHD system.

A 'firm' is identified primarily by their state UI
account number (SEIN). A single legal ``firm'' might have multiple SEINs but regardless of its
operations in other states a legal firm has a different unemployment
insurance account in each state in which it has statutory employees.
In particular the QWI are based exclusively on
SEIN-based entities and their associated establishments. Since a SEIN is
specific to a state, the QWI cannot account for movements of individuals
across state lines, but within the same company. Time-consistency is also
not guaranteed, since the tax number associated with a firm can change (see
later discussions).  Again, the coverage caveat mentioned also applies. 

The restriction to SEIN does not apply to the Infrastructure Files. For
some states, the federal Employer Identification Number, used for federal
tax purposes, is available, and reported on the Employer Characteristics
File (ECF).  Links to the Census Business Register (BR)
allow to map entities from the QCEW to larger companies across state lines
(see Section~\ref{sec:BRB} for more information on the Business Register Bridge).

\subsection{Error correction of person identifiers}
\label{sec:ssnedit}

Coding errors in the SSN can occur for a variety of
reasons. A survey of 53 state employment security agencies in the United
States over the 1996-1997 time period found that most errors are due to
coding errors by employers, but that when errors were attributable to state
agencies, data entry was the culprit \citep[pg. ii]{BLS1997}. %(Bureau of Labor Statistics 1997b, pg. ii). 
The report noted that 38\% of all records were entered by key entry, while
another 11\% were read in by optical character readers. OCR and magnetic
media tend to be less prone to errors. 

Errors can be random digit coding errors that do not persist, typically
generated when data are transferred from one format (paper) to another
(digital), or can be persistent, typically occurring when a firm's payroll
system contains an erroneous SSN. While the latter is hard to identify and
to correct, the LEHD system uses statistical matching techniques to correct
for spurious and non-persistent coding errors. Both the incidence of errors
and the success rate of the error correction methods differs widely by
state. In particular, it depends critically on the availability of name
information on the wage records. 

\citet{AbowdVilhuber2005} describe and analyze the process as it was
applied to data provided by the state of California. The process verified
over half a billion records. The number of records that are recoded is
slightly less than 10 percent of the total number of unique individuals
appearing in the original data, and only a little more than 0.5\% of all
wage records. The authors estimate that the true error rate in their data
is higher, in part due to the conservative setup of the process. Over
800,000 job history interruptions in the original data are eliminated,
representing 0.9\% of all jobs, but 11\% of all interrupted jobs. Despite the small number of records that are found to be miscoded, the impact on flow statistics can be large. Accessions in the uncorrected data are overestimated by 2\%, and recalls are biased upwards by nearly 6\%. Payroll for accessions and separations are biased upward by up to 7 percent. 

The wage record editing occurs prior to the construction of any of the
Infrastructure Files, for two reasons. First, the wage record edit process
requires access to the original Social Security Numbers as well as to the
names on the wage records, both of which are replaced by Personal
Identifier Numbers (PIK) or stripped off very early in the processing of
wage records. The wage record editing process takes place in a secure and
separate area from the rest of the LEHD processing, to avoid any
commingling of SSN-laden data with anonymized data. Second, because the
identifier changes underlying the wage record edit are deemed spurious, and
because individuals have no economic reason at all to change Social
Security Numbers, there is little
ambiguity about the applicability of the edit. This is different from the
editing of firm identifiers (see the next section).  

\subsection{Correcting for changes in firm identifiers}
\label{sec:input_spf}

Firms in the
QCEW system are identified by a (UI tax) account number attributed by the
state. As with all firm identities, an account number can change for a
number of reasons over time, not all of which are distinguishable economic
entities for the purpose of these statistics. State administrative units take great care to
follow the legal entities in their system, but account numbers may
nevertheless change for reasons which economists may not consider
legitimate economic reasons. For instance, a simple change in ownership of
a firm may lead to a change in the account number. 

Because changes in the firm identifiers are correlated with some elements
of economic choice, albeit imperfectly, they are not imposed on the entire
LEHD Infrastructure Files. Rather, an auxiliary file, the
Successor-Predecessor File, is created that allows for the  selective
application of such edits. This file is produced after the first of the Infrastructure
Files have been created, and is described later in this document.


%%% Local Variables: 
%%% mode: latex
%%% TeX-master: "qwi-overview"
%%% End: 

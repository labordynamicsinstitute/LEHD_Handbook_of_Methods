%TCIDATA{LaTeXparent=0,0,QWI_methods.tex}
% -*- latex -*- 
%
% Time-stamp: <01/10/17 23:25:20 vilhuber> 
%              Automatically adjusted if using Xemacs
%              Please adjust manually if using other editors
%
% future.tex
% Part of QWI_methods.tex

The ability of LEHD to combine unemployment insurance records from various 
states into a single data warehouse enables us to investigate questions that 
one state alone cannot pursue. One example of this type of research is the 
movement of workers across state borders. In addition to the industry where 
an individual is employed, Census data allow us to examine the demographic 
characteristics of these workers. Appendix~\Vref{app:mobility} shows preliminary results for 
workers that have moved across state lines and the proportion of those 
workers that can be identified in Census datasets.

The transition from SIC to NAICS based industry coding requires a relatively 
minor reconfiguration of the current job flow processing. Appendix~\Vref{app:sicnaics} 
contains a list of SIC codes and their NAICS equivalent. Additional 
information is also provided for detailed industry groups that have changed 
substantially under the NAICS classification system.

Census demographic surveys provide additional information on employees in 
each state, but the employer information at Census is also extensive. The 
first step towards linking employers from state data to Census business data 
has been taken for Illinois. The Bureau of the Census maintains a list of 
known business establishments in the United States called the Standard 
Statistical Establishments List (SSEL). The SSEL provides the crucial link 
from the federal EIN found on state data to Census internal firm 
identifiers. The overall match rate is very good (98{\%}), but there are 
differences in the classification of single and multi-unit firms. Detailed 
results are presented in Appendix~\Vref{app:linkuissel}. 

%%% Local Variables: 
%%% mode: latex
%%% TeX-master: "QWI_methods"
%%% End: 

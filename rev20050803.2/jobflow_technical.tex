%TCIDATA{LaTeXparent=0,0,sw-edit.tex}

% -*- latex -*- 
%
% Time-stamp: <02/04/09 11:08:32 vilhuber> 
%              Automatically adjusted if using Xemacs
%              Please adjust manually if using other editors
%
% jobflow_technical.tex
% Responsible: Paul/Bryce
% Part of QWI_methods.tex

The following briefly describes the programs used in the production of the
QWI estimates. 

\paragraph{jobflow01.sas }

(a) use EHF\index{EHF} (Appendix~\Vref{app:ehf_technical}) and
ICF\index{ICF} (Appendix~\Vref{app:icf_technical}) to create a complete
work history for each individual; (b) define age categories over which to
compute job and worker flows; (c) compute individual-level flow statistics.

\paragraph{jobflow02.sas }

Calculate job flow statistics at the ``firm-level,'' i.e. for each \textit{SEIN}, by 
year and quarter and by sex and age group.

\paragraph{jobflow03.sas }

(a) generate all intermediate statistics necessary for computing final 
statistics; (b) check identities for all relevant job and worker flow 
statistics.

\paragraph{jobflow04.sas }

(a) add UI-system county and industry information for each firm (\textit{SEIN%
}) from the employer characteristics file (ECF)\index{ECF}; (b) create a new series of
variables equal to the original variables multiplied by \textit{SEIN}
specific noise (or fuzz) factors.\index{fuzz factor}

\paragraph{jobflow05.sas }

Produce actual and noise adjusted flow statistics at the county and industry 
level.

\paragraph{jobflow06.sas }

(a) for each file (year-quarter-county and year-quarter-industry), produce
disclosure status flags for each variable; (b) calculate the percentage
difference between true and noisy values for each variable.

\paragraph{jobflow07a.sas -- jobflow07f.sas}

\index{raking} Rake employment dynamic estimates to be consistent with
BLS\index{BLS} published county and state totals from CEW\index{CEW}.
Output data files contain observations by year and quarter for: (a)
non-missing county codes (e.g., codes which do not correspond missing
county identifiers or other measures of geography), (b) valid county codes
for which the BLS raking totals are positive in the current and subsequent
period, and (c) state totals (indicated by county='000').  Tabulated
estimates are flagged where cell suppression\index{cell suppression} or
distortion\index{distortion} occurs.

\paragraph{jobflow08a.sas - jobflow08f.sas}

\index{raking} Rake employment dynamic estimates to be consistent with BLS
published industry division and state totals.  Output data files contain
observations by year and quarter for: (a) all industry divisions including
``other'' and (b) state totals (indicated by \textit{sic\_division}=' ',
blank).  Tabulated estimates are flagged where cell suppression\index{cell
  suppression} or distortion\index{distortion} occurs.

%%% Local Variables: 
%%% mode: latex
%%% TeX-master: "QWI_methods"
%%% End: 

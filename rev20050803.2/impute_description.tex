%TCIDATA{Version=4.00.0.2321}
%TCIDATA{LaTeXparent=0,0,sw-edit.tex}
                      
%
% Time-stamp: <05/03/09 21:53:10 vilhuber>

\subsection{A Probability Model for Employment Location}

\subsubsection{Definitions}

Let $i=1,...,I$ index workers (PIKs), $j=1,...,J$ index firms (SEINs), and $%
t=1,...,T$ index time (quarters). \ \ Let $R_{jt}$ denote the number of
active establishments (units) at firm $j$ in quarter $t,$ let $\mathfrak{R}%
=\max_{j,t}R_{jt},$ and $r=1,...,\mathfrak{R}$ index establishments
(SEINUNITs). Note the index $r$ is nested within $j.$ Let $N_{jrt}$ denote
the quarter $t$ employment of unit $r$ in firm $j.$ Finally, if worker $i$
was employed at firm $j$ in $t,$ denote by $y_{ijt}$ the unit at which she
was employed$.$

Now we define some useful sets. Let $\mathcal{J}_{t}$ denote the set of
firms active in quarter $t,$ let $\mathcal{I}_{jt}$ denote the set of
individuals employed at firm $j$ in quarter $t,$ let $\mathcal{R}_{jt}$
denote the set of active ($N_{jrt}>0$) units at firm $j$ in $t,$ and let $%
\mathcal{R}_{jt}^{i}\subset $ $\mathcal{R}_{jt}$ denote the set of active
units that are feasible for worker $i$. Feasibility is defined as follows. A
unit $r\in \mathcal{R}_{jt}^{i}$ if $N_{jrs}>0$ for every quarter $s$ that $%
i $ was employed at $j.$

\subsubsection{The Probability Model}

Let $p_{ijrt}=\Pr \left( y_{ijt}=r\right) .$ At the core of the model is the
probability statement:%
\begin{equation}
p_{ijrt}=\frac{e^{\alpha _{jrt}+x_{ijrt}^{\prime }\beta }}{\sum_{s\in 
\mathcal{R}_{jt}^{i}}e^{\alpha _{jst}+x_{ijst}^{\prime }\beta }}  \label{p}
\end{equation}%
where $\alpha _{jrt}$ is a unit- and quarter-specific effect, $x_{ijrt}$ is
a time-varying vector of characteristics of the worker and unit, and $\beta $
measures the effect of characteristics on the probability of being employed
at a particular establishment. In the current implementation, $x_{ijrt}$ is
a linear spline in the (great-circle) distance between worker $i$'s
residence and the physical location of unit $r.$ The spline has knots at 25,
50, and 100 miles. A convenient characteristic of (\ref{p}) is that the
log-odds ratio is linear:%
\begin{equation}
\log \left( \frac{p_{ijrt}}{p_{ijst}}\right) =\left( \alpha _{jrt}-\alpha
_{jst}\right) +\left( x_{ijrt}-x_{ijst}\right) ^{\prime }\beta
\end{equation}

We use (\ref{p}) to define the likelihood%
\begin{equation}
p\left( y|\alpha ,\beta ,x\right) =\prod_{t=1}^{T}\prod_{j\in \mathcal{J}%
_{t}}\prod_{i\in \mathcal{I}_{jt}}\prod_{r\in \mathcal{R}_{jt}^{i}}\left(
p_{ijrt}\right) ^{d_{ijrt}}  \label{likelihood}
\end{equation}%
where 
\begin{equation}
d_{ijrt}=\left\{ 
\begin{array}{cc}
1 & \text{if }y_{ijt}=r \\ 
0 & \text{otherwise}%
\end{array}%
\right.  \label{d}
\end{equation}%
and where $y$ is the appropriately-dimensioned vector of the outcome
variables $y_{ijt},$ $\alpha $ is the appropriately-dimensioned vector of
the $\alpha _{jrt},$ and $x$ is the appropriately-dimensioned matrix of
characteristics $x_{ijrt}$.

We adopt a hierarchical Bayesian model for the $\alpha _{jrt}$ based on
employment counts $N_{jrt}.$ Let $N_{jt}$ denote the $R_{jt}\times 1$ vector
of employment counts at firm $j$ in $t.$ Assume $N_{jt}\sim $Multin$\left(
\pi _{jt}\right) $, where $\pi _{jt}$ is an $R_{jt}\times 1$ vector of
parameters $\pi _{jrt}$ such that $\sum_{r\in \mathcal{R}_{jt}}\pi _{jrt}=1$
for all $j,t$ . That is,%
\begin{equation}
p\left( N_{jt}|\pi _{jt}\right) \varpropto \prod_{r\in \mathcal{R}%
_{jt}}\left( \pi _{jrt}\right) ^{N_{jrt}}.  \label{p(N|pi)}
\end{equation}%
We adopt a non-informative Dirichlet prior for the $\pi _{jt}$ with
hyperparameters $\theta _{jt}.$ That is, for all $j,t$ such that $R_{jt}=R,$%
\begin{equation}
p\left( \pi _{jt}\right) \varpropto \prod_{r=1}^{R}\left( \pi _{jrt}\right)
^{\theta _{jrt}-1}  \label{p(pi)}
\end{equation}%
with $\theta _{jrt}=0$ for all $j,r,t.$ The Dirichlet is of course the
conjugate prior for the Multinomial distribution, and when $\theta _{jrt}=0$
for all $j,r,t$ \ it is uniform in the $\log \left( \pi _{jt}\right) .$ It
is also improper in this case, but the resulting posterior distribution is
proper as long as there is at least one observation in each of the $R$
categories.\footnote{%
See Gelman, Carlin, Stern, and Rubin (1997, p.76)} This posterior is%
\begin{eqnarray}
p\left( \pi _{jt}|N_{jt}\right) &\varpropto &p\left( \pi _{jt}\right)
p\left( N_{jt}|\pi _{jt}\right)  \notag \\
&\varpropto &\prod_{r\in \mathcal{R}_{jt}}\left( \pi _{jrt}\right)
^{N_{jrt}-1}  \label{p(pi|N)}
\end{eqnarray}%
which is Dirichlet with ``parameter'' vector $N_{jt}.$ Now define%
\begin{equation}
\alpha _{jrt}=\left\{ 
\begin{array}{cc}
0 & r=1 \\ 
\log \left( \frac{\pi _{jrt}}{\pi _{j1t}}\right) & r=2,3,...,R_{jt}%
\end{array}%
\right. .  \label{a-def}
\end{equation}%
Thus the $\alpha _{jrt}$ are the inverse-logit of the Dirichlet random
variables $\pi _{jrt}.\footnote{%
The $\alpha _{jrt}$ can also be interpreted as the log-odds ratio of
employment at unit $r$ vs. unit 1 when $\beta =0$ (distance doesn't matter)
or when $x_{ijrt}=x_{ij1t}$ (the units are equidistant).}$ Let $\alpha _{jt}$
denote the $R_{jt}\times 1$ vector of the parameters $\alpha _{jrt}.$
Equations (\ref{p(pi)}) and (\ref{a-def}) implicitly define $p\left( \alpha
_{jt}|N_{jt}\right) .$ Taking the product of these across all $j$ and $t$
yields%
\begin{equation}
p\left( \alpha |N\right) =\prod_{t=1}^{T}\prod_{j\in \mathcal{J}_{t}}p\left(
\alpha _{jt}|N_{jt}\right)  \label{p(a|N)}
\end{equation}%
where $N$ is the appropriately-dimensioned vector of all the $N_{jt}.$

The object of interest is the joint posterior distribution of $\alpha $ and $%
\beta .$ We assume a uniform prior on $\beta ,$ $p\left( \beta \right)
\varpropto 1.$ Our characterization of $p\left( \alpha ,\beta |x,y,N\right) $
is based on the factorization 
\begin{eqnarray}
p\left( \alpha ,\beta |x,y,N\right) &=&p\left( \alpha |N\right) p\left(
\beta |\alpha ,x,y\right)  \notag \\
&\varpropto &p\left( \alpha |N\right) p\left( \beta \right) p\left( y|\alpha
,\beta ,x\right)  \notag \\
&\varpropto &p\left( \alpha |N\right) p\left( y|\alpha ,\beta ,x\right) .
\label{p(a,b|x,y,N)}
\end{eqnarray}
Thus the joint posterior (\ref{p(a,b|x,y,N)}) is completely characterized by
the posterior of $\alpha $ in (\ref{p(a|N)}) and the likelihood of $y$ in (%
\ref{likelihood}). Note (\ref{likelihood}) and (\ref{p(a,b|x,y,N)}) assume
that the employment counts $N$ affect employment location $y$ only through
the parameters $\alpha .$

\subsubsection{Estimation}

We approximate the joint posterior $p\left( \alpha ,\beta |x,y,N\right) $ at
the posterior mode. In particular, we estimate the posterior mode of $%
p\left( \beta |\alpha ,x,y\right) $ evaluated at the posterior mode of $%
\alpha .$ The posterior mode of $\alpha $ is simply defined -- the posterior
modal values of the $\pi _{jrt}$ are the employment shares at unit $r$ of
firm $j$ in quarter $t.$ From these we compute the posterior modal values of
the $\alpha _{jrt}$ using (\ref{a-def}). We then maximize the log posterior
density%
\begin{equation}
\log p\left( \beta |\alpha ,x,y\right) \varpropto \sum_{t=1}^{T}\sum_{j\in 
\mathcal{J}_{t}}\sum_{i\in \mathcal{I}_{jt}}\sum_{r\in \mathcal{R}%
_{jt}^{i}}d_{ijrt}\left( \alpha _{jrt}+x_{ijrt}^{\prime }\beta -\log \left(
\sum_{s\in \mathcal{R}_{jt}^{i}}e^{\alpha _{jst}+x_{ijst}^{\prime }\beta
}\right) \right)  \label{log-posterior}
\end{equation}%
evaluated at the posterior modal values of the $\alpha _{jrt},$ using a
modified Newton-Raphson method. The mode-finding exercise is based on the
gradient and Hessian of (\ref{log-posterior}). These are:%
\begin{eqnarray}
\frac{\partial \log p\left( \beta |\alpha ,x,y\right) }{\partial \beta }
&\varpropto &\sum_{t=1}^{T}\sum_{j\in \mathcal{J}_{t}}\sum_{i\in \mathcal{I}%
_{jt}}\sum_{r\in \mathcal{R}_{jt}^{i}}d_{ijrt}\left( x_{ijrt}-\frac{%
\sum_{s\in \mathcal{R}_{jt}^{i}}x_{ijst}e^{\alpha _{jst}+x_{ijst}^{\prime
}\beta }}{\sum_{s\in \mathcal{R}_{jt}^{i}}e^{\alpha _{jst}+x_{ijst}^{\prime
}\beta }}\right)  \notag \\
&=&\sum_{t=1}^{T}\sum_{j\in \mathcal{J}_{t}}\sum_{i\in \mathcal{I}%
_{jt}}\sum_{r\in \mathcal{R}_{jt}^{i}}d_{ijrt}\left( x_{ijrt}-\sum_{s\in 
\mathcal{R}_{jt}^{i}}p_{ijst}x_{ijst}\right)  \notag \\
&=&\sum_{t=1}^{T}\sum_{j\in \mathcal{J}_{t}}\sum_{i\in \mathcal{I}%
_{jt}}\sum_{r\in \mathcal{R}_{jt}^{i}}d_{ijrt}\left( x_{ijrt}-\bar{x}%
_{ijt}\right)  \label{gradient}
\end{eqnarray}%
\begin{eqnarray}
\frac{\partial ^{2}\log p\left( \beta |\alpha ,x,y\right) }{\partial \beta
^{2}} &\varpropto &-\sum_{t=1}^{T}\sum_{j\in \mathcal{J}_{t}}\sum_{i\in 
\mathcal{I}_{jt}}\sum_{r\in \mathcal{R}_{jt}^{i}}\left( 
\begin{array}{c}
\frac{d_{ijrt}}{\left( \sum_{s\in \mathcal{R}_{jt}^{i}}e^{\alpha
_{jst}+x_{ijst}^{\prime }\beta }\right) ^{2}} \\ 
\times \left( 
\begin{array}{c}
\sum_{s\in \mathcal{R}_{jt}^{i}}x_{ijst}x_{ijst}^{\prime }e^{\alpha
_{jst}+x_{ijst}^{\prime }\beta }\sum_{s\in \mathcal{R}_{jt}^{i}}e^{\alpha
_{jst}+x_{ijst}^{\prime }\beta } \\ 
-\sum_{s\in \mathcal{R}_{jt}^{i}}x_{ijst}e^{\alpha _{jst}+x_{ijst}^{\prime
}\beta }\sum_{s\in \mathcal{R}_{jt}^{i}}x_{ijst}^{\prime }e^{\alpha
_{jst}+x_{ijst}^{\prime }\beta }%
\end{array}%
\right)%
\end{array}%
\right)  \notag \\
&=&-\sum_{t=1}^{T}\sum_{j\in \mathcal{J}_{t}}\sum_{i\in \mathcal{I}%
_{jt}}\sum_{r\in \mathcal{R}_{jt}^{i}}d_{ijrt}\sum_{s\in \mathcal{R}%
_{jt}^{i}}p_{ijst}x_{ijst}\left( x_{ijst}-\bar{x}_{ijt}\right) ^{\prime } 
\notag \\
&=&-\sum_{t=1}^{T}\sum_{j\in \mathcal{J}_{t}}\sum_{i\in \mathcal{I}%
_{jt}}\sum_{r\in \mathcal{R}_{jt}^{i}}d_{ijrt}\sum_{s\in \mathcal{R}%
_{jt}^{i}}p_{ijst}\left( x_{ijst}-\bar{x}_{ijt}\right) \left( x_{ijst}-\bar{x%
}_{ijt}\right) ^{\prime }  \notag \\
&=&-\sum_{t=1}^{T}\sum_{j\in \mathcal{J}_{t}}\sum_{i\in \mathcal{I}%
_{jt}}\sum_{r\in \mathcal{R}_{jt}^{i}}p_{ijrt}\left( x_{ijrt}-\bar{x}%
_{ijt}\right) \left( x_{ijrt}-\bar{x}_{ijt}\right) ^{\prime }
\label{hessian}
\end{eqnarray}%
where $\bar{x}_{ijt}=\sum_{s\in \mathcal{R}_{jt}^{i}}p_{ijst}x_{ijst}$ is
the mean value of characteristics $x_{ijrt}$ across the units of firm $j$ in 
$t.$

In practice, (11) is estimated for three firm employment size classes: 1-100
employees, 101-500 employees, and greater than 500 employees, using data for
the State of Minnesota -- the only LEHD\ state data in which the physical
establishment of employment is recorded.

\subsection{Imputing Place of Work}

For all states other than Minnesota, place-of-work for employees is imputed.

\subsubsection{A Sketch of the Imputation Method}

As with all missing data in the LEHD database, place of work is
multiply-imputed. Ignoring temporal considerations, which are treated in the
next Section, we generate $M=10$ implicates of the missing data as follows.
First, using the mean and variance of $\beta $ estimated from those States
where place of work is available, we take 10 draws of $\beta $ from the
normal approximation (at the mode) to $p\left( \beta |\alpha ,x,y\right) .$
Next, using ES-202 employment counts for the units of firm $j$ in quarter $t$
(these are the $N_{jrt}$), we take 10 draws of $\pi _{jt}$ from the
Dirichlet distribution (\ref{p(pi|N)}). These are used to compute 10 values
of $\alpha _{jt}$ using (\ref{a-def}). Note these are draws from the exact
posterior distribution of the $\alpha _{jrt}.$ The drawn values of $\alpha $
and $\beta $ are used to draw 10 imputed values of place of work from the
normal approximation to the posterior predictive distribution%
\begin{equation}
p\left( \tilde{y}|x,y\right) =\int \int p\left( \tilde{y}|\alpha ,\beta
,x,y\right) p\left( \alpha |N\right) p\left( \beta |\alpha ,x,y\right)
d\alpha d\beta .  \label{post-pred}
\end{equation}%
The resulting imputed values reflect two sources of uncertainty: the
fundamental uncertainty of the model (i.e., the unexplained or residual
component of variation in place of work), and the posterior uncertainty in $%
\alpha $ and $\beta $.

\subsubsection{Details}

Using state-level ES-202 data, the set of firms (SEINs) that ever operate
more that one establishment in a given quarter are identified; these SEINs
represent the set of ever-multi-unit firms $\mathcal{J}_{t}$ \footnote{%
Firms with only one establishment represent the trivial case $p_{ijrt}=1$
for $r=1$.}. \ Establishments $r_{jt}$ operating under firms $j$ are
identified in the ES-202 data as well; those establishments with positive
first-month employment in time $t$ characterize the set $\mathcal{R}_{jt}$.
\ An establishment date-of-birth is identified and, in most cases, is the
first date in the ES-202 time series in which the establishment has positive
first-month employment\footnote{%
A different date-of-birth (always strictly less than the one of the first
quarter of positive first-month employment) may be assigned if a known or
suspected establishment identifier change has occured. \ The details of this
identification process is discussed in other LEHD\ technical documents.}. 

State Unemployment Insurance (UI) wage records provide the earnings
histories for employees of the ever-multi-unit firms.  \ From the UI wage
records, one observation -- a candidate for imputation --  is generated for
the \emph{end} of a job history, where a job history,  $h_{ijt}$, is defined
by a continuum of periods of positive earnings for worker $i$ at firm $j$,
during which there are no more than 3 consecutive periods of non-positive
earnings. \ The start-date of the job history is the first date of positive
earnings for the individual at the firm; the end-date is the last date date
of positive earnings. \ These job histories characterize the set $\mathcal{I}%
_{jt}$ defined in Section 1.1. \ 

A job history $h_{ijt}$ (where $t$ indexes the end-date) and an
establishment $r_{jt}\in \mathcal{R}_{jt}$ are paired for all $\ j\in 
\mathcal{J}_{t}$ in time $t$. \ The set of establishments  $\mathcal{R}%
_{jt}^{i}\subset $ $\mathcal{R}_{jt}$ are those which have dates-of-birth
that are less than or equal to the start-date of the job history $h_{ijt}$
for an individual $i$ separating from firm $j$ at time $t$. \ It is the set $%
\mathcal{R}_{jt}^{i}$ from which one unit $\hat{y}_{jt}\in \mathcal{R}%
_{jt}^{i}$ is imputed to job history $h_{ijt}$ and serves as the
establishment identifier for every period over which the job history is
defined\footnote{%
We explicitly rule-out within job history changes in establishment
identifiers in the estimation and implementation of the model. \ Though
transitions of this nature exist in the data, modeling them is beyond the
scope of our current approach.}.

\bigskip The imputation model is applied on a state-by-state basis. \ Once
the data are organized (as outlined above), a set of 10 imputed
establishment identifiers are generated for each job history $h_{ijt}$
ending in every period for which both ES-202 and UI wage records exist. \
Formally, for each implicate, $m=1,..,10$, and for each of the three size
classes, $s=1,2,3$, $\hat{\beta}_{m}^{s}$ is sampled from the normal
approximation 
\begin{equation}
p(\beta ^{s}|\alpha _{MN},x_{MN},y_{MN})  \label{sample beta}
\end{equation}
the posterior predictive distribution of $\beta ^{s}$ conditional\ on
Minnesota. \ The draws from this distribution varies across implicates, but
not across time, firms, and individuals.

Next, for each implicate $m$ and firm $j$ at time $t$, a set of $\alpha
_{jrt}^{\prime }s$ are drawn from%
\begin{equation}
p\left( \alpha _{ST}|N_{ST}\right)   \label{sample alpha}
\end{equation}%
These draws are based on the ES-202 first-month employment totals for all
establishments $r_{jt}\subset $ $\mathcal{R}_{jt}$at firm $j$ for the state
being processed. \ The inital draws of \ $\alpha _{jrt\text{ }}$from this
distribution vary across time and firms but not across job histories. \ \ 

Combining (\ref{sample beta}) and (\ref{sample alpha})

\begin{eqnarray}
&&p\left( \alpha _{ST}|N_{ST}\right) p(\beta ^{s}|\alpha _{MN},x_{MN},y_{MN})
\label{post mode ST} \\
&\approx &p\left( \alpha _{ST}|N_{ST}\right) p(\beta ^{s}|\alpha
_{ST},x_{ST},y_{ST})  \notag \\
&=&p\left( \alpha _{ST},\beta _{ST}|x_{ST},y_{ST},N_{ST}\right)   \notag
\end{eqnarray}%
an approximation of the joint posterior distribution of $\alpha $ and $\beta
^{s}$ (\ref{p(a,b|x,y,N)}) conditional on data from the state being
processed.

Finally, the draws $\hat{\alpha}$ and $\hat{\beta}^{s}$in conjunction with
the establishment, firm, and job history data in $x_{ijrt}$ are used to
construct the $p_{ijrt}$ in (\ref{p}) for all $r\in \mathcal{R}_{jt}^{i}$. \
Recall that the set of establishments $\mathcal{R}_{jt}^{i}$ are unique to a
given job history, so the draws of $\alpha _{jrt\text{ }}$ from (\ref{sample
alpha}) are approriately windsorized so that $\sum_{s\in \mathcal{R}%
_{jt}^{i}}$ $\alpha _{jst}=1$ for all $i$,$j$, and $t$.  A random number $q$%
, $q\backsim uniform[0,1]$, is drawn\ and compared to the cumulative
distribution (ordered by $r$ over $\mathcal{R}_{jt}^{i}$) of the $p_{ijrt}$;
the imputed unit is identified by index $r^{\ast }$ where $%
\sum\limits_{r=1}^{r^{\ast }}p_{ijrt}\leq q$ for $1\leq r^{\ast }\leq R_{ijt}
$.


%%% Local Variables: 
%%% mode: latex
%%% TeX-master: "u2w_master"
%%% TeX-master: "qwi-overview"
%%% End: 

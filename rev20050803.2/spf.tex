%TCIDATA{LaTeXparent=0,0,sw-edit.tex}

% -*- latex -*- 
%
% Time-stamp: <05/02/27 05:59:42 vilhuber> 

% from Vilhuber (2004)

Tracking firms in data, and in particular in administrative data, poses
some challenges. Firms can be born, split, merge, and disappear. Changes in
ownership, of legal and organizational form, and changes in products and
services offered can all lead to legitimate and legal changes in
administrative identifiers. The very boundaries of what constitutes a
single economic entity called a ``firm'' are often fluid.  For the purposes
of the QWI, the fundamental focus is on firms as places of work for
workers, i.e., the firm as employer, and the set of individuals which,
taken together, constitute the ``firm''. Under that premise, flows for a
particular employer should not be affected by purely administrative changes
of the employer identity.  But should it be affected by a merger or the
transfer of a plant from one firm to another? The identification of an
economic, rather than legal successor to a firm becomes an important
distinction.

Administrations are  interested in linking firms for their own
reasons. Payroll taxes can be experience rated, and firms with a higher
payroll tax rate have an incentive to change administrative identity,
becoming an apparently new entity not subject to the predecessor's higher
tax rates. Administrations follow-up on firms, and the QCEW data contain a
field that identifies a possible legal or legally obligated successor. This
data has been used to link firms (Spletzer 2000). 

The U.S. Bureau of Labor Statistics attributes about one third of the
quarter-to-quarter matches that are not directly linked through firm
identifiers to each of (a) the use of the administrative follow-up
described in the preceding paragraph, (b) probabilistic matching (c)
clerical review of otherwise unmatched records \citep{PivetzEtAl2001,ClaytonSpletzter2004}. At the time of writing of this
document, only the administrative follow-up information is available within
the LEHD system. 

Because the LEHD system has access to the wage records of a firm's
employees, it uses a different solution to identify a firm's
successor(s). Since workers can be followed from one employer to the next,
worker flows can be used to identify firms that are economically identical
despite changing administrative identifiers. At one extreme, if all workers
of firm $A$ simultaneously ``separate'', to then be collectively hired
by firm $B$, where they constitute the totality of employment, then
firms $A$ and $B$ are very likely to be the same firm having
changed administrative identifiers. More generally, in order for a firm
$B$ to be the economic successor of firm $A$, some fraction $f(A)$ of
workers leaving firm $A$ must be linkable to firm $B$, and possibly some
fraction $f(B)$ of workers at firm $B$ must have come from firm $A$. How to
set the cutoff levels $f(A)$ and $f(B)$ is the subject of ongoing research,
but for the purposes of the QWI, $f(A)=f(B)=80\%$ was chosen, based on
research by LEHD researchers \citep{tp-2003-09} and consistent
with similar methods in European countries (see \citet{Vilhuber2004} for an
overview of such methods for some European countries and the US).


%%% Local Variables: 
%%% mode: latex
%%% TeX-master: "qwi-overview"
%%% End: 

%TCIDATA{LaTeXparent=0,0,sw-edit.tex}
                      

% -*- latex -*- 
%
% Time-stamp: <02/05/13 23:58:14 vilhuber> 
%              Automatically adjusted if using Xemacs
%              Please adjust manually if using other editors
%
% icf_technical.tex
% Responsible: Fredrik/Martha
% Part of QWI_methods.tex

\index{ICF|(mainref}

\section{Overview}
\label{sec:icf_technical_overview}


This chapter describes how to process the ICF and how to verify that the
correct output is being produced. The purpose of the documentation is to
provide enough guidelines for someone with no prior experience to process
the ICF version 2.2%
%
\footnote{ICF version 2.2 is used in QWI versions 2.2 and 2.3.}
%
 in the absence of the normally responsible persons. The
documentation also contains information, although not in great detail, on
what each of the included programs in the program sequence does. This
information should hopefully be sufficient to at least direct the reader to
where in the program sequence any modifications should be applied if such
should become necessary.

\section{Versions of the ICF}

Up to three different versions of the ICF coexist for each of the states
that have been processed. The differences between version 2.0 and version
2.2 are that the latter rely on updated Numident data, the latest available
UI data, and geographical data from the STARs system. However, no further
processing of the geographical data takes place in version 2.2. Version 3.0
uses the very same input data as version 2.2 does to further refine the
geographical information on the ICF. Thus, the information contained in
version 2.2 is a strict subset of the information contained in version 3.0
and, hence, could be viewed as an intermediate product of version 3.0 . As
such, version 2.2 is to be deleted once version 3.0 has been finalized 
\underline{and} it has been verified that version 2.2 data indeed is a
strict subset of version 3.0. Version 2.2 is located in
''/data/master/Individual/v2.2/sasdata/'' and should be accessible from
''/data/master/Individual/current/sasdata/'', if it is the most current
version of the ICF\ available. To understand why version 2.2 is a necessary
intermediate product we need to understand what the input data are of each
version and how the ICF is related to the other products of LEHD.

\section{Input data}

Version 2.2 uses the Employment History File as one of its inputs and the
processing of version 2.2 can normally be initiated as soon as the EHF
exists. (Before processing the CPR and the STARs data must also have been
updated to include the specific state, but that can be done independently of
the status of EHF.) In addition, version 3.0 uses the Employer
Characteristics File as one of its inputs. However, the ECF uses the ICF
version 2.2 as one of its inputs and, hence, version 2.2 is a necessary
intermediate product.

\section{Program sequence}

The generic program `00.00.icfv2.2.sas' is located in
`/programs/projects/icf/icfv2.2/programs'. The program is completely
state-independent and all intermediate output files are deleted as soon as
possible in the program. It uses `/saswork4' as its working directory,
outputs `indmast{\&}state' and 'implicates{\&}state' to
`/data/working/individual/ui{\&}state.{\_}rev2/', and creates an index for
`pik' in `indmast{\&}state'. This program in turn includes several other
programs described briefly later on. Unless any modifications are needed
this program (or any if its underlying programs) should not be changed.

\begin{description}
\item The following is a step-by-step list of what needs to be done in order
to create the ICF version 2.2 for any state:

\begin{steps}
\item Create a directory in /programs/projects/icf/icfv2.2/{\&}state/ and
copy the program '01.01.icfv2.2.sas' from
`/programs/projects/icf/icfv2.2/programs' to that directory. This is the
program that contains all state-specific parameters and which calls upon
`00.00.icfv2.2.sas' when executed. Only the two letter state abbreviation 
\texttt{{\&}state} needs to be set, all other parameters are parsed from
global configuration files, in particular

/data/master/ui/{\&}state/conversions/config{\&}state.sas

However, visual verification of the time span defined in that file should be
done with the EHF and the Data Availability File (daf.xls in /data/doc).

\item Make sure enough space is available on `/saswork4' and `/data/master'.
If there is not (i.e. if `/saswork4/' has less than about 20 percent
available space), it's probably time to erase some temporary files. If that
is not possible change the libname `mywork' in `00.00.icfv2.2.sas' to
another directory with enough space, after having consulted the owner of the
alternative saswork directory specified.

\item Before executing '01.01.icfv2.2.sas' you probably want to make sure
that global macros and libnames are updated in `00.00.icfv2.2.sas'.

\item Execute '01.01.icfv2.2.sas'. (For California the cpu-time is about 6
hours and for Minnesota about 1 hour.)

\item Verify that the output is correct in the log and list file of
'01.01.icfv2.2.sas' (see Section~\Vref{icf_tech:what1}).

\item Create a symbolic link in
''/data/master/Individual/current/sasdata/''\ between
''indmast\&state..sas7bdat'' and ''indmast\&state..sas7bndx, on the on hand,
and ''/data/master/Indiviudual/v2.2/sasdata/indmast\&state..sas7bdat'' and
''/data/master/Indiviudual/v2.2/sasdata/indmast\&state..sas7bndx'', on the
other hand, assuming that v2.2 is the most current version of the ICF\
available for that state.\ 

\item Update `daf.xls' to reflect the fact that the ICF version 2.2 exists.

\item Notify the rest of the LEHD by e-mail that the ICF version 2.2 is done.
\end{steps}
\end{description}

\section{Description of sub-programs}

%The following is based on the documentation by Martha in /data/master/doc/public/Project-files.
The following is a description of sub-programs called from the main program
`00.00.icfv2.2.sas'. 
All of the list files associated with each of the included programs have
``Proc Contents'' printed in them to provide documentation on the exact
names and formats of variables. 

\paragraph{00.01.icfv2.2.sas}

This program reads in the Employment History File and counts the number of
unique SEINs with positive earnings for each individual in each year and
year/quarter. The resulting data set, persum01.sas7bdat contains the
variables

\begin{itemize}
\item \textit{PIK},

\item \textit{SY{\&}uifirstyear.ui{\&}uifirstquarter.}~-- \textit{SY{\&}%
uilastyear.ui{\&}uilastquarter.},

\item \textit{SPTNMF{\&}uifirstyear.}~-- \textit{SPTNMF{\&}uilastyear.},

\item \textit{source} (=FIPS code for {\&}state.), and

\item \textit{sourcetp} (=UI).
\end{itemize}

There is one observation per individual who ever worked in {\&}state. during
the time period {\&}uifirstyear.~-- {\&}uilastyear.. When there were no
employers in a particular quarter, the employer count variable, \textit{SY{\&%
}year.{\&}quarter.}, is set to zero. When the quarter was out of range (pre-{%
\&}uifirstyear. and post-{\&}uilastyear.), the employer count variables are
set to missing. The range of \textit{SY{\&}year.{\&}quarter.} is 1985
quarter 1 to 2001 quarter 4. (The range is defined in '00.00.icfv2.2.sas'.)

\paragraph{00.02.icfv2.2.sas}

This program reads in raw numident and STAR data and creates date of birth
and date of death sas-date format variables. Race, gender, place of birth,
and citizenship variables are labeled and the data is sorted by \textit{PIK}
in preparation for the merge with the UI data. The resulting data set is
persum02.sas7bdat.

\paragraph{00.03.icfv2.2.sas}

This program merges persum01.sas7bdat and persum02.sas7bdat by \textit{PIK}.
The resulting file is persum03.sas7bdat.

\paragraph{00.04.icfv2.2.sas}

This program matches on CPS household and person ID variables. The PIK/CPSID
crosswalk

\begin{center}
/data/working/individual/cpsuicw2.ssd04
\end{center}

was created by a series of programs in /programs/projects/cps/linkuimd.
These programs take the raw crosswalks, standardize them across years, and
deal with duplicates. Please see
/programs/projects/cps/linkuimd/jobsequence.txt  for details. This crosswalk
is merged with persum03.sas7bdat to form linkcps1.sas7bdat. Variables added
from crosswalk include \textit{hid1}, \textit{pposold1}, \textit{yearcps1}
and various indicators of duplicates. Since numerous people are in the March
CPS twice, variables \textit{hid2}, \textit{pposold2}, and \textit{yearcps2}
also exist.

\paragraph{00.05.icfv2.2.sas}

This program matches on SIPP person id variables. The PIK/SIPPID crosswalk

\begin{center}
/data/working/individual/sipp/xwalk/iu-pu-pikxwallyears\_final.sas7bdat
\end{center}

was created by a separate series of programs in
/programs/projects/sipp/xwalk. These programs standardize across years and
deal with duplicates. Variables added from the crosswalk include \textit{%
intid} (internal SIPP ID), panel year in the SIPP, and various indicators of
duplicates. {ntid} is a 14 character (19 character for 1996) identifier
which is created by concatenating \textit{pp\_psu} \textit{pp\_seg} \textit{%
pp\_ser} \textit{pp\_entry} \textit{pp\_pnum}. This crosswalk is merged with
linkcps1.sas7bdat to form linksip1.sas7bdat. Variables are then re-named and
labeled, extraneous variables are dropped, and the resulting file,
persum04.sas7bdat is indexed by \textit{PIK}.

\paragraph{00.06.icfv2.2.sas}

The output data, cps{\&}state.sas7bdat, contains the variables {IK}, \textit{%
a\_age}, \textit{a\_sex} and \textit{yearcps} from the CPS for those
individuals that match to the CPS.

\paragraph{00.07.icfv2.2.sas}

This program uses the employment history file

\begin{center}
/data/master/Employment\_history/sasdata/emphis{\&}state..sas7bdat
\end{center}

as its input  and calculates yearly earnings totals for each \textit{PIK} by
summing over all the jobs in a given year. Individuals who never have
positive earnings are dropped. The resulting data set is impute01.sas7bdat.

\paragraph{00.08.icfv2.2.sas}

This program merges impute01.sas7bdat and persum4.sas7bdat to form
impute02.sas7bdat. This program prepares variables to be used in the age and
gender impute, calculates age at first job (first job is defined as the
first time an individual appears in the UI wage records), and edits the 
\textit{dob} variable:

\begin{itemize}
\item Workers who are 101 years or older have 100 years added to their birth
year and 100 years subtracted from their age.

\item Workers ages 0 or younger and 90 or older (but still less than 101)
have their \textit{dob} and \textit{age} variables set to missing.
\end{itemize}

Then workers are assigned to one of nine age categories: 1-19, 20-29, 30-39,
40-49, 50-59, 60-69, 70-79, 80-89. Workers ages 14-89 or with missing ages
are output to impute148902.sas7bdat. All workers regardless of age are
output to imputeall02.sas7bdat. The first data set will be used in the logit
model which predicts age category. The second data set will be used in the
logit model which predicts gender. The program finishes by doing ``proc
freqs'' to get the distribution of ages within each age category and
outputing each distribution to a data set called freqagecatx.sas7bdat.

\paragraph{00.09.icfv2.2.sas}

This file takes the data sets freqagecat1.sas7bdat-freqagecat8.sas7bdat,
appends them all, and then creates a data set which is one observation long
and contains an array, \textit{agecprob}$\lbrace$8,10$\rbrace$, which
contains the probabilities of being less than or equal to a certain age
within each agecategory. The resulting data set, impute02freq.sas7bdat, will
be used in the age impute in impute05b.sas.

\paragraph{00.10.icfv2.2.sas}

This program uses impute148902.sas7bdat to run a multinomial logit model
which predicts the probability of an individual being in each of the 8 age
categories. Right-hand side variables are indicators for whether an
individual had a job in a given year and total earnings and total earnings
squared for each year. Results are output to impute03a.sas7bdat which
contains 7 observations per \textit{PIK}. Each observation contains the
variable \textit{agecathat} which is the probability that the person falls
into a given age group or younger.

\paragraph{00.11.icfv2.2.sas}

This program takes the data set impute03a.sas7bdat and collapses it into a
data set with only one observation per \textit{PIK}. At the same time it
creates an array of the cumulative probabilities, \textit{cprob1}- \textit{%
cprob7}, to be used in the age impute.

\paragraph{00.12.icfv2.2.sas}

This program uses imputeall02.sas7bdat to run a logit model where the
probability of being male is predicted. The output data set,
impute04.sas7bdat, contains all the original variables plus \textit{malehat}
- the probability of being male.

\paragraph{00.13.icfv2.2.sas}

This file merges impute04.sas7bdat, impute03b.sas7bdat, and cps{\&}%
state.sas7bdat. The output data set is called impute05a.sas7bdat.

\paragraph{00.14.icfv2.2.sas}

This program uses impute05a.sas7bdat to finally impute age and gender where
it is missing. First impute02freq.sas7bdat is merged on and then four arrays
of random numbers are generated for each individual. \textit{agecatrandom}
will be used to assign the age category, \textit{agerandom} will be used to
assign the age, \textit{dayrandom} will be used to assign the day of the
year for the dob variable, and \textit{x} will be used to assign gender. The
file has 9 loops to cover the 9 possibilities:

\begin{enumerate}
\item nothing missing

\item \textit{dob} not missing/\textit{sex} from CPS

\item \textit{dob} not missing/\textit{sex} missing,

\item \textit{dob} from CPS/\textit{sex} not missing,

\item \textit{dob} from CPS/\textit{sex} from CPS,

\item \textit{dob} from CPS/\textit{sex} missing,

\item \textit{dob} missing/\textit{sex} not missing,

\item \textit{dob} missing/\textit{sex} from CPS

\item \textit{dob} missing/\textit{sex} missing.
\end{enumerate}

When \textit{age} comes from the CPS, \textit{birthyear} is calculated, then 
\textit{age} at first job, and finally using the \textit{dayrandom}
variable, a random day of the year is assigned and the \textit{dob} variable
is created. When \textit{age} is imputed completely, age category is imputed
first by comparing \textit{agecatrandom} to the values of \textit{cprob1}- 
\textit{cprob7} for that individual. Then age within the category \textit{m}
is imputed by comparing \textit{agerandom} to \textit{agecprob$\lbrace$m,1$%
\rbrace$}-\textit{agecprob}$\lbrace$m,9$\rbrace$. Finally day of the year is
imputed using \textit{dayrandom}, and \textit{dob} is created. When \textit{%
sex} is missing, the value is assigned by comparing \textit{x} to \textit{%
malehat}. The imputing is done 10 times for each missing value, creating 10
'implicates' for individuals with missing values. Implicate 1 is output to
impute05b.sas7bdat along with observations with no missing values. The other
9 implicates are output to implicates.sas7bdat.

\paragraph{00.15.icfv2.2.sas}

This program merges impute05b.sas7bdat and persum4.sas7bdat to create the
final master file. This is necessary because most of the variables were
dropped from persum4 at the beginning of the imputing process to keep the
data sets small. The output file, indmast{\&}state..sas7bdat is written to
''/data/master/Individual/v2.2/sasdata/''.


\section{Verification and quality control\label{icf_tech:what1}}

There is an extensive list-file associated with `01.01.icfv2.2.sas'. Every
output is included for a reason, so make sure to understand how to read the
output. I'm afraid it is hard to be specific and that some experience is
necessary. However, looking at old list and log files can serve as a partial
substitute for experience.

\begin{itemize}
\item Check the number of observations. It should coincide with the number
of unique `piks' in the UI data, and in the EHF.

\item The time-span of data should coincide with the dimensions of the EHF
(located in `/data/master/Employment{\_}history/sasdata'), but double check
with `/data/doc/daf.xls'. It is also defined in `/data/master/ui/{\&}%
state/config{\&}state'. If it does not coincide, notify the person
processing the EHF.

\item Check for any error messages.
\end{itemize}


\index{ICF|)mainref} 

%%% Local Variables: 
%%% mode: latex
%%% TeX-master: "QWI_methods"
%%% End: 
